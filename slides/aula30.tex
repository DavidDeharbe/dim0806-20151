%% TODO: illustrations
\documentclass{beamer}
\setbeamertemplate{footline}[frame number]

\usepackage[utf8x]{inputenc}
\usepackage[brazil,british]{babel}
\usetheme{default} 
\usecolortheme{beaver}
\newtranslation[to=brazil]{Theorem}{Teorema}
\newtranslation[to=brazil]{Definition}{Definição}
\newtranslation[to=brazil]{Example}{Exemplo}
\newtranslation[to=brazil]{Problem}{Exercício}
\newtranslation[to=brazil]{Solution}{Resolução}
\newtranslation[to=brazil]{Lemma}{Lema}
\newtranslation[to=brazil]{Corollary}{Corolário}
\logo{\includegraphics[width=1cm]{../img/logo-ppgsc-icon-text.png}}

\usepackage{graphicx}
\usepackage{clrscode3e}
\usepackage{hyperref}

\usepackage{pgf}
\usepackage{tikz}
\newcommand{\assert}[1]{\textcolor{blue}{#1}}

\usepackage{xspace}

\newcommand{\semester}[0]{2015.2}


\title{Aula 30: Problemas P, NP, NP-árduos, e NP-completos}
\subtitle{Uma breve introdução}
\author{David Déharbe \\
  Programa de Pós-graduação em Sistemas e Computação \\
  Universidade Federal do Rio Grande do Norte \\
  Centro de Ciências Exatas e da Terra \\
  Departamento de Informática e Matemática Aplicada}
\date{}

\newcommand{\classP}[0]{\ensuremath{\mathcal{P}}\xspace}
\newcommand{\classNP}[0]{\ensuremath{\mathcal{NP}}\xspace}
\newcommand{\classcoNP}[0]{\ensuremath{\mbox{co-$\mathcal{NP}$}}\xspace}

\begin{document}
\selectlanguage{brazil}

%%%%% SLIDE 01 %%%%%

\begin{frame}
  \titlepage

  Download me from \url{http://DavidDeharbe.github.io}
\end{frame}

\begin{frame}
  \frametitle{Plano}

  \tableofcontents
Referência: Cormen, cap 36.
\end{frame}

\section{Introdução}

%%%%% SLIDE 02 %%%%%

\begin{frame}
\frametitle{Introdução}

\begin{itemize}

\item Considere os seguintes problemas:

\begin{itemize}
\item Determinar se um grafo $G=(V, E)$ é \alert{Hamiltoniano}:
  $\exists$ uma sequência de arestas que visita cada vértice
  exatamente uma vez, e volta ao vértice inicial

\item Dado um circuito combinacional, formado por portas lógicas (\textit{and},
  \textit{or}, \textit{not}), com $n$ entradas e uma única saída, determine se
  existe uma combinação de valores na entrada tal que a saída é 1.

\item Dado um conjunto de números $V_1, V_2, \ldots V_n$, e um valor $N$,
  determine se existe um subconjunto cuja soma é $V$.

\end{itemize}

\item Para nenhum desses problemas conhece-se um algoritmo de complexidade
  polinomial

\item Se existe um algoritmo de complexidade polinomial para um,
  então existe um algoritmo polinomial para os demais.

\item Qual é a teoria envolvida nesses resultados surpreendentes?

\item Classes de problemas \classP, \classNP; problemas
  \classNP-completos; problemas \classNP-árduos.

\end{itemize}

\end{frame}

%%%%% SLIDE 03 %%%%%

\begin{frame}
\frametitle{Princípios}

\begin{itemize}
\item É considerado \alert{tratável} um problema para o qual existe um
  algoritmo com complexidade polinomial ($O(n^k)$)
  \begin{itemize}
  \item na prática $k$ é geralmente ``pequeno''
  \end{itemize}
\item É considerado \alert{intratável} um problema para o qual só são conhecidos
  algoritmos com complexidade não polinomial ($O(k^n)$)
\item Motivação
  \begin{itemize}
  \item abordagens heurísticas
  \item consciência desta barreira teórica (provável)
  \end{itemize}
\end{itemize}

\end{frame}

\section{Arcabouço teórico}
%%%%% SLIDE 04 %%%%%

\begin{frame}
\frametitle{Arcabouço teórico}

\begin{itemize}
\item problemas de \alert{decisão}
  \begin{itemize}
  \item saída: sim / não
  \item exemplo (circuito combinacional): existe uma valoração das entradas tal
    que a saída é 1.
  \end{itemize}
\item instâncias de problemas são \alert{codificadas}, digamos em binário
  \begin{itemize}
  \item (prática: entrada é uma sequência de bits)
  \item complexidade é função do tamanho desta codificação
  \end{itemize}
\item um conjunto de instâncias
  \begin{itemize}
    \item $\equiv$ um conjunto de palavras binárias
    \item $\equiv$ uma linguagem sobre $\{0, 1\}$
  \end{itemize}
\item a cada problema de decisão $P$ corresponde uma linguagem $L$
\item decidir uma instância $I$ de $P$
  \begin{itemize}
  \item $\equiv$ decidir se uma palavra pertence à linguagem
  \end{itemize}
\end{itemize}

\end{frame}

\section{Arcabouço teórico}
%%%%% SLIDE 04 %%%%%

\begin{frame}
\frametitle{Problemas de decisão}

\begin{itemize}
\item E se o problema não for de decisão?

\item Geralmente existe um problema de decisão relacionado

\item Exemplo
  \begin{description}
  \item[original] Qual o tamanho do menor caminho entre dois
    vértices $u$ e $v$ de um grafo sem pesos?
  \item[decisão] Existe um caminho de $u$ até $v$ passando por
    exatamente $k$ arestas?
  \end{description}

\end{itemize}

\end{frame}

\section{A classe de problemas \classP}
%%%%% SLIDE 04 %%%%%

\begin{frame}
\frametitle{Decisão e aceitação}

\begin{definition}[Decisão]
O algoritmo $A$ \emph{decide} $L$ quando, dado uma palavra $x$, ou $A$ aceita $x$, ou $A$ rejeita $x$: $A$ termina e retorna $1$ ou $0$.
\end{definition}

\begin{definition}[Decisão em tempo polinomial]
O algoritmo $A$ \emph{decide} $L$ em tempo polinomial quando existe $k$ tal que,
dado $x$ uma palavra de tamanho $n$, $A$ decide se $x \in L$ em $O(n^k)$.
\end{definition}

\begin{definition}[Aceitação]
O algoritmo $A$ \emph{aceita} $L$ quando, dado uma palavra $x \in L$, $A$ aceita
$x$. Se $x \not\in L$, ou $A$ rejeita $x$, ou $A$ não termina.
\end{definition}

\begin{definition}[Aceitação em tempo polinomial]
O algoritmo $A$ \emph{aceita} $L$ em tempo polinomial quando existe $k$ tal que,
dado $x \in L$ uma palavra de tamanho $n$, $A$ aceita $x$ em $O(n^k)$.
\end{definition}

\pause
\begin{itemize}
\item Decidir é mais ``difícil'' que aceitar?
\end{itemize}

\end{frame}

%%%%% SLIDE 04 %%%%%

\begin{frame}
\frametitle{A classe de problemas \classP}

\begin{definition}[A classe de problemas \classP]
A classe de problemas de decisão polinomiais é $\classP = \{ L
\subseteq \{0, 1\}^* \mbox{ existe um algoritmo que decide $L$ em
  tempo polinomial} \}$.
\end{definition}

\begin{theorem}
A classe de problemas $\classP$ é a classe de problemas que são aceitos em tempo
polinomial.
\end{theorem}

\end{frame}

\section{A classe de problemas \classNP}

%%%%% SLIDE 04 %%%%%

\begin{frame}
\frametitle{Verificação}

\begin{definition}[Verificação]
O algoritmo $A$ \emph{verifica} $L$ quando, dado uma palavra $x \in L$, e um certificado $y$, $A(x,y) = 1$.
\end{definition}
O algoritmo de verificação confere se uma instância $x$ pertence à linguagem com
base uma evidência $y$.

\begin{definition}[A classe de problemas \classNP]
A classe de problemas \classNP é a classe de problemas $L$ tais que existe um
algoritmo de verificação polinomial.
\end{definition}

\begin{definition}[A classe de problemas \classcoNP]
A classe de problemas \classcoNP é a classe de problemas $L$ tais que
existe um algoritmo de verificação polinomial para $\bar{L}$.
\end{definition}

\end{frame}

\begin{frame}
\frametitle{Relação entre classes de problemas}

\begin{itemize}
\item $\classP \subseteq \classNP$, $\classP \subseteq \classcoNP$
\item \alert{$\classP = \classNP$?}
\item \alert{$\classNP = \classcoNP$?}
\item \alert{$\classP = \classNP \cap \classcoNP$?}
\end{itemize}

\end{frame}

\section{NP-completeza}

\subsection{Redução}

\begin{frame}
\frametitle{Redução entre problemas}

\begin{definition}[Redução entre problemas]
Um problema (linguagem) $L_1$ é \alert{redutível polinomialmente} em um
problema $L_2$ se existe uma função $f$ calculável em tempo polinomial
tal que $\forall x \in \{0, 1\}^* \cdot x \in L_1 \iff f(x) \in L_2$.
\end{definition}

\begin{itemize}
\item Solucionar $L_1$ não é mais difícil que solucionar $L_2$.
\item Notação: $L_1 \le_P L_2$
\end{itemize}
\end{frame}

\begin{frame}
\frametitle{Redução e $\classP$}

\begin{lemma}
Seja $L_1, L_2 \subseteq \{ 0, 1 \}^*$, tais que $L_1 \le_P L_2$. Então
$L_2 \in \classP \Rightarrow L_1 \in \classP$.
\end{lemma}

\end{frame}

\subsection{Definição e exemplo}

\begin{frame}
\frametitle{$\classNP$-completeza}

\begin{definition}[\classNP-árduo]
$L \subseteq \{ 0, 1 \}^*$ é $\classNP$-árduo se, para cada $L' \in \classNP$, $L' \le_P L$.
\end{definition}

\begin{definition}[$\classNP$-completo]
$L \subseteq \{ 0, 1 \}^*$ é $\classNP$-completo se
\begin{enumerate}
\item $L \in \classNP$, e
\item $L' \le_P L$, para cada $L' \in \classNP$ ($L$ é $NP$-árduo)
\end{enumerate}
\end{definition}

\begin{itemize}
\item O problema \textsc{Sat} de determinar se um dado circuito lógico
  pode ter a sua saída setada a um é um problema \classNP-completo.
\item Há centenas de outros problemas computacionais que foram
  mostrados como send \classNP-completos.
\end{itemize}

\end{frame}

\end{document}

