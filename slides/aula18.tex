\documentclass{beamer}
\setbeamertemplate{footline}[frame number]
%\documentclass{beamer}

\usepackage[utf8x]{inputenc}
\usepackage[brazil,british]{babel}
\usetheme{default} 
\usecolortheme{beaver}
\newtranslation[to=brazil]{Theorem}{Teorema}
\newtranslation[to=brazil]{Definition}{Definição}
\newtranslation[to=brazil]{Example}{Exemplo}
\newtranslation[to=brazil]{Problem}{Exercício}
\newtranslation[to=brazil]{Solution}{Resolução}
\newtranslation[to=brazil]{Lemma}{Lema}
\newtranslation[to=brazil]{Corollary}{Corolário}
\logo{\includegraphics[width=1cm]{../img/logo-ppgsc-icon-text.png}}

\usepackage{graphicx}
\usepackage{clrscode3e}
\usepackage{hyperref}

\usepackage{pgf}
\usepackage{tikz}
\newcommand{\assert}[1]{\textcolor{blue}{#1}}

\usepackage{xspace}

\newcommand{\semester}[0]{2015.2}


\title{Aula 17: Árvores rubro-negras}
\author{David Déharbe \\
  Programa de Pós-graduação em Sistemas e Computação \\
  Universidade Federal do Rio Grande do Norte \\
  Centro de Ciências Exatas e da Terra \\
  Departamento de Informática e Matemática Aplicada}
\date{22 de abril de 2015}

\newcommand{\negro}[1]{\colorbox{black}{\textcolor{white}{\textbf{#1}}}}
\newcommand{\rubro}[1]{\colorbox{red}{\textcolor{white}{\textbf{#1}}}}
\newcommand{\bh}[0]{\ensuremath{\id{bh}}}


\begin{document}
\selectlanguage{brazil}
\begin{frame}
  \titlepage
  Download me from \url{http://ufrn.academia.edu/DavidDeharbe}.
\end{frame}

\begin{frame}
  \frametitle{Plano da aula}


\begin{center}
\setlength{\unitlength}{0.4cm}
\begin{picture}(13,12.5)(0,0)

\put(6,11.2){\makebox(0,0)[b]{\negro{47}}}
\put(6,11){\circle*{.2}}
\put(6,11){\line(-4,-1){4}}
\put(6,11){\line(4,-1){4}}

\put(2,10.2){\makebox(0,0)[b]{\negro{32}}}
\put(2,10){\circle*{.2}}
\put(2,10){\line(-1,-1){2}}
\put(2,10){\line(1,-1){2}}

\put(10,10.2){\makebox(0,0)[b]{\rubro{68}}}
\put(10,10){\circle*{.2}}
\put(10,10){\line(-1,-1){2}}
\put(10,10){\line(1,-1){2}}

\put(0,8.2){\makebox(0,0)[b]{\negro{5}}}
\put(0,8){\circle*{.2}}
\put(0,8){\line(1,-4){1}}

\put(4,8.2){\makebox(0,0)[b]{\negro{40}}}
\put(4,8){\circle*{.2}}

\put(8,8.2){\makebox(0,0)[b]{\negro{60}}}
\put(8,8){\circle*{.2}}
\put(8,8){\line(-1,-4){1}}
\put(8,8){\line(1,-4){1}}

\put(12,8.2){\makebox(0,0)[b]{\negro{88}}}
\put(12,8){\circle*{.2}}
\put(12,8){\line(1,-4){1}}
\put(12,8){\line(-1,-4){1}}

\put(1,4.2){\makebox(0,0)[b]{\rubro{15}}}
\put(1,4){\circle*{.2}}

\put(7,4.2){\makebox(0,0)[b]{\negro{54}}}
\put(7,4){\circle*{.2}}
\put(7,4){\line(-1,-4){1}}

\put(9,4.2){\makebox(0,0)[b]{\negro{61}}}
\put(9,4){\circle*{.2}}

\put(11,4.2){\makebox(0,0)[b]{\negro{75}}}
\put(11,4){\circle*{.2}}

\put(13,4.2){\makebox(0,0)[b]{\negro{90}}}
\put(13,4){\circle*{.2}}

\put(6,0.2){\makebox(0,0)[b]{\rubro{50}}}
\put(6,0){\circle*{.2}}
\end{picture}
\end{center}

  \tableofcontents

\end{frame}

\section{Introdução}

\begin{frame}

  \frametitle{Árvores rubro-negra}
  \framesubtitle{Introdução}

  
  \begin{itemize}
    
  \item Em 1972, Rudolf Bayer projetou a estrutura de dados
    \alert{árvores B binárias simétricas}.

  \item Popularizada sob o nome árvores rubro-negras (Guibas \& Sedgewick 1978)

  \item Complexidade
    \begin{itemize}
    \item busca em $O(\lg n)$, 
    \item remoção em $\Theta(\lg n)$ e
    \item inserção em $\Theta(\lg n)$
    \end{itemize}

  \item Menos restritivas que árvores AVL, com alturas tendendo a ser maiores

    \begin{itemize}

      \item busca: menos eficiente

      \item inserção e remoção: mais eficiente

    \end{itemize}

  \item bastante usadas na prática (escalonador Linux, STL)

  \end{itemize}

\end{frame}

\section{Propriedades}

\begin{frame}

\frametitle{Árvores rubro-negra}
\framesubtitle{Especificação}

\begin{enumerate}

\item árvore binária de busca

\item os nós tem um atributo $\id{color} \in \{ \const{Red}, \const{Black} \}$.

Convenção: $\attrib{\const{Nil}}{color} = \const{Black}$

\item A raiz é negra $\attrib{root}{color} = \const{Black}$

\item Qualquer caminho da raiz até uma sub-árvore vazia tem o mesmo número de
  nós negros.
  \begin{eqnarray*}
    \id{bh}(\const{Nil}) & = & 0 \\
    \id{bh}(x) & = & 1 + \id{bh}(\attrib{x}{left}) \text{ se } \attrib{\attrib{x}{left}}{color} = \const{Black} \\
    \id{bh}(x) & = & \id{bh}(\attrib{x}{left}) \text{ se } \attrib{\attrib{x}{left}}{color} = \const{Red}
  \end{eqnarray*}

\item Os descendentes de um nó rubro são negros: 
$\attrib{x}{color} = \const{Red} \Rightarrow \attrib{\attrib{x}{left}}{color} = \const{Black} \land \attrib{\attrib{x}{right}}{color} = \const{Black}$.

\end{enumerate}

\end{frame}

\begin{frame}

\frametitle{Ilustração}

\begin{center}
\begin{tabular}{ccc}
\setlength{\unitlength}{0.35cm}
\begin{picture}(13,12.5)(0,0)

\put(6,11.2){\makebox(0,0)[b]{\negro{47}}}
\put(6,11){\circle*{.2}}
\put(6,11){\line(-4,-1){4}}
\put(6,11){\line(4,-1){4}}

\put(2,10.2){\makebox(0,0)[b]{\negro{32}}}
\put(2,10){\circle*{.2}}
\put(2,10){\line(-1,-1){2}}
\put(2,10){\line(1,-1){2}}

\put(10,10.2){\makebox(0,0)[b]{\rubro{68}}}
\put(10,10){\circle*{.2}}
\put(10,10){\line(-1,-1){2}}
\put(10,10){\line(1,-1){2}}

\put(0,8.2){\makebox(0,0)[b]{\negro{5}}}
\put(0,8){\circle*{.2}}
\put(0,8){\line(1,-4){1}}

\put(4,8.2){\makebox(0,0)[b]{\negro{40}}}
\put(4,8){\circle*{.2}}

\put(8,8.2){\makebox(0,0)[b]{\negro{60}}}
\put(8,8){\circle*{.2}}
\put(8,8){\line(-1,-4){1}}
\put(8,8){\line(1,-4){1}}

\put(12,8.2){\makebox(0,0)[b]{\negro{88}}}
\put(12,8){\circle*{.2}}
\put(12,8){\line(1,-4){1}}
\put(12,8){\line(-1,-4){1}}

\put(1,4.2){\makebox(0,0)[b]{\rubro{15}}}
\put(1,4){\circle*{.2}}

\put(7,4.2){\makebox(0,0)[b]{\negro{54}}}
\put(7,4){\circle*{.2}}
\put(7,4){\line(-1,-4){1}}

\put(9,4.2){\makebox(0,0)[b]{\negro{61}}}
\put(9,4){\circle*{.2}}

\put(11,4.2){\makebox(0,0)[b]{\negro{75}}}
\put(11,4){\circle*{.2}}

\put(13,4.2){\makebox(0,0)[b]{\negro{90}}}
\put(13,4){\circle*{.2}}

\put(6,0.2){\makebox(0,0)[b]{\rubro{50}}}
\put(6,0){\circle*{.2}}
\end{picture}

& \hspace{.2cm} &

\setlength{\unitlength}{0.35cm}
\begin{picture}(13,12.5)(0,0)

\put(6,11.2){\makebox(0,0)[b]{\negro{47}}}
\put(6,11){\circle*{.2}}
\put(6,11){\line(-4,-1){4}}
\put(6,11){\line(4,-1){4}}

\put(2,10.2){\makebox(0,0)[b]{\negro{32}}}
\put(2,10){\circle*{.2}}
\put(2,10){\line(-1,-1){2}}
\put(2,10){\line(1,-1){2}}

\put(10,10.2){\makebox(0,0)[b]{\rubro{68}}}
\put(10,10){\circle*{.2}}
\put(10,10){\line(-1,-1){2}}
\put(10,10){\line(1,-1){2}}

\put(0,8.2){\makebox(0,0)[b]{\negro{5}}}
\put(0,8){\circle*{.2}}
\put(0,8){\line(1,-4){1}}

\put(4,8.2){\makebox(0,0)[b]{\negro{40}}}
\put(4,8){\circle*{.2}}

\put(8,8.2){\makebox(0,0)[b]{\negro{60}}}
\put(8,8){\circle*{.2}}
\put(8,8){\line(-1,-4){1}}
\put(8,8){\line(1,-4){1}}

\put(12,8.2){\makebox(0,0)[b]{\negro{88}}}
\put(12,8){\circle*{.2}}
\put(12,8){\line(1,-4){1}}

\put(1,4.2){\makebox(0,0)[b]{\rubro{15}}}
\put(1,4){\circle*{.2}}

\put(7,4.2){\makebox(0,0)[b]{\negro{54}}}
\put(7,4){\circle*{.2}}
\put(7,4){\line(-1,-4){1}}

\put(9,4.2){\makebox(0,0)[b]{\negro{61}}}
\put(9,4){\circle*{.2}}

\put(13,4.2){\makebox(0,0)[b]{\negro{90}}}
\put(13,4){\circle*{.2}}

\put(6,0.2){\makebox(0,0)[b]{\rubro{50}}}
\put(6,0){\circle*{.2}}
\end{picture} \\
rubro-negra & & não rubro-negra

\end{tabular}
\end{center}
\end{frame}

\begin{frame}

\frametitle{Exercício}

Existe outras maneiras de colorir os nós da seguinte árvore de tal forma que
respeite as propriedades de árvores rubro-negras?

\begin{center}
\setlength{\unitlength}{0.35cm}
\begin{picture}(13,12.5)(0,0)

\put(6,11.2){\makebox(0,0)[b]{\negro{47}}}
\put(6,11){\circle*{.2}}
\put(6,11){\line(-4,-1){4}}
\put(6,11){\line(4,-1){4}}

\put(2,10.2){\makebox(0,0)[b]{\negro{32}}}
\put(2,10){\circle*{.2}}
\put(2,10){\line(-1,-1){2}}
\put(2,10){\line(1,-1){2}}

\put(10,10.2){\makebox(0,0)[b]{\rubro{68}}}
\put(10,10){\circle*{.2}}
\put(10,10){\line(-1,-1){2}}
\put(10,10){\line(1,-1){2}}

\put(0,8.2){\makebox(0,0)[b]{\negro{5}}}
\put(0,8){\circle*{.2}}
\put(0,8){\line(1,-4){1}}

\put(4,8.2){\makebox(0,0)[b]{\negro{40}}}
\put(4,8){\circle*{.2}}

\put(8,8.2){\makebox(0,0)[b]{\negro{60}}}
\put(8,8){\circle*{.2}}
\put(8,8){\line(-1,-4){1}}
\put(8,8){\line(1,-4){1}}

\put(12,8.2){\makebox(0,0)[b]{\negro{88}}}
\put(12,8){\circle*{.2}}
\put(12,8){\line(1,-4){1}}
\put(12,8){\line(-1,-4){1}}

\put(1,4.2){\makebox(0,0)[b]{\rubro{15}}}
\put(1,4){\circle*{.2}}

\put(7,4.2){\makebox(0,0)[b]{\negro{54}}}
\put(7,4){\circle*{.2}}
\put(7,4){\line(-1,-4){1}}

\put(9,4.2){\makebox(0,0)[b]{\negro{61}}}
\put(9,4){\circle*{.2}}

\put(11,4.2){\makebox(0,0)[b]{\negro{75}}}
\put(11,4){\circle*{.2}}

\put(13,4.2){\makebox(0,0)[b]{\negro{90}}}
\put(13,4){\circle*{.2}}

\put(6,0.2){\makebox(0,0)[b]{\rubro{50}}}
\put(6,0){\circle*{.2}}
\end{picture}
\end{center}

\end{frame}

\begin{frame}
\frametitle{Altura de árvores rubro-negras}

\begin{theorem}
Seja $T$ uma árvore rubro-negra, com $n$ nós, então $\alpha(\attrib{T}{root}) \in \Theta(\lg n)$.
\end{theorem}

\begin{proof}
\begin{itemize}
\item seja $p = \bh(T)$ (a altura negra de $T$)
\item há pelo menos $2^p - 1$ nós negros, logo $n \ge 2^p - 1$
\item a altura máxima $\alpha_m$ de $T$ é $2p$ (alternando nós rubros e negros)
\item $\alpha_m \le 2p$ e $n \ge 2^p - 1$, logo $\alpha_m \le 2 \log_2 (n+1)$.
\item em uma árvore binária $\alpha_m \in \Omega(\log n)$
\item logo $\alpha_m \in \Theta(\log n)$.
\end{itemize}
\end{proof}

\end{frame}

\begin{frame}

\frametitle{Implementação}

\begin{itemize}
\item $\attrib{x}{color} \in \{\const{Red}, \const{Black} \}$
\end{itemize}

\end{frame}

\section{Operações}

\begin{frame}

\frametitle{Operação de busca}

\begin{itemize}
\item A busca não modifica a árvore.

\item O algoritmo de busca em árvores rubro-negra é o mesmo da busca em árvores
  binárias de busca qualquer.

\item a complexidade é $O(\log n)$

\end{itemize}
\end{frame}

\begin{frame}

\frametitle{Operação de inserção}

\begin{enumerate}

\item inserção em árvore binária de busca

\item \alert{se necessário}, re-balanceamento

\end{enumerate}

\end{frame}

\begin{frame}

\frametitle{Re-balanceamento}
\framesubtitle{Inserção}

Propriedade 2: qual cor escolher para o novo nó?

\begin{itemize}

\item se for negro, e a árvore inicialmente não vazia, quebra a propriedade 4

\item se for rubro, e a árvore inicialmente vazia, quebra a propriedade 3

\end{itemize}

\pause
\alert{solução}

\begin{itemize}

\item é criado um nó, $q$

\item a cor defaut é rubro: $\attrib{q}{color} \gets \const{Red}$

\item após a inserção, a raiz é atribuída a cor negro
$\attrib{root}{color} \gets \const{Black}$

\end{itemize}

\end{frame}

\begin{frame}

\frametitle{Re-balanceamento}
\framesubtitle{Inserção}

Propriedade 5: nó rubro não pode ter descendente direto rubro

\begin{itemize}

\item seja $v$ o nó ascendente de $q$

\item se $v$ for negro, não ha desequilibro

\item se $v$ for rubro, a propriedade 5 é violada

\end{itemize}

\pause

\alert{solução}

\begin{itemize}

\item como $v$ é rubro, tem um nó ascendente, $w$, tal que $\attrib{w}{color} \isequal \const{Black}$

\item seja $t$ a outra sub-árvore de $w$

\begin{enumerate}

\item $\attrib{t}{color} \isequal \const{Red}$

\item $\attrib{t}{color} \isequal \const{Black}$

\end{enumerate}

\end{itemize}

\end{frame}

\begin{frame}

\begin{codebox}
\Procname{$\proc{Insert}(T, k)$}
\li \If $\attrib{T}{root} \isequal \const{Nil}$
\li \Then $\attrib{T}{root} \gets \proc{Make-Node}(k)$
\li \Else 
      $v \gets \const{Nil}$
\li   $w \gets \const{Nil}$
\li   $\proc{Insert-Aux}(\attrib{T}{root}, v, w, k)$
    \End
\li $\attrib{\attrib{T}{root}}{color} \gets \const{Black}$
\end{codebox}

\end{frame}


\begin{frame}

\frametitle{O caso $\attrib{t}{color} \isequal \const{Red}$}

$\attrib{t}{color} \isequal \const{Red}$

\begin{itemize}

\item logo $t \neq \const{Nil}$

\item inversão das cores de $v$, $w$ e $t$.
\end{itemize}

\begin{center}
\begin{tabular}{ccc}
(Antes) & & (Depois) \\
\\
\setlength{\unitlength}{0.35cm}
\begin{picture}(14,9)(0,0)

\put(7,8.2){\makebox(0,0)[b]{\negro{$w$}}}
\put(7,8){\circle*{.2}}
\put(7,8){\line(-4,-1){4}}
\put(7,8){\line(4,-1){4}}

\put(3,7.2){\makebox(0,0)[b]{\rubro{$v$}}}
\put(3,7){\circle*{.2}}
\put(3,7){\line(-1,-1){2}}
\put(3,7){\line(1,-1){2}}

\put(11,7.2){\makebox(0,0)[b]{\rubro{$t$}}}
\put(11,7){\circle*{.2}}
\put(11,7){\line(-1,-1){2}}
\put(11,7){\line(1,-1){2}}

\put(1,5.2){\makebox(0,0)[b]{\rubro{$q$}}}
\put(1,5){\circle*{.2}}
\put(1,5){\line(1,-4){1}}
\put(1,5){\line(-1,-4){1}}

\put(5,4.8){\makebox(0,0)[t]{\negro{$\gamma$}}}
\put(5,5){\circle*{.2}}

\put(9,4.8){\makebox(0,0)[t]{\negro{$\delta$}}}
\put(9,5){\circle*{.2}}

\put(13,4.8){\makebox(0,0)[t]{\negro{$\eta$}}}
\put(13,5){\circle*{.2}}

\put(2,0.8){\makebox(0,0)[t]{\negro{$\beta$}}}
\put(2,1){\circle*{.2}}

\put(0,0.8){\makebox(0,0)[t]{\negro{$\alpha$}}}
\put(0,1){\circle*{.2}}

\end{picture}
& & 
\setlength{\unitlength}{0.35cm}
\begin{picture}(14,9)(0,0)

\put(7,8.2){\makebox(0,0)[b]{\rubro{$w$}}}
\put(7,8){\circle*{.2}}
\put(7,8){\line(-4,-1){4}}
\put(7,8){\line(4,-1){4}}

\put(3,7.2){\makebox(0,0)[b]{\negro{$v$}}}
\put(3,7){\circle*{.2}}
\put(3,7){\line(-1,-1){2}}
\put(3,7){\line(1,-1){2}}

\put(11,7.2){\makebox(0,0)[b]{\negro{$t$}}}
\put(11,7){\circle*{.2}}
\put(11,7){\line(-1,-1){2}}
\put(11,7){\line(1,-1){2}}

\put(1,5.2){\makebox(0,0)[b]{\rubro{$q$}}}
\put(1,5){\circle*{.2}}
\put(1,5){\line(1,-4){1}}
\put(1,5){\line(-1,-4){1}}

\put(5,4.8){\makebox(0,0)[t]{\negro{$\gamma$}}}
\put(5,5){\circle*{.2}}

\put(9,4.8){\makebox(0,0)[t]{\negro{$\delta$}}}
\put(9,5){\circle*{.2}}

\put(13,4.8){\makebox(0,0)[t]{\negro{$\eta$}}}
\put(13,5){\circle*{.2}}

\put(2,0.8){\makebox(0,0)[t]{\negro{$\beta$}}}
\put(2,1){\circle*{.2}}

\put(0,0.8){\makebox(0,0)[t]{\negro{$\alpha$}}}
\put(0,1){\circle*{.2}}

\end{picture}
\end{tabular}
\end{center}

\end{frame}

\begin{frame}

\frametitle{O caso $\attrib{t}{color} \isequal \const{Red}$}

$\attrib{t}{color} \isequal \const{Red}$

\begin{center}
\begin{tabular}{ccc}
(Antes) & & (Depois) \\
\\
\setlength{\unitlength}{0.32cm}
\begin{picture}(14,9)(0,0)

\put(7,8.2){\makebox(0,0)[b]{\negro{$w$}}}
\put(7,8){\circle*{.2}}
\put(7,8){\line(-4,-1){4}}
\put(7,8){\line(4,-1){4}}

\put(3,7.2){\makebox(0,0)[b]{\rubro{$v$}}}
\put(3,7){\circle*{.2}}
\put(3,7){\line(-1,-1){2}}
\put(3,7){\line(1,-1){2}}

\put(11,7.2){\makebox(0,0)[b]{\rubro{$t$}}}
\put(11,7){\circle*{.2}}
\put(11,7){\line(-1,-1){2}}
\put(11,7){\line(1,-1){2}}

\put(1,5.2){\makebox(0,0)[b]{\rubro{$q$}}}
\put(1,5){\circle*{.2}}
\put(1,5){\line(1,-4){1}}
\put(1,5){\line(-1,-4){1}}

\put(5,4.8){\makebox(0,0)[t]{\negro{$\gamma$}}}
\put(5,5){\circle*{.2}}

\put(9,4.8){\makebox(0,0)[t]{\negro{$\delta$}}}
\put(9,5){\circle*{.2}}

\put(13,4.8){\makebox(0,0)[t]{\negro{$\eta$}}}
\put(13,5){\circle*{.2}}

\put(2,0.8){\makebox(0,0)[t]{\negro{$\beta$}}}
\put(2,1){\circle*{.2}}

\put(0,0.8){\makebox(0,0)[t]{\negro{$\alpha$}}}
\put(0,1){\circle*{.2}}

\end{picture}
& & 
\setlength{\unitlength}{0.32cm}
\begin{picture}(14,9)(0,0)

\put(7,8.2){\makebox(0,0)[b]{\rubro{$w$}}}
\put(7,8){\circle*{.2}}
\put(7,8){\line(-4,-1){4}}
\put(7,8){\line(4,-1){4}}

\put(3,7.2){\makebox(0,0)[b]{\negro{$v$}}}
\put(3,7){\circle*{.2}}
\put(3,7){\line(-1,-1){2}}
\put(3,7){\line(1,-1){2}}

\put(11,7.2){\makebox(0,0)[b]{\negro{$t$}}}
\put(11,7){\circle*{.2}}
\put(11,7){\line(-1,-1){2}}
\put(11,7){\line(1,-1){2}}

\put(1,5.2){\makebox(0,0)[b]{\rubro{$q$}}}
\put(1,5){\circle*{.2}}
\put(1,5){\line(1,-4){1}}
\put(1,5){\line(-1,-4){1}}

\put(5,4.8){\makebox(0,0)[t]{\negro{$\gamma$}}}
\put(5,5){\circle*{.2}}

\put(9,4.8){\makebox(0,0)[t]{\negro{$\delta$}}}
\put(9,5){\circle*{.2}}

\put(13,4.8){\makebox(0,0)[t]{\negro{$\eta$}}}
\put(13,5){\circle*{.2}}

\put(2,0.8){\makebox(0,0)[t]{\negro{$\beta$}}}
\put(2,1){\circle*{.2}}

\put(0,0.8){\makebox(0,0)[t]{\negro{$\alpha$}}}
\put(0,1){\circle*{.2}}

\end{picture}
\end{tabular}
\end{center}

\begin{enumerate}
\item se $w$ era a raiz, é atribuído a cor negro
\item se $w$ não era a raiz, opera-se eventual rebalanceamento em nível superior
\item a operação não alterou $\id{bh}(w)$
\end{enumerate}
\end{frame}

\begin{frame}

\frametitle{O caso $\attrib{t}{color} \isequal \const{Black}$}

$\attrib{t}{color} \isequal \const{Black}$

\begin{enumerate}
\item $\attrib{w}{left} \isequal v$ e $\attrib{v}{left} \isequal q$
\item $\attrib{w}{left} \isequal v$ e $\attrib{v}{right} \isequal q$
\item $\attrib{w}{right} \isequal v$ e $\attrib{v}{right} \isequal q$ (simétrico de 1)
\item $\attrib{w}{right} \isequal v$ e $\attrib{v}{left} \isequal q$ (simétrico de 2)
\end{enumerate}

\end{frame}

\begin{frame}

\frametitle{O caso $\attrib{t}{color} \isequal \const{Black}$ e $\attrib{w}{left} \isequal v$ e $\attrib{v}{left} \isequal q$}

$\attrib{t}{color} \isequal \const{Black}$ e $\attrib{w}{left} \isequal v$ e $\attrib{v}{left} \isequal q$

\begin{center}
\begin{tabular}{ccc}
(Antes) & & (Depois) \\
\\
\setlength{\unitlength}{0.35cm}
\begin{picture}(14,9)(0,0)
\put(7,8.2){\makebox(0,0)[b]{\negro{$w$}}}
\put(7,8){\circle*{.2}}
\put(7,8){\line(-4,-1){4}}
\put(7,8){\line(4,-1){4}}

\put(3,7.2){\makebox(0,0)[b]{\rubro{$v$}}}
\put(3,7){\circle*{.2}}
\put(3,7){\line(-1,-1){2}}
\put(3,7){\line(1,-1){2}}

\put(11,6.8){\makebox(0,0)[t]{\negro{$\delta$}}}
\put(11,7){\circle*{.2}}

\put(1,5.2){\makebox(0,0)[b]{\rubro{$q$}}}
\put(1,5){\circle*{.2}}
\put(1,5){\line(1,-4){1}}
\put(1,5){\line(-1,-4){1}}

\put(5,4.8){\makebox(0,0)[t]{\negro{$\gamma$}}}
\put(5,5){\circle*{.2}}

\put(2,0.8){\makebox(0,0)[t]{\negro{$\beta$}}}
\put(2,1){\circle*{.2}}

\put(0,0.8){\makebox(0,0)[t]{\negro{$\alpha$}}}
\put(0,1){\circle*{.2}}

\end{picture}
& & 
\setlength{\unitlength}{0.35cm}
\begin{picture}(14,9)(0,0)

\put(7,8.2){\makebox(0,0)[b]{\negro{$v$}}}
\put(7,8){\circle*{.2}}
\put(7,8){\line(-4,-1){4}}
\put(7,8){\line(4,-1){4}}

\put(3,7.2){\makebox(0,0)[b]{\rubro{$q$}}}
\put(3,7){\circle*{.2}}
\put(3,7){\line(-1,-1){2}}
\put(3,7){\line(1,-1){2}}

\put(11,7.2){\makebox(0,0)[b]{\rubro{$w$}}}
\put(11,7){\circle*{.2}}
\put(11,7){\line(-1,-1){2}}
\put(11,7){\line(1,-1){2}}

\put(1,4.8){\makebox(0,0)[t]{\negro{$\alpha$}}}
\put(1,5){\circle*{.2}}

\put(5,4.8){\makebox(0,0)[t]{\negro{$\beta$}}}
\put(5,5){\circle*{.2}}

\put(9,4.8){\makebox(0,0)[t]{\negro{$\gamma$}}}
\put(9,5){\circle*{.2}}

\put(13,4.8){\makebox(0,0)[t]{\negro{$\delta$}}}
\put(13,5){\circle*{.2}}

\end{picture}
\end{tabular}
\end{center}

rotação simples a direita + inversão cores de $v$ e $w$

\only<2>{O resultado é uma árvore rubro-negra? Justifique.}
\end{frame}

\begin{frame}

\frametitle{O caso $\attrib{t}{color} \isequal \const{Black}$ e $\attrib{w}{left} \isequal v$ e $\attrib{v}{right} \isequal q$}

$\attrib{t}{color} \isequal \const{Black}$ e $\attrib{w}{left} \isequal v$ e $\attrib{v}{right} \isequal q$

\begin{center}
\begin{tabular}{ccc}
(Antes) & & (Depois) \\
\\
\setlength{\unitlength}{0.35cm}
\begin{picture}(14,9)(0,0)

\put(7,8.2){\makebox(0,0)[b]{\negro{$w$}}}
\put(7,8){\circle*{.2}}
\put(7,8){\line(-4,-1){4}}
\put(7,8){\line(4,-1){4}}

\put(3,7.2){\makebox(0,0)[b]{\rubro{$v$}}}
\put(3,7){\circle*{.2}}
\put(3,7){\line(-1,-1){2}}
\put(3,7){\line(1,-1){2}}

\put(11,6.8){\makebox(0,0)[t]{\negro{$\delta$}}}
\put(11,7){\circle*{.2}}

\put(1,4.8){\makebox(0,0)[t]{\negro{$\alpha$}}}
\put(1,5){\circle*{.2}}

\put(5,5.2){\makebox(0,0)[b]{\rubro{$q$}}}
\put(5,5){\circle*{.2}}
\put(5,5){\line(-1,-1){2}}
\put(5,5){\line(1,-1){2}}

\put(3,2.8){\makebox(0,0)[t]{\negro{$\beta$}}}
\put(3,3){\circle*{.2}}

\put(7,2.8){\makebox(0,0)[t]{\negro{$\gamma$}}}
\put(7,3){\circle*{.2}}

\end{picture}
& & 
\setlength{\unitlength}{0.35cm}
\begin{picture}(14,9)(0,0)

\put(7,8.2){\makebox(0,0)[b]{\negro{$q$}}}
\put(7,8){\circle*{.2}}
\put(7,8){\line(-4,-1){4}}
\put(7,8){\line(4,-1){4}}

\put(3,7.2){\makebox(0,0)[b]{\rubro{$v$}}}
\put(3,7){\circle*{.2}}
\put(3,7){\line(-1,-1){2}}
\put(3,7){\line(1,-1){2}}

\put(11,7.2){\makebox(0,0)[b]{\rubro{$w$}}}
\put(11,7){\circle*{.2}}
\put(11,7){\line(-1,-1){2}}
\put(11,7){\line(1,-1){2}}

\put(1,4.8){\makebox(0,0)[t]{\negro{$\alpha$}}}
\put(1,5){\circle*{.2}}

\put(5,4.8){\makebox(0,0)[t]{\negro{$\beta$}}}
\put(5,5){\circle*{.2}}

\put(9,4.8){\makebox(0,0)[t]{\negro{$\gamma$}}}
\put(9,5){\circle*{.2}}

\put(13,4.8){\makebox(0,0)[t]{\negro{$\delta$}}}
\put(13,5){\circle*{.2}}

\end{picture}
\end{tabular}
\end{center}

rotação dupla a direita + inversão cores de $q$ e $w$

\only<2>{O resultado é uma árvore rubro-negra? Justifique.}
\end{frame}

\begin{frame}
\frametitle{Algoritmo de inserção}
\begin{codebox}
\Procname{$\proc{Insert-Aux}(k, q, v, w)$}
\li \If $q \isequal \const{Nil}$
\li \Then 
      $q \gets \proc{Make-Node}(k)$
\li \ElseIf $k < \attrib{q}{key}$
\li \Then
      $\proc{Insert-Aux}(l, \attrib{q}{left}, q, v)$
\li   \If $\attrib{q}{color} \isequal \const{Red}$ and $\attrib{\attrib{q}{left}}{color} \isequal \const{Red}$
\li   \Then
        $\proc{Balance}(\attrib{q}{left}, q, v)$
      \End
\li \ElseIf $k > \attrib{q}{key}$
\li \Then
      $\cdots$
    \End    
\end{codebox}

\end{frame}

\begin{frame}
\frametitle{Algoritmo de inserção}
\begin{small}
\begin{codebox}
\Procname{$\proc{Balance}(q, v, w)$}
\li \If $v \isequal \attrib{w}{esq}$
\li \Then 
      $t \gets \attrib{w}{right}$
\li \Else
      $t \gets \attrib{w}{left}$
    \End
\li \If $\attrib{t}{color} \isequal \const{Red}$
\li \Then
      $\attrib{t}{color} \gets \attrib{v}{color} \gets \const{Black}$
\li   $\attrib{w}{color} \gets \const{Red}$
\li \Else
      \If $q \isequal \attrib{v}{left}$ and $v \isequal \attrib{w}{left}$
\li   \Then
        $\attrib{v}{color} \gets \const{Black}$, $\attrib{w}{color} \gets \const{Red}$
\li     $\proc{Rotate-Simple-Right}(w)$
\li   \ElseIf $q \isequal \attrib{v}{right}$ and $v \isequal \attrib{w}{left}$
\li     \Then
          $\attrib{q}{color} \gets \const{Black}$, $\attrib{w}{color} \gets \const{Red}$
\li     $\proc{Rotate-Double-Right}(w)$
\li   \ElseIf $q \isequal \attrib{v}{right}$ and $v \isequal \attrib{w}{right}$
\li     $\cdots$
\li   \ElseNoIf
\li     $\cdots$
      \End
    \End
\end{codebox}
\end{small}
\end{frame}

\begin{frame}
\frametitle{A operação de remoção}

\begin{enumerate}

\item remoção em árvore binária de busca

\item \alert{se necessário}, re-balanceamento

\item o nó removido tem pelo menos uma sub-árvore vazia

\end{enumerate}

\end{frame}

\begin{frame}

\frametitle{Análise da remoção}

O que fazer quando o nó a remover tem a cor
\begin{description}
\item[rubro] ? \only<2->{Nada, pois nenhuma propriedade pode ser quebrada.}

\only<3>{Exercício: remover os nós com valores 15 e 50 na árvore seguinte:}

\begin{center}
\setlength{\unitlength}{0.4cm}
\begin{picture}(13,12.5)(0,0)

\put(6,11.2){\makebox(0,0)[b]{\negro{47}}}
\put(6,11){\circle*{.2}}
\put(6,11){\line(-4,-1){4}}
\put(6,11){\line(4,-1){4}}

\put(2,10.2){\makebox(0,0)[b]{\negro{32}}}
\put(2,10){\circle*{.2}}
\put(2,10){\line(-1,-1){2}}
\put(2,10){\line(1,-1){2}}

\put(10,10.2){\makebox(0,0)[b]{\rubro{68}}}
\put(10,10){\circle*{.2}}
\put(10,10){\line(-1,-1){2}}
\put(10,10){\line(1,-1){2}}

\put(0,8.2){\makebox(0,0)[b]{\negro{5}}}
\put(0,8){\circle*{.2}}
\put(0,8){\line(1,-4){1}}

\put(4,8.2){\makebox(0,0)[b]{\negro{40}}}
\put(4,8){\circle*{.2}}

\put(8,8.2){\makebox(0,0)[b]{\negro{60}}}
\put(8,8){\circle*{.2}}
\put(8,8){\line(-1,-4){1}}
\put(8,8){\line(1,-4){1}}

\put(12,8.2){\makebox(0,0)[b]{\negro{88}}}
\put(12,8){\circle*{.2}}
\put(12,8){\line(1,-4){1}}
\put(12,8){\line(-1,-4){1}}

\put(1,4.2){\makebox(0,0)[b]{\rubro{15}}}
\put(1,4){\circle*{.2}}

\put(7,4.2){\makebox(0,0)[b]{\negro{54}}}
\put(7,4){\circle*{.2}}
\put(7,4){\line(-1,-4){1}}

\put(9,4.2){\makebox(0,0)[b]{\negro{61}}}
\put(9,4){\circle*{.2}}

\put(11,4.2){\makebox(0,0)[b]{\negro{75}}}
\put(11,4){\circle*{.2}}

\put(13,4.2){\makebox(0,0)[b]{\negro{90}}}
\put(13,4){\circle*{.2}}

\put(6,0.2){\makebox(0,0)[b]{\rubro{50}}}
\put(6,0){\circle*{.2}}
\end{picture}
\end{center}

\item[negro] ? \only<4>{\alert{Rebalancear}}
\end{description}

\end{frame}

\begin{frame}
\frametitle{Remoção de um nó negro}

Seja $y$ o nó a remover.
\begin{itemize}
\item todos os caminhos da raiz até uma folha passando por $y$ tem um nó negro a
  menos
\item seja $x$ a sub-árvore após a remoção de $y$ 
  \begin{itemize}
    \item $x$ é a sub-árvore vazia ($y$ era uma folha), ou
    \item $x$ era a sub-árvore não vazia de $y$
  \end{itemize}
\item solução: incrementar a altura negra de $x$.
  \begin{itemize}
  \item se a raiz de $x$ era de cor rubro, passa a ser de cor negro
  \item se a raiz de $x$ era de cor negra, a cor é duplamente negro
  \end{itemize}
\item se $x$ é a raiz da árvore (global): atribuir negro à cor de $x$
\item se $x$ não é a raiz da árvore:
  \begin{itemize}
  \item seja $v : \attrib{x}{up}$, o ascendente direto de $x$
  \item seja $w$ a outra sub-árvore de $v$

    $w$ não pode ser a árvore vazia \pause explique o motivo.
  \end{itemize}
\end{itemize}
\end{frame}

\begin{frame}
\frametitle{Os diferentes casos}
\framesubtitle{$x \neq \id{root}, \attrib{x}{color} \isequal \const{Black}, \attrib{x}{up} \isequal v, \attrib{w}{up} \isequal v, w \neq x$}

\begin{enumerate}
\item $\attrib{w}{color} \isequal \const{Red}$
\item $\attrib{w}{color} \isequal \const{Black}$, as sub-árvores de $w$ são de cor negro.
\item $\attrib{w}{color} \isequal \const{Black}$, 
$\attrib{\attrib{w}{left}}{color} \isequal \const{Red}$, 
$\attrib{\attrib{w}{right}}{color} \isequal \const{Black}$.
\item $\attrib{w}{color} \isequal \const{Black}$, 
$\attrib{\attrib{w}{right}}{color} \isequal \const{Red}$.
\end{enumerate}
\end{frame}

\begin{frame}
\frametitle{Caso 1}
\framesubtitle{$x \neq \id{root}, \attrib{x}{color} \isequal \const{Black}, \attrib{x}{up} \isequal v, \attrib{w}{up} \isequal v, w \neq x, \attrib{w}{color} \isequal \const{Red}$}

\begin{center}
\begin{tabular}[t]{ccc}
(Antes) & & (Depois) \\
\setlength{\unitlength}{0.35cm}
\begin{picture}(14,6)(0,0)

\put(7,5.2){\makebox(0,0)[b]{\negro{$v$}}}
\put(7,5){\circle*{.2}}
\put(7,5){\line(-4,-1){4}}
\put(7,5){\line(4,-1){4}}

\put(3,4.2){\makebox(0,0)[b]{\negro{$x$}$\bullet$}}
\put(3,4){\circle*{.2}}
\put(3,4){\line(-1,-1){2}}
\put(3,4){\line(1,-1){2}}

\put(11,4.2){\makebox(0,0)[b]{\rubro{$w$}}}
\put(11,4){\circle*{.2}}
\put(11,4){\line(-1,-1){2}}
\put(11,4){\line(1,-1){2}}

\put(1,1.8){\makebox(0,0)[b]{$\alpha$}}
\put(1,2){\circle*{.2}}

\put(5,1.8){\makebox(0,0)[t]{$\beta$}}
\put(5,2){\circle*{.2}}

\put(9,0.8){\makebox(0,0)[b]{\negro{$\gamma$}}}
\put(9,2){\circle*{.2}}

\put(13,0.8){\makebox(0,0)[b]{\negro{$\delta$}}}
\put(13,2){\circle*{.2}}

\end{picture}
& & 
\setlength{\unitlength}{0.35cm}
\begin{picture}(14,7)(0,0)
\put(7,6.2){\makebox(0,0)[b]{\negro{$w$}}}
\put(7,6){\circle*{.2}}
\put(7,6){\line(-4,-1){4}}
\put(7,6){\line(4,-1){4}}

\put(3,5.2){\makebox(0,0)[b]{\rubro{$v$}}}
\put(3,5){\circle*{.2}}
\put(3,5){\line(-1,-1){2}}
\put(3,5){\line(1,-1){2}}

\put(11,4.2){\makebox(0,0)[b]{\negro{$\delta$}}}
\put(11,5){\circle*{.2}}

\put(1,2.8){\makebox(0,0)[b]{\negro{$x$}$\bullet$}}
\put(1,3){\circle*{.2}}
\put(1,3){\line(-1,-4){.5}}
\put(1,3){\line(1,-4){.5}}

\put(0.5,0.8){\makebox(0,0)[t]{$\alpha$}}
\put(0.5,1){\circle*{.2}}

\put(1.5,0.8){\makebox(0,0)[t]{$\beta$}}
\put(1.5,1){\circle*{.2}}

\put(5,1.8){\makebox(0,0)[b]{\negro{$\gamma$}}}
\put(5,3){\circle*{.2}}

\end{picture}
\end{tabular}
\end{center}

O nó irmão de $x$ tem a cor negra: caso 2, 3 ou 4.
\end{frame}

\begin{frame}
\frametitle{Caso 2}
\framesubtitle{$x \neq \id{root}, \attrib{x}{color} \isequal \const{Black}, \attrib{x}{up} \isequal v, \attrib{w}{up} \isequal v, w \neq x, \attrib{w}{color} \isequal \const{Black}, \attrib{\attrib{w}{left}}{color} \isequal \const{Black}, \attrib{\attrib{w}{right}}{color} \isequal \const{Black}$}

\begin{center}
\begin{tabular}{ccc}
(Antes) & & (Depois) \\
\setlength{\unitlength}{0.35cm}
\begin{picture}(14,6)(0,0)

\put(7,5.2){\makebox(0,0)[b]{$v_c$}}
\put(7,5){\circle*{.2}}
\put(7,5){\line(-4,-1){4}}
\put(7,5){\line(4,-1){4}}

\put(3,4.2){\makebox(0,0)[b]{\negro{$x$}$\bullet$}}
\put(3,4){\circle*{.2}}
\put(3,4){\line(-1,-1){2}}
\put(3,4){\line(1,-1){2}}

\put(11,4.2){\makebox(0,0)[b]{\negro{$w$}}}
\put(11,4){\circle*{.2}}
\put(11,4){\line(-1,-1){2}}
\put(11,4){\line(1,-1){2}}

\put(1,1.8){\makebox(0,0)[b]{$\alpha$}}
\put(1,2){\circle*{.2}}

\put(5,1.8){\makebox(0,0)[t]{$\beta$}}
\put(5,2){\circle*{.2}}

\put(9,0.8){\makebox(0,0)[b]{\negro{$\gamma$}}}
\put(9,2){\circle*{.2}}

\put(13,0.8){\makebox(0,0)[b]{\negro{$\delta$}}}
\put(13,2){\circle*{.2}}

\end{picture}
& & 
\setlength{\unitlength}{0.35cm}
\begin{picture}(14,6)(0,0)

\put(7,5.2){\makebox(0,0)[b]{$v_c \bullet$}}
\put(7,5){\circle*{.2}}
\put(7,5){\line(-4,-1){4}}
\put(7,5){\line(4,-1){4}}

\put(3,4.2){\makebox(0,0)[b]{\negro{$x$}}}
\put(3,4){\circle*{.2}}
\put(3,4){\line(-1,-1){2}}
\put(3,4){\line(1,-1){2}}

\put(11,4.2){\makebox(0,0)[b]{\rubro{$w$}}}
\put(11,4){\circle*{.2}}
\put(11,4){\line(-1,-1){2}}
\put(11,4){\line(1,-1){2}}

\put(1,1.8){\makebox(0,0)[b]{$\alpha$}}
\put(1,2){\circle*{.2}}

\put(5,1.8){\makebox(0,0)[t]{$\beta$}}
\put(5,2){\circle*{.2}}

\put(9,0.8){\makebox(0,0)[b]{\negro{$\gamma$}}}
\put(9,2){\circle*{.2}}

\put(13,0.8){\makebox(0,0)[b]{\negro{$\delta$}}}
\put(13,2){\circle*{.2}}

\end{picture}
\end{tabular}
\end{center}

\end{frame}

\begin{frame}
\frametitle{Caso 3}
\framesubtitle{$x \neq \id{root}, \attrib{x}{color} \isequal \const{Black}, \attrib{x}{up} \isequal v, \attrib{w}{up} \isequal v, w \neq x, \attrib{w}{color} \isequal \const{Black}, \attrib{\attrib{w}{left}}{color} \isequal \const{Red}, \attrib{\attrib{w}{right}}{color} \isequal \const{Black}$}

\begin{center}
\begin{tabular}{ccc}
(Antes) & & (Depois) \\
\\
\setlength{\unitlength}{0.35cm}
\begin{picture}(14,11)(0,0)

\put(7,10.2){\makebox(0,0)[b]{$v_c$}}
\put(7,10){\circle*{.2}}
\put(7,10){\line(-4,-1){4}}
\put(7,10){\line(4,-1){4}}

\put(3,9.2){\makebox(0,0)[b]{\negro{$x$}$\bullet$}}
\put(3,9){\circle*{.2}}
\put(3,9){\line(-1,-1){2}}
\put(3,9){\line(1,-1){2}}

\put(11,9.2){\makebox(0,0)[b]{\negro{$w$}}}
\put(11,9){\circle*{.2}}
\put(11,9){\line(-1,-1){2}}
\put(11,9){\line(1,-1){2}}

\put(1,6.8){\makebox(0,0)[b]{$\alpha$}}
\put(1,7){\circle*{.2}}

\put(5,6.8){\makebox(0,0)[t]{$\beta$}}
\put(5,7){\circle*{.2}}

\put(9,7.2){\makebox(0,0)[b]{\rubro{$z$}}}
\put(9,7){\circle*{.2}}
\put(9,7){\line(1,-4){1}}
\put(9,7){\line(-1,-4){1}}

\put(13,5.8){\makebox(0,0)[b]{\negro{$\eta$}}}
\put(13,7){\circle*{.2}}

\put(8,2.8){\makebox(0,0)[t]{\negro{$\gamma$}}}
\put(8,3){\circle*{.2}}

\put(10,2.8){\makebox(0,0)[t]{\negro{$\delta$}}}
\put(10,3){\circle*{.2}}

\end{picture}
& & 
\setlength{\unitlength}{0.35cm}
\begin{picture}(14,11)(0,0)

\put(7,10.2){\makebox(0,0)[b]{$v_c$}}
\put(7,10){\circle*{.2}}
\put(7,10){\line(-4,-1){4}}
\put(7,10){\line(4,-1){4}}

\put(3,9.2){\makebox(0,0)[b]{\negro{$x$}$\bullet$}}
\put(3,9){\circle*{.2}}
\put(3,9){\line(-1,-1){2}}
\put(3,9){\line(1,-1){2}}

\put(11,9.2){\makebox(0,0)[b]{\negro{$w$}}}
\put(11,9){\circle*{.2}}
\put(11,9){\line(-1,-1){2}}
\put(11,9){\line(1,-1){2}}

\put(1,6.8){\makebox(0,0)[b]{$\alpha$}}
\put(1,7){\circle*{.2}}

\put(5,6.8){\makebox(0,0)[t]{$\beta$}}
\put(5,7){\circle*{.2}}

\put(9,6.8){\makebox(0,0)[b]{\negro{$\gamma$}}}

\put(13,7.2){\makebox(0,0)[b]{\rubro{$z$}}}
\put(13,7){\circle*{.2}}
\put(13,7){\line(1,-4){1}}
\put(13,7){\line(-1,-4){1}}

\put(12,2.8){\makebox(0,0)[t]{\negro{$\delta$}}}
\put(12,3){\circle*{.2}}

\put(14,2.8){\makebox(0,0)[t]{\negro{$\eta$}}}
\put(14,3){\circle*{.2}}

\end{picture}
\end{tabular}
\end{center}

\end{frame}

\begin{frame}
\frametitle{Caso 4}
\framesubtitle{$x \neq \id{root}, \attrib{x}{color} \isequal \const{Black}, \attrib{x}{up} \isequal v, \attrib{w}{up} \isequal v, w \neq x, \attrib{w}{color} \isequal \const{Black}, \attrib{\attrib{w}{right}}{color} \isequal \const{Red}$}.

\begin{center}
\begin{tabular}{ccc}
(Antes) & & (Depois) \\
\\
\setlength{\unitlength}{0.35cm}
\begin{picture}(14,9)(0,0)

\put(7,8.2){\makebox(0,0)[b]{$v_c$}}
\put(7,8){\circle*{.2}}
\put(7,8){\line(-4,-1){4}}
\put(7,8){\line(4,-1){4}}

\put(3,7.2){\makebox(0,0)[b]{\negro{$x$}$\bullet$}}
\put(3,7){\circle*{.2}}
\put(3,7){\line(-1,-1){2}}
\put(3,7){\line(1,-1){2}}

\put(11,7.2){\makebox(0,0)[b]{\negro{$w$}}}
\put(11,7){\circle*{.2}}
\put(11,7){\line(-1,-1){2}}
\put(11,7){\line(1,-1){2}}

\put(1,4.8){\makebox(0,0)[b]{$\alpha$}}
\put(1,5){\circle*{.2}}

\put(5,4.8){\makebox(0,0)[t]{$\beta$}}
\put(5,5){\circle*{.2}}

\put(9,4.8){\makebox(0,0)[t]{$\gamma$}}
\put(9,5){\circle*{.2}}

\put(13,5.2){\makebox(0,0)[b]{\rubro{$z$}}}
\put(13,5){\circle*{.2}}
\put(13,5){\line(1,-4){1}}
\put(13,5){\line(-1,-4){1}}

\put(12,0.8){\makebox(0,0)[t]{\negro{$\delta$}}}
\put(12,1){\circle*{.2}}

\put(14,0.8){\makebox(0,0)[t]{\negro{$\eta$}}}
\put(14,1){\circle*{.2}}

\end{picture}
& & 
\setlength{\unitlength}{0.35cm}
\begin{picture}(14,9)(0,0)

\put(7,8.2){\makebox(0,0)[b]{$w_c$}}
\put(7,8){\circle*{.2}}
\put(7,8){\line(-4,-1){4}}
\put(7,8){\line(4,-1){4}}

\put(3,7.2){\makebox(0,0)[b]{\negro{$v$}}}
\put(3,7){\circle*{.2}}
\put(3,7){\line(-1,-1){2}}
\put(3,7){\line(1,-1){2}}

\put(11,7.2){\makebox(0,0)[b]{\negro{$z$}}}
\put(11,7){\circle*{.2}}
\put(11,7){\line(-1,-1){2}}
\put(11,7){\line(1,-1){2}}

\put(1,5.2){\makebox(0,0)[b]{\negro{$x$}}}
\put(1,5){\circle*{.2}}
\put(1,5){\line(-1,-4){1}}
\put(1,5){\line(1,-4){1}}

\put(0,0.8){\makebox(0,0)[t]{$\alpha$}}
\put(0,1){\circle*{.2}}

\put(2,0.8){\makebox(0,0)[t]{$\beta$}}
\put(2,1){\circle*{.2}}

\put(5,4.8){\makebox(0,0)[t]{$\gamma$}}
\put(5,5){\circle*{.2}}

\put(9,4.8){\makebox(0,0)[t]{\negro{$\delta$}}}
\put(9,5){\circle*{.2}}

\put(13,4.8){\makebox(0,0)[t]{\negro{$\eta$}}}
\put(13,5){\circle*{.2}}

\end{picture}
\end{tabular}
\end{center}

\end{frame}

\begin{frame}
\frametitle{Exercícios}

\begin{center}
\setlength{\unitlength}{0.4cm}
\begin{picture}(13,12.5)(0,0)

\put(6,11.2){\makebox(0,0)[b]{\negro{47}}}
\put(6,11){\circle*{.2}}
\put(6,11){\line(-4,-1){4}}
\put(6,11){\line(4,-1){4}}

\put(2,10.2){\makebox(0,0)[b]{\negro{32}}}
\put(2,10){\circle*{.2}}
\put(2,10){\line(-1,-1){2}}
\put(2,10){\line(1,-1){2}}

\put(10,10.2){\makebox(0,0)[b]{\rubro{68}}}
\put(10,10){\circle*{.2}}
\put(10,10){\line(-1,-1){2}}
\put(10,10){\line(1,-1){2}}

\put(0,8.2){\makebox(0,0)[b]{\negro{5}}}
\put(0,8){\circle*{.2}}
\put(0,8){\line(1,-4){1}}

\put(4,8.2){\makebox(0,0)[b]{\negro{40}}}
\put(4,8){\circle*{.2}}

\put(8,8.2){\makebox(0,0)[b]{\negro{60}}}
\put(8,8){\circle*{.2}}
\put(8,8){\line(-1,-4){1}}
\put(8,8){\line(1,-4){1}}

\put(12,8.2){\makebox(0,0)[b]{\negro{88}}}
\put(12,8){\circle*{.2}}
\put(12,8){\line(1,-4){1}}
\put(12,8){\line(-1,-4){1}}

\put(1,4.2){\makebox(0,0)[b]{\rubro{15}}}
\put(1,4){\circle*{.2}}

\put(7,4.2){\makebox(0,0)[b]{\negro{54}}}
\put(7,4){\circle*{.2}}
\put(7,4){\line(-1,-4){1}}

\put(9,4.2){\makebox(0,0)[b]{\negro{61}}}
\put(9,4){\circle*{.2}}

\put(11,4.2){\makebox(0,0)[b]{\negro{75}}}
\put(11,4){\circle*{.2}}

\put(13,4.2){\makebox(0,0)[b]{\negro{90}}}
\put(13,4){\circle*{.2}}

\put(6,0.2){\makebox(0,0)[b]{\rubro{50}}}
\put(6,0){\circle*{.2}}
\end{picture}
\end{center}

\begin{enumerate}
\item Repetidamente remove o valor na raiz até a árvore se tornar vazia
\item Repetidamente remove o menor valor até a árvore se tornar vazia
\item Repetidamente remove o maior valor até a árvore se tornar vazia
\pause
\item Escrever o algoritmo de remoção
\end{enumerate}

\end{frame}

\end{document}

