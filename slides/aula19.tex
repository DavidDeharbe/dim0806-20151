\documentclass{beamer}
\setbeamertemplate{footline}[frame number]
%\documentclass{beamer}

\usepackage[utf8x]{inputenc}
\usepackage[brazil,british]{babel}
\usetheme{default} 
\usecolortheme{beaver}
\newtranslation[to=brazil]{Theorem}{Teorema}
\newtranslation[to=brazil]{Definition}{Definição}
\newtranslation[to=brazil]{Example}{Exemplo}
\newtranslation[to=brazil]{Problem}{Exercício}
\newtranslation[to=brazil]{Solution}{Resolução}
\newtranslation[to=brazil]{Lemma}{Lema}
\newtranslation[to=brazil]{Corollary}{Corolário}
\logo{\includegraphics[width=1cm]{../img/logo-ppgsc-icon-text.png}}

\usepackage{graphicx}
\usepackage{clrscode3e}
\usepackage{hyperref}

\usepackage{pgf}
\usepackage{tikz}
\newcommand{\assert}[1]{\textcolor{blue}{#1}}

\usepackage{xspace}

\newcommand{\semester}[0]{2015.2}


\title{Aula 17: Árvores B}
\author{David Déharbe \\
  Programa de Pós-graduação em Sistemas e Computação \\
  Universidade Federal do Rio Grande do Norte \\
  Centro de Ciências Exatas e da Terra \\
  Departamento de Informática e Matemática Aplicada}
\date{27 de abril de 2015}

\newcommand{\negro}[1]{\colorbox{black}{\textcolor{white}{\textbf{#1}}}}
\newcommand{\rubro}[1]{\colorbox{red}{\textcolor{white}{\textbf{#1}}}}
\newcommand{\bh}[0]{\ensuremath{\id{bh}}}


\begin{document}
\selectlanguage{brazil}
\begin{frame}
  \titlepage
  Download me from \url{http://ufrn.academia.edu/DavidDeharbe}.
\end{frame}

\begin{frame}
  \frametitle{Plano da aula}

\begin{center}
\hspace*{-1cm}
\setlength{\unitlength}{.8cm}
\scriptsize
\begin{picture}(15, 5)(0,0)
\put(07.0, 3.5){\framebox(0.8,0.8){~50~}}
\put(07.1, 3.7){\vector(-3,-1){2.7}}
\put(07.7, 3.7){\vector(3,-1){2.7}}
\put(03.5, 2.0){\framebox(1.2,0.8){~27~~38~}}
\put(03.6, 2.2){\vector(-1,-1){1}}
\put(04.0, 2.2){\vector(0,-1){1}}
\put(04.5, 2.2){\vector(1,-1){1}}
\put(10.3, 2.0){\framebox(1.6,0.8){~72~~85~~96~}}
\put(10.4, 2.2){\vector(-2,-1){2}}
\put(10.9, 2.2){\vector(-1,-2){.5}}
\put(11.3, 2.2){\vector(1,-2){.5}}
\put(11.8, 2.2){\vector(2,-1){2}}
\put(01.5, 0.5){\framebox(1.2,0.8){~3~~12~}}
\put(03.5, 0.5){\framebox(1.2,0.8){~30~~34~}}
\put(05.5, 0.5){\framebox(1.6,0.8){~40~~43~~47~}}
\put(07.8, 0.5){\framebox(1.2,0.8){~60~~70~}}
\put(09.8, 0.5){\framebox(1.2,0.8){~77~~83~}}
\put(11.8, 0.5){\framebox(1.2,0.8){~88~~90~}}
\put(13.8, 0.5){\framebox(1.2,0.8){~99~~102~}}
\end{picture}
\end{center}

  \tableofcontents

\end{frame}

\section{Introdução}

\begin{frame}

  \frametitle{Árvores B}
  \framesubtitle{Introdução}

  
  \begin{itemize}
    
  \item Em 1971, Rudolf Bayer e Ed McCreight projetaram a estrutura de dados
    \alert{árvores B binárias simétricas}.

  \item Complexidade
    \begin{itemize}
    \item busca em $O(\lg n)$, 
    \item remoção em $O(\lg n)$ e
    \item inserção em $O(\lg n)$
    \end{itemize}

  \item Motivação: hierarquia de memória

  \item Árvores de busca, balanceadas 

    \begin{itemize}

      \item aumentar grau de ramificação

      \item diminuir altura

    \end{itemize}

  \item banco de dados

  \end{itemize}

\end{frame}

\section{Propriedades}

\begin{frame}

\frametitle{Árvores B}
\framesubtitle{Especificação}

  \begin{enumerate}
  \item raiz $r$, tal que:
  \item atributos do nó $x$:
    \begin{enumerate}
    \item $\attrib{x}{n}$: número de chaves de $x$;
    \item $\attrib{x}{keys}$: vetor ordenado das chaves
      $$\forall i | 1 \leq i < \attrib{x}{n} \cdot \attrib{x}{keys}[i] < \attrib{x}{keys}[i+1];$$
    \item $\attrib{x}{leaf}$: booleano indicando se $x$ é uma folha;
    \item se $\neg \attrib{x}{leaf}$: $\attrib{x}{sub}$ vetor de sub-árvores;
    \end{enumerate}
  \item $\attrib{x}{leaf} \lor \id{len}(\attrib{x}{sub}) = 1 + \attrib{x}{n}$
  \item $k_i \in \id{keys}(\attrib{x}{sub}[i]) \land k_{i+1} \in \id{keys}(\attrib{x}{sub}[i+1]) \Rightarrow k_i < k_{i+1}$;
  \item $\forall x \cdot \attrib{x}{leaf} \Rightarrow \id{level}(x) = \alpha(r)$;
  \item $g > 1$: grau de ramificação \emph{mínimo} da árvore:
    \begin{enumerate}
    \item $x \neq r \Rightarrow g-1 \le \attrib{x}{n} \le 2g - 1$;
    \item $1 \le \attrib{r}{n} \le 2g - 1$,
    \end{enumerate}
  \end{enumerate}
  $g = 2$: árvore 2-3-4.

\end{frame}

\begin{frame}

\frametitle{Ilustração}

\begin{center}
\hspace*{-1cm}
\setlength{\unitlength}{.8cm}
\scriptsize
\begin{picture}(15, 5)(0,0)
\put(07.0, 3.5){\framebox(0.8,0.8){~50~}}
\put(07.1, 3.7){\vector(-3,-1){2.7}}
\put(07.7, 3.7){\vector(3,-1){2.7}}
\put(03.5, 2.0){\framebox(1.2,0.8){~27~~38~}}
\put(03.6, 2.2){\vector(-1,-1){1}}
\put(04.0, 2.2){\vector(0,-1){1}}
\put(04.5, 2.2){\vector(1,-1){1}}
\put(10.3, 2.0){\framebox(1.6,0.8){~72~~85~~96~}}
\put(10.4, 2.2){\vector(-2,-1){2}}
\put(10.9, 2.2){\vector(-1,-2){.5}}
\put(11.3, 2.2){\vector(1,-2){.5}}
\put(11.8, 2.2){\vector(2,-1){2}}
\put(01.5, 0.5){\framebox(1.2,0.8){~3~~12~}}
\put(03.5, 0.5){\framebox(1.2,0.8){~30~~34~}}
\put(05.5, 0.5){\framebox(1.6,0.8){~40~~43~~47~}}
\put(07.8, 0.5){\framebox(1.2,0.8){~60~~70~}}
\put(09.8, 0.5){\framebox(1.2,0.8){~77~~83~}}
\put(11.8, 0.5){\framebox(1.2,0.8){~88~~90~}}
\put(13.8, 0.5){\framebox(1.2,0.8){~99~~102~}}
\end{picture}
\end{center}
\end{frame}

\begin{frame}
\frametitle{Grau de ramificação}

\begin{itemize}
  \item $N$: nó
  \item $K$: chave
  \item $A_n$: endereço de um nó
  \item $A_r$: endereço de um registro de dados
  \item $g_{max} \times A_n + g_{max} - 1 \times (K + A_r) \le N$
  \item Supondo: $N = 4096, K = A_n = A_r = 4$
  \item $g_{max} = 342$
  \item \begin{tabular}[t]{c|cl}
      altura & capacidade \\
      \hline
      1 & $1 \times 341$ & $=341$ \\
      2 & $(1 + 342) \times 341$ & $= 116.963$ \\
      3 & $(1 + 342 + 342^2) \times 341$ & $= 40.001.674$
    \end{tabular}
\end{itemize}

\end{frame}

\begin{frame}

\frametitle{Altura de árvore B}

\begin{theorem}
  A altura $a$ de uma árvore B com $n$ elementos e de grau mínimo $g$ é
  tal que
  $$
  a \leq \log_g \frac{n-1}{2} + 1.
  $$
\end{theorem}

\end{frame}

\begin{frame}

\frametitle{Altura de árvore B}

\begin{proof}
  
\only<1>{
  Pior caso: 
  \begin{itemize}
  \item raiz: uma chave e dois filhos,
  \item outros nós internos: $g-1$ chaves e $g$ filhos
  \item folhas: $g-1$ chaves.
  \end{itemize}

  \begin{tabular}{r|l}
    nível & quantidade de nós \\
    \hline
    1 & 1 \\
    2 & 2 \\
    3 & $2g$ \\
    4 & $2g^2$ \\
    $a$ & $2g^{a-2}$
  \end{tabular}
}

\only<2>{
  \begin{eqnarray*}
    n & \geq & 1 + (g-1)\times 2 \times \Sigma_{i = 0}^{a-2} g^i \\
    n & \geq & 1 + (g-1)\times 2 \times \frac{1-g^{a-1}}{1-g} \\
    n & \geq & 1 + 2 \times(g^{a-1}-1) \\
    \frac{n+1}{2} & \geq & g^{a-1} \\
    a & \leq & \log_g \frac{n-1}{2} + 1
  \end{eqnarray*}
}
\end{proof}

\end{frame}

\section{Operações}

\begin{frame}
\frametitle{Implementação}

Escrita/leitura de nós
\begin{itemize}
  \item $\proc{Write-Node}$
  \item $\proc{Read-Node}$
\end{itemize}

\end{frame}

\begin{frame}
\frametitle{Busca}

\begin{codebox}
\Procname{$\proc{Search-Node}(T, x, k)$}
\li $i \gets 0$
\li \While $i < \attrib{x}{n}$ and $\attrib{x}{keys}[i] < k$
\li \Do $i \gets i + 1$
    \End
\li \If $i \isequal \attrib{x}{n}$ and $\attrib{x}{keys}[i] \isequal k$
\li \Then \Return $\attrib{x}{refs}[i]$
    \End
\li \If $\attrib{x}{leaf}$
\li \Then \Return $\const{Nil}$
    \End
\li $\proc{Read-Node}(T, \attrib{x}{sub}[i])$
\li \Return $\proc{Search-Node}(T, \attrib{x}{sub}[i], k)$
\end{codebox}

\pause
\begin{itemize}
\item uma chamada: $O(1)$ acesso disco, $O(g)$ busca sequencial;
\item total: $O(a) = O(\log_{g} n)$ acessos disco, e $O(ag) = O(g\log_g n)$ 
operações processador.
\end{itemize}
\end{frame}

\begin{frame}
\frametitle{Divisão}

\begin{itemize}
\item operação auxiliar para a inserção
\item entrada: nó cheio $x$ com $2g-1$ chaves, tal que 
  \begin{itemize}
    \item $x$ é a raiz ou
    \item o nó pai de $x$ não é cheio
  \end{itemize}
\item seja $k$ é a chave mediana de $x$
\item efeito:
  \begin{itemize}
    \item se $x = r$, um novo nó raiz é criado, com a chave $k$
    \item se $x \neq r$:
      \begin{itemize}
        \item a chave $k$ é inserida no nó pai
        \item um novo nó $y$ é criado, e recebe $g-1$ chaves de $x$
        \item $x$ permanece com $g-1$ chaves
      \end{itemize}
  \end{itemize}
\end{itemize}

\end{frame}

\begin{frame}

\frametitle{Inserção}
\framesubtitle{Princípios}

\begin{itemize}
\item todas as folhas devem ter o mesmo nível antes e \alert{depois} da inserção
\item na descida recursiva na árvore B, garante-se que há espaço para armazenar o novo registro
\item se um nó a visitar estiver cheio, ele é \alert{dividido} em dois
\end{itemize}

\end{frame}

\begin{frame}

\frametitle{Divisão}
\framesubtitle{Exemplo}

$g = 3$

\begin{center}
\scriptsize
\begin{tabular}{cc}
\setlength{\unitlength}{.8cm}
\begin{picture}(6, 3)(0,0)
\put(2.0, 2.0){\framebox(1.2,0.8){~N~~W~}}
\put(1.8, 2.3){$p$}
\put(2.1, 2.2){\vector(-3,-2){1.5}}
\put(3.1, 2.2){\vector(3,-2){1.5}}
\put(2.6, 2.2){\vector(0,-1){1.4}}
\put(1.4, 0.0){\framebox(2.4,0.8){~P~~Q~~R~~S~~T~}}
\put(0.8, 0.3){$x$}
\end{picture}
&
\setlength{\unitlength}{.8cm}
\begin{picture}(6, 3)(0,0)
\put(2.5, 2.0){\framebox(1.6,0.8){~N~~R~~W~}}
\put(2.3, 2.3){$p$}
\put(2.6, 2.2){\vector(-3,-2){2}}
\put(3.0, 2.2){\vector(-2,-3){0.9}}
\put(3.5, 2.2){\vector(2,-3){0.9}}
\put(4.0, 2.2){\vector(3,-2){2}}
\put(1.2, 0.0){\framebox(1.2,0.8){~P~~Q~}}
\put(0.8, 0.3){$x$}
\put(4.2, 0.0){\framebox(1.2,0.8){~S~~T~}}
\put(3.8, 0.3){$y$}
\end{picture}
\\
(antes) & (depois)
\end{tabular}
\end{center}

\end{frame}

\begin{frame}

\frametitle{Divisão}
\framesubtitle{Algoritmo}

{
\footnotesize
\begin{codebox}
\Procname{$\proc{Divide-Node}(T, p, i, x)$}
\li $y \gets \proc{Make-Node}(T)$
\li $\attrib{y}{n} = \attrib{T}{degree}-1$
\li \For $j \gets 0$ \To $\attrib{y}{n}$
\li \Do $\attrib{y}{keys}[j] \gets \attrib{x}{keys}[j + \attrib{y}{n}]$
    \End
\li \If not $\attrib{x}{leaf}$
\li \Then \For $j \gets 0$ \To $\attrib{y}{n}+1$
\li    \Do $\attrib{y}{sub}[j] \gets \attrib{x}{sub}[j + \attrib{y}{n}]$
    \End \End
\li $\attrib{x}{n} = \attrib{T}{degree}-1$
\li \For $j \gets \attrib{p}{n}$ \Downto $i$
\li \Do $\attrib{p}{sub}[j+1] \gets \attrib{p}{sub}[j]$
    \End
\li $\attrib{p}{sub}[i] \gets y$
\li \For $j \gets \attrib{p}{n}$ \Downto $i-1$
\li \Do $\attrib{p}{keys}[j+1] \gets \attrib{p}{keys}[j]$
    \End
\li $\attrib{p}{keys}[i-1] \gets \attrib{x}{keys}[\attrib{y}{n}]$
\li $\attrib{p}{n} \gets \attrib{p}{n}+1$
\li $\proc{Write-Node}(T, x)$
\li $\proc{Write-Node}(T, y)$
\li $\proc{Write-Node}(T, p)$
\end{codebox}
}
\end{frame}

\begin{frame}
\frametitle{Inserção}
\framesubtitle{Exemplo}

{\footnotesize
\begin{center}
\begin{tabular}{cc}
Estado inicial &
\setlength{\unitlength}{.8cm}
\begin{picture}(8.8, 2.8)(0,0)
\put(3.4, 2.0){\framebox(2.0,0.8){~G~~M~~P~~X~}}
\put(3.5, 2.2){\vector(-3,-2){2}}
\put(3.9, 2.2){\vector(-2,-3){0.9}}
\put(4.4, 2.2){\vector(-1,-4){0.34}}
\put(4.9, 2.2){\vector(2,-3){0.9}}
\put(5.3, 2.2){\vector(2,-1){2.7}}
\put(0.0, 0.0){\framebox(2.0,0.8){~A~~C~~D~~E}}
\put(2.2, 0.0){\framebox(1.2,0.8){~J~~K~}}
\put(3.6, 0.0){\framebox(1.2,0.8){~N~~O~}}
\put(5.0, 0.0){\framebox(2.4,0.8){~R~~S~~T~~U~~V~}}
\put(7.6, 0.0){\framebox(1.2,0.8){~Y~~Z~}}
\end{picture}\\
\\
Inserção de B &
\setlength{\unitlength}{.8cm}
\begin{picture}(9.2, 2.8)(0.0,0)
\put(3.8, 2.0){\framebox(2.0,0.8){~G~~M~~P~~X~}}
\put(3.9, 2.2){\vector(-3,-2){2}}
\put(4.3, 2.2){\vector(-2,-3){0.9}}
\put(4.8, 2.2){\vector(-1,-4){0.34}}
\put(5.3, 2.2){\vector(2,-3){0.9}}
\put(5.7, 2.2){\vector(2,-1){2.7}}
\put(0.0, 0.0){\framebox(2.4,0.8){~A~~\textbf{B}~~C~~D~~E}}
\put(2.6, 0.0){\framebox(1.2,0.8){~J~~K~}}
\put(4.0, 0.0){\framebox(1.2,0.8){~N~~O~}}
\put(5.4, 0.0){\framebox(2.4,0.8){~R~~S~~T~~U~~V~}}
\put(8.0, 0.0){\framebox(1.2,0.8){~Y~~Z~}}
\end{picture}
\end{tabular}
\end{center}
}
\end{frame}

\begin{frame}
\frametitle{Inserção}
\framesubtitle{Exemplo}

{\footnotesize
\begin{center}
\begin{tabular}{cc}
 &
\setlength{\unitlength}{.8cm}
\begin{picture}(9.2, 2.8)(0.0,0)
\put(3.8, 2.0){\framebox(2.0,0.8){~G~~M~~P~~X~}}
\put(3.9, 2.2){\vector(-3,-2){2}}
\put(4.3, 2.2){\vector(-2,-3){0.9}}
\put(4.8, 2.2){\vector(-1,-4){0.34}}
\put(5.3, 2.2){\vector(2,-3){0.9}}
\put(5.7, 2.2){\vector(2,-1){2.7}}
\put(0.0, 0.0){\framebox(2.4,0.8){~A~~\textbf{B}~~C~~D~~E}}
\put(2.6, 0.0){\framebox(1.2,0.8){~J~~K~}}
\put(4.0, 0.0){\framebox(1.2,0.8){~N~~O~}}
\put(5.4, 0.0){\framebox(2.4,0.8){~R~~S~~T~~U~~V~}}
\put(8.0, 0.0){\framebox(1.2,0.8){~Y~~Z~}}
\end{picture}\\
\\
Inserção de Q &
\setlength{\unitlength}{.8cm}
\begin{picture}(10.2, 2.8)(0.0,0)
\put(3.8, 2.0){\framebox(2.4,0.8){~G~~M~~P~~T~~X~}}
\put(3.9, 2.2){\vector(-3,-2){2}}
\put(4.3, 2.2){\vector(-2,-3){0.9}}
\put(4.8, 2.2){\vector(-1,-4){0.34}}
\put(5.3, 2.2){\vector(2,-3){0.9}}
\put(5.7, 2.2){\vector(2,-1){2.7}}
\put(6.1, 2.2){\vector(3,-1){4.0}}
\put(0.0, 0.0){\framebox(2.4,0.8){~A~~B~~C~~D~~E}}
\put(2.6, 0.0){\framebox(1.2,0.8){~J~~K~}}
\put(4.0, 0.0){\framebox(1.2,0.8){~N~~O~}}
\put(5.4, 0.0){\framebox(1.6,0.8){~\textbf{Q}~~R~~S~}}
\put(7.2, 0.0){\framebox(1.6,0.8){~U~~V~}}
\put(9.0, 0.0){\framebox(1.2,0.8){~Y~~Z~}}
\end{picture}
\end{tabular}
\end{center}
}
\end{frame}

\begin{frame}
\frametitle{Inserção}
\framesubtitle{Exemplo}

{\footnotesize
\begin{center}
\begin{tabular}{cc}
 &
\setlength{\unitlength}{.8cm}
\begin{picture}(10.2, 2.8)(0.0,0)
\put(3.8, 2.0){\framebox(2.4,0.8){~G~~M~~P~~T~~X~}}
\put(3.9, 2.2){\vector(-3,-2){2}}
\put(4.3, 2.2){\vector(-2,-3){0.9}}
\put(4.8, 2.2){\vector(-1,-4){0.34}}
\put(5.3, 2.2){\vector(2,-3){0.9}}
\put(5.7, 2.2){\vector(2,-1){2.7}}
\put(6.1, 2.2){\vector(3,-1){4.0}}
\put(0.0, 0.0){\framebox(2.4,0.8){~A~~B~~C~~D~~E}}
\put(2.6, 0.0){\framebox(1.2,0.8){~J~~K~}}
\put(4.0, 0.0){\framebox(1.2,0.8){~N~~O~}}
\put(5.4, 0.0){\framebox(1.6,0.8){~\textbf{Q}~~R~~S~}}
\put(7.2, 0.0){\framebox(1.6,0.8){~U~~V~}}
\put(9.0, 0.0){\framebox(1.2,0.8){~Y~~Z~}}
\end{picture}\\
\\
Inserção de L &
\setlength{\unitlength}{.8cm}
\begin{picture}(10.6, 4.8)(0.0,0)
\put(5.1, 4.0){\framebox(0.8,0.8){~P~}}
\put(5.2, 4.2){\vector(-3,-2){2}}
\put(5.8, 4.2){\vector(3,-2){2}}
\put(2.2, 2.0){\framebox(1.2,0.8){~G~~M~}}
\put(2.3, 2.2){\vector(-1,-1){1.4}}
\put(2.8, 2.2){\vector(0,-1){1.4}}
\put(3.3, 2.2){\vector(1,-1){1.4}}
\put(0.0, 0.0){\framebox(2.4,0.8){~A~~B~~C~~D~~E}}
\put(2.6, 0.0){\framebox(1.6,0.8){~J~~K~~\textbf{L}~}}
\put(4.4, 0.0){\framebox(1.2,0.8){~N~~O~}}
\put(7.6, 2.0){\framebox(1.2,0.8){~T~~X~}}
\put(7.7, 2.2){\vector(-1,-1){1.4}}
\put(8.2, 2.2){\vector(0,-1){1.4}}
\put(8.7, 2.2){\vector(1,-1){1.4}}
\put(5.8, 0.0){\framebox(1.6,0.8){~Q~~R~~S~}}
\put(7.6, 0.0){\framebox(1.6,0.8){~U~~V~}}
\put(9.4, 0.0){\framebox(1.2,0.8){~Y~~Z~}}
\end{picture}
\end{tabular}
\end{center}
}
\end{frame}

\begin{frame}
\frametitle{Inserção}
\framesubtitle{Exemplo}

\vspace*{-5mm}
{\footnotesize
\begin{center}
\begin{tabular}{cc}
 &
\setlength{\unitlength}{.8cm}
\vspace*{-5mm}
\begin{picture}(10.6, 4.8)(0.0,0)
\put(5.1, 4.0){\framebox(0.8,0.8){~P~}}
\put(5.2, 4.2){\vector(-3,-2){2}}
\put(5.8, 4.2){\vector(3,-2){2}}
\put(2.2, 2.0){\framebox(1.2,0.8){~G~~M~}}
\put(2.3, 2.2){\vector(-1,-1){1.4}}
\put(2.8, 2.2){\vector(0,-1){1.4}}
\put(3.3, 2.2){\vector(1,-1){1.4}}
\put(0.0, 0.0){\framebox(2.4,0.8){~A~~B~~C~~D~~E}}
\put(2.6, 0.0){\framebox(1.6,0.8){~J~~K~~\textbf{L}~}}
\put(4.4, 0.0){\framebox(1.2,0.8){~N~~O~}}
\put(7.6, 2.0){\framebox(1.2,0.8){~T~~X~}}
\put(7.7, 2.2){\vector(-1,-1){1.4}}
\put(8.2, 2.2){\vector(0,-1){1.4}}
\put(8.7, 2.2){\vector(1,-1){1.4}}
\put(5.8, 0.0){\framebox(1.6,0.8){~Q~~R~~S~}}
\put(7.6, 0.0){\framebox(1.6,0.8){~U~~V~}}
\put(9.4, 0.0){\framebox(1.2,0.8){~Y~~Z~}}
\end{picture}\\
\\
Inserção de F &
\setlength{\unitlength}{.7cm}
\vspace*{-5mm}
\begin{picture}(10.6, 4.8)(0.0,0)
\put(5.3, 4.0){\framebox(0.8,0.8){~P~}}
\put(5.4, 4.2){\vector(-3,-2){2}}
\put(6.0, 4.2){\vector(3,-2){2.2}}
\put(2.4, 2.0){\framebox(1.6,0.8){~C~~G~~M~}}
\put(2.5, 2.2){\vector(-1,-1){1.4}}
\put(2.9, 2.2){\vector(-1,-2){0.7}}
\put(3.4, 2.2){\vector(1,-2){0.7}}
\put(3.9, 2.2){\vector(1,-1){1.4}}
\put(0.0, 0.0){\framebox(1.2,0.8){~A~~B~}}
\put(1.4, 0.0){\framebox(1.6,0.8){~D~~E~~\textbf{F}~}}
\put(3.2, 0.0){\framebox(1.6,0.8){~J~~K~~L~}}
\put(5.0, 0.0){\framebox(1.2,0.8){~N~~O~}}
\put(8.2, 2.0){\framebox(1.2,0.8){~T~~X~}}
\put(8.3, 2.2){\vector(-1,-1){1.4}}
\put(8.8, 2.2){\vector(0,-1){1.4}}
\put(9.3, 2.2){\vector(1,-1){1.4}}
\put(6.4, 0.0){\framebox(1.6,0.8){~Q~~R~~S~}}
\put(8.2, 0.0){\framebox(1.6,0.8){~U~~V~}}
\put(10.0, 0.0){\framebox(1.2,0.8){~Y~~Z~}}
\end{picture}
\end{tabular}
\end{center}
}
\end{frame}


\begin{frame}
\frametitle{Inserção}
\framesubtitle{Algoritmo}

{
\begin{codebox}
\Procname{$\proc{Insert}(T, r)$}
\li \If $\attrib{T}{root} \isequal \const{Nil}$
\li \Then $y \gets \proc{Make-Node}(T)$
\li   $\attrib{y}{n} \gets 1$
\li   $\attrib{y}{leaf} \gets \const{False}$
\li   $\attrib{y}{keys}[0] \gets \attrib{r}{key}$
\li   $\attrib{y}{refs}[0] \gets r$
\li \ElseIf $\attrib{\attrib{T}{root}}{n} \isequal 2 \times \attrib{T}{degree} - 1 $
\li   \Then $y \gets \proc{Make-Node}(T)$
\li   $\attrib{y}{n} \gets 1$
\li   $\attrib{y}{leaf} \gets \const{True}$
\li   $\attrib{y}{sub}[0] \gets \attrib{T}{root}$
\li   $\attrib{T}{root} \gets y$
\li   $\proc{Write-Node}(T, y)$
\li   $\proc{Divide-Node}(T, y, 0, \attrib{y}{sub}[0])$
    \End
\li $\proc{Node-Insert}(T, \attrib{T}{root}, r)$
\end{codebox}
}
\end{frame}

\begin{frame}
\frametitle{Inserção}

{
\begin{codebox}
\Procname{$\proc{Node-Insert}(T, x, r)$}
\li \If $\attrib{x}{leaf} \isequal \const{True}$
\li \Then $\proc{Leaf-Insert}(T, x, r)$
\li \Else $\proc{Interior-Insert}(T, x, r)$
    \End
\end{codebox}
}
\end{frame}

\begin{frame}
\frametitle{Inserção}
\framesubtitle{Sub-rotina auxiliar para inserção em folha}
{
\begin{codebox}
\Procname{$\proc{Leaf-Insert}(T, x, r)$}
\li $i \gets \attrib{x}{n}-1$
\li \While $i \ge 0$ and $\attrib{r}{key} < \attrib{x}{keys}[i]$
\li \Do $\attrib{x}{keys}[i+1] \gets \attrib{x}{keys}[i]$
\li   $i \gets i - 1$
    \End
\li $\attrib{x}{keys}[i+1] \gets \attrib{r}{key}$
\li $\attrib{x}{refs}[i+1] \gets r$
\li $\proc{Node-Write}(T, x)$
\end{codebox}
}
\end{frame}

\begin{frame}
\frametitle{Inserção}
\framesubtitle{Sub-rotina auxiliar para inserção em nó interno}

{
\begin{codebox}
\Procname{$\proc{Interior-Insert}(T, x, r)$}
\li $i \gets \attrib{x}{n}-1$
\li \While $i \ge 0$ and $\attrib{r}{key} < \attrib{x}{keys}[i]$
\li \Do $\attrib{x}{keys}[i+1] \gets \attrib{x}{keys}[i]$
\li   $i \gets i - 1$
    \End
\li $i \gets i + 1$
\li $y \gets \proc{Read-Node}(T, \attrib{x}{sub}[i])$
\li \If $\attrib{y}{n} \isequal 2 \times \attrib{T}{degree} - 1$
\li \Then $\proc{Node-Divide}(T, y)$
\li   \If $\attrib{x}{keys}[i] < c$
\li   \Then $i \gets i + 1$
      \End
    \End
\li $\proc{Node-Insert}(T, y, r)$
\end{codebox}
}
\end{frame}

\begin{frame}

\frametitle{Remoção}
\framesubtitle{Princípios}

\begin{itemize}
\item todas as folhas devem ter o mesmo nível antes e \alert{depois} da remoção
\item na descida recursiva na árvore B, garante-se que pode se remover um registro sem quebrar o invariante da sub-árvore
\item se um nó a visitar estiver com $g-1$ chaves, opera-se uma operação de
\alert{fusão} com um nó vizinho
\end{itemize}

\end{frame}

\begin{frame}
\frametitle{Fusão}
\framesubtitle{Princípios}

\begin{itemize}
\item dois nós filhos vizinhos $x$ e $y$ de um nó $p$
\item $g-1$ chaves cada
\item seja $k$ a chave ``entre'' $x$ e $y$
\item as chaves de $x$, $k$, e as chaves de $y$ formam o novo nó
\end{itemize}
\end{frame}

\begin{frame}
\frametitle{Fusão}
\framesubtitle{Exemplo, $g = 3$}

\begin{center}
\begin{tabular}{cc}
\setlength{\unitlength}{1cm}
\begin{picture}(6, 3)(0,0)
\put(2.5, 2.0){\framebox(1.6,0.8){~N~~R~~W~}}
\put(2.3, 2.3){$p$}
\put(2.6, 2.2){\vector(-3,-2){2}}
\put(3.0, 2.2){\vector(-2,-3){0.9}}
\put(3.5, 2.2){\vector(2,-3){0.9}}
\put(4.0, 2.2){\vector(3,-2){2}}
\put(1.2, 0.0){\framebox(1.2,0.8){~P~~Q~}}
\put(0.8, 0.3){$x$}
\put(4.2, 0.0){\framebox(1.2,0.8){~S~~T~}}
\put(3.8, 0.3){$y$}
\end{picture}
&
\setlength{\unitlength}{1cm}
\begin{picture}(6, 3)(0,0)
\put(2.0, 2.0){\framebox(1.2,0.8){~N~~W~}}
\put(1.8, 2.3){$p$}
\put(2.1, 2.2){\vector(-3,-2){1.5}}
\put(3.1, 2.2){\vector(3,-2){1.5}}
\put(2.6, 2.2){\vector(0,-1){1.4}}
\put(1.4, 0.0){\framebox(2.4,0.8){~P~~Q~~R~~S~~T~}}
\put(0.8, 0.3){$x$}
\end{picture}
\\
(antes) & (depois)
\end{tabular}
\end{center}
\end{frame}

\begin{frame}
\frametitle{Remoção}
\framesubtitle{Princípios}

\begin{itemize}
  \only<1>{\item se $x = \attrib{T}{root}$ fica sem chaves
    \begin{itemize}
      \item se tem filhos, só tem um, e estetorna-se a nova raiz
      \item caso contrário, a árvore torna-se vazia
    \end{itemize}
  }
  \only<2>{\item se $k \in \attrib{x}{keys}$, e $\attrib{x}{leaf}$, $k$ é eliminado de $x$.}
  \only<3-5>{\item se $k \in \attrib{x}{keys}$, e $\neg \attrib{x}{leaf}$:
    \begin{itemize}
      \only<3>{
      \item se o filho $y$ que precede $k$ em $x$ tem pelo menos $g$
        chaves
        \begin{itemize}
        \item achar o predecessor $k'$ de $k$ na subárvore $y$
        \item remover recursivamente $k'$,
        \item substituir $k$ por $k'$ em $x$.
        \end{itemize}
      }
      \only<4>{
      \item senão, se o filho $z$ que sucede $k$ no nó $x$ tem
        pelo menos $g$ chaves:
        \begin{itemize}
        \item achar o sucessor $k'$ de $k$ na subárvore $z$
        \item remover recursivamente $k'$,
        \item substituir $k$ por $k'$ em $x$.
        \end{itemize}
      }
      \only<5>{
      \item senão, $\attrib{y}{n} = \attrib{z}{n} = g-1$:
        \begin{itemize}
        \item $y$ e $z$ são fusionados em $y$, com chave mediana $k$
        \item $k$ e $z$ são eliminados de $x$
        \item $y$ tem $2g-1$ chaves
        \item liberar o nó $z$ e
        \item remover recursivamente $k$ de $y$.
        \end{itemize}
      }
    \end{itemize}
  }
  \only<6->{
  \item se $k \not\in \attrib{x}{keys}$ 
    \begin{itemize}
    \item determinar a sub-árvore $y$ que deve conter $k$
      \only<6>{
      \item se $\attrib{y}{n} \ge n$, eliminar $k$ de $y$
      }
      \only<7>{
      \item se $\attrib{y}{n} = g-1$ e tem um vizinho $z$ com $\attrib{z}{n} \ge g$
        \begin{itemize}
        \item deslocar uma chave e um filho de $x$ em $y$
        \item deslocar uma chave e um filho de $z$ para $x$
        \item liminar recursivamente $k$ de $y$.
        \end{itemize}
      }
      \only<8>{
      \item se os vizinhos de $y$ tem $g-1$ chaves,
        \begin{itemize}
        \item  fusionar $y$ com um dos irmãos
        \item liminar recursivamente $k$ de $y$.
        \end{itemize}
      }
    \end{itemize}
    }
\end{itemize}

\end{frame}

\begin{frame}
\frametitle{Remoção}
\framesubtitle{Exemplos}

{\scriptsize
\begin{center}
\begin{tabular}{cc}
Estado inicial &
\setlength{\unitlength}{.7cm}
\begin{picture}(10.6, 4.8)(0.0,0)
\put(5.3, 4.0){\framebox(0.8,0.8){~P~}}
\put(5.4, 4.2){\vector(-3,-2){2}}
\put(6.0, 4.2){\vector(3,-2){2.2}}
\put(2.4, 2.0){\framebox(1.6,0.8){~C~~G~~M~}}
\put(2.5, 2.2){\vector(-1,-1){1.4}}
\put(2.9, 2.2){\vector(-1,-2){0.7}}
\put(3.4, 2.2){\vector(1,-2){0.7}}
\put(3.9, 2.2){\vector(1,-1){1.4}}
\put(0.0, 0.0){\framebox(1.2,0.8){~A~~B~}}
\put(1.4, 0.0){\framebox(1.6,0.8){~D~~E~~F~}}
\put(3.2, 0.0){\framebox(1.6,0.8){~J~~K~~L~}}
\put(5.0, 0.0){\framebox(1.2,0.8){~N~~O~}}
\put(8.2, 2.0){\framebox(1.2,0.8){~T~~X~}}
\put(8.3, 2.2){\vector(-1,-1){1.4}}
\put(8.8, 2.2){\vector(0,-1){1.4}}
\put(9.3, 2.2){\vector(1,-1){1.4}}
\put(6.4, 0.0){\framebox(1.6,0.8){~Q~~R~~S~}}
\put(8.2, 0.0){\framebox(1.6,0.8){~U~~V~}}
\put(10.0, 0.0){\framebox(1.2,0.8){~Y~~Z~}}
\end{picture}\\
Remoção de F &
\setlength{\unitlength}{.7cm}
\begin{picture}(10.6, 4.8)(0.0,0)
\put(5.3, 4.0){\framebox(0.8,0.8){~P~}}
\put(5.4, 4.2){\vector(-3,-2){2}}
\put(6.0, 4.2){\vector(3,-2){2.2}}
\put(2.4, 2.0){\framebox(1.6,0.8){~C~~G~~M~}}
\put(2.5, 2.2){\vector(-1,-1){1.4}}
\put(2.9, 2.2){\vector(-1,-2){0.7}}
\put(3.4, 2.2){\vector(1,-2){0.7}}
\put(3.9, 2.2){\vector(1,-1){1.4}}
\put(0.0, 0.0){\framebox(1.2,0.8){~A~~B~}}
\put(1.4, 0.0){\framebox(1.6,0.8){~D~~E~}}
\put(3.2, 0.0){\framebox(1.6,0.8){~J~~K~~L~}}
\put(5.0, 0.0){\framebox(1.2,0.8){~N~~O~}}
\put(8.2, 2.0){\framebox(1.2,0.8){~T~~X~}}
\put(8.3, 2.2){\vector(-1,-1){1.4}}
\put(8.8, 2.2){\vector(0,-1){1.4}}
\put(9.3, 2.2){\vector(1,-1){1.4}}
\put(6.4, 0.0){\framebox(1.6,0.8){~Q~~R~~S~}}
\put(8.2, 0.0){\framebox(1.6,0.8){~U~~V~}}
\put(10.0, 0.0){\framebox(1.2,0.8){~Y~~Z~}}
\end{picture}
\end{tabular}
\end{center}
}
\end{frame}

\begin{frame}

{
\scriptsize
\begin{center}
\begin{tabular}{cc}
 &
\setlength{\unitlength}{.7cm}
\begin{picture}(10.6, 4.8)(0.0,0)
\put(5.3, 4.0){\framebox(0.8,0.8){~P~}}
\put(5.4, 4.2){\vector(-3,-2){2}}
\put(6.0, 4.2){\vector(3,-2){2.2}}
\put(2.4, 2.0){\framebox(1.6,0.8){~C~~G~~M~}}
\put(2.5, 2.2){\vector(-1,-1){1.4}}
\put(2.9, 2.2){\vector(-1,-2){0.7}}
\put(3.4, 2.2){\vector(1,-2){0.7}}
\put(3.9, 2.2){\vector(1,-1){1.4}}
\put(0.0, 0.0){\framebox(1.2,0.8){~A~~B~}}
\put(1.4, 0.0){\framebox(1.6,0.8){~D~~E~}}
\put(3.2, 0.0){\framebox(1.6,0.8){~J~~K~~L~}}
\put(5.0, 0.0){\framebox(1.2,0.8){~N~~O~}}
\put(8.2, 2.0){\framebox(1.2,0.8){~T~~X~}}
\put(8.3, 2.2){\vector(-1,-1){1.4}}
\put(8.8, 2.2){\vector(0,-1){1.4}}
\put(9.3, 2.2){\vector(1,-1){1.4}}
\put(6.4, 0.0){\framebox(1.6,0.8){~Q~~R~~S~}}
\put(8.2, 0.0){\framebox(1.6,0.8){~U~~V~}}
\put(10.0, 0.0){\framebox(1.2,0.8){~Y~~Z~}}
\end{picture}\\
Remoção de M &
\setlength{\unitlength}{.7cm}
\begin{picture}(10.6, 4.8)(0.0,0)
\put(5.3, 4.0){\framebox(0.8,0.8){~P~}}
\put(5.4, 4.2){\vector(-3,-2){2}}
\put(6.0, 4.2){\vector(3,-2){2.2}}
\put(2.4, 2.0){\framebox(1.6,0.8){~C~~G~~L~}}
\put(2.5, 2.2){\vector(-1,-1){1.4}}
\put(2.9, 2.2){\vector(-1,-2){0.7}}
\put(3.4, 2.2){\vector(1,-2){0.7}}
\put(3.9, 2.2){\vector(1,-1){1.4}}
\put(0.0, 0.0){\framebox(1.2,0.8){~A~~B~}}
\put(1.4, 0.0){\framebox(1.6,0.8){~D~~E~}}
\put(3.2, 0.0){\framebox(1.6,0.8){~J~~K~}}
\put(5.0, 0.0){\framebox(1.2,0.8){~N~~O~}}
\put(8.2, 2.0){\framebox(1.2,0.8){~T~~X~}}
\put(8.3, 2.2){\vector(-1,-1){1.4}}
\put(8.8, 2.2){\vector(0,-1){1.4}}
\put(9.3, 2.2){\vector(1,-1){1.4}}
\put(6.4, 0.0){\framebox(1.6,0.8){~Q~~R~~S~}}
\put(8.2, 0.0){\framebox(1.6,0.8){~U~~V~}}
\put(10.0, 0.0){\framebox(1.2,0.8){~Y~~Z~}}
\end{picture}
\end{tabular}
\end{center}
}
\end{frame}


\begin{frame}

{
\scriptsize
\begin{center}
\begin{tabular}{cc}
 &
\setlength{\unitlength}{.7cm}
\begin{picture}(10.6, 4.8)(0.0,0)
\put(5.3, 4.0){\framebox(0.8,0.8){~P~}}
\put(5.4, 4.2){\vector(-3,-2){2}}
\put(6.0, 4.2){\vector(3,-2){2.2}}
\put(2.4, 2.0){\framebox(1.6,0.8){~C~~G~~L~}}
\put(2.5, 2.2){\vector(-1,-1){1.4}}
\put(2.9, 2.2){\vector(-1,-2){0.7}}
\put(3.4, 2.2){\vector(1,-2){0.7}}
\put(3.9, 2.2){\vector(1,-1){1.4}}
\put(0.0, 0.0){\framebox(1.2,0.8){~A~~B~}}
\put(1.4, 0.0){\framebox(1.6,0.8){~D~~E~}}
\put(3.2, 0.0){\framebox(1.6,0.8){~J~~K~}}
\put(5.0, 0.0){\framebox(1.2,0.8){~N~~O~}}
\put(8.2, 2.0){\framebox(1.2,0.8){~T~~X~}}
\put(8.3, 2.2){\vector(-1,-1){1.4}}
\put(8.8, 2.2){\vector(0,-1){1.4}}
\put(9.3, 2.2){\vector(1,-1){1.4}}
\put(6.4, 0.0){\framebox(1.6,0.8){~Q~~R~~S~}}
\put(8.2, 0.0){\framebox(1.6,0.8){~U~~V~}}
\put(10.0, 0.0){\framebox(1.2,0.8){~Y~~Z~}}
\end{picture}\\
Remoção de G &
\setlength{\unitlength}{.7cm}
\begin{picture}(10.6, 4.8)(0.0,0)
\put(5.3, 4.0){\framebox(0.8,0.8){~P~}}
\put(5.4, 4.2){\vector(-3,-2){2}}
\put(6.0, 4.2){\vector(3,-2){2.2}}
\put(2.4, 2.0){\framebox(1.2,0.8){~C~~L~}}
\put(2.5, 2.2){\vector(-1,-1){1.4}}
\put(2.9, 2.2){\vector(-1,-2){0.7}}
\put(3.4, 2.2){\vector(1,-1){1.4}}
\put(0.0, 0.0){\framebox(1.2,0.8){~A~~B~}}
\put(1.4, 0.0){\framebox(2.0,0.8){~D~~E~~J~~K~}}
\put(3.8, 0.0){\framebox(1.2,0.8){~N~~O~}}
\put(8.2, 2.0){\framebox(1.2,0.8){~T~~X~}}
\put(8.3, 2.2){\vector(-1,-1){1.4}}
\put(8.8, 2.2){\vector(0,-1){1.4}}
\put(9.3, 2.2){\vector(1,-1){1.4}}
\put(6.4, 0.0){\framebox(1.6,0.8){~Q~~R~~S~}}
\put(8.2, 0.0){\framebox(1.6,0.8){~U~~V~}}
\put(10.0, 0.0){\framebox(1.2,0.8){~Y~~Z~}}
\end{picture}
\end{tabular}
\end{center}
}
\end{frame}

\begin{frame}

{
\scriptsize
\begin{center}
\begin{tabular}{cc}
 &
\setlength{\unitlength}{.7cm}
\begin{picture}(10.6, 4.8)(0.0,0)
\put(5.3, 4.0){\framebox(0.8,0.8){~P~}}
\put(5.4, 4.2){\vector(-3,-2){2}}
\put(6.0, 4.2){\vector(3,-2){2.2}}
\put(2.4, 2.0){\framebox(1.2,0.8){~C~~L~}}
\put(2.5, 2.2){\vector(-1,-1){1.4}}
\put(2.9, 2.2){\vector(-1,-2){0.7}}
\put(3.4, 2.2){\vector(1,-1){1.4}}
\put(0.0, 0.0){\framebox(1.2,0.8){~A~~B~}}
\put(1.4, 0.0){\framebox(2.0,0.8){~D~~E~~J~~K~}}
\put(3.8, 0.0){\framebox(1.2,0.8){~N~~O~}}
\put(8.2, 2.0){\framebox(1.2,0.8){~T~~X~}}
\put(8.3, 2.2){\vector(-1,-1){1.4}}
\put(8.8, 2.2){\vector(0,-1){1.4}}
\put(9.3, 2.2){\vector(1,-1){1.4}}
\put(6.4, 0.0){\framebox(1.6,0.8){~Q~~R~~S~}}
\put(8.2, 0.0){\framebox(1.6,0.8){~U~~V~}}
\put(10.0, 0.0){\framebox(1.2,0.8){~Y~~Z~}}
\end{picture} \\
Remoção de D &
\setlength{\unitlength}{.7cm}
\begin{picture}(10.6, 4.8)(0.0,0)
\put(4.0, 4.0){\framebox(0.8,0.8){~~}}
\put(4.4, 4.2){\vector(0,-1){1.4}}
\put(3.2, 2.0){\framebox(2.4,0.8){~C~~L~~P~~T~~X~}}
\put(3.3, 2.2){\vector(-3,-2){2.1}}
\put(3.8, 2.2){\vector(-1,-1){1.4}}
\put(4.2, 2.2){\vector(-1,-2){0.7}}
\put(4.6, 2.2){\vector(1,-2){0.7}}
\put(5.0, 2.2){\vector(1,-1){1.4}}
\put(5.5, 2.2){\vector(2,-1){2.8}}
\put(0.0, 0.0){\framebox(1.2,0.8){~A~~B~}}
\put(1.4, 0.0){\framebox(1.6,0.8){~E~~J~~K~}}
\put(3.2, 0.0){\framebox(1.2,0.8){~N~~O~}}
\put(4.6, 0.0){\framebox(1.6,0.8){~Q~~R~~S~}}
\put(6.4, 0.0){\framebox(1.2,0.8){~U~~V~}}
\put(7.8, 0.0){\framebox(1.2,0.8){~Y~~Z~}}
\end{picture}
\end{tabular}
\end{center}
}
\end{frame}

\begin{frame}

{
\scriptsize
\begin{center}
\begin{tabular}{cc}
A altura diminui &
\setlength{\unitlength}{.7cm}
\begin{picture}(10.6, 2.8)(0.0,0)
\put(3.2, 2.0){\framebox(2.4,0.8){~C~~L~~P~~T~~X~}}
\put(3.3, 2.2){\vector(-3,-2){2.1}}
\put(3.8, 2.2){\vector(-1,-1){1.4}}
\put(4.2, 2.2){\vector(-1,-2){0.7}}
\put(4.6, 2.2){\vector(1,-2){0.7}}
\put(5.0, 2.2){\vector(1,-1){1.4}}
\put(5.5, 2.2){\vector(2,-1){2.8}}
\put(0.0, 0.0){\framebox(1.2,0.8){~A~~B~}}
\put(1.4, 0.0){\framebox(1.6,0.8){~E~~J~~K~}}
\put(3.2, 0.0){\framebox(1.2,0.8){~N~~O~}}
\put(4.6, 0.0){\framebox(1.6,0.8){~Q~~R~~S~}}
\put(6.4, 0.0){\framebox(1.2,0.8){~U~~V~}}
\put(7.8, 0.0){\framebox(1.2,0.8){~Y~~Z~}}
\end{picture}\\
\\
B é removido &
\setlength{\unitlength}{.7cm}
\begin{picture}(10.6, 2.8)(0.0,0)
\put(3.2, 2.0){\framebox(2.4,0.8){~E~~L~~P~~T~~X~}}
\put(3.3, 2.2){\vector(-3,-2){2.1}}
\put(3.8, 2.2){\vector(-1,-1){1.4}}
\put(4.2, 2.2){\vector(-1,-2){0.7}}
\put(4.6, 2.2){\vector(1,-2){0.7}}
\put(5.0, 2.2){\vector(1,-1){1.4}}
\put(5.5, 2.2){\vector(2,-1){2.8}}
\put(0.0, 0.0){\framebox(1.2,0.8){~A~~C~}}
\put(1.4, 0.0){\framebox(1.6,0.8){~E~~J~~K~}}
\put(3.2, 0.0){\framebox(1.2,0.8){~N~~O~}}
\put(4.6, 0.0){\framebox(1.6,0.8){~Q~~R~~S~}}
\put(6.4, 0.0){\framebox(1.2,0.8){~U~~V~}}
\put(7.8, 0.0){\framebox(1.2,0.8){~Y~~Z~}}
\end{picture}
\end{tabular}
\end{center}
}
\end{frame}

\end{document}

