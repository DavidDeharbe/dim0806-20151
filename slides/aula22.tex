\documentclass{beamer}
%\documentclass{beamer}
\setbeamertemplate{footline}[frame number]

\usepackage[utf8x]{inputenc}
\usepackage[brazil,british]{babel}
\usetheme{default} 
\usecolortheme{beaver}
\newtranslation[to=brazil]{Theorem}{Teorema}
\newtranslation[to=brazil]{Definition}{Definição}
\newtranslation[to=brazil]{Example}{Exemplo}
\newtranslation[to=brazil]{Problem}{Exercício}
\newtranslation[to=brazil]{Solution}{Resolução}
\newtranslation[to=brazil]{Lemma}{Lema}
\newtranslation[to=brazil]{Corollary}{Corolário}
\logo{\includegraphics[width=1cm]{../img/logo-ppgsc-icon-text.png}}

\usepackage{graphicx}
\usepackage{clrscode3e}
\usepackage{hyperref}

\usepackage{pgf}
\usepackage{tikz}
\newcommand{\assert}[1]{\textcolor{blue}{#1}}

\usepackage{xspace}

\newcommand{\semester}[0]{2015.2}


\usepackage{pgf}
\usepackage{tikz}

\title{Aula 22: Árvores chanfradas}
\subtitle{\textit{Splay trees\/}}
\author{David Déharbe \\
  Programa de Pós-graduação em Sistemas e Computação \\
  Universidade Federal do Rio Grande do Norte \\
  Centro de Ciências Exatas e da Terra \\
  Departamento de Informática e Matemática Aplicada}
\date{}

\begin{document}
\selectlanguage{brazil}

\begin{frame}
  \titlepage
  Download me from \url{http://DavidDeharbe.github.io}.
\end{frame}

\begin{frame}
  \frametitle{Introdução}
  \tableofcontents

  \begin{enumerate}
    \item Avaliação de estruturas de dados: análise amortizada
      \begin{itemize}
        \item Contador binário
        \item Arranjos dinâmicos
      \end{itemize}
  \end{enumerate}

\end{frame}

\section{Análise amortizada}

\begin{frame}
\frametitle{Análise amortizada}

\begin{itemize}
\item Medição da complexidade ao longo de $n$ operações
\item Custo de execução, no pior caso: $T(n)$
\item Custo amortizado: $T(n)/n$
\item Exemplos
\begin{enumerate}
  \item contador binário
  \item arranjo dinâmico
\end{enumerate}
\end{itemize}
\end{frame}

\subsection{Método agregado}

\begin{frame}

\frametitle{Método agregado}

\begin{itemize}
\item Considere uma série de $n$ operações em alguma estrutura de dados, $n$
  qualquer.

\item Determina o custo $T(n)$ desta série de operações.

\item O custo médio de cada operação é $T(n)/n$.

\item Exemplo: contador binário
\end{itemize}

\end{frame}

\begin{frame}

\frametitle{Método agregado: um exemplo}

  \begin{codebox}
    \Procname{$\proc{Increment}(A)$}
    \zi \Comment $A$ é um arranjo de $k$ bits
    \li $i \gets 0$
    \li \While $i < \id{length}[A] \mbox{ and } A[i] = 1$
    \li \Do $A[i] \gets 0$
    \li   $i \gets i + 1$
        \End
    \li \If $i < \id{length}[A]$
    \li \Then $A[i] \gets 1$
        \End
  \end{codebox}

\begin{itemize}
\item O tamanho da entrada é $k$;
\item O custo da execução da $\proc{Increment}$ é proporcional ao número de bits
  invertidos.
\item No pior caso, a complexidade é proporcional a $k$;
\item Logo, o custo de uma sequência de $n$ operações é $O(n \times k)$.
\item Podemos prover uma análise mais precisa?
\end{itemize}

\end{frame}

\begin{frame}

\frametitle{Método agregado: um exemplo}

Em uma série de $n$ execuções de $\proc{Increment}$:
\begin{itemize}
\item $A[0]$ é invertido $n$ vezes;
\item $A[1]$ é invertido $n/2$ vezes;
\item $A[2]$ é invertido $n/4$ vezes;
\item $A[k-1]$ é invertido $n/2^{k-1}$ vezes;
\end{itemize}
O custo total para $n$ operações é 
\begin{eqnarray*}
T(n) & = & \sum_{i=1}^{\lfloor \lg n \rfloor} n/2^{i-1} \\
T(n) & < & \sum_{i=1}^{\infty} n/2^{i-1} \\
T(n) & < & n \times \sum_{i=1}^{\infty} 1/2^{i-1} = 2n
\end{eqnarray*}

Logo o custo amortizado de $\proc{Increment}$ é $T(n)/n = 2 \in \Theta(1)$: é \alert{constante}.

\end{frame}

\subsection{Método do contador}

\begin{frame}

\frametitle{Método do contador}

\begin{quote}
Quanto pagar para executar $n$ operações?
\end{quote}

\begin{itemize}

\item cada operação tem um custo real

\item a cada operação é atribuído um preço ($\approx$ custo amortizado individual)

\item os preços cobrados devem cobrir os custos reais

\item se o preço é maior que o custo:
\begin{itemize}
\item crédito é associado a elementos da estrutura de dados
\end{itemize}

\item se o custo é maior que o preço
\begin{itemize}
\item a diferença deve poder ser paga com os créditos disponíveis
\end{itemize}

\item custo amortizado total: soma dos preços cobrados
\end{itemize}

\end{frame}

\begin{frame}

\frametitle{Critérios de definição do custo}

\begin{itemize}

\item o custo amortizado total deve ser uma cota superior do custo real
\begin{itemize}
\item a estrutura de dados não ``empresta''
\end{itemize}

\item o crédito total associado aos elementos nunca pode ser negativo

\end{itemize}

\end{frame}

\begin{frame}

\frametitle{Método do contador: o contador binário}

\begin{itemize}

  \item a unidade de custo é inverter um bit

  \item preço para $0 \rightarrow 1$: 2
   
  \begin{itemize}

    \item 1 é o custo

    \item sobra $2 - 1 = 1$ de crédito, associado ao bit setado a 1

  \end{itemize}

  \item preço para $1 \rightarrow 0$: 0

    \begin{itemize}

      \item pago com o crédito

    \end{itemize}
  
\end{itemize}

\end{frame}

\begin{frame}

\frametitle{Método do contador: o contador binário}

  \begin{codebox}
    \Procname{$\proc{Increment}(A)$}
    \zi \Comment $A$ é um arranjo de $k$ bits
    \li $i \gets 0$
    \li \While $i < \id{length}[A] \mbox{ and } A[i] = 1$
    \li \Do $A[i] \gets 0$
    \li   $i \gets i + 1$
        \End
    \li \If $i < \id{length}[A]$
    \li \Then $A[i] \gets 1$
        \End
  \end{codebox}

\begin{itemize}
\item O custo do laço é pago usando os créditos dos bits setados a 1.

\item no máximo um único bit é setado a 1: custo 1, mais 1 que fica como crédito

\item logo, o custo de uma execução da operação $\proc{Increment}$ é 2

\item o custo de $n$ execuções da operação $\proc{Increment}$ é $O(n)$.

\end{itemize}

\end{frame}

\subsection{Método do potencial}

\begin{frame}

\frametitle{Método do potencial}

\begin{itemize}

\item crédito $\Longrightarrow$ \alert{potencial} para executar futuras operações

\item estrutura de dados inicia em um estado $D_0$

\item $n$ operações são executadas, 

\begin{itemize}

\item custo real $c_1, \ldots c_i, \ldots c_n$

\item levando aos estados $D_1, \ldots D_i, \ldots D_n$

\end{itemize}

\item A função $\Phi$ associa um número real $\Phi_i$ (potencial) ao estado
  $D_i$

\item \alert{Custo amortizado} $a_i = c_i + \Phi_i - \Phi_{i-1}$

\item Custo amortizado total
$$
\sum_{i=1}^n a_i = \sum_{i=1}^n c_i + \Phi_i - \Phi_{i-1} = \sum_{i=1}^n c_i + \Phi_n - \Phi_0
$$
\item \alert{se $\forall i \cdot \Phi_i \ge \Phi_0$, então $\sum_{i=1}^n a_i$ é uma cota superior do custo total real}.

\end{itemize}

\end{frame}

\begin{frame}

\frametitle{Método do potencial}
\framesubtitle{Relação com o método do contador}

\begin{itemize}

\item $\Phi_0 = 0$

\item mostrar que $\Phi_i \ge 0$, $\forall i$.

\item $\Phi_i > \Phi_{i-1} \approx \text{crédito}$ (potencial aumenta)

\item $\Phi_i < \Phi_{i-1} \approx \text{débito}$ (potencial diminui)

\end{itemize}

\end{frame}

\begin{frame}

\frametitle{Método do potencial: o contador binário}

\begin{itemize}

  \item a $i$-ésima chamada de $\proc{Increment}$ zera $t_i$ bits

  \item custo real $c_i = t_i + 1$ (zera $i$, seta 1)

  \item potencial: $\Phi_i = \text{quantidade de bits setados a 1}$

  \item naturalmente, $\forall i \cdot \Phi_i \ge 0$

  \item logo $\Phi_i \le \Phi_{i-1} - t_i + 1$

  \item $\Phi_i - \Phi_{i-1} \le 1 - t_i$

  \item custo amortizado de uma operação
    $$a_i = c_i + \Phi_i - \Phi_{i-1} \le (t_i + 1) - (1 - t_i) = 2$$

  \item custo amortizado de $n$ operações $O(n)$

\end{itemize}

\end{frame}

\section{Arranjos dinâmicos}

\begin{frame}
\frametitle{Arranjos dinâmicos}

\begin{itemize}

\item contêiner tabela, cuja capacidade adapta-se ao tamanho da coleção
\begin{itemize}
\item inserção e remoção: modelo pilha
\end{itemize}

\item inserção em uma tabela está cheia
\begin{itemize}
\item aloca uma nova tabela de capacidade maior
\item copia o conteúdo da tabela original para a nova tabela
\item libera o espaço ocupado pela tabela original
\item insere o novo elemento
\end{itemize}

\item política de expansão: dobra o tamanho

\item remoção, quando o fator de carga fica baixo
\begin{itemize}
\item aloca uma nova tabela de capacidade menor
\item copia o conteúdo da tabela original para a nova tabela
\item libera o espaço ocupado pela tabela original
\item remove o elemento
\end{itemize}

\item política de contração: divide a capacidade por dois quando fator de carga chega a $1/4$.
\end{itemize}
\end{frame}

\begin{frame}
\frametitle{Implementação: dados}

\begin{itemize}
\item $\attrib{T}{table}$ tabela com os elementos
\item $\attrib{T}{num}$ quantidade de elementos armazenados na tabela
\item $\attrib{T}{size}$ capacidade máxima
\end{itemize}

\end{frame}

\begin{frame}
\frametitle{Inserção}

\begin{codebox}
\Procname{$\proc{Insert}(T, k)$}
\li \If $\attrib{T}{size} \isequal 0$ \Comment alocação inicial
\li \Then $\attrib{T}{table} \gets \proc{Alloc-Table}(1)$
\li       $\attrib{T}{size} \gets 1$
\li \ElseIf $\attrib{T}{num} \isequal \attrib{T}{size}$ \Comment expansão
\li \Then $\id{tab} \gets \attrib{T}{table}$
\li   $\attrib{T}{table} \gets \proc{Alloc-Table}(\attrib{T}{size} \times 2)$
\li   \For $i \gets 1$ \To $\attrib{T}{size}$
\li   \Do $\attrib{T}{table}[i] \gets \id{tab}[i]$
      \End
\li   $\attrib{T}{size} \gets \attrib{T}{size} \times 2$
\li   $\proc{Free-Table}(tab)$
    \End
\li $\attrib{T}{num} \gets \attrib{T}{num}+1$ \Comment inserção
\li $\attrib{T}{table}[\attrib{T}{num}] \gets k$
\end{codebox}

\end{frame}

\begin{frame}
\frametitle{Análise}

Cenário: $n$ inserções em uma tabela inicialmente vazia
\begin{itemize}
\item custo da operação $i$
\begin{itemize}
\item sem expansão: 1
\item com expansão: $i$
\end{itemize}
\item Pior caso $O(n^2)$
\item Melhoria desta análise com análise amortizada
\end{itemize}

\end{frame}

\begin{frame}
\frametitle{Método agregado}
\begin{itemize}
\item Há expansão quando $i$ é uma potência de 2.
\item Custo total:
\begin{eqnarray*}
\sum_{i=1}^{n} c_i & \le & n + \sum_{j=0}^{\lfloor \log_2 n \rfloor} 2^j \\
& < & n + 2n \\
& = & 3n
\end{eqnarray*}
\end{itemize}

\end{frame}

\begin{frame}
\frametitle{Método do contador}
\begin{itemize}
\item Custo 1 para atribuir uma posição de $\attrib{T}{table}$
\item Preço de uma operação de inserção: 3
\item Pagamento de uma operação de inserção, sem expansão:
\begin{itemize}
\item 1 vai para a inserção (custo real) $i = \attrib{T}{size}/2 + j$
\item sobra 2 de crédito
\item 1 crédito na posição de inserção $i$
\item 1 crédito para um outro elemento da tabela $j$
\end{itemize}
\item Pagamento da expansão:
\begin{itemize}
\item todos os itens tem um crédito
\item o crédito paga a expansão
\end{itemize}
\end{itemize}

\end{frame}

\begin{frame}
\frametitle{Método do potencial}
\begin{itemize}
\item Após cada expansão: $\Phi = 0$
\item $\Phi \uparrow$ quando a tabela é preenchida
\item até a tabela ser preenchida, e $\Phi$ é gasto na expansão
\item $\Phi(T) = 2 \times \attrib{T}{num} - \attrib{T}{size}$
\begin{itemize}
\item Inicialmente, e após cada expansão $\Phi(T) = 0$
\item Antes de uma expansão $\Phi(T) = 2 \times \attrib{T}{num} - \attrib{T}{size} = \attrib{T}{size}$
\item A cada momento $\Phi(T) \ge 0$
\item Logo, a soma dos custos amortizados é uma cota superior dos custos reais.
\end{itemize}
\end{itemize}

\end{frame}

\begin{frame}

\frametitle{Método do potencial}

\begin{itemize}
\item $n_i$: número de elementos após operação $i$
\item $s_i$: capacidade após a operação $i$
\item inicialmente $n_0 = s_0 = 0$, $\Phi_0 = 0$
\item $n_i = n_{i-1} + 1$
\end{itemize}

\end{frame}

\begin{frame}

\frametitle{Método do potencial}

se a operação $i$ não tem expansão:
\begin{itemize}
\item $s_i = s_{i-1}$ e 
\end{itemize}
\begin{eqnarray*}
a_i & = & c_i + \Phi_i - \Phi_{i-1} \\
    & = & 1 + (2 \cdot n_i - s_i) + (2 \cdot n_{i-1} - s_{i-1}) \\
    & = & 1 + (2 \cdot n_i - s_i) + (2 \cdot (n_i - 1) - s_i) \\
    & = & 3
\end{eqnarray*}

\end{frame}

\begin{frame}

\frametitle{Método do potencial}

se a operação $i$ tem expansão:
\begin{itemize}
\item $s_i = 2 \times s_{i-1}$, $s_{i-1} = n_{i-1} = n_i - 1$ e 
\end{itemize}
\begin{eqnarray*}
a_i & = & c_i + \Phi_i - \Phi_{i-1} \\
    & = & n_i + (2 \cdot n_i - s_i) - (2 \cdot n_{i-1} - s_{i-1}) \\
    & = & n_i + (2 \cdot n_i - 2 \cdot (n_i - 1)) - (2 \cdot (n_i - 1) - (n_i - 1)) \\
    & = & 3
\end{eqnarray*}

\end{frame}

\begin{frame}

\frametitle{Remoção com contração de capacidade}

\begin{itemize}
\item expansão dobra capacidade
\item contração ocorre quando capacidade chega a $1/4$ 
\item por quê não escolher $1/2$?
\end{itemize}

\begin{codebox}
\Procname{$\proc{Remove}(T)$}
\li \If $\attrib{T}{num} \isequal 0$ \Then
\li   $\attrib{T}{size} \gets 0$
\li   $\proc{Free-Table}(\attrib{T}{table})$
\li \ElseIf $\attrib{T}{num} \neq 0$ and $\attrib{T}{num} * 4 < \attrib{T}{size}$ \Comment contração
\li \Then $\id{tab} \gets \attrib{T}{table}$
\li   $\attrib{T}{table} \gets \proc{Alloc-Table}(\attrib{T}{size} / 2)$
\li   \For $i \gets 1$ \To $\attrib{T}{num}$
\li   \Do $\attrib{T}{table}[i] \gets \id{tab}[i]$
      \End
\li   $\attrib{T}{size} \gets \attrib{T}{size} / 2$
\li   $\proc{Free-Table}(tab)$
    \End
\li $\attrib{T}{num} \gets \attrib{T}{num}-1$ \Comment remoção
\end{codebox}

\end{frame}

\begin{frame}
\frametitle{Análise com o método do potencial}

\begin{itemize}
\item $\Phi = 0$ logo após contração ou expansão
\item Fator de carga $\alpha(T) = \attrib{T}{num} / \attrib{T}{size}$, se $T$ não for vazia, 1 se for.
\item $\attrib{T}{num} = \alpha(T) \cdot \attrib{T}{size}$
\item Função potencial
\begin{eqnarray*}
\Phi(T) & = & \left\{ \begin{array}{ll}
2 \cdot \attrib{T}{num} - \attrib{T}{size} & \text{se } \alpha{T} \ge 1/2 \\
\attrib{T}{size}/2 - \attrib{T}{num} & \text{se } 1/4 \le \alpha{T} < 1/2 \\
\end{array}
\right.
\end{eqnarray*}
\end{itemize}
A função potencial nunca é negativa: obteremos uma cota superior do custo real
\end{frame}

\begin{frame}

\frametitle{Análise com o método do potencial}
\framesubtitle{Justificativa intuitiva}

\begin{itemize}
\item antes de uma expansão:
\begin{itemize}
\item $\alpha(T) = 1$
\item $\attrib{T}{num} = \attrib{T}{size}$,
\item $\Phi(T) = \attrib{T}{num}$
\item permite cópia de $\attrib{T}{num}$ elementos.
\end{itemize}
\item antes de uma contração:
\begin{itemize}
\item $\alpha(T) = 1/4$
\item $\attrib{T}{num} \cdot 4 = \attrib{T}{size}$,
\item $\Phi(T) = \attrib{T}{size}/2 - \attrib{T}{num} = \attrib{T}{num}$
\item permite cópia de $\attrib{T}{num}$ elementos.
\end{itemize}
\end{itemize}

\end{frame}

\begin{frame}

\frametitle{Análise com o método do potencial}
\framesubtitle{Notações}

\begin{center}
\begin{tabular}{rcl}
$s_i$ & : & capacidade após operação $i$ \\
$n_i$ & : & número de elementos após operação $i$ \\
$\Phi_i$ & : & potencial após operação $i$ \\
$\alpha_i$ & : & fator de carga após operação $i$
\end{tabular}
\end{center}
Inicialmente: $s_0 = n_0 = \Phi_0 = 0, \alpha_0 = 1$

\end{frame}

\begin{frame}

\frametitle{Análise com o método do potencial}
\framesubtitle{Inserção}

A operação $i$ é uma inserção:
\begin{itemize}
\item já fizemos a análise no caso $\alpha_{i-1} \ge 1/2$: $a_i = 3$
\item se $\alpha_{i-1} < 1/2$, não há expansão
\only<2>{
\item se $\alpha_i < 1/2$:
\begin{eqnarray*}
a_i & = & c_i + \Phi_i - \Phi_{i-1} \\
& = & 1 + (s_i/2 - n_i) - (s_{i-1}/2 - n_{i-1}) \\
& = & 1 + (s_i/2 - n_i) - (s_i/2 - (n_i-1)) \\
& = & 0
\end{eqnarray*}
O custo amortizado de uma inserção é \alert{3} no máximo.
}
\only<1>{
\item se $\alpha_i \ge 1/2$:
\begin{eqnarray*}
a_i & = & c_i + \Phi_i - \Phi_{i-1} \\
& = & 1 + (2\cdot n_i - s_i) - (s_{i-1}/2 - n_{i-1}) \\
& = & 1 + (2\cdot (n_{i-1}+1) - s_{i-1}) - (s_{i-1}/2 - n_{i-1}) \\
& = & 3 \cdot n_{i-1} - \frac{3}{2}\cdot s_{i-1} + 3 \\
& = & 3 \cdot \alpha_{i-1} \cdot s_{i-1} - \frac{3}{2}\cdot s_{i-1} + 3 \\
& < & \frac{3}{2} \cdot s_{i-1} - \frac{3}{2}\cdot s_{i-1} + 3 \\
& = & 3
\end{eqnarray*}
}
\end{itemize}

\end{frame}

\begin{frame}
\frametitle{Análise com o método do potencial}
\framesubtitle{Remoção}
$$n_i = n_{i-1} - 1$$
\begin{itemize}
\only<1>{
\item se $\alpha_{i-1} < 1/2$,
\begin{itemize}
\item sem contração: $s_i = s_{i-1}$
\begin{eqnarray*}
a_i & = & c_i + \Phi_i - \Phi_{i-1} \\
& = & 1 + (s_i/2 - n_i) - (s_{i-1}/2 - n_{i-1}) \\
& = & 1 + (s_i/2 - (n_{i-1} - 1)) - (s_i/2 - n_{i-1}) \\
& = & 2
\end{eqnarray*}
\item com contração: $n_{i-1} = n_i + 1 = s_i/2 = s_{i-1}/4$ e $c_i = n_i + 1$
\begin{eqnarray*}
a_i & = & c_i + \Phi_i - \Phi_{i-1} \\
& = & n_i + 1 + (s_i/2 - n_i) - (s_{i-1}/2 - n_{i-1}) \\
& = & n_{i-1} + (n_{i-1} - (n_{i-1} - 1)) - (2\cdot n_{i-1} - n_{i-1}) \\
& = & 1
\end{eqnarray*}
\end{itemize}
}
\only<2-3>{
\item se $\alpha_{i-1} \ge 1/2$,
\item não há contração: $s_i = s_{i-1}$, $n_i = n_{i-1} - 1$
}
\only<2>{
\item se $\alpha_i \ge 1/2$
\begin{eqnarray*}
a_i & = & c_i + \Phi_i - \Phi_{i-1} \\
& = & 1 + (2 \cdot n_i - s_i) - (2 \cdot n_{i-1} - s_{i-1}) \\
& = & 1 + (2 \cdot (n_{i-1} - 1) - s_{i-1}) - (2 \cdot n_{i-1} - s_{i-1}) \\
& = & -1
\end{eqnarray*}
}
\only<3>{
\item se $\alpha_i < 1/2$
\begin{eqnarray*}
a_i & = & c_i + \Phi_i - \Phi_{i-1} \\
& = & 1 + (s_i/2 - n_i) - (2 \cdot n_{i-1} - s_{i-1}) \\
& = & 1 + (s_{i-1}/2 - (n_{i-1} - 1)) - (2 \cdot n_{i-1} - s_{i-1}) \\
& = & 2 + \frac{3}{2} \cdot s_{i-1} - 3 \cdot n_{i-1} \\
& = & 2 + \frac{3}{2} \cdot s_{i-1} - 3 \cdot \alpha_{i-1} \cdot s_{i-1} \\
& \le & 2
\end{eqnarray*}
O custo amortizado de uma remoção é \alert{2} no máximo.
}
\end{itemize}
\end{frame}

\begin{frame}
\frametitle{Arranjos dinâmicos}
\framesubtitle{Síntese da análise de complexidade amortizada}
\begin{itemize}
  \item inserção tem custo amortizado $O(1)$
  \item remoção tem custo amortizado $O(1)$
\end{itemize}
\end{frame}

\begin{frame}
\frametitle{Arranjos dinâmicos}
\framesubtitle{Exercício}

\begin{enumerate}
\item Projetar uma estrutura de dados para arranjos dinâmicos, que tem apenas
  operação de inserção, tal que esta tenha complexidade constante, \emph{no pior
    caso}.
  \begin{itemize}
  \item hipótese: supõe-se que o custo de uma alocação dinâmica é $\Theta(1)$.
  \item dica: pode inspirar-se na aplicação do método do contado desta aula.
  \end{itemize}
\item Estender a estrutura projetada com uma operação de remoção do último
  elemento (\proc{Pop-Back}), tal que tanto a inserção quanto a remoção
  tenham complexidade constante, no pior caso.
\end{enumerate}

\end{frame}

\section{Árvores chanfradas}

\begin{frame}
\frametitle{Árvores chanfradas}
\framesubtitle{\it splay trees}

\begin{itemize}
\item árvores binárias de busca
\item auto-ajustáveis
\item não necessitam de atributos adicionais
\item complexidade amortizada $O(\log n)$
\end{itemize}
\end{frame}

\begin{frame}
\frametitle{Árvores chanfradas}

\begin{itemize}
\item uma única operação básica: chanfrar
\begin{itemize}
\item $\proc{Splay}(T, k)$: chanfra a árvore $T$ com relação ao valor $k$
\item no término da operação $k$ está na raiz (ou o maior valor $< k$,
ou o menor valor $> k$
\end{itemize}
\item busca, inserção e remoção são todas realizadas após chanfrar a árvore
\end{itemize}
\end{frame}

\begin{frame}
\frametitle{Realização das operações e o chanframento}

\begin{itemize}
\item uma única operação básica: chanfrar
\begin{itemize}
\item $\proc{Splay}(T, k)$: chanfra a árvore $T$ com relação ao valor $k$
\item busca a chave $k$
\item remaneja $T$ tal que $k$ fica na raiz (ou o maior valor $< k$,
ou o menor valor $> k$
\end{itemize}
\item busca, inserção e remoção são todas realizadas após chanfrar a árvore
\begin{itemize}
\item $\proc{Search}(T, k)$: $\proc{Splay}(T, k)$ e consulta a raiz
\item $\proc{Insert}(T, k)$: $\proc{Splay}(T, k)$ e insere $k$ na raiz
\item $\proc{Remove}(T, k)$: $\proc{Splay}(T, k)$, remove a raiz, $T_L$ e $T_R$ árvores resultantes, $\proc{Splay}(T_L, k)$, e torne $T_R$ a sub-árvore da nova raiz de $T_L$.
\item $O(1)$ aplicações de $\proc{Splay}$ + $O(1)$ operações.
\end{itemize}
\end{itemize}
\end{frame}

\begin{frame}
\frametitle{Busca}

\begin{itemize}
\item $\proc{Search}(T, x)$: $\proc{Splay}(T, x)$ e consulta a raiz
\end{itemize}

\begin{tabular}{c}
\includegraphics{fig/splay-remove-1.pdf} \pause \\
$\proc{Splay}(T, x)$ \pause \\
\includegraphics{fig/splay-remove-2.pdf}
\end{tabular}

\end{frame}

\begin{frame}
\frametitle{Inserção}

\begin{itemize}
\item $\proc{Insert}(T, k)$: $\proc{Splay}(T, k)$ e insere $k$ na raiz
\end{itemize}

\begin{tabular}{c}
\only<1-3>{\includegraphics{fig/splay-insert-1.pdf} \\}
\only<2-3>{$\proc{Splay}(T, x)$ \\}
\only<3>{\includegraphics{fig/splay-insert-2.pdf} \\}
\only<4>{\includegraphics[height=.2\textheight]{fig/splay-insert-2.pdf} \\}
\only<5>{\includegraphics[height=.15\textheight]{fig/splay-insert-2.pdf} \\}
\only<4-5>{$\attrib{n}{val} \gets x;$ \\}
\only<4-5>{$\attrib{n}{left} \gets \attrib{\attrib{T}{root}}{left};$ \\}
\only<4-5>{$\attrib{\attrib{T}{root}}{left} \gets \const{Nil};$ \\}
\only<4-5>{$\attrib{n}{right} \gets \attrib{T}{root};$ \\}
\only<4-5>{$\attrib{T}{root} \gets n$ \\}
\only<5>{\includegraphics[height=.35\textheight]{fig/splay-insert-3.pdf}}
\end{tabular}

\end{frame}

\begin{frame}
\frametitle{Remoção}

\begin{itemize}
\item $\proc{Remove}(T, k)$: $\proc{Splay}(T, k)$, remove a raiz, $T_L$ e $T_R$ árvores resultantes, $\proc{Splay}(T_L, k)$, e torne $T_R$ a sub-árvore da nova raiz de $T_L$.
\end{itemize}

\begin{tabular}{c}
\only<1-3>{\includegraphics{fig/splay-remove-1.pdf} \\}
\only<2-3>{$\proc{Splay}(T, k)$ \\}
\only<3-5>{\includegraphics{fig/splay-remove-2.pdf} \\}
\only<4-5>{remove root: $n = \attrib{T}{root}; T_L = \attrib{n}{left}; T_R = \attrib{n}{right}; \proc{Free}(n)$ \\}
\only<5-7>{\includegraphics{fig/splay-remove-3.pdf} \\}
\only<6-7>{$\proc{Splay}(T_L, k)$ \\}
\only<7-9>{\includegraphics{fig/splay-remove-4.pdf} \\}
\only<8-9>{$\attrib{T_L}{right} \gets T_R$ \\}
\only<9-10>{\includegraphics{fig/splay-remove-5.pdf}}
\end{tabular}

\end{frame}

\begin{frame}
\frametitle{Chanfrar um valor}

\begin{itemize}
\item aplicar rotações até chegar à raiz
\item se $x$ for um filho da raiz: rotação simples (\alert{caso 1})
\item $\attrib{x}{up} = y$ e $\attrib{y}{up} = z$
\begin{itemize}
\item se $x = \attrib{y}{left}$ e $y = \attrib{z}{left}$: rotação simples de $y$, rotação simples de $x$ (\alert{caso 2})
\item se $x = \attrib{y}{right}$ e $y = \attrib{z}{right}$: rotação simples de $y$, rotação simples de $x$  (\alert{caso 2})
\item se $x = \attrib{y}{right}$ e $y = \attrib{z}{left}$: rotação dupla de $x$  (\alert{caso 3})
\item se $x = \attrib{y}{left}$ e $y = \attrib{z}{right}$: rotação dupla de $x$  (\alert{caso 3})
\end{itemize}
\item complexidade no pior caso $O(n)$
\item análise amortizada mostra que todas as operações de chanframento tem
custo amortizado $O(\log n)$
\end{itemize}
\end{frame}

\begin{frame}
\frametitle{Chanfrar um valor --- caso 1}
\framesubtitle{se $x$ for um filho da raiz: rotação simples}

\begin{tabular}{c}
\includegraphics[height=.4\textheight]{fig/splay-1-1.pdf} \\
$\proc{Rotate-Simple-Right}(x, y)$ \\
\includegraphics[height=.4\textheight]{fig/splay-1-2.pdf}
\end{tabular}

\end{frame}

\begin{frame}
\frametitle{Chanfrar um valor --- caso 2}
\framesubtitle{$x = \attrib{y}{left}$ e $y = \attrib{z}{left}$ (e simétrico)}

\begin{tabular}{c}
\includegraphics[height=.35\textheight]{fig/splay-2-1.pdf} \\
$\proc{Rotate-Simple-Right}(y, z)$ \\
$\proc{Rotate-Simple-Right}(x, y)$ \\
\includegraphics[height=.35\textheight]{fig/splay-2-2.pdf}
\end{tabular}

\end{frame}

\begin{frame}
\frametitle{Chanfrar um valor --- caso 3}
\framesubtitle{$x = \attrib{y}{right}$ e $y = \attrib{z}{left}$}

\begin{tabular}{c}
\includegraphics[height=.4\textheight]{fig/splay-3-1.pdf} \\
$\proc{Rotate-Double-Right}(x, y, z)$ \\
\includegraphics[height=.4\textheight]{fig/splay-3-2.pdf}
\end{tabular}

\end{frame}

\begin{frame}
\frametitle{Chanfrar um valor: exemplo}

\begin{tabular}[t]{ccc}
\includegraphics[height=.8\textheight]{fig/splay-do-1.pdf} &
Caso 3 &
\includegraphics[height=.7\textheight]{fig/splay-do-2.pdf}
\end{tabular}

\end{frame}

\begin{frame}
\frametitle{Chanfrar um valor: exemplo}

\begin{tabular}[t]{ccc}
\includegraphics[height=.7\textheight]{fig/splay-do-2.pdf} &
Caso 2 &
\includegraphics[height=.6\textheight]{fig/splay-do-3.pdf}
\end{tabular}

\end{frame}

\begin{frame}
\frametitle{Chanfrar um valor: exemplo}

\begin{tabular}[t]{ccc}
\includegraphics[height=.6\textheight]{fig/splay-do-3.pdf} &
Caso 1 &
\includegraphics[height=.5\textheight]{fig/splay-do-4.pdf}
\end{tabular}

\end{frame}

\begin{frame}
\frametitle{Interlúdio matemático}
\framesubtitle{Seja $a$, $b$ dois números reais quaisquer}

\begin{eqnarray*}
0 & \le & (a-b)^2 = a^2-2ab+b^2\\
4ab & \le & a^2+2ab+b^2 \\
ab & \le & \frac{(a+b)^2}{4}\\
ab^{\frac{1}{2}} & \le & \frac{a+b}{2} \\
\log (ab^{\frac{1}{2}}) & \le & \log\frac{a+b}{2} \\
\log (a^{\frac{1}{2}}) + \log(b^{\frac{1}{2}}) & \le & \log\frac{a+b}{2} \\
\frac{1}{2}\times\log a + \frac{1}{2}\times\log b & \le & \log\frac{a+b}{2} \\
\frac{(\log a + \log b)}{2} & \le & \log\frac{a+b}{2}
\end{eqnarray*}

\end{frame}

\begin{frame}
\frametitle{Análise amortizada: método do potencial}
\begin{itemize}
\item seja $T_i(x)$ a sub-árvore enraizada em $x$ após a operação $i$.
\item $|T|$: número de nós em $T$
\item o potencial fica distribuído entre os nós ($\phi_i = \sum_x \gamma_i(x)$)
\item invariante: $\gamma_i(x) = \lfloor \log | T_i(x) | \rfloor$
\item a complexidade amortizada de $\proc{Splay}$ é $O(\log n)$

  $T(x)$: sub-árvore enraizada em $x$

\item propriedade: o custo de $\proc{Splay}(T, x)$ e manter o invariante
  é menor ou igual a $3 \times (\gamma(T) - \gamma(x)) + O(1)$. 
(\alert{vamos mostrar isto depois})
\item logo $\proc{Splay}$ custa no máximo $3 \lfloor \log n \rfloor + O(1) \in |\Theta(\log n)$.
\item para pagar inserção, busca ou remoção, basta pagar $O(\log n)$.
\end{itemize}
\end{frame}

\begin{frame}
\frametitle{Análise amortizada: método do potencial}
\begin{itemize}
\item $\gamma(x)$ e $\gamma'(x)$: potencial em $x$ antes e depois das rotações
  nos casos 1, 2 e 3
\item $\proc{Splay}$ cai $k \ge 0$ vezes nos casos 2 e 3 e possivelmente uma vez
  no caso 1.
\item vamos mostrar que o custo em cada caso é
\begin{itemize}
\item caso 1: $\gamma'(x) - \gamma(x) + O(1)$
\item caso 2: $3 \times(\gamma'(x) - \gamma(x))$
\item caso 3: $3 \times(\gamma'(x) - \gamma(x))$
\end{itemize}
\item somando, obtemos $3(\gamma(T) - \gamma(x)) + O(1)$

\end{itemize}
\end{frame}

\begin{frame}
\frametitle{Caso 1}
\framesubtitle{$y = \attrib{x}{up}, \attrib{y}{up} = \const{Nil}$}

$\gamma'$: potencial depois, $\gamma$: potencial antes

\begin{itemize}
\item os únicos nós que alteram seu potencial são $x$ e $y$,
\item logo $\phi' - \phi = (\gamma'(x)+\gamma'(y))-(\gamma(x)+\gamma(y))$
\item $\gamma'(x) = \gamma(y)$, $\gamma'(y) \le \gamma'(x)$, $\gamma(x) \le \gamma'(x)$
\item logo 
\begin{eqnarray*}
\gamma'(x) + \gamma'(y) - \gamma(x) - \gamma(y) & = & \gamma'(y) - \gamma(x) \\
& \le & \gamma'(x) - \gamma(x) \\
& \le & 3 \times(\gamma'(x) - \gamma(x))
\end{eqnarray*}
\item é necessária apenas uma rotação:
$$a = c + \phi'- \phi \le 1 + \gamma'(x) - \gamma(x)$$
\end{itemize}
\end{frame}

\begin{frame}
\frametitle{Caso 2}
\framesubtitle{$y = \attrib{x}{up}, x = \attrib{y}{left}, z = \attrib{y}{up}, y = \attrib{z}{left}$}

\begin{itemize}
\item temos: $\gamma'(x) = \gamma(z)$, $\gamma'(y) \le \gamma'(x)$,
$\gamma'(z) \le \gamma'(x)$, $\gamma(y) \ge \gamma(x)$
\item logo 
\begin{eqnarray*}
\lefteqn{a = c + \phi'(x) - \phi(x)} \\
& = & 2 + \gamma'(x) - \gamma(x) + \gamma'(y) - \gamma(y) + \gamma'(z) - \gamma(z) \\
 & = & 2 + (\gamma'(x) - \gamma(z)) + \gamma'(y) + \gamma'(z) - \gamma(x) - \gamma(y) \\
& = & 2 + \gamma'(y) + \gamma'(z) - \gamma(x)) - \gamma(y) \\
& \le & 2 + \gamma'(x) + \gamma'(z) - \gamma(x) - \gamma(x) \\
& = & 2 + \gamma'(x) + \gamma'(z) - 2 \times \gamma(x)
\end{eqnarray*}
\item Vamos substituir $\gamma'(z)$ nesta expressão...
\end{itemize}
\end{frame}

\begin{frame}
\frametitle{Caso 2 (continuação)}

\begin{eqnarray*}
\frac{\log a + \log b} 2 & \le & \log \left( \frac{a+b}{2} \right)
\end{eqnarray*}

\begin{eqnarray*}
\gamma(x) + \gamma'(z) & = & \log |T(x)| + \log |T'(z)| \\
& \le & 2 \times \log \left( \frac{|T(x)| + |T'(z)|}{2} \right)
\end{eqnarray*}
\begin{itemize}
\item observe que $|T(x)| + |T'(z)| \le |T'(x)|$
\end{itemize}
\begin{eqnarray*}
\gamma(x) + \gamma'(z) & \le & 2 \times \log \left( \frac{|T'(x)|}{2} \right) \\
 & = & 2 \times \log |T'(x)| - 2 \times \log 2 \\
 & = & 2 \times \gamma'(x) - 2
\end{eqnarray*}
Logo:
\begin{eqnarray*}
\gamma'(z) & \le & 2 \times \gamma'(x) - 2 - \gamma(x)
\end{eqnarray*}
\end{frame}

\begin{frame}
\frametitle{Caso 2 (continuação)}

Temos:
\begin{eqnarray*}
\lefteqn{a = c + \phi'(x) - \phi(x)} \\
& \le & 2 + \gamma'(x) + \gamma'(z) - 2 \times \gamma(x)
\end{eqnarray*}
e:
\begin{eqnarray*}
\gamma'(z) & \le & 2 \times \gamma'(x) - 2 - \gamma(x).
\end{eqnarray*}
Logo, podemos prosseguir:
\begin{eqnarray*}
a & \le & 2 + \gamma'(x) + \gamma'(z) - 2 \times \gamma(x) \\
  & \le & 2 + \gamma'(x) + 2 \times \gamma'(x) - 2 - \gamma(x) - 2 \times \gamma(x) \\
  & = & 3 \times \gamma'(x) - 3 \times \gamma(x) \\
  & = & 3 \times (\gamma'(x) - \gamma(x))
\end{eqnarray*}
\end{frame}

\begin{frame}

\frametitle{Caso 3}

\begin{itemize}
\item Análogo ao caso 2
\item Lemas úteis:
\begin{enumerate}
\item $\gamma'(x) = \gamma(z)$;
\item $\gamma(y) \ge \gamma(x)$;
\item $\gamma'(y)+\gamma'(z) \le 2 \times \gamma'(x) - 2$.
\item $\gamma(x) \le \gamma'(x)$.
\end{enumerate}
\end{itemize}

\begin{eqnarray*}
a & = & c + [\gamma'(x)+\gamma'(y)+\gamma'(z)] - [\gamma(x)+\gamma(y)+\gamma(z)]\\
& = & 2 + \gamma'(y)+\gamma'(z)-\gamma(x)-\gamma(y) \quad \mbox{ por 1)}\\
& \le & 2 + \gamma'(y)+\gamma'(z) - 2 \times \gamma(x) \quad \mbox{ por 2)}\\
& \le & 2 + 2 \times \gamma'(x) - 2 - 2 \times \gamma(x) \quad \mbox{ por 3)}\\
& = & 2 \times (\gamma'(x) - \gamma(x)) \\
& \le & 3 \times (\gamma'(x) - \gamma(x))
\end{eqnarray*}

\end{frame}

\begin{frame}

\frametitle{Conclusões}

\begin{itemize}
\item Análise amortizada pode fornecer resultados mais apurados
\item outro exemplo: árvores rubro-negras
\begin{itemize}
\item inserção: $O(\log n)$
\item mas o custo amortizado de $m$ inserções em uma árvore de $n$ nós
  é $O(n+m)$
\end{itemize}
\end{itemize}

\end{frame}

\end{document}
