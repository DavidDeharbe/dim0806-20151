\documentclass{beamer}
%\documentclass{beamer}
\setbeamertemplate{footline}[frame number]

\usepackage[utf8x]{inputenc}
\usepackage[brazil,british]{babel}
\usetheme{default} 
\usecolortheme{beaver}
\newtranslation[to=brazil]{Theorem}{Teorema}
\newtranslation[to=brazil]{Definition}{Definição}
\newtranslation[to=brazil]{Example}{Exemplo}
\newtranslation[to=brazil]{Problem}{Exercício}
\newtranslation[to=brazil]{Solution}{Resolução}
\newtranslation[to=brazil]{Lemma}{Lema}
\newtranslation[to=brazil]{Corollary}{Corolário}
\logo{\includegraphics[width=1cm]{../img/logo-ppgsc-icon-text.png}}

\usepackage{graphicx}
\usepackage{clrscode3e}
\usepackage{hyperref}

\usepackage{pgf}
\usepackage{tikz}
\newcommand{\assert}[1]{\textcolor{blue}{#1}}

\usepackage{xspace}

\newcommand{\semester}[0]{2015.2}


\usepackage{pgf}
\usepackage{tikz}

\title{Aula 22: Extensão de estruturas de dados}
\author{David Déharbe \\
  Programa de Pós-graduação em Sistemas e Computação \\
  Universidade Federal do Rio Grande do Norte \\
  Centro de Ciências Exatas e da Terra \\
  Departamento de Informática e Matemáica Aplicada}
\date{06 de maio de 2015}

\begin{document}
\selectlanguage{brazil}

\begin{frame}
  \titlepage
\end{frame}

\begin{frame}
  \frametitle{Plano}
  \tableofcontents

Referência: Cormen, cap 15.
\end{frame}

\section{Introdução}

\begin{frame}
\frametitle{Motivação}

\begin{itemize}
\item Estruturas de dados prontas não resolvem todos os problemas práticos
\item Discernimento para saber adaptar uma estrutura de dados ao problema
\item Conservar as boas propriedades da estrutura de dados
\begin{enumerate}
  \item complexidade
  \item propriedades de ordenação
\end{enumerate}
\end{itemize}
\end{frame}

\section{O problema da seleção}

\begin{frame}

\frametitle{Descrição}

\begin{itemize}
\item Coleção dinâmica de dados ordenados
\begin{itemize}
\item inserção
\item busca
\item remoção
\end{itemize}
\item + seleção do dado na $k$-ésima posição da classificação
\item + cálculo da classificação de um dado
\end{itemize}
\pause
\alert{propostas?}

\end{frame}

\begin{frame}

\frametitle{Propostas}

\begin{enumerate}
\item arranjo
\item árvore balanceada
\end{enumerate}

\end{frame}

\begin{frame}

\frametitle{Arranjo}

\begin{enumerate}
\item ordenado: acesso direto
  \begin{itemize}
    \item inserção, remoção são ineficientes
  \end{itemize}
\item não ordenado: algoritmo similar ao \textit{quick sort\/}
  \begin{itemize}
    \item busca, remoção são ineficientes
  \end{itemize}
\end{enumerate}

\end{frame}

\begin{frame}

\frametitle{Árvore balanceada}

\begin{enumerate}
\item árvore rubro-negra
\item associar a cada nó o tamanho da sub-árvore enraizada nele
\end{enumerate}

\begin{codebox}
\Procname{$\proc{Size}(n)$}
\li \If $n \isequal \const{Nil}$
\li \Then \Return 0
\li \Else \Return $\attrib{n}{size}$
\End
\end{codebox}

\end{frame}

\begin{frame}
\frametitle{Seleção do dado na posição $k$}

\begin{codebox}
\Procname{$\proc{Select}(n, k)$}
\li $\id{sl} \gets \proc{Size}(\attrib{n}{left})$
\li \If $k \le \id{sl}$
\li \Then \Return $\proc{Select}(\attrib{n}{left}, k)$
\li \ElseIf $k \isequal \id{sl}+1$
\li \Then \Return $n$
\li \ElseNoIf \Return $\proc{Select}(\attrib{n}{right}, k - 1 - \id{sl})$
  \End
\end{codebox}

\pause
\alert{Complexidade?}

\end{frame}

\begin{frame}
\frametitle{Cálculo da posição de um dado $x$}

\begin{codebox}
\Procname{$\proc{Rank}(n, x)$}
\li \If $x \isequal \attrib{n}{key}$
\li \Then \Return $\proc{Size}(\attrib{n}{left}) + 1$
\li \ElseIf $x < \attrib{n}{key}$
\li \Then \Return $\proc{Rank}(\attrib{n}{left}, x)$
\li \ElseNoIf \Return $\proc{Size}(\attrib{n}{left}) + 1 + \proc{Rank}(\attrib{n}{right}, x)$
  \End
\end{codebox}

\pause
\alert{Complexidade?}

\end{frame}

\begin{frame}
\frametitle{Como atualizar o atributo $\id{size}$?}

\begin{itemize}
\item Invariant: $\attrib{x}{size} = 1 + \proc{Size}(\attrib{x}{left}) + \proc{Size}(\attrib{x}{right})$
\item Inserção:
\begin{enumerate}
\item descida recursiva na árvore até o ponto de inserção
\item criação de um novo vértice
\item subida até a raiz, com rotações
\end{enumerate}
\item Remoção:
\begin{enumerate}
\item descida recursiva na árvore até o ponto de remoção
\item possivelmente remoção auxiliar
\item subida até a raiz, com rotações
\end{enumerate}
\end{itemize}
\end{frame}

\begin{frame}
\frametitle{Como atualizar o atributo $\id{size}$?}
\framesubtitle{Inserção}

\begin{itemize}
\item descida até o ponto de inserção: incremento de $\id{size}$
\pause\item criação de um novo vértice
\begin{codebox}
\Procname{$\proc{Make-Node}(x)$}
\li $n \gets \proc{Alloc-Node}()$
\li $\attrib{n}{val} \gets x$
\li $\attrib{n}{color} \gets \const{Red}$
\li $\attrib{n}{size} \gets 1$
\end{codebox}
\pause\item subida até a raiz, com rotações
\begin{codebox}
\Procname{$\proc{Rotate-Simple-Right}(x, y)$}
\zi \assert{\Comment $\attrib{y}{left} \isequal x$ and $\attrib{x}{up \isequal y}$}
\li $\attrib{y}{left} \gets \attrib{x}{right}$
\li $\attrib{x}{left} \gets y$
\li $\attrib{x}{size} \gets \attrib{y}{size}$
\li $\attrib{y}{size} \gets \proc{Size}(\attrib{y}{left}) + \proc{Size}(\attrib{y}{right}) + 1$
\end{codebox}
\pause\alert{Complexidade?}
\end{itemize}
\end{frame}

\begin{frame}
\frametitle{Como atualizar o atributo $\id{size}$?}
\framesubtitle{Remoção}

\begin{itemize}
\item descida até o ponto de inserção: decremento de $\id{size}$  \only<2>{\alert{$O(\log n)$}}
\item possívelmente remoção auxiliar
\item eliminação de um vértice \only<2>{\alert{$O(1)$}}
\item número finito de rotações \only<2>{\alert{$O(1)$}}
\end{itemize}

\end{frame}

\section{Receita}
\begin{frame}
\frametitle{Receita para extender uma estrutura de dados}

\begin{enumerate}
\item escolha da estrutura mais adequada
\item identificação dos novos atributos
\item adaptar operações existentes
\item novas operações
\end{enumerate}

\end{frame}

\begin{frame}
\frametitle{Extensão de árvores rubro-negra}

\begin{theorem}
Seja $a$ um novo atributo para uma árvore rubro-negra com $n$ vértices.

Se, para qualquer nó $x$, $\attrib{x}{a}$ pode ser calculado a partir
de $x$, $\attrib{x}{left}$ e $\attrib{x}{right}$, então podemos 
atualizar os valores de $a$ para todos os nós da árvore sem alterar
a complexidade assintótica das operações básicas.
\end{theorem}

\pause

\alert{Justificativa}: 
\begin{itemize}
\item Mudar $\attrib{x}{a}$ só implica em alterar
$\attrib{\attrib{x}{up}}{a}$, etc. até $\attrib{\attrib{T}{root}}{a}$, ou seja $O(\log n)$ alterações.

\item Rotações só implicam em alterar $\id{a}$ em um número limitado de vértices.

\item Cada alteração de $\id{a}$ é $O(1)$.
\end{itemize}

\end{frame}

\section{Coleções de intervalos}

\begin{frame}
\frametitle{Exemplo: coleção de intervalos}

$i$: intervalo 
\begin{itemize}
\item $\attrib{i}{low}$: limite inferior
\item $\attrib{i}{upp}$: limite superior
\end{itemize}

comparação de intervalos
\begin{enumerate}
\item $\attrib{i}{low} \le \attrib{i'}{high}$ e $\attrib{i'}{low} \le \attrib{i}{high}$: $i$ e $i'$ tem sobreposição
\item $\attrib{i}{high} < \attrib{i'}{low}$: $i$ antes de $i'$
\item $\attrib{i'}{low} < \attrib{i}{high}$: $i$ depois de $i'$
\end{enumerate}

\end{frame}

\begin{frame}
\frametitle{Exemplo: operações}

Operações:
\begin{itemize}
\item $\proc{Interval-Insert}(T, x)$: $x$ com $\attrib{x}{int}$ um intervalo;
\item $\proc{Interval-Delete}(T, x)$: remove $x$;
\item $\proc{Interval-Search}(T, i)$: retorna um elemento $x$ de $T$ tal que
  $i$ e $\attrib{x}{int}$ se sobrepõem.
\end{itemize}

\pause
\alert{Árvores rubro-negras}

\end{frame}

\begin{frame}
\frametitle{Árvores rubro-negras extendidas para intervalos}

\begin{itemize}
\item chave: $\attrib{\attrib{x}{int}}{low}$
\item percurso em ordem: intervalos em ordem crescente de limite inferior
\item atributo extra: $\attrib{x}{max}$
\begin{itemize}
\item valor máximo de qualquer intervalo na sub-árvore enraizada em $x$
\end{itemize}
\end{itemize}

\begin{center}
\includegraphics{fig/interval-tree}
\end{center}

\end{frame}

\begin{frame}
\frametitle{Atualização de novos atributos}

\begin{itemize}
\item $\attrib{x}{max} = \max \{ \attrib{\attrib{x}{left}}{max}, \attrib{\attrib{x}{right}}{max}, \attrib{\attrib{x}{int}}{upp} \}$
\item complexidade: $O(1)$
\item preserva complexidade assintótica das operações básicas (Teorema)
\end{itemize}

\end{frame}

\begin{frame}
\frametitle{Novas operações}

\begin{codebox}
\Procname{$\proc{Interval-Search}(T, i)$}
\li $x \gets \attrib{T}{root}$
\li \While $x \neq \const{Nil}$ and not $\proc{Overlap}(i, \attrib{x}{int})$ 
\li \Do \If $\attrib{x}{left} \neq \const{Nil}$ and $\attrib{\attrib{x}{left}}{max} \ge \attrib{i}{low}$
\li      \Then $x \gets \attrib{x}{left}$
\li      \Else $x \gets \attrib{x}{right}$
         \End
    \End
\li \Return $x$
\end{codebox}

\end{frame}

\begin{frame}
\frametitle{Busca}

\vspace*{-5mm}
\begin{small}
\begin{codebox}
\Procname{$\proc{Interval-Search}(T, i)$}
\li $x \gets \attrib{T}{root}$
\li \While $x \neq \const{Nil}$ and not $\proc{Overlap}(i, \attrib{x}{int})$ 
\li \Do \If $\attrib{x}{left} \neq \const{Nil}$ and $\attrib{\attrib{x}{left}}{max} \ge i$
\li      \Then $x \gets \attrib{x}{left}$
\li      \Else $x \gets \attrib{x}{right}$
         \End
    \End
\li \Return $x$
\end{codebox}
\end{small}

\pause

\only<2-5>{
\begin{center}
\begin{tabular}{c}
$i = [22, 25]$ \\
\only<2>{\includegraphics[height=.4\textheight]{fig/interval-tree-search-1.pdf}}
\only<3>{\includegraphics[height=.4\textheight]{fig/interval-tree-search-2.pdf}}
\only<4>{\includegraphics[height=.4\textheight]{fig/interval-tree-search-3.pdf}}
\end{tabular}
\end{center}
}
\only<6-8>{
\begin{center}
\begin{tabular}{c}
$i = [11, 14]$ \\
\only<6>{\includegraphics[height=.4\textheight]{fig/interval-tree-search-1.pdf}}
\only<7>{\includegraphics[height=.4\textheight]{fig/interval-tree-search-2.pdf}}
\only<8>{\includegraphics[height=.4\textheight]{fig/interval-tree-search-3.pdf}}
\end{tabular}
\end{center}
}

\end{frame}

\begin{frame}
\frametitle{Busca}

\begin{codebox}
\Procname{$\proc{Interval-Search}(T, i)$}
\li $x \gets \attrib{T}{root}$
\li \While $x \neq \const{Nil}$ and not $\proc{Overlap}(i, \attrib{x}{int})$ 
\li \Do \If $\attrib{x}{left} \neq \const{Nil}$ and $\attrib{\attrib{x}{left}}{max} \ge i$
\li      \Then $x \gets \attrib{x}{left}$
\li      \Else $x \gets \attrib{x}{right}$
         \End
    \End
\li \Return $x$
\end{codebox}

\alert{Complexidade}: $O(\log n)$
\pause
\alert{Corretude?}
\end{frame}

\begin{frame}
\frametitle{Exercícios}

\begin{enumerate}
\item Adaptar a busca para funcionar com intervalos abertos
\item Escrever uma operação que
\begin{itemize}
\item tem como parâmetro um intervalo $i$ e,
\item retorna um intervalo que se sobrepões a $i$, com o menor limite
  inferior, ou $\const{Nil}$ se não existir.
\end{itemize}
\item Adapte a estrutura de dados união/fusão para ter a operação que
  retorna o número de elementos de um conjunto.
\end{enumerate}
\end{frame}

\end{document}
