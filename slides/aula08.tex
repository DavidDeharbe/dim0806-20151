\documentclass{beamer}
\setbeamertemplate{footline}[frame number]
%\documentclass{beamer}

\usepackage[utf8x]{inputenc}
\usepackage[brazil,british]{babel}
\usetheme{default} 
\usecolortheme{beaver}
\newtranslation[to=brazil]{Theorem}{Teorema}
\newtranslation[to=brazil]{Definition}{Definição}
\newtranslation[to=brazil]{Example}{Exemplo}
\newtranslation[to=brazil]{Problem}{Exercício}
\newtranslation[to=brazil]{Solution}{Resolução}
\newtranslation[to=brazil]{Lemma}{Lema}
\newtranslation[to=brazil]{Corollary}{Corolário}
\logo{\includegraphics[width=1cm]{../img/logo-ppgsc-icon-text.png}}

\usepackage{graphicx}
\usepackage{clrscode3e}
\usepackage{hyperref}

\usepackage{pgf}
\usepackage{tikz}
\newcommand{\assert}[1]{\textcolor{blue}{#1}}

\usepackage{xspace}

\newcommand{\semester}[0]{2015.2}



\title{Aula 08: Algoritmos de ordenação em arranjos}
\subtitle{Ordenação por inserção}
\author{David Déharbe \\
  Programa de Pós-graduação em Sistemas e Computação \\
  Universidade Federal do Rio Grande do Norte \\
  Centro de Ciências Exatas e da Terra \\
  Departamento de Informática e Matemática Aplicada}
\date{}


\begin{document}
\selectlanguage{brazil}
\begin{frame}
  \titlepage
  Download me from \url{http://DavidDeharbe.github.io}.
\end{frame}

\section{Introdução}

\begin{frame}

  \frametitle{Algoritmos}
  \framesubtitle{Ordenação de arranjo}

  \begin{description}
    \item[Entrada]
      \begin{itemize}
        \item uma sequência linear $A = \langle a_1 \cdots a_n \rangle$,
      \end{itemize}
    \item[Saída] 
      \begin{itemize}
        \item $A$ é uma permutação de $a_1 \cdots a_n$ tal que 
          $A[i] \le A[i+1]$ para $1 \le i < n$.
      \end{itemize}
    \item[Algoritmos]
      \begin{itemize}
      \item ordenação por inserção (\textit{insertion sort})
      \item ordenação por fusão (\textit{merge sort})
      \item ordenação por \textit{heap} (\textit{heap sort})
      \item ordenação \textit{quick-sort}
      \item ordenação por contadores (\textit{counting sort})
      \end{itemize}
  \end{description}

\end{frame}

\begin{frame}
  \frametitle{Plano da aula}
  \tableofcontents
\end{frame}

\section{Ordenação por inserção}

\subsection{Algoritm}

\begin{frame}

  \frametitle{Ordenação por inserção \textit{insertion sort}}
  \framesubtitle{O algoritmo}

(Cormen et al. 1990)

\begin{codebox}
\Procname{$\proc{Insertion-Sort}(A)$}
\li \For $j \gets 2$ \To $\id{length}[A]$
\li     \Do
$\id{key} \gets A[j]$
\zi
\Comment Insert $A[j]$ into the sorted sequence
    $A[1 \twodots j-1]$.
\li $i \gets j-1$
\li \While $i > 0$ and $A[i] > \id{key}$
\li    \Do
        $A[i+1] \gets A[i]$
\li     $i \gets i-1$
    \End
\li $A[i+1] \gets \id{key}$
\End
\end{codebox}  

\end{frame}

\subsection{Simulação}

\begin{frame}

  \frametitle{Ordenação por inserção}
  \framesubtitle{Simulação}
\begin{small}
\begin{codebox}
\Procname{$\proc{Insertion-Sort}(A)$}
\li \For $j \gets 2$ \To $\id{length}[A]$
\li     \Do
$\id{key} \gets A[j]$
\li $i \gets j-1$
\li \While $i > 0$ and $A[i] > \id{key}$
\li    \Do
        $A[i+1] \gets A[i]$
\li     $i \gets i-1$
    \End
\li $A[i+1] \gets \id{key}$
\End
\end{codebox}  
\end{small}
\only<1->{$
\begin{array}{l|l}
j & A \\
\hline
  & \langle 5, 2, 4, 6, 1, 3 \rangle \pause \\
2 & \langle \alert{\times} 5, \alert{2}, 4, 6, 1, 3 \rangle \pause \\
3 & \langle 2, \alert{\times} 5, \alert{4}, 6, 1, 3 \rangle \pause \\
4 & \langle 2, 4, 5, \alert{\times} \alert{6}, 1, 3 \rangle \pause \\
5 & \langle \alert{\times} 2, 4, 5, 6, \alert{1}, 3 \rangle \pause \\
6 & \langle 1, 2, \alert{\times} 4, 5, 6, \alert{3} \rangle \pause \\
7 & \langle 1, 2, 3, 4, 5, 6 \rangle \\
\end{array}
$}
\end{frame}

\subsection{Complexidade}

\begin{frame}

  \frametitle{Ordenação por inserção}
  \framesubtitle{Complexidade}

  \begin{itemize}
    \item Pior caso
      \only<2->{
        \begin{itemize}
          \item Quando $A$ está inicialmente em ordem inversa
          \item $\Theta(n^2)$
        \end{itemize}
      }
    \item Melhor caso
      \only<3>{
        \begin{itemize}
          \item Quando $A$ está inicialmente ordenado
          \item $\Theta(n)$
        \end{itemize}
      }
  \end{itemize}
\end{frame}

\subsection{Correção}

\begin{frame}

  \frametitle{Ordenação por inserção}
  \framesubtitle{Correção}

\begin{description}
\item[contrato] o que é ser correto?
  \begin{description}
    \item[pré-condição] descrição dos valores legais para os parâmetros
    \item[pós-condição] descrição do resultado e efeitos colaterais da
      sub-rotina
  \end{description}
\item[correção (parcial)] invariantes de laço: propriedades sobre as
  variáveis mantidas durante a execução dos laços.
\item[término] variantes de laço: número natural que decresce a cada
  execução do laço
\end{description}

\end{frame}


\begin{frame}

  \frametitle{Ordenação por inserção}
  \framesubtitle{Correção}

\begin{codebox}
\Procname{\only<1>{$\proc{Insertion-Sort}(A)$}}
\only<2->{\zi \assert{$\{A = \langle a_1, \ldots, a_n \rangle\}$}}
\li \For $j \gets 2$ \To $\id{length}[A]$
\only<4->{
\zi \assert{
  $\{\begin{array}[t]{l}
    2 \le j \le n+1 \land \\
    A [ 1 \twodots j-1 ] \in \id{permutation}(\langle a_1, \ldots, a_{j-1} \rangle) \land \\
    \forall k \mid 1 \le k < j-1 \cdot A[k] \le A[k+1] \land \\
    \forall k \mid j \le k \le n \cdot A[k] = a_k \}
    \end{array}$}}
\li     \Do $\id{key} \gets A[j]$
\li   $i \gets j-1$
\li   \While $i > 0$ and $A[i] > \id{key}$
\li   \Do $A[i+1] \gets A[i]$
\li     $i \gets i-1$
      \End
\li   $A[i+1] \gets \id{key}$
    \End
\only<3->{
\zi \assert{
  $\{A \in \id{permutation}(\langle a_1, \ldots, a_n \rangle) \land \forall k \mid 1 \le k < n \cdot A[k] \le A[k+1] \}$}}
\end{codebox}  
\only<5->{Considere $j = n+1$ (término da execução do laço).}
\end{frame}

\begin{frame}
  \frametitle{Ordenação por inserção}
  \framesubtitle{Subsídios para a correção}

\begin{codebox}
\zi \assert{
  $\{\begin{array}[t]{l}
    2 \le j \le n+1 \land \\
    A [ 1 \twodots j-1 ] \in \id{permutation}(\langle a_1, \ldots, a_{j-1} \rangle) \land \\
    \forall k \mid 1 \le k < j-1 \cdot A[k] \le A[k+1] \land \\
    \forall k \mid j \le k \le n \cdot A[k] = a_k \}
    \end{array}$}
\li     \Do $\id{key} \gets A[j]$
\li   $i \gets j-1$
\li   \While $i > 0$ and $A[i] > \id{key}$
\only<2->{
\zi \assert{
  $\{\begin{array}[t]{l}
    2 \le j \le n+1 \land 0 \le i < j \land \id{key} = a_j \land \\
    \only<3->{A[1..j] - \{A[i+1]\} \cup \{\id{key}\} \in \id{permutation}(\langle a_1, \ldots, a_j \rangle) \land \\}
    \only<4->{\forall k \mid 1 \le k < j \cdot A[k] \le A[k+1] \land \\}
    \only<5->{\forall k \mid i+1 \le k \le j \cdot \id{key} \le A[k]  \land \\}
    \forall k \mid j+1 \le k \le n \cdot A[k] = a_k\}
    \end{array}$}}
\li   \Do $A[i+1] \gets A[i]$
\li     $i \gets i-1$
      \End
\li   $A[i+1] \gets \id{key}$
    \End
\end{codebox}  

\end{frame}

\subsection{Exercício}

\begin{frame}

  \frametitle{Exercício}

  A ordenação por inserção pode ser realizada de forma
  recursiva: dado um arranjo $A[1 \twodots n]$, \alert{1)} ordenar $A[1
    \twodots n-1]$ e \alert{2)} inserir $A[n]$ no arranjo resultante.
  \begin{enumerate}
  \item Escreva o algoritmo recursivo;
  \item Determine a complexidade deste algoritmo;
  \item Escreve a pré-condição e a pós-condição deste algoritmo.
  \end{enumerate}
  
\end{frame}

\end{document}
