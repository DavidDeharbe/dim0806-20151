\documentclass{beamer}
\setbeamertemplate{footline}[frame number]
%\documentclass{beamer}

\usepackage[utf8x]{inputenc}
\usepackage[brazil,british]{babel}
\usetheme{default} 
\usecolortheme{beaver}
\newtranslation[to=brazil]{Theorem}{Teorema}
\newtranslation[to=brazil]{Definition}{Definição}
\newtranslation[to=brazil]{Example}{Exemplo}
\newtranslation[to=brazil]{Problem}{Exercício}
\newtranslation[to=brazil]{Solution}{Resolução}
\newtranslation[to=brazil]{Lemma}{Lema}
\newtranslation[to=brazil]{Corollary}{Corolário}
\logo{\includegraphics[width=1cm]{../img/logo-ppgsc-icon-text.png}}

\usepackage{graphicx}
\usepackage{clrscode3e}
\usepackage{hyperref}

\usepackage{pgf}
\usepackage{tikz}
\newcommand{\assert}[1]{\textcolor{blue}{#1}}

\usepackage{xspace}

\newcommand{\semester}[0]{2015.2}


\title{Aula 15: Árvores binárias de busca}
\subtitle{Tabelas de dispersão, tabelas \textit{hash\/}}
\author{David Déharbe \\
  Programa de Pós-graduação em Sistemas e Computação \\
  Universidade Federal do Rio Grande do Norte \\
  Centro de Ciências Exatas e da Terra \\
  Departamento de Informática e Matemática Aplicada}
\date{15 de abril de 2015}


\begin{document}
\selectlanguage{brazil}
\begin{frame}
  \titlepage
  Download me from \url{http://ufrn.academia.edu/DavidDeharbe}.
\end{frame}

\begin{frame}
  \frametitle{Plano da aula}
  \tableofcontents

  \alert{Referência}: Cormen et al. Capítulo 13.
\end{frame}

\section{Árvores binárias}

\begin{frame}

  \frametitle{Árvores binárias}
  \framesubtitle{Introdução}

  \begin{center}
    \includegraphics[width=.5\textwidth]{fig/ab.pdf}
  \end{center}
  \begin{itemize}
    
  \item uma \alert{árvore binária} é formada por células (\alert{nós})
    com atributos
          
    \begin{itemize}

    \item $\attrib{x}{key}$ (chave do dado armazenado),

    \item $\attrib{x}{left}$ (sub-árvore esquerda),
      
    \item $\attrib{x}{right}$  (sub-árvore direita),
      
    \item $\attrib{x}{up}$ (nó pai),

    \end{itemize}

  \item a \alert{raiz} é um nó destacado da árvore.

  \item $\const{Nil}$ representa á árvore binária vazia (nenhuma célula)

  \end{itemize}

\end{frame}

\begin{frame}

  \frametitle{Árvores binárias}
  \framesubtitle{Definições}

  \begin{definition}
    Nós descendentes a partir de $x$:

    $\begin{array}{rcl}
      \id{nodes}(x) & = & \{ x \} \cup
    \id{nodes}(\attrib{x}{left}) \cup \id{nodes}(\attrib{x}{right}) \\
      \id{nodes}(\const{Nil}) & = & \emptyset
    \end{array}$
  \end{definition}

  \begin{definition}
    Chaves armazenados em uma árvore enraizada em $x$:
    $
    \begin{array}[t]{rcl}
      \id{values}(x) & = & \{ \{ \attrib{x}{key} \, | \, x \in \id{nodes}(\id{root}) \} \} \\
      \id{values}(\const{Nil}) & = & \emptyset
    \end{array}
    $
  \end{definition}

  \begin{definition}
    Altura da sub-árvore enraizada em $x$: 

    $\alpha(x) = 
    \begin{array}[t]{l}
      0 \mbox{ se } x = \const{Nil}, \\
      1 + \max \{ \alpha(\attrib{x}{left}), \alpha(\attrib{y}{right}) \} \mbox{ senão}
    \end{array}$
  \end{definition}

\end{frame}

\begin{frame}

  \frametitle{Árvores binárias}
  \framesubtitle{Propriedades}

  \begin{itemize}

  \item célula raiz: $\attrib{\id{root}}{up} = \const{Nil}$

  \item ausência de ciclos

    $x \not\in \id{nodes}(\attrib{x}{left})$
    
    $x \not\in \id{nodes}(\attrib{x}{right})$
    
    $\attrib{x}{left} \cap \attrib{x}{right} = \emptyset$

  \item $\attrib{x}{left} \neq \const{Nil} \Rightarrow 
    x = \attrib{\attrib{x}{left}}{up}$

    $\attrib{x}{right} \neq \const{Nil} \Rightarrow 
    x = \attrib{\attrib{x}{right}}{up}$
        

  \item O número de atributos $\id{left}$ e $\id{right}$
    iguais a $\const{Nil}$ é o número de nós mais um.
  \end{itemize}

\end{frame}

\begin{frame}

\frametitle{Aplicação de árvore binária}

\begin{itemize}

  \item Qualquer árvore pode ser representada através de uma árvore binária

    \begin{itemize}

      \item $\attrib{n}{left}$: primeiro nó descendente

      \item $\attrib{n}{right}$: próximo nó descendente

      \item $\attrib{n}{up}$: nó ancestral

    \end{itemize}

  \item Árvores binárias de busca

\end{itemize}

\end{frame}

%% TODO : Using binary trees to represent any tree

\section{Árvores binárias de busca}

\begin{frame}

\frametitle{Árvores binárias de busca}
\framesubtitle{Introdução}

\begin{itemize}

\item Representa coleção de dados $\id{values}(\id{root})$

\item Um iterador sobre a coleção é uma referência a um nó da árvore.

\item Operações

      \begin{itemize}

        \item Inserção de um dado;

        \item Remoção de um dado;

        \item Busca na coleção de um dado com uma determinada chave

        \item Maior elemento

        \item Menor elemento

        \item Elemento seguinte

        \item Elemento anterior

\end{itemize}

    \item $h = \alpha(\id{root})$: altura da árvore:

      Custo no pior caso: $\Theta(h)$ em média.

\end{itemize}

\end{frame}

\begin{frame}

\frametitle{Árvores binárias de busca}
\framesubtitle{Especificação}

\begin{itemize}
\item representa a coleção $\id{values}(\id{root})$
\item árvore binária
\item com a seguinte propriedade de ordenação:
$$
\forall x \cdot 
\forall y \cdot
\begin{array}[t]{rl}
y \in \id{nodes}(\attrib{x}{left}) \Rightarrow \attrib{x}{key} \ge \attrib{y}{key} & \land \\
y \in \id{nodes}(\attrib{x}{right}) \Rightarrow \attrib{x}{key} \le \attrib{y}{key} &
\end{array}
$$

\begin{eqnarray*}
\id{bst}(x) & \equiv &
x = \const{Nil} \lor
\begin{array}[t]{ll}
  (\attrib{x}{key} \ge \max \id{values}(\attrib{x}{left}) & \land \\
  \attrib{x}{key} \le \min \id{values}(\attrib{x}{right}) & \land \\
  \id{bst}{(\attrib{x}{left}}) \land \id{bst}({\attrib{x}{right}}) )
\end{array}
\end{eqnarray*}
\end{itemize}

\end{frame}

\begin{frame}

\frametitle{Árvores binárias de busca}
\framesubtitle{Ilustração}

  Coleção: $2, 3, 5, 5, 7, 8$

  \begin{itemize}
\only<1>{
    \item Uma árvore binária de busca de altura 3
      \begin{center}
        \includegraphics[width=.7\textwidth]{fig/abb-1.pdf}
      \end{center}
}
\only<2>{
    \item Uma árvore binária de busca de altura 4
      \begin{center}
        \includegraphics[width=.6\textwidth]{fig/abb-2.pdf}
      \end{center}
}
  \end{itemize}

\end{frame}

\begin{frame}

\frametitle{Árvores binárias de busca}
\framesubtitle{Processamento}

\begin{itemize}
\item Processamento \alert{em ordem} permite visitar os valores da
  coleção em ordem crescente de chaves.

\begin{codebox}
  \Procname{$\proc{In-Order}(x, f)$}
  \li \If $x \neq \const{Nil}$
  \li \Then $\proc{In-Order}(\attrib{x}{left})$
  \li $f(x)$
  \li $\proc{In-Order}(\attrib{x}{right})$
      \End
\end{codebox}

\item Complexidade: $\Theta(n)$
\end{itemize}

\end{frame}

\begin{frame}

\frametitle{Exercícios}

\begin{enumerate}

  \item Qual a altura máxima possível para uma árvore binária de busca
    representando uma coleção de $n$ valores?

  \item Seja $\proc{Make-Node}(k, l, r, u)$ uma sub-rotina que
    constroi um nó $x$ de árvore binária, tal que $\attrib{x}{key} =
    k$, $\attrib{x}{left} = l$, $\attrib{x}{right} = r$,
    $\attrib{x}{up} = u$.

    Projete um algoritmo utilize esta sub-rotina e ordenação em
    arranjo para construir uma árvore binária de busca a partir
    de uma coleção inicialmente em um arranjo. O algoritmo deverá
    ter complexidade $\Theta(n \lg n)$.

  \item Projete um algoritmo não recursivo para processar os valores
    de uma árvore binária de busca em ordem, e em tempo $\Theta(n)$.
\end{enumerate}
\end{frame}

\begin{frame}

\frametitle{Busca}
\frametitle{Operações de consulta}

\begin{codebox}
  \Procname{$\proc{Search}(x, k)$}
  \zi \assert{\Comment $\id{bst}(x)$}
  \li \If $x \isequal \const{Nil}$ or $\attrib{x}{key} \isequal k$
  \li \Then \Return $x$
  \li \Else 
  \li   \If $k < \attrib{x}{key}$
  \li   \Then \Return $\proc{Search}(\attrib{x}{left}, k)$
  \li   \Else \Return $\proc{Search}(\attrib{x}{right}, k)$
        \End
      \End
  \zi \assert{\Comment $\proc{Search}(x, k) = \const{Nil} \Leftrightarrow k \not\in \id{values}(x)$}
  \zi \assert{\Comment $\proc{Search}(x, k) = y \neq \const{Nil} \Leftrightarrow \attrib{y}{key} = k \land y \in \id{nodes}(x)$}
\end{codebox}

Complexidade: $O(\alpha(\id{root}))$
\pause

Exercício: escrever um algoritmo não recursivo.

\end{frame}

\section{Operações de consulta}

\begin{frame}

\frametitle{Extremos}
\frametitle{Operações de consulta}

\begin{codebox}
  \Procname{$\proc{Minimum}(x)$}
  \zi \assert{\Comment $\id{bst}(x) \land x \neq \const{Nil}$}
  \li \If $\attrib{x}{left} \isequal \const{Nil}$
  \li \Then \Return $x$
  \li \Else \Return $\proc{Minimum}(\attrib{x}{left}, x)$
      \End
  \zi \assert{\Comment $\proc{Minimum}(x) = \min \id{values}(x)$}
\end{codebox}

\begin{codebox}
  \Procname{$\proc{Maximum}(x)$}
  \zi \assert{\Comment $\id{bst}(x) \land x \neq \const{Nil}$}
  \li \If $\attrib{x}{right} \isequal \const{Nil}$
  \li \Then \Return $x$
  \li \Else \Return $\proc{Maximum}(\attrib{x}{right}, x)$
      \End
  \zi \assert{\Comment $\proc{Maximum}(x) = \max \id{values}(x)$}
\end{codebox}

Complexidade: $O(\alpha(\id{root}))$
\pause

Exercício: escrever algoritmos não recursivas.

\end{frame}

\begin{frame}

\frametitle{Sucessor}
\frametitle{Operações de consulta}

\begin{codebox}
  \Procname{$\proc{Successor}(x)$}
  \zi \assert{\Comment $x \neq \const{Nil} \land \id{bst}(x)$}
  \li \If $\attrib{x}{right} \neq \const{Nil}$
  \li \Then \Return $\proc{Minimum}(\attrib{x}{right})$
      \End
  \li $y \gets \attrib{x}{up}$
  \li \While $y \neq \const{Nil}$ and $x \isequal \attrib{y}{right}$
  \li \Do $x \gets y$
  \li   $y \gets \attrib{x}{up}$
      \End
  \li \Return $y$
\end{codebox}

Complexidade: $O(\alpha(\id{root}))$
\pause

Exercício: escrever o algoritmo que encontra o nó predecessor (se existir).

\end{frame}

%% TODO : EXERCICIOS SOBRE CONSULTA

\section{Operações de mutação}

\begin{frame}

\frametitle{Inserção}
\framesubtitle{Operações de mutação}

\begin{codebox}
  \Procname{$\proc{Insert}(A, k)$}
  \zi \assert{\Comment $\id{bst}(\attrib{A}{root})$}
  \li $\attrib{A}{root} \gets \proc{Insert-Aux}(k, \attrib{A}{root}, \const{Nil})$
  \zi \assert{\Comment $\id{bst}(\attrib{A'}{root}) \land \id{values}(\attrib{A'}{root}) = \id{values}(\attrib{A}{root}) \cup \{ \{ k \} \}$}
\end{codebox}

\begin{codebox}
  \Procname{$\proc{Insert-Aux}(k, x, p)$}
  \zi \assert{\Comment $\id{bst}(x) \land $}
  \zi \assert{\Comment $(x = Nil \lor \attrib{x}{up} = \const{Nil} \lor$}
  \zi \assert{\Comment $\attrib{x}{up} = p \land (\attrib{p}{key} \le k \land \attrib{p}{right} = x \lor
      \attrib{p}{key} \ge k \land \attrib{p}{left} = x )\,)$}
  \li \If $x \isequal \const{Nil}$
  \li \Then \Return $\proc{Make-Node}(k, \const{Nil}, \const{Nil}, \id{p})$
  \li \Else
  \li    \If $k < \attrib{x}{key}$
  \li    \Then $\attrib{x}{left} \gets \proc{Insert-Aux}(p, \attrib{x}{left}, x)$
  \li    \Else $\attrib{x}{right} \gets \proc{Insert-Aux}(p, \attrib{x}{right}, x)$
         \End
       \End
\end{codebox}

\end{frame}

\begin{frame}

\frametitle{Inserção}
\framesubtitle{Operações de mutação}

\begin{itemize}

  \item $O(h)$ para encontrar a posição de inserção

  \item $\Theta(1)$ para criar o nó

  \item $O(h)$ para atualizar as referências para as sub-árvores

\end{itemize}


\end{frame}

%% TODO: ilustrate remove scenarios

\begin{frame}

\frametitle{Remoção}
\framesubtitle{Operações de mutação}

\begin{itemize}

  \item Objetivo: remover a chave $k$ da coleção representada

  \item Preâmbulo: buscar o nó, digamos $z$, com a chave $k$

  \item Remover o nó $z$. 

\pause

\only<2>{
  \assert{$z$ é uma folha:}

  \begin{itemize}
        
  \item eliminar $z$, e 
            
  \item modificar o nó $\attrib{z}{up}$ (se existir) para ter
    $\const{Nil}$ como sub-árvore, ao invés de $z$;

  \end{itemize}
}

\only<3>{
  \assert{$z$ tem apenas uma sub-árvore:}

  \begin{itemize}

  \item eliminar $z$
        
  \item modificar o nó $\attrib{z}{up}$ (se existir) para
    ter a sub-árvore de $z$ como sub-árvore, ao invés de $z$.

  \end{itemize}
}

\only<4>{
  \assert{$z$ tem duas sub-árvores:}

  \begin{itemize}

  \item procurar o nó $y$, elemento sucessor de $z$ na árvore

    necessariamente existe $y$ (\alert{por quê}?)

  \item copiar $\attrib{y}{key}$ em $\attrib{z}{key}$

  \item eliminar o nó $y$.

    necessariamente $y$ tem pelo menos uma sub-árvore vazia.
    (\alert{por quê}?)
  \end{itemize}
}

\end{itemize}

\end{frame}

\begin{frame}

\frametitle{Remoção}
\framesubtitle{Operações de mutação}

\begin{footnotesize}
\begin{codebox}
  \Procname{$\proc{Remove-Node}(A, k)$}
  \li $z \gets \proc{Search}(\attrib{A}{root}, k)$ \Comment $z$: nó com o valor a eliminar
  \li \If $z \isequal \const{Nil}$
  \li \Then \Return
      \End
  \li \If $\attrib{z}{left} \isequal \const{Nil}$
  or $\attrib{z}{right} \isequal \const{Nil}$
  \li \Then $y \gets z$
  \li \Else $y \gets \proc{Successor}(z)$ \Comment $y$: nó a eliminar
      \End
  \li \If $\attrib{y}{left} \neq \const{Nil}$
  \li \Then $x \gets \attrib{y}{left}$
  \li \Else $x \gets \attrib{y}{right}$   \Comment $x$: sub-árvore não vazia de $y$ ou $\const{Nil}$
      \End
  \li \If $x \neq \const{Nil}$
  \li \Then $\attrib{x}{up} \gets \attrib{y}{up}$
       \End
  \li \If $\attrib{y}{up} \isequal \const{Nil}$
  \li \Then $\attrib{A}{root} \gets x$
  \li \Else \If $y \isequal \attrib{\attrib{y}{up}}{left}$
  \li   \Then $\attrib{\attrib{y}{up}}{left} \gets x$
  \li   \Else $\attrib{\attrib{y}{up}}{right} \gets x$ \Comment $y$ foi desconectado da árvore
        \End
      \End
  \li \If $y \neq z$
  \li \Then $\attrib{z}{key} \gets \attrib{y}{key}$
      \End
  \li $\proc{Free-Node}(y)$
\end{codebox}
\end{footnotesize}
\end{frame}

\begin{frame}

\frametitle{Inserção}
\framesubtitle{Operações de mutação}

\begin{itemize}

  \item $O(h)$ para encontrar o nó a remover

  \item $O(h)$ para encontrar o nó sucessor

  \item $Theta(1)$ para realizar as modificações nas referências

\end{itemize}


\end{frame}

\begin{frame}

\frametitle{Exercícios}

\begin{enumerate}

  \item O algoritmo de inserção apresentado realiza $O(h)$
    atualizações de sub-árvore.

    Modificar o algoritmo para realizar $\Theta(1)$ atualizações
    de sub-árvore

  \item Projetar um algoritmo de inserção não recursivo.

  \item Verdadeiro ou falso?

    Remover a chave $k_1$, e então remover a chave $k_2$ deixa a árvore no
    mesmo estado de que remover primeiro $k_2$ e então $k_1$.

    Justifique.

  \item Um algoritmo de ordenação de um arranjo consiste em 
    \begin{enumerate}
      \item inserir sucessivamente os valores do arranjo em uma árvore binária de busca
      \item percorrer a árvore em ordem, inserindo os valores no arranjo.

    \end{enumerate}

    Qual a complexidade deste algoritmo?

\end{enumerate}

\end{frame}
%% TODO : EXERCICIOS SOBRE MUTACAO

\begin{frame}

  \frametitle{Conclusões}

  \begin{itemize}

    \item Operações: $O(\alpha(\attrib{A}{root}))$

    \item Em geral:

      \begin{itemize}
      \item $\alpha(\attrib{A}{root}) \in O( | \id{values}(\attrib{A}{root}) |)$

      \item $\alpha(\attrib{A}{root}) \in \Omega( \lg | \id{values}(\attrib{A}{root}) |)$
      \end{itemize}

    \item Podemos fazer melhor?

      \pause

      \begin{itemize}

        \item \alert{Sim!}

          $\Longrightarrow \alpha(\attrib{A}{root}) \in \Theta( \lg | \id{values}(\attrib{A}{root}) |)$ 

          \item Árvores AVL, árvores rubro-negras.

      \end{itemize}
  \end{itemize}

\end{frame}

\end{document}


