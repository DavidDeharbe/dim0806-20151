\documentclass{beamer}
\setbeamertemplate{footline}[frame number]
%\documentclass{beamer}

\usepackage[utf8x]{inputenc}
\usepackage[brazil,british]{babel}
\usetheme{default} 
\usecolortheme{beaver}
\newtranslation[to=brazil]{Theorem}{Teorema}
\newtranslation[to=brazil]{Definition}{Definição}
\newtranslation[to=brazil]{Example}{Exemplo}
\newtranslation[to=brazil]{Problem}{Exercício}
\newtranslation[to=brazil]{Solution}{Resolução}
\newtranslation[to=brazil]{Lemma}{Lema}
\newtranslation[to=brazil]{Corollary}{Corolário}
\logo{\includegraphics[width=1cm]{../img/logo-ppgsc-icon-text.png}}

\usepackage{graphicx}
\usepackage{clrscode3e}
\usepackage{hyperref}

\usepackage{pgf}
\usepackage{tikz}
\newcommand{\assert}[1]{\textcolor{blue}{#1}}

\usepackage{xspace}

\newcommand{\semester}[0]{2015.2}


\title{Aula 20: Estruturas para união e busca}
\author{David Déharbe \\
  Programa de Pós-graduação em Sistemas e Computação \\
  Universidade Federal do Rio Grande do Norte \\
  Centro de Ciências Exatas e da Terra \\
  Departamento de Informática e Matemática Aplicada}
\date{}

\begin{document}
\selectlanguage{brazil}
\begin{frame}
  \titlepage
  Download me from \url{http://DavidDeharbe.github.io}.
\end{frame}

\begin{frame}
  \frametitle{Plano da aula}

\begin{center}
  \includegraphics[width=\textwidth]{fig/hp2-300-300}
\end{center}

  \tableofcontents

\end{frame}

\section{Introdução}

\begin{frame}

  \frametitle{União e busca}
  \framesubtitle{Introdução}

  
  \begin{itemize}
    
  \item Cenário de aplicação:

    \begin{itemize}

      \item registros devem ser mantidos em grupos

      \item teste se dois elementos pertencem ao mesmo grupo (\alert{busca})

      \item juntar dois grupos (\alert{fusão})

      \item e também: 

        \begin{itemize}

          \item quantos grupos diferentes? 

          \item qual o tamanho de cada grupo?

          \item qual o maior grupo? 

        \end{itemize}

      \item exemplo: componentes conexos em um grafo

    \end{itemize}

\end{itemize}

\end{frame}

\begin{frame}

  \frametitle{Ilustração}

\begin{center}
\includegraphics[height=6cm]{fig/connex.pdf}
\end{center}

\begin{itemize}

\item $G = (V, E)$, 
\item $V = \{1,2,3,4,5,6,7,8,9,10\}$, e 
\item $E = \{(1,3),(2,6),(3,4),(3,5),(6,7),(6,9),(8,10)\}$.
\end{itemize}

\end{frame}

\section{Propriedades}

\begin{frame}
\frametitle{Propriedades}

\begin{itemize}

  \item Os grupos formam uma \alert{partição} do conjunto dos registros

  \item Esta partição induz uma relação $R$ sobre os registros:

    $$a \sim b \text{ sse } \id{grupo}(A) = \id{grupo}(B)$$

  \item $\sim$ é uma relação de equivalência

    \begin{itemize}

      \item reflexiva $\forall e \cdot e \sim e$

      \item simétrica $\forall e, e' \cdot e \sim e' \Rightarrow e' \sim e$

      \item transitiva $\forall e_0, e_1, e_2 \cdot e_0 \sim e_1 \land e_1 \sim e_2 \Rightarrow e_0 \sim e_2$.

    \end{itemize}

\end{itemize}

\end{frame}

\begin{frame}
\frametitle{Especificação das operações}

\begin{itemize}

  \item Criação de um novo grupo $\proc{Make-Set}(x)$

    \begin{itemize}

      \item registro $x$

      \item resultado: o grupo criado

    \end{itemize}

  \item Busca do grupo $\proc{Find-Set}(x)$

    \begin{itemize}

      \item registro $x$

      \item resultado: o grupo de $x$

    \end{itemize}
    
  \item Junção de dois grupos $\proc{Union-Set}(x, y)$

    \begin{itemize}

      \item registros $x$, $y$

      \item resultado: o grupo com $x$, $y$, e todos os elementos
        que estavam com $x$ e com $y$

    \end{itemize}

\end{itemize}

\end{frame}

\begin{frame}

\frametitle{Aplicação}

\begin{codebox}
\Procname{$\proc{Graph-SCC}(G)$}
\li \For $v \in \attrib{G}{V}$
\li \Do $\proc{Make-Set}(v)$
    \End
\li \For $e \in \attrib{G}{E}$
\li \Do $\proc{Union-Set}(\attrib{e}{src}, \attrib{e}{dst})$
    \End
\end{codebox}
\pause
\begin{center}
\begin{tabular}{|l|l|}
\hline
inicial & $\{1\},\{2\},\{3\},\{4\},\{5\},\{6\},\{7\},\{8\},\{9\},\{10\}$ \\
\hline
$(1,3)$ & $\{1,3\},\{2\},\{4\},\{5\},\{6\},\{7\},\{8\},\{9\},\{10\}$ \\
\hline
$(2,6)$ & $\{1,3\},\{2,6\},\{4\},\{5\},\{7\},\{8\},\{9\},\{10\}$ \\
\hline
$(3,4)$ & $\{1,3,4\},\{2,6\},\{5\},\{7\},\{8\},\{9\},\{10\}$ \\
\hline
$(3,5)$ & $\{1,3,4,5\},\{2,6\},\{7\},\{8\},\{9\},\{10\}$ \\
\hline
$(6,7)$ & $\{1,3,4,5\},\{2,6,7\},\{8\},\{9\},\{10\}$ \\
\hline
\end{tabular}
\end{center}

\end{frame}

\section{Implementação}

\begin{frame}
\frametitle{Estruturas de dados}

\begin{description}
\item[listas encadeadas] busca mais eficiente
\item[árvores] união mais eficiente
\end{description}

Princípio comum: 
\begin{itemize}
\item cada grupo elege um registro \alert{representante}
\item o representante identifica o grupo
\end{itemize}


\end{frame}

\subsection{Implementação por listas}

\begin{frame}
\frametitle{Estado}
\framesubtitle{Implementação por lista}

Para cada registro $x$
\begin{itemize}
  \item $\attrib{x}{rep}$ é o grupo de $x$
  \item $\attrib{x}{next}$ é o próximo elemento do grupo
\end{itemize}
\end{frame}

\begin{frame}
\frametitle{Criação}
\framesubtitle{Implementação por lista}

\begin{codebox}
\Procname{$\proc{Make-Set}(x)$}
\li $\attrib{x}{rep} \gets x$
\li $\attrib{x}{next} \gets x$
\end{codebox}

\end{frame}

\begin{frame}
\frametitle{Busca}
\framesubtitle{Implementação por lista}

\begin{codebox}
\Procname{$\proc{Find-Set}(x)$}
\li \Return $\attrib{x}{rep}$
\end{codebox}

\end{frame}

\begin{frame}
\frametitle{União}
\framesubtitle{Implementação por lista}

\begin{codebox}
\Procname{$\proc{Union-Set}(x_1, x_2)$}
\li $r_1 \gets \proc{Find-Set}(x_1)$
\li $r_2 \gets \proc{Find-Set}(x_2)$
\li $\id{tmp} \gets \attrib{r_1}{next}$
\li $\attrib{r_1}{next} \gets r_2$
\li $x \gets r_2$
\li \Repeat
\li   $\attrib{x}{rep} \gets r_1$
\li   $x \gets \attrib{x}{next}$
\li \Until $\attrib{x}{next} \isequal r_2$
\li $\attrib{x}{rep} \gets r_1$
\li $\attrib{x}{next} \gets tmp$
\li \Return $r_1$
\end{codebox}
\pause
Complexidade: $\Theta( | \id{grupo}(x_2) |)$
\end{frame}

\begin{frame}
\frametitle{Ilustração da união}
\framesubtitle{Implementação por lista}

\begin{tabular}[t]{c}
\only<1-3>{\includegraphics[width=.9\textwidth]{fig/union-find-0} \\}
\only<2-3>{$1 \equiv 2$ \\}
\only<3-6>{\includegraphics[width=.6\textwidth]{fig/union-find-1} \\}
\only<5-6>{$3 \equiv 4$ \\}
\only<6-9>{\includegraphics[width=.6\textwidth]{fig/union-find-2} \\}
\only<8-9>{$5 \equiv 2$ \\}
\only<9-12>{\includegraphics[width=.4\textwidth]{fig/union-find-3} \\}
\only<11-12>{$1 \equiv 3$ \\}
\end{tabular}
\only<12>{\includegraphics[width=.4\textwidth]{fig/union-find-4} \\}

\end{frame}

\begin{frame}

\frametitle{União por tamanho}

\begin{itemize}
\item a solução apresentada pode ser otimizada facilmente
\item incluir o menor grupo no maior
\item $\Theta(\min(|\id{grupo}(x_1)|, |\id{grupo}(x_2)|))$
\item realização:
\begin{itemize}
  \item estado: $\attrib{x}{rank}$ tamanho do grupo
    (só precisa ser atualizado quando $\attrib{x}{rep} = x$)

  \item união: 
    \begin{enumerate}
    \item troque $r_1$ e $r_2$ quando $\attrib{r_1}{rank} < \attrib{r_2}{rank}$
    \item $\attrib{r_1}{rank} \gets \attrib{r_1}{rank} + \attrib{r_2}{rank}$
    \end{enumerate}
\end{itemize}
\end{itemize}

\end{frame}

\subsection{Implementação por árvores}

\begin{frame}
\frametitle{Estado}
\framesubtitle{Implementação por árvores}

\begin{itemize}
\item $\attrib{x}{up}$ é um registro do mesmo grupo que $x$
\item $r$ é representante de um grupo quando $\attrib{r}{up} = r$
\item o fecho transitivo da relação entre nós induzida por $\id{up}$ 
 relaciona cada registro com seu representante
\end{itemize}
\pause
\begin{center}
\includegraphics[height=4cm]{fig/union-tree-4}
\end{center}

\end{frame}

\begin{frame}
\frametitle{Criação}
\framesubtitle{Implementação por árvores}

\begin{codebox}
\Procname{$\proc{Make-Set}(x)$}
\li $\attrib{x}{up} \gets x$
\end{codebox}

\end{frame}

\begin{frame}
\frametitle{Busca}
\framesubtitle{Implementação por árvores}

\begin{codebox}
\Procname{$\proc{Find-Set}(x)$}
\li \While $x \neq \attrib{x}{up}$
\li \Do $x \gets \attrib{x}{up}$
    \End
\li \Return $x$
\end{codebox}

\end{frame}

\begin{frame}
\frametitle{União}
\framesubtitle{Implementação por árvores}

\begin{codebox}
\Procname{$\proc{Union-Set}(x_1, x_2)$}
\li $r_1 \gets \proc{Find-Set}(x_1)$
\li $r_2 \gets \proc{Find-Set}(x_2)$
\li $\attrib{r_2}{up} \gets r_1$
\li \Return $r_1$
\end{codebox}
\pause
\alert{Complexidade?}
\end{frame}

\begin{frame}
\frametitle{Ilustração da união}
\framesubtitle{Implementação por árvores}

\only<1-9>{
\begin{tabular}[t]{c}
\only<1-3>{\includegraphics[width=.8\textwidth]{fig/union-tree-0} \\}
\only<2-3>{$1 \equiv 2$ \\}
\only<3-6>{\includegraphics[width=.5\textwidth]{fig/union-tree-1} \\}
\only<5-6>{$3 \equiv 4$ \\}
\only<6-9>{\includegraphics[width=.5\textwidth]{fig/union-tree-2} \\}
\only<8-9>{$5 \equiv 2$ \\}
\end{tabular}
}
\begin{tabular}[t]{c}
\only<9-12>{\includegraphics[width=.3\textwidth]{fig/union-tree-3} \\}
\only<11-12>{$1 \equiv 3$ \\}
\end{tabular}
\only<12>{\includegraphics[width=.3\textwidth]{fig/union-tree-4} \\}

\end{frame}

\begin{frame}
\frametitle{União por tamanho}

\begin{itemize}
\item $\proc{Union-Set}(x_1, x_2)$ aumenta a altura de todos os nós 
  do grupo de $x_2$
\item o custo da operação de busca depende da altura
\item o custo da operação de fusão depende da altura
\item para minimizar o custo ao longo da execução, sempre escolher como
  grupo destino o que era inicialmente maior
\item mesma técnica que para as listas
\end{itemize}
\end{frame}

\begin{frame}
\frametitle{União por tamanho}
\framesubtitle{Estudo experimental}

\begin{itemize}
\item é gerado aleatoriamente um grafo de 300 vertices e 300 arestas
\item são impressas as árvores formadas pelos grupos depois de aplicar
  $\proc{Graph-SCC}$
\begin{itemize}
\item sem união por tamanho
\item com união por tamanho
\end{itemize}
\end{itemize}

\end{frame}

\begin{frame}
\frametitle{União por tamanho}
\framesubtitle{Estudo experimental}

\begin{tabular}[t]{cc}
sem & com \\
\\
\includegraphics[width=.25\textwidth]{fig/qu-300-300}
&
\includegraphics[width=.7\textwidth]{fig/wqu-300-300}
\end{tabular}

\end{frame}

\begin{frame}
\frametitle{Compressão de caminhos}

\begin{itemize}
\item a busca é uma operação essencial
\item o custo depende da altura do valor procurado
\item como diminuir a altura de forma eficiente?
\pause
\item aproveitar a busca para \alert{encurtar} o caminho até o representante
\begin{itemize}
\item no final da busca, atribuir $\id{up}$ com o representante do grupo
\item durante a busca, encurtar o caminho: $\attrib{x}{up} \gets \attrib{\attrib{x}{up}}{up}$
\end{itemize}
\item pode ser combinado com a união por tamanho
\end{itemize}
\end{frame}

\begin{frame}
\frametitle{Algoritmos revisitado}
\framesubtitle{Compressão de caminhos}

\begin{itemize}
\only<1>{
\item compressão completa:
\begin{codebox}
\Procname{$\proc{Find-Set}(x)$}
\li $\id{tmp} \gets x$
\li \While $x \neq \attrib{x}{up}$
\li \Do $x \gets \attrib{x}{up}$
    \End
\li \While $tmp \neq x$
\li \Do $y \gets \attrib{tmp}{up}$
\li   $\attrib{tmp}{up} \gets x$
\li   $\id{tmp} \gets y$
    \End
\li \Return $x$
\end{codebox}
}
\only<2>{
\item compressão por divisão:
\begin{codebox}
\Procname{$\proc{Find-Set}(x)$}
\li \While $x \neq \attrib{x}{up}$
\li \Do $\attrib{x}{up} \gets \attrib{\attrib{x}{up}}{up}$
\li   $x \gets \attrib{x}{up}$
    \End
\li \Return $x$
\end{codebox}
}
\end{itemize}

\end{frame}

\begin{frame}
\frametitle{Ilustração}
\framesubtitle{Compressão de caminhos}

\begin{tabular}{l||l}
\begin{tabular}{c}
\includegraphics[width=.05\textwidth]{fig/compressao-init} \\
Estado inicial
\end{tabular}
&
\begin{tabular}{c}
\includegraphics[width=.6\textwidth]{fig/compressao-completa} \\
Após busca de 1, com compressão completa \\
\hline
\hline
\begin{tabular}{p{.25\textwidth}p{.30\textwidth}}
\includegraphics[width=.2\textwidth]{fig/compressao-divisao} &
\includegraphics[width=.25\textwidth]{fig/compressao-divisao-2} \\
Após busca de 1, com compressão por divisão &
Após duas buscas de 1, com compressão por divisão
\end{tabular}
\end{tabular}
\end{tabular}
\end{frame}

\begin{frame}
\frametitle{Estudo experimental}

\begin{itemize}
\item é gerado aleatoriamente um grafo de 300 vertices e 300 arestas
\item são impressas as árvores formadas pelos grupos depois de aplicar
  $\proc{Graph-SCC}$
\begin{itemize}
\item compressão completa
\item compressão por divisão
\item sem compressão
\end{itemize}
\item são impressas as árvores formadas pelos grupos depois de aplicar
  $\proc{Graph-SCC}$ e 300 buscas aleatórias
\begin{itemize}
\item compressão completa
\item compressão por divisão
\item união por tamanho
\end{itemize}

\end{itemize}

\end{frame}

\begin{frame}
\frametitle{Estudo experimental}
\framesubtitle{Após aplicação de $\proc{Graph-SCC}$}
\begin{center}
\begin{tabular}[t]{c}
Compressão completa: \\
\\
\includegraphics[width=\textwidth]{fig/fp-300-300} \\
\\
Compressão por divisão: \\
\\
\includegraphics[width=\textwidth]{fig/hp-300-300} \\
\\
União por tamanho, sem compressão: \\
\\
\includegraphics[width=\textwidth]{fig/wqu-300-300}
\end{tabular}
\end{center}

\end{frame}

\begin{frame}
\frametitle{Estudo experimental}
\framesubtitle{Após aplicação de $\proc{Graph-SCC}$ e 300 buscas}

\begin{center}
\begin{tabular}[t]{c}
Compressão completa: \\
\\
\includegraphics[width=\textwidth]{fig/fp2-300-300} \\
\\
Compressão por divisão: \\
\\
\includegraphics[width=\textwidth]{fig/hp2-300-300} \\
\\
União por tamanho, sem compressão: \\
\\
\includegraphics[width=\textwidth]{fig/wqu-300-300}
\end{tabular}
\end{center}

\end{frame}

\end{document}


