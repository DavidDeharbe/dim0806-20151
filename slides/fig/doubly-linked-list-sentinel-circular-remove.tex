\documentclass{article}
\usepackage[pdftex,active,tightpage]{preview}
\setlength\PreviewBorder{2mm}
 
\usepackage{tikz}
\usetikzlibrary{calc,shapes.multipart,chains,arrows}

\tikzset{
    squarecross/.style={
        draw, rectangle,minimum size=18pt, fill=orange!80,
        inner sep=0pt, text=black,
        path picture = {
            \draw[black]
            (path picture bounding box.north west) -- 
            (path picture bounding box.south east)
            (path picture bounding box.south west) -- 
            (path picture bounding box.north east);
        }
    }
}

\begin{document}

%% initial

\begin{preview}
\begin{tikzpicture}[
        list/.style={
            very thick, rectangle split, 
            rectangle split parts=3, draw, 
            rectangle split horizontal, minimum size=24pt,
            inner sep=4pt, text=black,
            rectangle split part fill={blue!20, red!20, blue!20}
        }, 
        sentinel/.style={
            very thick, rectangle split, 
            rectangle split parts=3, draw, 
            rectangle split horizontal, minimum size=24pt,
            inner sep=4pt, text=black,
            rectangle split part fill={blue!20, red!80, blue!20}
        }, 
        ->, start chain, very thick
      ]
  \node[sentinel,on chain] (S) {
    \nodepart{two} 
  };
  \node[list,on chain] (A) {
    \nodepart{two} 12
  };
  \node[list,on chain] (B) {
    \nodepart{two} 99
  };
  \node[list,on chain] (C) {
    \nodepart{two} 37
  };
  \node (nil) [below=of S] {\sc Nil};
  \node (it) [below=of A] {\sc it};
  \draw[->] let \p1 = (S.three), \p2 = (A.center), \p3 = (A.one), \p4 = (A.center) in (\x1,\y2+2) to [bend left=45] (\x3,\y4+2);
  \draw[->] let \p1 = (A.three), \p2 = (A.center), \p3 = (B.one), \p4 = (B.center) in (\x1,\y2+2) to [bend left=45] (\x3,\y4+2);
  \draw[->] let \p1 = (B.three), \p2 = (B.center), \p3 = (C.one), \p4 = (C.center) in (\x1,\y2+2) to [bend left=45] (\x3,\y4+2);
  \draw[->] let \p1 = (C.three), \p2 = (C.center) in (\x1,\y2) -- +(0, .7) -| (S);
  \draw[->] let \p1 = (S.one), \p2 = (S.center) in (\x1,\y2) -- +(0, -.7) -| (C);
  \draw[->] let \p1 = (A.one), \p2 = (A.center), \p3 = (S.three), \p4 = (S.center) in (\x1,\y2-2) to [bend left=45] (\x3, \y4-2);
  \draw[->] let \p1 = (B.one), \p2 = (B.center), \p3 = (A.three), \p4 = (A.center) in (\x1,\y2-2) to [bend left=45] (\x3, \y4-2);
  \draw[->] let \p1 = (C.one), \p2 = (C.center), \p3 = (B.three), \p4 = (B.center) in (\x1,\y2-2) to [bend left=45] (\x3, \y4-2);
  \draw[->] (nil) -- (S);
  \draw[->] (it) -- (A);

\end{tikzpicture}
\end{preview}

%% it.prev.next = it.next

\begin{preview}
\begin{tikzpicture}[
        list/.style={
            very thick, rectangle split, 
            rectangle split parts=3, draw, 
            rectangle split horizontal, minimum size=24pt,
            inner sep=4pt, text=black,
            rectangle split part fill={blue!20, red!20, blue!20}
        }, 
        sentinel/.style={
            very thick, rectangle split, 
            rectangle split parts=3, draw, 
            rectangle split horizontal, minimum size=24pt,
            inner sep=4pt, text=black,
            rectangle split part fill={blue!20, red!80, blue!20}
        }, 
        ->, start chain, very thick
      ]
  \node[sentinel,on chain] (S) {
    \nodepart{two} 
  };
  \node[list,on chain] (A) {
    \nodepart{two} 12
  };
  \node[list,on chain] (B) {
    \nodepart{two} 99
  };
  \node[list,on chain] (C) {
    \nodepart{two} 37
  };
  \node (nil) [below=of S] {\sc Nil};
  \node (it) [below=of A] {\sc it};
  \draw[red!20,dashed,->] let \p1 = (S.three), \p2 = (S.center), \p3 = (A.one), \p4 = (A.center) in (\x1,\y2+2) to [bend left=45] (\x3,\y4+2);
  \draw[->] let \p1 = (S.three), \p2 = (S.center), \p3 = (B.one), \p4 = (B.center) in (\x1,\y2+2) to [bend left=45] (\x3,\y4+2);
  \draw[->] let \p1 = (A.three), \p2 = (A.center), \p3 = (B.one), \p4 = (B.center) in (\x1,\y2+2) to [bend left=45] (\x3,\y4+2);
  \draw[->] let \p1 = (B.three), \p2 = (B.center), \p3 = (C.one), \p4 = (C.center) in (\x1,\y2+2) to [bend left=45] (\x3,\y4+2);
  \draw[->] let \p1 = (C.three), \p2 = (C.center) in (\x1,\y2) -- +(0, .7) -| (S);
  \draw[->] let \p1 = (S.one), \p2 = (S.center) in (\x1,\y2) -- +(0, -.7) -| (C);
  \draw[->] let \p1 = (A.one), \p2 = (A.center), \p3 = (S.three), \p4 = (S.center) in (\x1,\y2-2) to [bend left=45] (\x3, \y4-2);
  \draw[->] let \p1 = (B.one), \p2 = (B.center), \p3 = (A.three), \p4 = (A.center) in (\x1,\y2-2) to [bend left=45] (\x3, \y4-2);
  \draw[->] let \p1 = (C.one), \p2 = (C.center), \p3 = (B.three), \p4 = (B.center) in (\x1,\y2-2) to [bend left=45] (\x3, \y4-2);
  \draw[->] (nil) -- (S);
  \draw[->] (it) -- (A);

\end{tikzpicture}
\end{preview}

%% it.next.prev = it.prev

\begin{preview}
\begin{tikzpicture}[
        list/.style={
            very thick, rectangle split, 
            rectangle split parts=3, draw, 
            rectangle split horizontal, minimum size=24pt,
            inner sep=4pt, text=black,
            rectangle split part fill={blue!20, red!20, blue!20}
        }, 
        sentinel/.style={
            very thick, rectangle split, 
            rectangle split parts=3, draw, 
            rectangle split horizontal, minimum size=24pt,
            inner sep=4pt, text=black,
            rectangle split part fill={blue!20, red!80, blue!20}
        }, 
        ->, start chain, very thick
      ]
  \node[sentinel,on chain] (S) {
    \nodepart{two} 
  };
  \node[list,on chain] (A) {
    \nodepart{two} 12
  };
  \node[list,on chain] (B) {
    \nodepart{two} 99
  };
  \node[list,on chain] (C) {
    \nodepart{two} 37
  };
  \node (nil) [below=of S] {\sc Nil};
  \node (it) [below=of A] {\sc it};
  \draw[->] let \p1 = (S.three), \p2 = (S.center), \p3 = (B.one), \p4 = (B.center) in (\x1,\y2+2) to [bend left=45] (\x3,\y4+2);
  \draw[->] let \p1 = (A.three), \p2 = (A.center), \p3 = (B.one), \p4 = (B.center) in (\x1,\y2+2) to [bend left=45] (\x3,\y4+2);
  \draw[->] let \p1 = (B.three), \p2 = (B.center), \p3 = (C.one), \p4 = (C.center) in (\x1,\y2+2) to [bend left=45] (\x3,\y4+2);
  \draw[->] let \p1 = (C.three), \p2 = (C.center) in (\x1,\y2) -- +(0, .7) -| (S);
  \draw[->] let \p1 = (S.one), \p2 = (S.center) in (\x1,\y2) -- +(0, -.7) -| (C);
  \draw[->] let \p1 = (A.one), \p2 = (A.center), \p3 = (S.three), \p4 = (S.center) in (\x1,\y2-2) to [bend left=45] (\x3, \y4-2);
  \draw[red!20,dashed,->] let \p1 = (B.one), \p2 = (B.center), \p3 = (A.three), \p4 = (A.center) in (\x1,\y2-2) to [bend left=45] (\x3, \y4-2);
  \draw[->] let \p1 = (B.one), \p2 = (B.center), \p3 = (S.three), \p4 = (S.center) in (\x1,\y2-2) to [bend left=45] (\x3, \y4-2);
  \draw[->] let \p1 = (C.one), \p2 = (C.center), \p3 = (B.three), \p4 = (B.center) in (\x1,\y2-2) to [bend left=45] (\x3, \y4-2);
  \draw[->] (nil) -- (S);
  \draw[->] (it) -- (A);

\end{tikzpicture}
\end{preview}

%% it.next.prev = it.prev

\begin{preview}
\begin{tikzpicture}[
        list/.style={
            very thick, rectangle split, 
            rectangle split parts=3, draw, 
            rectangle split horizontal, minimum size=24pt,
            inner sep=4pt, text=black,
            rectangle split part fill={blue!20, red!20, blue!20}
        }, 
        sentinel/.style={
            very thick, rectangle split, 
            rectangle split parts=3, draw, 
            rectangle split horizontal, minimum size=24pt,
            inner sep=4pt, text=black,
            rectangle split part fill={blue!20, red!80, blue!20}
        }, 
        free/.style={
            very thin, rectangle split, 
            rectangle split parts=3, draw, 
            rectangle split horizontal, minimum size=24pt,
            inner sep=4pt, text=black,
            rectangle split part fill={blue!10, red!10, blue!10}
        }, 
        ->, start chain, very thick
      ]
  \node[sentinel,on chain] (S) {
    \nodepart{two} 
  };
  \node[free,on chain] (A) {
    \nodepart[gray]{two} 12
  };
  \node[list,on chain] (B) {
    \nodepart{two} 99
  };
  \node[list,on chain] (C) {
    \nodepart{two} 37
  };
  \node (nil) [below=of S] {\sc Nil};
  \node (it) [below=of A] {\sc it};
  \draw[->] let \p1 = (S.three), \p2 = (S.center), \p3 = (B.one), \p4 = (B.center) in (\x1,\y2+2) to [bend left=45] (\x3,\y4+2);
  \draw[gray,->] let \p1 = (A.three), \p2 = (A.center), \p3 = (B.one), \p4 = (B.center) in (\x1,\y2+2) to [bend left=45] (\x3,\y4+2);
  \draw[->] let \p1 = (B.three), \p2 = (B.center), \p3 = (C.one), \p4 = (C.center) in (\x1,\y2+2) to [bend left=45] (\x3,\y4+2);
  \draw[->] let \p1 = (C.three), \p2 = (C.center) in (\x1,\y2) -- +(0, .7) -| (S);
  \draw[->] let \p1 = (S.one), \p2 = (S.center) in (\x1,\y2) -- +(0, -.7) -| (C);
  \draw[gray,->] let \p1 = (A.one), \p2 = (A.center), \p3 = (S.three), \p4 = (S.center) in (\x1,\y2-2) to [bend left=45] (\x3, \y4-2);
  \draw[red!20,dashed,->] let \p1 = (B.one), \p2 = (B.center), \p3 = (A.three), \p4 = (A.center) in (\x1,\y2-2) to [bend left=45] (\x3, \y4-2);
  \draw[->] let \p1 = (B.one), \p2 = (B.center), \p3 = (S.three), \p4 = (S.center) in (\x1,\y2-2) to [bend left=45] (\x3, \y4-2);
  \draw[->] let \p1 = (C.one), \p2 = (C.center), \p3 = (B.three), \p4 = (B.center) in (\x1,\y2-2) to [bend left=45] (\x3, \y4-2);
  \draw[->] (nil) -- (S);
  \draw[->] (it) -- (A);

\end{tikzpicture}
\end{preview}

\end{document}
