\documentclass{article}
\usepackage[pdftex,active,tightpage]{preview}
\setlength\PreviewBorder{2mm}
 
\usepackage{tikz}
\usetikzlibrary{calc,shapes.multipart,chains,arrows}

\tikzstyle{squarecross}=[
        draw, rectangle,minimum size=18pt, fill=orange!80,
        inner sep=0pt, text=black,
        path picture = {
            \draw[black]
            (path picture bounding box.north west) -- 
            (path picture bounding box.south east)
            (path picture bounding box.south west) -- 
            (path picture bounding box.north east);
        }
    ]

\tikzstyle{list}=[
            very thick, rectangle split, 
            rectangle split parts=2, draw, 
            rectangle split horizontal, minimum size=18pt,
            inner sep=4pt, text=black,
            rectangle split part fill={red!20, blue!20}]

\begin{document}

%% initial

\begin{preview}
\begin{tikzpicture}[
        ->, start chain, very thick
      ]
  \node[list,on chain] (A) {12};
  \node[list,on chain] (B) {99};
  \node[list,on chain] (C) {37};
  \node[list] [below=of B] (N) {25
  \nodepart{two} ?
  };
  \node (hd) [below=of A] (hd) {hd};
  \node (it) [above=of B] {it};
  \node (c) [right=of N] {c};
  \node[squarecross]   (D) [right=of C] {};
  \draw[->] (hd) -- (A);
  \draw[->] let \p1 = (A.two), \p2 = (A.center) in (\x1,\y2) -- (B);
  \draw[->] let \p1 = (B.two), \p2 = (B.center) in (\x1,\y2) -- (C);
  \draw[->] let \p1 = (C.two), \p2 = (C.center) in (\x1,\y2) -- (D);
  \draw[->] (it) -- (B);
  \draw[->] (c) -- (N);
\end{tikzpicture}
\end{preview}

%% c.next = it.next

\begin{preview}
\begin{tikzpicture}[
        ->, start chain, very thick
      ]
  \node[list,on chain] (A) {12};
  \node[list,on chain] (B) {99};
  \node[list,on chain] (C) {37};
  \node[list] [below=of B] (N) {25};
  \node (hd) [below=of A] (hd) {hd};
  \node (it) [above=of B] {it};
  \node (c) [right=of N] {c};
  \node[squarecross]   (D) [right=of C] {};
  \draw[->] (hd) -- (A);
  \draw[->] let \p1 = (A.two), \p2 = (A.center) in (\x1,\y2) -- (B);
  \draw[->] let \p1 = (B.two), \p2 = (B.center) in (\x1,\y2) -- (C);
  \draw[->] let \p1 = (C.two), \p2 = (C.center) in (\x1,\y2) -- (D);
  \draw[red,->] let \p1 = (N.two), \p2 = (N.center) in (\x1,\y2) -- (C);
  \draw[->] (it) -- (B);
  \draw[->] (c) -- (N);
\end{tikzpicture}
\end{preview}


\begin{preview}
\begin{tikzpicture}[
        ->, start chain, very thick
      ]
  \node[list,on chain] (A) {12};
  \node[list,on chain] (B) {99};
  \node[list,on chain] (C) {37};
  \node[list] [below=of B] (N) {25};
  \node (hd) [below=of A] (hd) {hd};
  \node (it) [above=of B] {it};
  \node (c) [right=of N] {c};
  \node[squarecross]   (D) [right=of C] {};
  \draw[->] (hd) -- (A);
  \draw[->] let \p1 = (A.two), \p2 = (A.center) in (\x1,\y2) -- (B);
  \draw[red,->] let \p1 = (B.two), \p2 = (B.center) in (\x1,\y2) -- (N);
  \draw[red!20,dashed,->,bend left=30] let \p1 = (B.two), \p2 = (B.center) in (\x1,\y2) -- (C);
  \draw[->] let \p1 = (C.two), \p2 = (C.center) in (\x1,\y2) -- (D);
  \draw[->] let \p1 = (N.two), \p2 = (N.center) in (\x1,\y2) -- (C);
  \draw[->] (it) -- (B);
  \draw[->] (c) -- (N);
\end{tikzpicture}
\end{preview}

\end{document}
