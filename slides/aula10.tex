\documentclass{beamer}
\setbeamertemplate{footline}[frame number]
%\documentclass{beamer}

\usepackage[utf8x]{inputenc}
\usepackage[brazil,british]{babel}
\usetheme{default} 
\usecolortheme{beaver}
\newtranslation[to=brazil]{Theorem}{Teorema}
\newtranslation[to=brazil]{Definition}{Definição}
\newtranslation[to=brazil]{Example}{Exemplo}
\newtranslation[to=brazil]{Problem}{Exercício}
\newtranslation[to=brazil]{Solution}{Resolução}
\newtranslation[to=brazil]{Lemma}{Lema}
\newtranslation[to=brazil]{Corollary}{Corolário}
\logo{\includegraphics[width=1cm]{../img/logo-ppgsc-icon-text.png}}

\usepackage{graphicx}
\usepackage{clrscode3e}
\usepackage{hyperref}

\usepackage{pgf}
\usepackage{tikz}
\newcommand{\assert}[1]{\textcolor{blue}{#1}}

\usepackage{xspace}

\newcommand{\semester}[0]{2015.2}


\usepackage{tikz}
\usetikzlibrary{arrows,positioning, calc} 
\tikzstyle{vertex}=[draw,fill=black!15,circle,minimum size=18pt,inner sep=0pt]

\title{Aula 10: Algoritmos de ordenação em arranjos}
\subtitle{Ordenação por \textit{heap\/}}
\author{David Déharbe \\
  Programa de Pós-graduação em Sistemas e Computação \\
  Universidade Federal do Rio Grande do Norte \\
  Centro de Ciências Exatas e da Terra \\
  Departamento de Informática e Matemáica Aplicada}
\date{18 de março de 2015}


\begin{document}
\selectlanguage{brazil}
\begin{frame}
  \titlepage
  Download me from \url{http://ufrn.academia.edu/DavidDeharbe}.
\end{frame}

\begin{frame}
  \frametitle{Plano da aula}
  \tableofcontents
\end{frame}

\begin{frame}
  \frametitle{Introdução}

  \begin{itemize}
    \item Ordenação em $\Theta(n \log n)$ (como ordenação por fusão);
    \item Sem arranjo auxiliar (como ordenação por inserção);
    \item Baseado no conceito de \textit{heap\/}.
      \begin{itemize}
        \item Listas de prioridade
      \end{itemize}
  \end{itemize}
\end{frame}

\section{Heap}

\begin{frame}

  \frametitle{\textit{Heap}}
  \framesubtitle{Conceito}

{\center \includegraphics[height=.5\textheight]{fig/heap.pdf}}

\begin{itemize}
\only<1>{
\item Árvore binária
  \begin{itemize}
  \item $\id{value}(N)$: valor no nó $N$;
  \item $\id{left}(N)$: filho esquerdo do nó $N$ (ou $\const{Nil}$);
  \item $\id{right}(N)$: filho direito do nó $N$ (ou $\const{Nil}$).
  \end{itemize}
}
\only<2>{
\item Árvore binária cheia, e o último nível é preenchido da esquerda
  para a direita. \alert{Propriedade estrutural}
}
\only<3>{
\item Em cada nó, o valor é maior ou igual aos valores nos nós filhos
  \begin{itemize}
  \item \alert{Propriedade de ordenação}
  \item $\forall N \cdot \id{left}(N) = \const{Nil} \lor \id{value}(N) \ge
    \id{value}(\id{left}(N))$;
  \item $\forall N \cdot \id{right}(N) = \const{Nil} \lor \id{value}(N) \ge
    \id{value}(\id{right}(N))$.
  \item Corolário: o maior valor encontra-se sempre na raiz.
  \end{itemize}
}
\end{itemize}

\end{frame}

\begin{frame}

  \frametitle{\textit{Heap}}
  \framesubtitle{Exercício}

{\center \includegraphics[height=.5\textheight]{fig/heap.pdf}}

\begin{itemize}
\item Seja um \textit{heap\/} com $n$ elementos.
\item Qual é a quantidade máxima de elementos no nível $i$ (assumindo que 1 é o nível da raiz, 2 o nível seguinte, etc.)?
\item Qual é a altura máxima do \textit{heap\/}?
\item O que podemos dizer de cada sub-árvore de um \textit{heap\/}?
\end{itemize}

\end{frame}

\begin{frame}

  \frametitle{\textit{Heap}}
  \framesubtitle{Representação dos dados}

{\center \includegraphics[height=.4\textheight]{fig/heap.pdf}}

\begin{itemize}
\item Representação em um arranjo, digamos $\id{A}$:
  \begin{itemize}
    \item seja $N_i$ o nó na posição $i$;
    \item $N_i$: $\id{value}(N_i) = \id{A}[i]$, $\id{left}(N_i) = 2i$, $\id{right}(N_i) = 2i+1$, $\id{up}(N_i) = \lfloor i / 2 \rfloor$.
    \item A raiz fica na posição $1$.
  \end{itemize}
\end{itemize}

{\center \includegraphics[width=\textwidth]{fig/heap-array.pdf}}
\end{frame}

\section{Listas de prioridade}

\begin{frame}

  \frametitle{\textit{Heap} para listas de prioridade}
  \framesubtitle{Especificação}

A lista de prioridade é uma coleção de elementos com as seguintes operações:
\begin{enumerate}
\item Obter elemento de maior prioridade
\item Inserir um novo elemento
\item Retirar o elemento de maior prioridade
\item Construir a lista a partir de uma sequência qualquer inicial
\end{enumerate}

\end{frame}

\begin{frame}

  \frametitle{\textit{Heap} para listas de prioridade}
  \framesubtitle{Representação dos dados}

Um arranjo $A$ onde
\begin{itemize}
\item $\id{num}(A)$: número de elementos na lista
\item $\id{size}(A)$: capacidade máxima da lista
\item $0 \le \id{num}(A) \le \id{size}(A)$
\item $\forall i | 2 \le i \le \id{num}(A) \cdot A(i) \le A(\lfloor i/2 \rfloor)$
\item Exemplo:
{\center \includegraphics[width=\textwidth]{fig/heap-array-2.pdf}}
\begin{itemize}
\item $\id{num} = 12$
\item $\id{size} = 16$
\end{itemize}
\end{itemize}

\end{frame}

\begin{frame}

  \frametitle{\textit{Heap} para listas de prioridade}
  \framesubtitle{Obtenção do maior elemento}

\begin{codebox}
\Procname{$\proc{Get-Min}(H)$}
\zi \Comment \assert{$H = \langle a_1, \ldots, a_{\id{size}(H)} \rangle \land 0 < \id{num}(H)$}
\li \Return $H[1]$
\zi \Comment \assert{$\id{r} = \mathbf{max} \{a_1, \ldots, a_{\id{num}(H)}\}$}
    \End
\end{codebox}  

\end{frame}

\begin{frame}{\textit{Heap} para listas de prioridade}
  \framesubtitle{Inserção de um elemento: princípio e exemplo}

\only<1>{
  {\center \includegraphics[width=\textwidth]{fig/heap-insert-1.pdf}}
  \begin{itemize}
    \item Inserir 60.
  \end{itemize}
}
\only<2>{
  {\center \includegraphics[width=\textwidth]{fig/heap-insert-2.pdf}}
  \begin{itemize}
  \item Restrição: Preservar a propriedade estrutural.
  \item Criar uma nova folha a direita no último nível.
  \end{itemize}
}
\only<3>{
  {\center \includegraphics[width=\textwidth]{fig/heap-insert-3.pdf}}
  \begin{itemize}
    \item Restrição 2: Reestabelecer a propriedade de ordenação.
  \item Ideia: trocar valores com pai.
  \end{itemize}
}
\only<4>{
  {\center \includegraphics[width=\textwidth]{fig/heap-insert-4.pdf}}
  Continuar enquanto o valor ``subindo''
  \begin{itemize}
  \item está em um nó com nó pai;
  \item é maior que o valor do pai.
  \end{itemize}
}
\only<5>{
  {\center \includegraphics[width=\textwidth]{fig/heap-insert-5.pdf}}
}
\only<6->{
  {\center \includegraphics[width=\textwidth]{fig/heap-insert-6.pdf}}
\pause
Exercício:
\begin{itemize}
  \item Desenhar o estado do \textit{heap\/} após uma, duas, três,... remoções do maior elemento.
\end{itemize}
}

\end{frame}

\begin{frame}
  \frametitle{\textit{Heap} para listas de prioridade}
  \framesubtitle{Algoritmo}

\begin{codebox}
\Procname{$\proc{Insert}(H, v)$}
\zi \Comment \assert{$H = \langle a_1, \ldots, a_{\id{size}(H)} \rangle \land \id{num}(H) < \id{size}(H)$}
\li $\id{num}(H) \gets \id{num}(H) + 1$
\li $H[\id{num}(H)] \gets v$
\li $\proc{Sift-Up}(H, \id{num}(H))$
    \End
\end{codebox}

\begin{codebox}
\Procname{$\proc{Sift-Up}(H, i)$}
\li \If $i > 1$ and $H[i] > H[\id{up}(i)]$
\li   \Then
        $\proc{Swap}(H, i, \id{up}(i))$
\li     $\proc{Sift-Up}(H, \id{up}(i))$
    \End
\end{codebox}

\begin{itemize}
\item Complexidade no pior caso
\begin{itemize}
  \item Uma chamada por nível da árvore
  \item $\Theta(\log \id{Num}(H))$
\end{itemize}
\end{itemize}
\end{frame}

\begin{frame}{\textit{Heap} para listas de prioridade}
  \framesubtitle{Remoção do maior elemento: princípio e exemplo}

\only<1>{
  {\center \includegraphics[width=\textwidth]{fig/heap-2-1.pdf}}
  \begin{itemize}
    \item O elemento 67 deve ser removido.
  \end{itemize}
}
\only<2>{
  {\center \includegraphics[width=\textwidth]{fig/heap-2-2.pdf}}
  \begin{itemize}
  \item Restrição: Preserva a propriedade estrutural.
  \item A folha a mais a direita é o único nó que pode ser removido.
  \item Mover o elemento nesta para raiz e eliminar o nó.
  \end{itemize}
}
\only<3>{
  {\center \includegraphics[width=\textwidth]{fig/heap-2-3.pdf}}
  \begin{itemize}
    \item Restrição 2: Reestabelecer a propriedade de ordenação.
  \end{itemize}
}
\only<4>{
  {\center \includegraphics[width=\textwidth]{fig/heap-2-4.pdf}}
  \begin{itemize}
  \item Restrição 2: Reestabelecer a propriedade de ordenação.
  \item Ideia: trocar valores com filho que tem maior elemento.
  \end{itemize}
}
\only<5>{
  {\center \includegraphics[width=\textwidth]{fig/heap-2-5.pdf}}
}
\only<6>{
  {\center \includegraphics[width=\textwidth]{fig/heap-2-6.pdf}}
  Continuar enquanto o valor ``descendo''
  \begin{itemize}
  \item tem um ou dois filhos;
  \item é menor que o maior dos filhos.
  \end{itemize}
}
\only<7>{
  {\center \includegraphics[width=\textwidth]{fig/heap-2-7.pdf}}
}
\only<8>{
  {\center \includegraphics[width=\textwidth]{fig/heap-2-8.pdf}}
}
\only<9>{
  {\center \includegraphics[width=\textwidth]{fig/heap-2-9.pdf}}
}
\only<10>{
  {\center \includegraphics[width=\textwidth]{fig/heap-2-10.pdf}}
}
\only<11->{
  {\center \includegraphics[width=\textwidth]{fig/heap-2-11.pdf}}
\pause
Exercício:
\begin{itemize}
  \item Desenhar o estado do \textit{heap\/} após inserir 50, 73, 65.
\end{itemize}
}

\end{frame}

\begin{frame}
  \frametitle{\textit{Heap} para listas de prioridade}
  \framesubtitle{Algoritmo}

\begin{codebox}
\Procname{$\proc{Remove-Min}(H)$}
\zi \Comment \assert{$H = \langle a_1, \ldots, a_{\id{size}(H)} \rangle \land 0 < \id{num}(H)$}
\li $H[1] \gets H[\id{num}(H)]$
\li $\id{num}(H) \gets \id{num}(H) - 1$
\li $\proc{Sift-Down}(H, 1)$
    \End
\end{codebox}

\end{frame}

\begin{frame}
  \frametitle{\textit{Heap} para listas de prioridade}
  \framesubtitle{Algoritmo}

\begin{codebox}
\Procname{$\proc{Sift-Down}(H, i)$}
\li \If $\proc{Left}(i) > \id{num}(H)$
\li   \Then \Return
\li   \ElseIf $\proc{Left}(i) = \id{num}(H)$ and $H[i] > H[\proc{Left}(i)]$
\li   \Then 
         $\proc{Swap}(H, i, \proc{Left}(i))$
\li      $\proc{Sift-Down}(H, \proc{Left}(i))$
\li   \ElseNoIf
\li       \If $H[\proc{Left}(i)] \ge H[\proc{Right}(i)]$
\li       \Then $m \gets \proc{Left}(i)$
\li       \Else $m \gets \proc{Right}(i)$
        \End
\li     \If $H[\proc{Left}(i)] > H[m]$
\li       \Then 
            $\proc{Swap}(H, i, \proc{Left}(i))$
\li         $\proc{Sift-Down}(H, \proc{Left}(i))$
          \End 
      \End
\end{codebox}  

\begin{itemize}
\item Complexidade no pior caso
\begin{itemize}
  \item Uma chamada por nível da árvore: $\Theta(\log \id{Num}(H))$
\end{itemize}
\end{itemize}
\end{frame}


\begin{frame}
  \frametitle{\textit{Heap} para listas de prioridade}
  \framesubtitle{Construção}

  \only<1>{
    \includegraphics[width=\textwidth]{fig/heap-build-1.pdf}

    \begin{itemize}
      \item Considerando uma sequência de valores em uma ordem qualquer, 
        como construir um \textit{heap\/}?
      \item Se os valores estão em um arranjo, a propriedade estrutural
        é inicialmente satisfeita.
    \end{itemize}
  }

  \only<2>{
    \includegraphics[width=\textwidth]{fig/heap-build-2.pdf}
    \begin{itemize}
    \item Inicialmente, as folhas formam sub-árvores que são
      \textit{heaps\/}.
    \end{itemize}
  }

  \only<3>{
    \includegraphics[width=\textwidth]{fig/heap-build-3.pdf}
    \begin{itemize}
    \item Cada nó interno cujas sub-árvores são \textit{heaps} é analizado.
    \end{itemize}
  }

  \only<4>{
    \includegraphics[width=\textwidth]{fig/heap-build-4.pdf}
    \begin{itemize}
    \item Se é maior que os filhos, nada é feito.
    \end{itemize}
  }

  \only<5>{
    \includegraphics[width=\textwidth]{fig/heap-build-5.pdf}
    \begin{itemize}
    \item Senão, a sub-árvore é corrigida aplicando $\proc{Sift-Down}$.
    \end{itemize}
  }

  \only<6>{
    \includegraphics[width=\textwidth]{fig/heap-build-6.pdf}
    \begin{itemize}
    \item A sub-árvore passa a ser um \textit{heap\/}.
    \end{itemize}
  }

  \only<7>{
    \includegraphics[width=\textwidth]{fig/heap-build-7.pdf}
    \begin{itemize}
    \item E assim sucessivamente...
    \end{itemize}
  }

  \only<8>{
    \includegraphics[width=\textwidth]{fig/heap-build-8.pdf}
  }

  \only<9>{
    \includegraphics[width=\textwidth]{fig/heap-build-9.pdf}
  }

  \only<10>{
    \includegraphics[width=\textwidth]{fig/heap-build-10.pdf}
  }

  \only<11>{
    \includegraphics[width=\textwidth]{fig/heap-build-11.pdf}
  }

  \only<12>{
    \includegraphics[width=\textwidth]{fig/heap-build-12.pdf}
  }

  \only<13>{
    \includegraphics[width=\textwidth]{fig/heap-build-13.pdf}
  }

  \only<14>{
    \includegraphics[width=\textwidth]{fig/heap-build-14.pdf}
  }

  \only<15>{
    \includegraphics[width=\textwidth]{fig/heap-build-15.pdf}
  }

  \only<16>{
    \includegraphics[width=\textwidth]{fig/heap-build-16.pdf}
  }

  \only<17>{
    \includegraphics[width=\textwidth]{fig/heap-build-17.pdf}
    \begin{itemize}
      \item Finalmente, o arranjo forma um \textit{heap\/}.
    \end{itemize}
  }

\end{frame}

\begin{frame}
  \frametitle{\textit{Heap} para listas de prioridade}
  \framesubtitle{Algoritmo de construção}

\begin{codebox}
\Procname{$\proc{Build}(H)$}
\li \For $i \gets \id{num}(H)/2$ \Downto $1$
\li   \Do 
        $\proc{Sift-Down}(H, i)$
      \End
\end{codebox}

\begin{itemize}
\item Complexidade
\begin{itemize}
\item $n$ nós,
\item Há, no máximo, $n/2^{h+1}$ nós na altura $h$,
\item O custo na altura $h$ é $O(h)$,
\item $\begin{array}[t]{rcl}
    T(n) & = & \sum_{h=0}^{\lfloor \lg n \rfloor} \lceil n / 2^{h+1} \rceil O(h) \\
    & = & O(n \sum_{h=0}^{\lfloor \lg n \rfloor} h/2^h) \\
    & = & O(n \sum_{h=0}^{\infty} h/2^h)
  \end{array}$
\item $\sum_{k=0}^{\infty} k x^k = \frac{x}{(1-x)^2}$ ($|x| < 1$)
\item $T(n) = O(n)$
\end{itemize}
\end{itemize}
  
\end{frame}

\begin{frame}

\frametitle{Listas de prioridade e \textit{heaps\/}}
\framesubtitle{Síntese da complexidade no pior caso}

\begin{enumerate}
\item Obter elemento de maior prioridade $\Theta(1)$
\item Inserir um novo elemento $\Theta(\lg n)$
\item Retirar o elemento de maior prioridade  $\Theta(\lg n)$
\item Construir a lista a partir de uma sequência qualquer inicial $\Theta(n)$
\end{enumerate}

\end{frame}

\begin{frame}

\frametitle{Ordenação e \textit{heaps\/}}

{\center \includegraphics[height=.4\textheight]{fig/heap.pdf}}

{\center \includegraphics[width=\textwidth]{fig/heap-array-2.pdf}}

\begin{itemize}
\item Como utilizar o conceito de \textit{heap\/} para ordenar o arranjo em
  ordem crescente?
\end{itemize}

\end{frame}

\begin{frame}

\frametitle{Ordenação e \textit{heaps\/}}
\framesubtitle{Princípios}

{\center \includegraphics[height=.4\textheight]{fig/heap.pdf}}

\begin{itemize}
\item O maior valor está na posição 1 do \textit{heap\/}.
\item Deve ficar na última posição do arranjo ordenado.
\pause
\item Inverter o conteúdo das posições 1 e $\id{num}(H)$.
\item Decrementar $\id{num}(H)$.
\item Reestabelecer a propriedade de ordenação aplicando $\proc{Sift-Down}$
  à posição 1.
\item Repetir enquanto $\id{num}(H) > 2$.
\end{itemize}

\end{frame}

\begin{frame}

\frametitle{Ordenação e \textit{heaps\/}}
\framesubtitle{Complexidade}

\begin{itemize}
\item Há $\Theta(n)$ chamadas a $\proc{Sift-Down}$
\item Cada chamada custa $O(\lg n)$.
\item Logo o custo total é $O(n \lg n)$
\end{itemize}

\end{frame}

\end{document}

