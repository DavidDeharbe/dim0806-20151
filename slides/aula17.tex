\documentclass{beamer}
\setbeamertemplate{footline}[frame number]
%\documentclass{beamer}

\usepackage[utf8x]{inputenc}
\usepackage[brazil,british]{babel}
\usetheme{default} 
\usecolortheme{beaver}
\newtranslation[to=brazil]{Theorem}{Teorema}
\newtranslation[to=brazil]{Definition}{Definição}
\newtranslation[to=brazil]{Example}{Exemplo}
\newtranslation[to=brazil]{Problem}{Exercício}
\newtranslation[to=brazil]{Solution}{Resolução}
\newtranslation[to=brazil]{Lemma}{Lema}
\newtranslation[to=brazil]{Corollary}{Corolário}
\logo{\includegraphics[width=1cm]{../img/logo-ppgsc-icon-text.png}}

\usepackage{graphicx}
\usepackage{clrscode3e}
\usepackage{hyperref}

\usepackage{pgf}
\usepackage{tikz}
\newcommand{\assert}[1]{\textcolor{blue}{#1}}

\usepackage{xspace}

\newcommand{\semester}[0]{2015.2}


\title{Aula 17: Árvores AVL}
\author{David Déharbe \\
  Programa de Pós-graduação em Sistemas e Computação \\
  Universidade Federal do Rio Grande do Norte \\
  Centro de Ciências Exatas e da Terra \\
  Departamento de Informática e Matemática Aplicada}
\date{}


\begin{document}
\selectlanguage{brazil}
\begin{frame}
  \titlepage
  Download me from \url{http://DavidDeharbe.github.io}.
\end{frame}

\begin{frame}
  \frametitle{Plano da aula}


  \begin{center}
    \includegraphics[width=.5\textwidth]{fig/avl.pdf}
  \end{center}

  \tableofcontents

\end{frame}

\section{Introdução}

\begin{frame}

  \frametitle{Árvores AVL}
  \framesubtitle{Introdução}

  
  \begin{itemize}
    
  \item Em 1962, Adelson-Velskii e Landis inventaram essa estrutura de dados:
    \begin{itemize}
    \item busca em $O(\lg n)$, 
    \item remoção em $\Theta(\lg n)$ e
    \item inserção em $\Theta(\lg n)$
    \end{itemize}

  \item uma \alert{árvore AVL} é uma árvore binária de busca 

  +restrição estrutural: altura das sub-árvores

  \item após inserção e remoção, a árvore pode se auto-balancear 

    caso médio: $\Theta(1)$, pior caso: $\Theta(\lg n)$
    
  \item a altura da árvore é $\Theta(\lg \langle \text{número de nós} \rangle)$

  \end{itemize}

\end{frame}

\begin{frame}

\frametitle{Árvores AVL}
\framesubtitle{Especificação}

\begin{eqnarray*}
\id{avl}(x) & \equiv & \begin{array}[t]{l}
\id{bst}(x) \land \\
(\begin{array}[t]{l}
x = \const{Nil} \lor \\
(\begin{array}[t]{ll}
  \left| \alpha(\attrib{x}{left}) - \alpha(\attrib{x}{right}) \right| \le 1 & \land \\
  \id{avl}{(\attrib{x}{left}}) \land \id{avl}({\attrib{x}{right}})\, )\, )
\end{array}
\end{array}
\end{array}
\end{eqnarray*}

\begin{itemize}
\item árvore binária de busca, e
\item árvore vazia, ou
\item para qualquer sub-árvore, a diferença de altura entre as duas sub-árvores
  não ultrapassa um (1).
\end{itemize}

\end{frame}

\begin{frame}

\frametitle{Implementação}

\begin{itemize}
\item $\attrib{x}{bal} \in \{-1, 0, 1\}$: balanço da árvore enraizada em $x$
\item $\attrib{x}{bal} = \alpha(\attrib{x}{right}) - \alpha(\attrib{x}{left})$
\end{itemize}

\end{frame}

\begin{frame}

\frametitle{Ilustração}

  Coleção: $2, 3, 5, 5, 7, 8$

  \begin{itemize}
\only<1>{
    \item Uma árvore AVL
      \begin{center}
        \includegraphics[width=.7\textwidth]{fig/abb-1.pdf}
      \end{center}
}
\only<2>{
    \item Uma árvore binária de busca que não é AVL
      \begin{center}
        \includegraphics[width=.6\textwidth]{fig/abb-2.pdf}
      \end{center}
}
  \end{itemize}

\end{frame}

\begin{frame}

\frametitle{Exercício}

\begin{enumerate}

  \item Desenhar a estrutura de árvores AVL com o menor número de nós
    possível com alturas de 0 até 5.

  \item Como definir a relação entre a altura e o menor número de nós
    possível? e o maior?

\end{enumerate}

\end{frame}

\section{Propriedades}

\begin{frame}

\frametitle{Propriedade de altura}

\only<1>{
$$\phi = \frac{1 + \sqrt{5}}{2}$$ (número de ouro)

\begin{theorem}
  Uma árvore AVL, enraizada em $x$, e com $n$ nós tem a sua altura tal que
  \begin{eqnarray*}
  \alpha(x) & \le & \log_\phi (\sqrt{5} \cdot (n+2)) - 2 \approx 1,44 \ln (n+1) - 0,328
  \end{eqnarray*}
\end{theorem}
}

\only<2>{
\begin{proof}(Esbouço)
Seja $\min$ a função que a uma dada altura $a$ associa o número mínimo de nós em uma árvore AVL de altura $a$:

$$\min 0 = 0, \min 1 = 1, \min n = 1 + \min (n-1) + \min (n-2)$$

As propriedades da função $\min$ tem como consequência o teorema enunciado. Não
entraremos mais em detalhes sobre este assunto.
\end{proof}
}

\end{frame}

\section{Operações}

\begin{frame}

\frametitle{Operação de busca}

\begin{itemize}
\item A busca não modifica a árvore.

\item O algoritmo de busca em árvores AVL é o mesmo da busca em árvores binárias
  de busca qualquer.

\end{itemize}
\end{frame}

\begin{frame}

\frametitle{Operação de inserção}

\begin{enumerate}

\item inserção em árvore binária de busca

\item \alert{se necessário}, re-balanceamento

\end{enumerate}

\end{frame}

\begin{frame}

\frametitle{Re-balanceamento}
\framesubtitle{Inserção}

aparece desequilibro quando:

\begin{itemize}

\item há pelo menos um nó entre a raiz e o nó criado tal que $\left|\attrib{n}{bal}\right| > 1$ após a inserção

\item seja $n$ o nó de maior nível (mais próximo do nó inserido) com
desequilibro

\item antes da inserção: $\attrib{n}{bal} = 1$ ou $\attrib{n}{bal} = -1$.

\item $n$ possui pelo uma sub-árvore 

  \begin{itemize}

    \item com altura maior que a outra sub-árvore

    \item logo não é vazia

  \end{itemize}

\item análise por caso: 

  \begin{enumerate}

  \item sub-árvore esquerda $E$, ou

  \item sub-árvore direita $D$

  \end{enumerate}

\end{itemize}

\end{frame}

\begin{frame}

\frametitle{Inserção em $E$}
\framesubtitle{Re-balanceamento}

$\alpha(E) = \alpha(D)+1$, $E \neq \const{Nil}$

\begin{tabular}{rcl}
$a$ & : & $\alpha(D)$ (e $\alpha(E) = a+1$), \\
$n_e$ & : & a raiz de $E$, \\
$EE$ & : & a sub-árvore esquerda de $n_e$, \\ 
$ED$ & : & a sub-árvore direita de $n_e$.
\end{tabular}

Logo, $a$ é a altura da sub-árvore mais alta entre $EE$ e $ED$.

\begin{enumerate}

\item inserção: em $EE$, ou

\item inserção: em $ED$.

\end{enumerate}

\end{frame}

\begin{frame}

\frametitle{Inserção em $EE$}
\framesubtitle{Re-balanceamento}

$\alpha(E) = \alpha(D)+1$, $E \neq \const{Nil}$

\begin{tabular}{rcl}
$a$ & : & $\alpha(D)$ (e $\alpha(E) = a+1$), \\
$n_e$ & : & a raiz de $E$, \\
$EE$ & : & a sub-árvore esquerda de $n_e$, \\ 
$ED$ & : & a sub-árvore direita de $n_e$.
\end{tabular}

inserção em $EE$

\begin{itemize}

\item altura de $EE$ antes: $a$

\item altura de $EE$ depois: $a+1$

\item altura de $ED$ antes: $a$ ou $a-1$

\begin{itemize}

\item se fosse $a-1$, $n_e$ estaria desbalanceado

\item contradiz hipótese que $n$ é o nó desbalanceado de maior nível

\item altura de $ED$ é $a$

\end{itemize}

\end{itemize}

\end{frame}

\begin{frame}

\frametitle{Inserção em $EE$: ilustração}
\framesubtitle{Re-balanceamento}

\begin{center}
\setlength{\unitlength}{0.55cm}
\begin{picture}(13,8)(0,0)
\put(1,3.00){\vector(0,1){2}}
\put(1,2.50){\vector(0,-1){2}}
\put(1,2.75){\makebox(0,0){$a+1$}}

\put(2.5,5){\circle*{.2}}
\put(2,5.7){\makebox(0,0)[b]{$n_{ee}$}}
\put(1.5,1){\line(1,4){1}}
\put(1.5,1){\line(1,0){2}}
\put(3.5,1){\line(-1,4){1}}
\put(2.5,3){\makebox(0,0){$EE'$}}

\put(2.5,0.5){\vector(0,1){0.5}}
\put(2.5,0.4){\makebox(0,0)[t]{$c$}}

\put(5.5,5){\circle*{.2}}
\put(4.5,1){\line(1,4){1}}
\put(4.5,1){\line(1,0){2}}
\put(6.5,1){\line(-1,4){1}}
\put(5.5,3){\makebox(0,0){$ED$}}

\put(7,3.25){\vector(0,1){1.75}}
\put(7,2.75){\vector(0,-1){1.75}}
\put(7,3){\makebox(0,0){$a$}}

\put(10,6.5){\circle*{.2}}
\put(9,2.5){\line(1,4){1}}
\put(9,2.5){\line(1,0){2}}
\put(11,2.5){\line(-1,4){1}}
\put(10,4.5){\makebox(0,0){$D$}}

\put(12,4.75){\vector(0,1){1.75}}
\put(12,4.25){\vector(0,-1){1.75}}
\put(12,4.5){\makebox(0,0){$a$}}

\put(2.5,5){\line(1,1){1.5}}
\put(5.5,5){\line(-1,1){1.5}}
\put(4,6.5){\circle*{.2}}
\put(4,6.7){\makebox(0,0)[b]{$n_e$}}
\put(4,5.9){\makebox(0,0)[b]{$E$}}

\put(4,6.5){\line(3,1){3}}
\put(10,6.5){\line(-3,1){3}}
\put(7,7.5){\circle*{.2}}
\put(7,7.7){\makebox(0,0)[b]{$n$}}
\put(7,6.9){\makebox(0,0)[b]{$S$}}

\end{picture}
\end{center}

\alert{rotação simples a direita}

\end{frame}

\begin{frame}
\frametitle{Rotação simples a direita}
\framesubtitle{Remanejamento}

\begin{tabular}{cc}
\setlength{\unitlength}{0.45cm}
\begin{picture}(13,8)(0,0)
\put(1,3.00){\vector(0,1){2}}
\put(1,2.50){\vector(0,-1){2}}
\put(1,2.75){\makebox(0,0){$a+1$}}

\put(2.5,5){\circle*{.2}}
\put(1.5,1){\line(1,4){1}}
\put(1.5,1){\line(1,0){2}}
\put(3.5,1){\line(-1,4){1}}
\put(2.5,3){\makebox(0,0){$EE'$}}

\put(2.5,0.5){\vector(0,1){0.5}}
\put(2.5,0.4){\makebox(0,0)[t]{$c$}}

\put(5.5,5){\circle*{.2}}
\put(4.5,1){\line(1,4){1}}
\put(4.5,1){\line(1,0){2}}
\put(6.5,1){\line(-1,4){1}}
\put(5.5,3){\makebox(0,0){$ED$}}

\put(7,3.25){\vector(0,1){1.75}}
\put(7,2.75){\vector(0,-1){1.75}}
\put(7,3){\makebox(0,0){$a$}}

\put(10,6.5){\circle*{.2}}
\put(9,2.5){\line(1,4){1}}
\put(9,2.5){\line(1,0){2}}
\put(11,2.5){\line(-1,4){1}}
\put(10,4.5){\makebox(0,0){$D$}}

\put(12,4.75){\vector(0,1){1.75}}
\put(12,4.25){\vector(0,-1){1.75}}
\put(12,4.5){\makebox(0,0){$a$}}

\put(2.5,5){\line(1,1){1.5}}
\put(5.5,5){\line(-1,1){1.5}}
\put(4,6.5){\circle*{.2}}
\put(4,6.7){\makebox(0,0)[b]{$n_e$}}

\put(4,6.5){\line(3,1){3}}
\put(10,6.5){\line(-3,1){3}}
\put(7,7.5){\circle*{.2}}
\put(7,7.7){\makebox(0,0)[b]{$n$}}

\end{picture}
&
\setlength{\unitlength}{0.45cm}
\begin{picture}(13,8)(0,0)
\put(0,2.00){\begin{picture}(13,6)(0,0)

\put(1,3.00){\vector(0,1){2}}
\put(1,2.50){\vector(0,-1){2}}
\put(1,2.75){\makebox(0,0){$a+1$}}

\put(2.5,5){\circle*{.2}}
\put(1.5,1){\line(1,4){1}}
\put(1.5,1){\line(1,0){2}}
\put(3.5,1){\line(-1,4){1}}
\put(2.5,3){\makebox(0,0){$EE'$}}

\put(2.5,0.5){\vector(0,1){0.5}}
\put(2.5,0.4){\makebox(0,0)[t]{$c$}}

\put(4.5,2.75){\vector(0,1){1.75}}
\put(4.5,2.25){\vector(0,-1){1.75}}
\put(4.5,2.5){\makebox(0,0){$a$}}

\put(6,4.5){\circle*{.2}}
\put(5,0.5){\line(1,4){1}}
\put(5,0.5){\line(1,0){2}}
\put(7,0.5){\line(-1,4){1}}
\put(6,2.5){\makebox(0,0){$ED$}}

\put(9,4.5){\circle*{.2}}
\put(8,0.5){\line(1,4){1}}
\put(8,0.5){\line(1,0){2}}
\put(10,0.5){\line(-1,4){1}}
\put(9,2.5){\makebox(0,0){$D$}}

\put(10.5,3.00){\vector(0,1){2}}
\put(10.5,2.50){\vector(0,-1){2}}
\put(10.5,2.75){\makebox(0,0){$a+1$}}

\put(6,4.5){\line(3,1){1.5}}
\put(9,4.5){\line(-3,1){1.5}}
\put(7.5,5){\circle*{.2}}
\put(7.5,5.2){\makebox(0,0)[b]{$n$}}

\put(2.5,5){\line(5,1){2.5}}
\put(7.5,5){\line(-5,1){2.5}}
\put(5,5.5){\circle*{.2}}
\put(5,5.7){\makebox(0,0)[b]{$n_e$}}

\end{picture}}
\end{picture} \\
Antes & Depois
\end{tabular}

\end{frame}

\begin{frame}

\frametitle{Rotação simples a direita: algoritmo}
\framesubtitle{Remanejamento}

\begin{codebox}
\Procname{$\proc{Rotate-Simple-Right}(n)$}
\li $ne \gets \attrib{n}{left}$
\li $\attrib{n}{left} \gets \attrib{ne}{right}$
\li $\attrib{ne}{right} \gets n$
\li $\attrib{ne}{bal} \gets 0$
\li $\attrib{n}{bal} \gets 0$
\li \Return $ne$
\end{codebox}

\only<1>{
\begin{itemize}
\item o parâmetro é a raiz da sub-árvore a remanejar
\item o resultado é a raiz da sub-árvore após o remanejamento
\end{itemize}
$\Theta(1)$ operações.
}

\only<2>{
Exercício:
\begin{enumerate}
\item termine o algoritmo com cálculo de $\attrib{}{up}$.
\item argumente que a árvore resultante é uma árvore binária de busca
\end{enumerate}
}


\end{frame}

\begin{frame}

\frametitle{Inserção em $ED$}
\framesubtitle{Re-balanceamento}

$\alpha(E) = \alpha(D)+1$, $E \neq \const{Nil}$

\begin{tabular}{rcl}
$a$ & : & $\alpha(D)$ (e $\alpha(E) = a+1$), \\
$n_e$ & : a raiz de $E$, \\
$EE$ & a sub-árvore esquerda de $n_e$, \\ 
$ED$ & a sub-árvore direita de $n_e$.
\end{tabular}

inserção em $ED$

\begin{itemize}

\item altura de $ED$ antes: $a$

\item altura de $ED$ depois: $a+1$

\item altura de $EE$ antes: $a$ ou $a-1$

\begin{itemize}

\item se fosse $a-1$, $n_e$ estaria desbalanceado

\item contradiz hipótese que $n$ é o nó desbalanceado de maior nível

\item altura de $EE$ é $a$

\end{itemize}

\end{itemize}

\end{frame}

\begin{frame}

\frametitle{Inserção em $ED$}
\framesubtitle{Re-balanceamento}

\begin{center}
\setlength{\unitlength}{0.61cm}
\begin{picture}(13,8)(0,0)
\put(1,3.25){\vector(0,1){1.75}}
\put(1,2.75){\vector(0,-1){1.75}}
\put(1,3){\makebox(0,0){$a$}}

\put(2.5,5){\circle*{.2}}
\put(1.5,1){\line(1,4){1}}
\put(1.5,1){\line(1,0){2}}
\put(3.5,1){\line(-1,4){1}}
\put(2.5,3){\makebox(0,0){$EE$}}

\put(5.5,5){\circle*{.2}}
\put(5.5,5.2){\makebox(0,0)[b]{$n_{ed}$}}
\put(4.5,1){\line(1,4){1}}
\put(4.5,1){\line(1,0){2}}
\put(6.5,1){\line(-1,4){1}}
\put(5.5,3){\makebox(0,0){$ED'$}}

\put(5.5,0.5){\vector(0,1){0.5}}
\put(5.5,0.4){\makebox(0,0)[t]{$c$}}

\put(7,3.00){\vector(0,1){2}}
\put(7,2.50){\vector(0,-1){2}}
\put(7,2.75){\makebox(0,0){$a+1$}}

\put(10,6.5){\circle*{.2}}
\put(9,2.5){\line(1,4){1}}
\put(9,2.5){\line(1,0){2}}
\put(11,2.5){\line(-1,4){1}}
\put(10,4.5){\makebox(0,0){$D$}}

\put(12,4.75){\vector(0,1){1.75}}
\put(12,4.25){\vector(0,-1){1.75}}
\put(12,4.5){\makebox(0,0){$a$}}

\put(2.5,5){\line(1,1){1.5}}
\put(5.5,5){\line(-1,1){1.5}}
\put(4,6.5){\circle*{.2}}
\put(4,6.7){\makebox(0,0)[b]{$n_e$}}
\put(4,5.9){\makebox(0,0)[b]{$E$}}

\put(4,6.5){\line(3,1){3}}
\put(10,6.5){\line(-3,1){3}}
\put(7,7.5){\circle*{.2}}
\put(7,7.7){\makebox(0,0)[b]{$n$}}
\put(7,6.9){\makebox(0,0)[b]{$S$}}

\end{picture}
\end{center}

\alert{rotação dupla a direita}

\end{frame}

\begin{frame}
\frametitle{Rotação dupla a direita}
\framesubtitle{Remanejamento}

\begin{tabular}{c}
\setlength{\unitlength}{0.45cm}
\begin{picture}(15,8)(0,0)

\put(1,3.75){\vector(0,1){1.75}}
\put(1,3.25){\vector(0,-1){1.75}}
\put(1,3.50){\makebox(0,0){\footnotesize $a$}}

\put(2.5,5.5){\circle*{.2}}
\put(1.5,1.5){\line(1,4){1}}
\put(1.5,1.5){\line(1,0){2}}
\put(3.5,1.5){\line(-1,4){1}}
\put(2.5,3.5){\makebox(0,0){\footnotesize $EE$}}

\put(4.5,2.75){\vector(0,1){2.25}}
\put(4.5,2.25){\vector(0,-1){2.25}}
\put(4.5,2.50){\makebox(0,0){\footnotesize $a$}}

\put(6,5){\circle*{.2}}
\put(5,1){\line(1,4){1}}
\put(5,1){\line(1,0){2}}
\put(7,1){\line(-1,4){1}}
\put(6,3){\makebox(0,0){\footnotesize $EDE'$}}

\put(6,0.5){\vector(0,1){0.5}}
\put(6,0.4){\makebox(0,0)[t]{\footnotesize $c$}}

\put(9,5){\circle*{.2}}
\put(8,1){\line(1,4){1}}
\put(8,1){\line(1,0){2}}
\put(10,1){\line(-1,4){1}}
\put(9,3){\makebox(0,0){\footnotesize $EDD'$}}

\put(10.5,3.25){\vector(0,1){1.75}}
\put(10.5,2.75){\vector(0,-1){1.75}}
\put(10.5,3.00){\makebox(0,0){\footnotesize $a-1$}}

\put(13,6){\circle*{.2}}
\put(12,2){\line(1,4){1}}
\put(12,2){\line(1,0){2}}
\put(14,2){\line(-1,4){1}}
\put(13,4){\makebox(0,0){\footnotesize $D$}}

\put(14.5,4.25){\vector(0,1){1.75}}
\put(14.5,3.75){\vector(0,-1){1.75}}
\put(14.5,4.00){\makebox(0,0){\footnotesize $a$}}

\put(6,5){\line(3,1){1.5}}
\put(9,5){\line(-3,1){1.5}}
\put(7.5,5.5){\circle*{.2}}
\put(7.5,5.7){\makebox(0,0)[b]{\footnotesize $n_{ed}$}}

\put(2.5,5.5){\line(5,1){2.5}}
\put(7.5,5.5){\line(-5,1){2.5}}
\put(5.0,6.0){\circle*{.2}}
\put(5.0,6.2){\makebox(0,0)[b]{\footnotesize $n_e$}}

\put(5,6){\line(4,1){4}}
\put(13,6){\line(-4,1){4}}
\put(9,7){\circle*{.2}}
\put(9,7.2){\makebox(0,0)[b]{\footnotesize $n$}}

\end{picture}
\\
antes \\
\setlength{\unitlength}{0.45cm}
\begin{picture}(12,6)(0,0)

\put(1,2.25){\vector(0,1){1.75}}
\put(1,1.75){\vector(0,-1){1.75}}
\put(1,2.00){\makebox(0,0){\footnotesize $a$}}

\put(2.5,4){\circle*{.2}}
\put(1.5,0){\line(1,4){1}}
\put(1.5,0){\line(1,0){2}}
\put(3.5,0){\line(-1,4){1}}
\put(2.5,2){\makebox(0,0){\footnotesize $EE$}}

\put(4,2.25){\vector(0,1){1.75}}
\put(4,1.75){\vector(0,-1){1.75}}
\put(4,2.00){\makebox(0,0){\footnotesize $a$}}

\put(5.5,4){\circle*{.2}}
\put(4.5,0){\line(1,4){1}}
\put(4.5,0){\line(1,0){2}}
\put(6.5,0){\line(-1,4){1}}
\put(5.5,2){\makebox(0,0){\footnotesize $EDE'$}}

\put(2.5,4.0){\line(3,1){1.5}}
\put(5.5,4.0){\line(-3,1){1.5}}
\put(4.0,4.5){\circle*{.2}}
\put(4.0,4.7){\makebox(0,0)[b]{\footnotesize $n_e$}}

\put(7,2.25){\vector(0,1){1.75}}
\put(7,1.75){\vector(0,-1){1.75}}
\put(7,2.00){\makebox(0,0){\footnotesize $a-1$}}

\put(8.5,4){\circle*{.2}}
\put(7.5,0){\line(1,4){1}}
\put(7.5,0){\line(1,0){2}}
\put(9.5,0){\line(-1,4){1}}
\put(8.5,2){\makebox(0,0){\footnotesize $EDD'$}}

\put(10,2.25){\vector(0,1){1.75}}
\put(10,1.75){\vector(0,-1){1.75}}
\put(10,2.00){\makebox(0,0){\footnotesize $a$}}

\put(11.5,4){\circle*{.2}}
\put(10.5,0){\line(1,4){1}}
\put(10.5,0){\line(1,0){2}}
\put(12.5,0){\line(-1,4){1}}
\put(11.5,2){\makebox(0,0){\footnotesize $D$}}

\put(8.5,4.0){\line(3,1){1.5}}
\put(11.5,4.0){\line(-3,1){1.5}}
\put(10.0,4.5){\circle*{.2}}
\put(10.0,4.7){\makebox(0,0)[b]{\footnotesize $n$}}

\put(04.0,4.5){\line(3,1){3}}
\put(10.0,4.5){\line(-3,1){3}}
\put(07.0,5.5){\circle*{.2}}
\put(07.0,5.7){\makebox(0,0)[b]{\footnotesize $n_{ed}$}}

\end{picture} \\
depois
\end{tabular}

\end{frame}

\begin{frame}

\frametitle{Rotação dupla a direita: algoritmo}
\framesubtitle{Remanejamento}

\begin{tabular}{cc}
\begin{minipage}[t]{.5\textwidth}
\begin{codebox}
\Procname{$\proc{Rotate-Double-Right}(n)$}
\zi $ne \gets \attrib{n}{left}$
\zi $ned \gets \attrib{ne}{right}$
\zi $\attrib{n}{left} \gets \attrib{ned}{right}$
\zi $\attrib{ne}{right} \gets \attrib{ned}{left}$
\zi $\attrib{ned}{left} \gets ne$
\zi $\attrib{ned}{right} \gets n$
\end{codebox}
\end{minipage}
&\begin{minipage}[t]{.45\textwidth}
\begin{codebox}
\zi \If $\attrib{ned}{bal} \isequal 1$
\zi \Then $\attrib{n}{bal} \gets 0$
\zi   $\attrib{ne}{bal} \gets -1$
\zi \Else \If $\attrib{ned}{bal} \isequal 0$
\zi \Then $\attrib{n}{bal} \gets 0$
\zi   $\attrib{ne}{bal} \gets 0$
\zi \Else
\zi   $\attrib{n}{bal} \gets 1$
\zi   $\attrib{ne}{bal} \gets 0$
\zi $\attrib{ned}{bal} \gets 0$
\zi \Return $ned$
\end{codebox}
\end{minipage}
\end{tabular}

\only<1>{
\begin{itemize}
\item o parâmetro é a raiz da sub-árvore a remanejar
\item o resultado é a raiz da sub-árvore após o remanejamento
\end{itemize}
$\Theta(1)$ operações.
}

\only<2>{
Exercício:
\begin{enumerate}
\item termine o algoritmo com cálculo de $\attrib{}{up}$.
\item argumente que a árvore resultante é uma árvore binária de busca
\end{enumerate}
}

\end{frame}

\begin{frame}

\frametitle{Inserção em $D$}
\framesubtitle{Re-balanceamento}

\begin{itemize}

\item A situação é simétrica à inserção em $E$.

\item A análise é similar.

\item O tratamento é simétrico, com 2 sub-casos ($DD$ e $DE$).

\end{itemize}

\end{frame}

\begin{frame}

\frametitle{Inserção em $DD$: ilustração}
\framesubtitle{Re-balanceamento}

\begin{center}
\setlength{\unitlength}{0.55cm}
\begin{picture}(13,8)(0,0)
\put(12,2.50){\vector(0,-1){2.25}}
\put(12,3.00){\vector(0,1){2.00}}
\put(12,2.75){\makebox(0,0){$a+1$}}

\put(10.5,5){\circle*{.2}}
\put(11.5,1){\line(-1,4){1}}
\put(11.5,1){\line(-1,0){2}}
\put(9.5,1){\line(1,4){1}}
\put(10.5,3){\makebox(0,0){$DD$}}

\put(10.5,0.5){\vector(0,1){0.5}}
\put(10.5,0.4){\makebox(0,0)[t]{$c$}}

\put(7.5,5){\circle*{.2}}
\put(8.5,1){\line(-1,4){1}}
\put(8.5,1){\line(-1,0){2}}
\put(6.5,1){\line(1,4){1}}
\put(7.5,3){\makebox(0,0){$DE$}}

\put(6,3.25){\vector(0,1){1.75}}
\put(6,2.75){\vector(0,-1){1.75}}
\put(6,3){\makebox(0,0){$a$}}

\put(3,6.5){\circle*{.2}}
\put(4,2.5){\line(-1,4){1}}
\put(4,2.5){\line(-1,0){2}}
\put(2,2.5){\line(1,4){1}}
\put(3,4.5){\makebox(0,0){$D$}}

\put(1,4.75){\vector(0,1){1.75}}
\put(1,4.25){\vector(0,-1){1.75}}
\put(1,4.5){\makebox(0,0){$a$}}

\put(10.5,5){\line(-1,1){1.5}}
\put(7.5,5){\line(1,1){1.5}}
\put(9,6.5){\circle*{.2}}
\put(9,6.7){\makebox(0,0)[b]{$n_d$}}
\put(9,5.9){\makebox(0,0)[b]{$D$}}

\put(9,6.5){\line(-3,1){3}}
\put(3,6.5){\line(3,1){3}}
\put(6,7.5){\circle*{.2}}
\put(6,7.7){\makebox(0,0)[b]{$n$}}
\put(6,6.9){\makebox(0,0)[b]{$S$}}

\end{picture}
\end{center}

\alert{rotação simples a esquerda}

\end{frame}

\begin{frame}

\frametitle{Inserção em $DE$}
\framesubtitle{Re-balanceamento}

\begin{center}
\setlength{\unitlength}{0.61cm}
\begin{picture}(13,8)(0,0)
\put(12,3.25){\vector(0,1){1.75}}
\put(12,2.70){\vector(0,-1){2.20}}
\put(12,3){\makebox(0,0){$a$}}

\put(10.5,5){\circle*{.2}}
\put(11.5,1){\line(-1,4){1}}
\put(11.5,1){\line(-1,0){2}}
\put(9.5,1){\line(1,4){1}}
\put(10.5,3){\makebox(0,0){$DD$}}

\put(7.5,5){\circle*{.2}}
\put(7.5,5.2){\makebox(0,0)[b]{$n_{de}$}}
\put(8.5,1){\line(-1,4){1}}
\put(8.5,1){\line(-1,0){2}}
\put(6.5,1){\line(1,4){1}}
\put(7.5,3){\makebox(0,0){$DE'$}}

\put(7.5,0.5){\vector(0,1){0.5}}
\put(7.5,0.4){\makebox(0,0)[t]{$c$}}

\put(6,3.00){\vector(0,1){2}}
\put(6,2.50){\vector(0,-1){2.25}}
\put(6,2.75){\makebox(0,0){$a+1$}}

\put(3,6.5){\circle*{.2}}
\put(4,2.5){\line(-1,4){1}}
\put(4,2.5){\line(-1,0){2}}
\put(2,2.5){\line(1,4){1}}
\put(3,4.5){\makebox(0,0){$E$}}

\put(1,4.75){\vector(0,1){1.75}}
\put(1,4.25){\vector(0,-1){1.75}}
\put(1,4.5){\makebox(0,0){$a$}}

\put(10.5,5){\line(-1,1){1.5}}
\put(7.5,5){\line(1,1){1.5}}
\put(9,6.5){\circle*{.2}}
\put(9,6.7){\makebox(0,0)[b]{$n_d$}}
\put(9,5.9){\makebox(0,0)[b]{$D$}}

\put(9,6.5){\line(-3,1){3}}
\put(3,6.5){\line(3,1){3}}
\put(6,7.5){\circle*{.2}}
\put(6,7.7){\makebox(0,0)[b]{$n$}}
\put(6,6.9){\makebox(0,0)[b]{$S$}}
\end{picture}
\end{center}

\alert{rotação dupla a esquerda}

\end{frame}

\begin{frame}

\frametitle{Inserção: algoritmo}

\begin{tabular}{rcl}
$n$ & : & raiz da sub-árvore onde a inserção ocorrerá (\alert{ref}) \\
$v$ & : & valor a inserir \\
& $\rightarrow$ & Booleano indicando se a altura aumentou
\end{tabular}

\begin{codebox}
\only<1>{
\Procname{$\proc{Insert}(n, v)$}
\li \If $n \isequal \const{Nil}$
\li \Then $n \gets \proc{Make-Avl-Node}(v, \const{Nil}, \const{Nil}, 0)$
\li   \Return $\const{True}$
    \End
\li \If $\attrib{n}{val} \isequal v$
\li \Then \Return $\const{False}$
    \End
\li \If $v < \attrib{n}{val}$
\li \Then  $\id{inc} \gets \proc{Insert}(\attrib{n}{left}, v)$
\li   \If $\id{inc}$
\zi      \alert{$\langle \text{tratamento de incremento de altura na sub-árvore } \rangle$}
      \End
\li \Else $\cdots$
    \End
}
\only<2>{
\li   \If $\id{inc}$
\li   \Then \If $\attrib{n}{bal} \isequal 0$
\li     \Then
          $\attrib{n}{bal} \gets -1$
\li       \Return $\const{True}$
\li     \ElseIf $\attrib{n}{bal} \isequal +1$
\li     \Then
          $\attrib{n}{bal} \gets 0$
\li       \Return $\const{False}$
\li     \Else
\li       \If $\attrib{\attrib{n}{left}}{bal} \isequal -1$
\li       \Then $n \gets \proc{Rotate-Simple-Right}(n)$
\li       \Else $n \gets \proc{Rotate-Double-Right}(n)$
          \End
\li       \Return $\const{False}$
        \End
      \End
\li \Else \alert{$\langle \text{Inserção na sub-árvore direita} \rangle$}
    \End
}
\end{codebox}

\only<3>{Exercícios:
\begin{enumerate}
\item escrever o tratamento correspondente a $v > \attrib{n}{val}$.
\item verifique que, quando há remanejamento, a altura da árvore após a inserção
  é a mesma da altura antes da inserção.
\end{enumerate}
}

\end{frame}

\begin{frame}
\frametitle{Remoção}
\framesubtitle{Introdução}
\begin{itemize}

\item remove valor como em árvore binária de busca

\item reequilibrar a árvore para reestabelecer a propriedade AVL

\item mais de uma rotação pode ser necessária

  $O(\lg n)$ no pior caso

\end{itemize}
\end{frame}

\begin{frame}
\frametitle{Remoção}
\framesubtitle{Exemplo}

\begin{center}
\begin{tabular}{ccc}
\setlength{\unitlength}{0.3cm}
\begin{picture}(13,12)(0,0)

\put(6,11.2){\makebox(0,0)[b]{$47^{(1)}$}}
\put(6,11){\circle*{.2}}
\put(6,11){\line(-4,-1){4}}
\put(6,11){\line(4,-1){4}}

\put(2,10.2){\makebox(0,0)[b]{$32^{(-1)}$}}
\put(2,10){\circle*{.2}}
\put(2,10){\line(-1,-1){2}}
\put(2,10){\line(1,-1){2}}

\put(10,10.2){\makebox(0,0)[b]{$68^{(-1)}$}}
\put(10,10){\circle*{.2}}
\put(10,10){\line(-1,-1){2}}
\put(10,10){\line(1,-1){2}}

\put(0,8.2){\makebox(0,0)[b]{$5^{(1)}$}}
\put(0,8){\circle*{.2}}
\put(0,8){\line(1,-4){1}}

\put(4,8.2){\makebox(0,0)[b]{$40^{(0)}$}}
\put(4,8){\circle*{.2}}

\put(8,8.2){\makebox(0,0)[b]{$60^{(-1)}$}}
\put(8,8){\circle*{.2}}
\put(8,8){\line(-1,-4){1}}
\put(8,8){\line(1,-4){1}}

\put(12,8.2){\makebox(0,0)[b]{$88^{(1)}$}}
\put(12,8){\circle*{.2}}
\put(12,8){\line(1,-4){1}}

\put(1,4.2){\makebox(0,0)[b]{$15^{(0)}$}}
\put(1,4){\circle*{.2}}

\put(7,4.2){\makebox(0,0)[b]{$54^{(-1)}$}}
\put(7,4){\circle*{.2}}
\put(7,4){\line(-1,-4){1}}

\put(9,4.2){\makebox(0,0)[b]{$61^{(0)}$}}
\put(9,4){\circle*{.2}}

\put(13,4.2){\makebox(0,0)[b]{$90^{(0)}$}}
\put(13,4){\circle*{.2}}

\put(6,0.2){\makebox(0,0)[b]{$50^{(0)}$}}
\put(6,0){\circle*{.2}}
\end{picture}

& 
\hspace{1cm}

&
\setlength{\unitlength}{0.4cm}
\begin{picture}(13,12)(0,0)

\put(6,11.2){\makebox(0,0)[b]{$40^{(1)}$}}
\put(6,11){\circle*{.2}}
\put(6,11){\line(-4,-1){4}}
\put(6,11){\line(4,-1){4}}

\put(2,10.2){\makebox(0,0)[b]{\alert{$32^{(-2)}$}}}
\put(2,10){\alert{\circle*{.2}}}
\put(2,10){\line(-1,-1){2}}

\put(10,10.2){\makebox(0,0)[b]{$68^{(-1)}$}}
\put(10,10){\circle*{.2}}
\put(10,10){\line(-1,-1){2}}
\put(10,10){\line(1,-1){2}}

\put(0,8.2){\makebox(0,0)[b]{$5^{(1)}$}}
\put(0,8){\circle*{.2}}
\put(0,8){\line(1,-4){1}}

\put(8,8.2){\makebox(0,0)[b]{$60^{(-1)}$}}
\put(8,8){\circle*{.2}}
\put(8,8){\line(-1,-4){1}}
\put(8,8){\line(1,-4){1}}

\put(12,8.2){\makebox(0,0)[b]{$88^{(1)}$}}
\put(12,8){\circle*{.2}}
\put(12,8){\line(1,-4){1}}

\put(1,4.2){\makebox(0,0)[b]{$15^{(0)}$}}
\put(1,4){\circle*{.2}}

\put(7,4.2){\makebox(0,0)[b]{$54^{(-1)}$}}
\put(7,4){\circle*{.2}}
\put(7,4){\line(-1,-4){1}}

\put(9,4.2){\makebox(0,0)[b]{$61^{(0)}$}}
\put(9,4){\circle*{.2}}

\put(13,4.2){\makebox(0,0)[b]{$90^{(0)}$}}
\put(13,4){\circle*{.2}}

\put(6,0.2){\makebox(0,0)[b]{$50^{(0)}$}}
\put(6,0){\circle*{.2}}
\end{picture} \\

& & \\

(1) árvore inicial & & (2) após remoção de 47

\end{tabular}
\end{center}

\end{frame}

\begin{frame}
\frametitle{Remoção}
\framesubtitle{Exemplo}

\begin{center}
\begin{tabular}{ccc}
\setlength{\unitlength}{0.3cm}
\begin{picture}(13,12)(0,0)

\put(6,11.2){\makebox(0,0)[b]{$40^{(1)}$}}
\put(6,11){\circle*{.2}}
\put(6,11){\line(-4,-1){4}}
\put(6,11){\line(4,-1){4}}

\put(2,10.2){\makebox(0,0)[b]{$32^{(-2)}$}}
\put(2,10){\circle*{.2}}
\put(2,10){\line(-1,-1){2}}

\put(10,10.2){\makebox(0,0)[b]{$68^{(-1)}$}}
\put(10,10){\circle*{.2}}
\put(10,10){\line(-1,-1){2}}
\put(10,10){\line(1,-1){2}}

\put(0,8.2){\makebox(0,0)[b]{$5^{(1)}$}}
\put(0,8){\circle*{.2}}
\put(0,8){\line(1,-4){1}}

\put(8,8.2){\makebox(0,0)[b]{$60^{(-1)}$}}
\put(8,8){\circle*{.2}}
\put(8,8){\line(-1,-4){1}}
\put(8,8){\line(1,-4){1}}

\put(12,8.2){\makebox(0,0)[b]{$88^{(1)}$}}
\put(12,8){\circle*{.2}}
\put(12,8){\line(1,-4){1}}

\put(1,4.2){\makebox(0,0)[b]{$15^{(0)}$}}
\put(1,4){\circle*{.2}}

\put(7,4.2){\makebox(0,0)[b]{$54^{(-1)}$}}
\put(7,4){\circle*{.2}}
\put(7,4){\line(-1,-4){1}}

\put(9,4.2){\makebox(0,0)[b]{$61^{(0)}$}}
\put(9,4){\circle*{.2}}

\put(13,4.2){\makebox(0,0)[b]{$90^{(0)}$}}
\put(13,4){\circle*{.2}}

\put(6,0.2){\makebox(0,0)[b]{$50^{(0)}$}}
\put(6,0){\circle*{.2}}
\end{picture} 
& &
\setlength{\unitlength}{0.3cm}
\begin{picture}(13,12)(0,0)

\put(6,11.2){\makebox(0,0)[b]{\alert{$40^{(2)}$}}}
\put(6,11){\alert{\circle*{.2}}}
\put(6,11){\line(-4,-1){4}}
\put(6,11){\line(4,-1){4}}

\put(2,10.2){\makebox(0,0)[b]{$15^{(0)}$}}
\put(2,10){\circle*{.2}}
\put(2,10){\line(-1,-1){2}}
\put(2,10){\line(1,-1){2}}

\put(10,10.2){\makebox(0,0)[b]{$68^{(-1)}$}}
\put(10,10){\circle*{.2}}
\put(10,10){\line(-1,-1){2}}
\put(10,10){\line(1,-1){2}}

\put(0,8.2){\makebox(0,0)[b]{$5^{(0)}$}}
\put(0,8){\circle*{.2}}

\put(4,8.2){\makebox(0,0)[b]{$32^{(0)}$}}
\put(4,8){\circle*{.2}}

\put(8,8.2){\makebox(0,0)[b]{$60^{(-1)}$}}
\put(8,8){\circle*{.2}}
\put(8,8){\line(-1,-4){1}}
\put(8,8){\line(1,-4){1}}

\put(12,8.2){\makebox(0,0)[b]{$88^{(1)}$}}
\put(12,8){\circle*{.2}}
\put(12,8){\line(1,-4){1}}

\put(7,4.2){\makebox(0,0)[b]{$54^{(-1)}$}}
\put(7,4){\circle*{.2}}
\put(7,4){\line(-1,-4){1}}

\put(9,4.2){\makebox(0,0)[b]{$61^{(0)}$}}
\put(9,4){\circle*{.2}}

\put(13,4.2){\makebox(0,0)[b]{$90^{(0)}$}}
\put(13,4){\circle*{.2}}

\put(6,0.2){\makebox(0,0)[b]{$50^{(0)}$}}
\put(6,0){\circle*{.2}}
\end{picture}
\\

& & \\

após remoção de 47 
& & 
após rotação dupla a direita de 32
\end{tabular}
\end{center}

\end{frame}

\begin{frame}
\frametitle{Remoção}
\framesubtitle{Exemplo}

\begin{center}
\begin{tabular}{cc}
\setlength{\unitlength}{0.3cm}
\begin{picture}(13,12)(0,0)

\put(6,11.2){\makebox(0,0)[b]{\footnotesize $40^{(2)}$}}
\put(6,11){\circle*{.2}}
\put(6,11){\line(-4,-1){4}}
\put(6,11){\line(4,-1){4}}

\put(2,10.2){\makebox(0,0)[b]{\footnotesize $15^{(0)}$}}
\put(2,10){\circle*{.2}}
\put(2,10){\line(-1,-1){2}}
\put(2,10){\line(1,-1){2}}

\put(10,10.2){\makebox(0,0)[b]{\footnotesize $68^{(-1)}$}}
\put(10,10){\circle*{.2}}
\put(10,10){\line(-1,-1){2}}
\put(10,10){\line(1,-1){2}}

\put(0,8.2){\makebox(0,0)[b]{\footnotesize $5^{(0)}$}}
\put(0,8){\circle*{.2}}

\put(4,8.2){\makebox(0,0)[b]{\footnotesize $32^{(0)}$}}
\put(4,8){\circle*{.2}}

\put(8,8.2){\makebox(0,0)[b]{\footnotesize $60^{(-1)}$}}
\put(8,8){\circle*{.2}}
\put(8,8){\line(-1,-4){1}}
\put(8,8){\line(1,-4){1}}

\put(12,8.2){\makebox(0,0)[b]{\footnotesize $88^{(1)}$}}
\put(12,8){\circle*{.2}}
\put(12,8){\line(1,-4){1}}

\put(7,4.2){\makebox(0,0)[b]{\footnotesize $54^{(-1)}$}}
\put(7,4){\circle*{.2}}
\put(7,4){\line(-1,-4){1}}

\put(9,4.2){\makebox(0,0)[b]{\footnotesize $61^{(0)}$}}
\put(9,4){\circle*{.2}}

\put(13,4.2){\makebox(0,0)[b]{\footnotesize $90^{(0)}$}}
\put(13,4){\circle*{.2}}

\put(6,0.2){\makebox(0,0)[b]{\footnotesize $50^{(0)}$}}
\put(6,0){\circle*{.2}}
\end{picture}

&

\setlength{\unitlength}{0.3cm}
\begin{picture}(13,12)(0,0)

\put(0,0.2){\makebox(0,0)[b]{\footnotesize $5^{(0)}$}}
\put(0,0){\circle*{.2}}

\put(2,0.2){\makebox(0,0)[b]{\footnotesize $32^{(0)}$}}
\put(2,0){\circle*{.2}}

\put(4,0.2){\makebox(0,0)[b]{\footnotesize $50^{(0)}$}}
\put(4,0){\circle*{.2}}

\put(14,0.2){\makebox(0,0)[b]{\footnotesize $90^{(0)}$}}
\put(14,0){\circle*{.2}}

\put(1,4.2){\makebox(0,0)[b]{\footnotesize $15^{(0)}$}}
\put(1,4){\circle*{.2}}
\put(1,4){\line(-1,-4){1}}
\put(1,4){\line(1,-4){1}}

\put(5,4.2){\makebox(0,0)[b]{\footnotesize $54^{(-1)}$}}
\put(5,4){\circle*{.2}}
\put(5,4){\line(-1,-4){1}}

\put(9,4.2){\makebox(0,0)[b]{\footnotesize $61^{(0)}$}}
\put(9,4){\circle*{.2}}

\put(13,4.2){\makebox(0,0)[b]{\footnotesize $88^{(1)}$}}
\put(13,4){\circle*{.2}}
\put(13,4){\line(1,-4){1}}

\put(3,6.2){\makebox(0,0)[b]{\footnotesize $40^{(0)}$}}
\put(3,6){\circle*{.2}}
\put(3,6){\line(-1,-1){2}}
\put(3,6){\line(1,-1){2}}

\put(11,6.2){\makebox(0,0)[b]{\footnotesize $68^{(1)}$}}
\put(11,6){\circle*{.2}}
\put(11,6){\line(-1,-1){2}}
\put(11,6){\line(1,-1){2}}

\put(7,7.2){\makebox(0,0)[b]{\footnotesize $60^{(0)}$}}
\put(7,7){\circle*{.2}}
\put(7,7){\line(-4,-1){4}}
\put(7,7){\line(4,-1){4}}

\end{picture}

\\

& 

\\

rotação dupla a direita de 32

&

rotação dupla a esquerda de 40
\end{tabular}
\end{center}

\end{frame}

\begin{frame}

\frametitle{Algoritmo de remoção}
\framesubtitle{Principais etapas}

Entradas: a raiz $n$ de uma árvore AVL, um valor $v$

\begin{enumerate}
\item encontrar o valor $v$ em $n$
\item se a busca for mal sucedida: término
\item remover o nó $m$ com o valor $v$
\begin{itemize}
\item nó folha: remover nó $m$
\item nó interno com uma sub-árvore vazia: remover nó $m$
\item nó interno sem sub-árvore vazia: remover nó com valor mínimo $v'$ da sub-árvore direita, substituir $v$ por $v'$ em $m$.
\end{itemize}
\item possivelmente reequilibrar a árvore
\end{enumerate}
\pause
algoritmos auxiliares:
\begin{itemize}
\item $\proc{Remove-Min}$
\item $\proc{Balance-Right}$, $\proc{Balance-Left}$: rebalanceamento da sub-árvore direita, esquerda
\end{itemize}
\end{frame}

\begin{frame}
\frametitle{Remoção}

\begin{small}
\only<1>{
\begin{tabular}{rcl}
$n$ & : & raiz da sub-árvore cujo valor mínimo deve ser removido (\alert{ref}) \\
$v$ & : & valor a remover \\
& $\rightarrow$ & Booleano indicando se a altura diminuiu
\end{tabular}
}
\only<2>{
\begin{codebox}
\Procname{$\proc{Remove}(n, v)$}
\li \If $n \isequal \const{Nil}$
\li \Then \Return $\const{False}$
    \End
\li \If $v < \attrib{n}{val}$
\li \Then
      \Return $\proc{Balance-Left}(n, \proc{Remove}(\attrib{n}{left}))$
    \End
\li \If $v > \attrib{n}{val}$
\li \Then
      \Return $\proc{Balance-Right}(n, \proc{Remove}(\attrib{n}{right}))$
    \End
\li \If $\attrib{n}{left} \isequal \const{Nil}$
\li \Then
      $\id{tmp} \gets n$
\li   $n \gets \attrib{n}{right}$
\li   $\proc{Free-AVL-Node}(\id{tmp})$
\li   \Return $\const{True}$
    \End
\li \If $\attrib{n}{right} \isequal \const{Nil}$
\li \Then
      $\id{tmp} \gets n$
\li   $n \gets \attrib{n}{left}$
\li   $\proc{Free-AVL-Node}(tmp)$
\li   \Return $\const{True}$
    \End
\li  \Return $\proc{Balance-Right}(n, \proc{Remove-Min}(\attrib{n}{right}, \attrib{n}{val}))$
\end{codebox}
}
\end{small}

\end{frame}

\begin{frame}
\frametitle{Remoção do nó com valor mínimo}

\begin{tabular}{rcl}
$n$ & : & raiz da sub-árvore cujo valor mínimo deve ser removido (ref) \\
$v$ & : & variável onde o valor mínimo será armazenado (ref) \\
& $\rightarrow$ & Booleano indicando se a altura diminuiu
\end{tabular}

\begin{codebox}
\Procname{$\proc{Remove-Min}(n, v)$}
\li \If $\attrib{n}{left} \isequal \const{Nil}$
\li \Then
      $v \gets \attrib{n}{val}$
\li   $\id{tmp} \gets n$
\li   $n \gets \attrib{n}{right}$
\li   $\proc{Free-AVL-Node}(tmp)$
\li   \Return $\const{True}$
\li \Else
\li   \Return $\proc{Balance-Left}(n, \proc{Remove-Min}(\attrib{n}{left}, v))$
    \End  
\end{codebox}

\end{frame}

\begin{frame}
\frametitle{Rebalanceamento da sub-árvore direita}

\begin{small}
\begin{tabular}{rcl}
$n$ & : & a sub-árvore direita de $n$ sofreu remoção (ref) \\
$dec$ & : & indica se a altura diminuiu na remoção na sub-árvore \\
& $\rightarrow$ & indica se a altura diminuiu na remoção em $n$
\end{tabular}

\begin{codebox}
\Procname{$\proc{Balance-Right}(n, \id{dec})$}
\li \If not $\id{dec}$
\li \Then \Return $\const{False}$
    \End
\li \If $\attrib{n}{bal} \isequal -1$ \> \> \> \> \> \Comment desequilibro em $n$
\li \Then
      \If $\attrib{\attrib{n}{left}}{bal} \isequal -1$ or
          $\attrib{\attrib{n}{left}}{bal} \isequal 0$
\li   \Then
        $n \gets \proc{Rotate-Simple-Right}(n)$
\li   \ElseNoIf 
        $n \gets \proc{Rotate-Double-Right}(n)$
      \End
\li   \Return $\const{True}$
\li \ElseIf $\attrib{n}{bal} \isequal 0$
\li \Then
      $\attrib{n}{bal} \gets -1$
\li   \Return $\const{False}$
\li \ElseNoIf
      $\attrib{n}{bal} \gets 0$
\li   \Return $\const{True}$
    \End
\end{codebox}
\end{small}

\end{frame}

\begin{frame}
\frametitle{Exercícios}

\begin{enumerate}
\item Escrever o algoritmo $\proc{Balance-Left}$
\item Aplicar repetidamente o algoritmo de remoção à árvore inicial do exemplo
  até ter removido todos os valores da árvore:
  \begin{itemize}
    \item em ordem decrescente de valor
    \item em ordem crescente de valor
    \item escolhendo sempre o valor mediano da coleção representada
  \end{itemize}
\end{enumerate}
\end{frame}
\end{document}



