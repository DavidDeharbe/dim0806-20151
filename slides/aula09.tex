\documentclass{beamer}
\setbeamertemplate{footline}[frame number]
%\documentclass{beamer}

\usepackage[utf8x]{inputenc}
\usepackage[brazil,british]{babel}
\usetheme{default} 
\usecolortheme{beaver}
\newtranslation[to=brazil]{Theorem}{Teorema}
\newtranslation[to=brazil]{Definition}{Definição}
\newtranslation[to=brazil]{Example}{Exemplo}
\newtranslation[to=brazil]{Problem}{Exercício}
\newtranslation[to=brazil]{Solution}{Resolução}
\newtranslation[to=brazil]{Lemma}{Lema}
\newtranslation[to=brazil]{Corollary}{Corolário}
\logo{\includegraphics[width=1cm]{../img/logo-ppgsc-icon-text.png}}

\usepackage{graphicx}
\usepackage{clrscode3e}
\usepackage{hyperref}

\usepackage{pgf}
\usepackage{tikz}
\newcommand{\assert}[1]{\textcolor{blue}{#1}}

\usepackage{xspace}

\newcommand{\semester}[0]{2015.2}



\title{Aula 09: Algoritmos de ordenação em arranjos}
\subtitle{Ordenação por fusão}
\author{David Déharbe \\
  Programa de Pós-graduação em Sistemas e Computação \\
  Universidade Federal do Rio Grande do Norte \\
  Centro de Ciências Exatas e da Terra \\
  Departamento de Informática e Matemáica Aplicada}
\date{16 de março de 2015}


\begin{document}
\selectlanguage{brazil}
\begin{frame}
  \titlepage
  Download me from \url{http://ufrn.academia.edu/DavidDeharbe}.
\end{frame}

\begin{frame}
  \frametitle{Plano da aula}
  \tableofcontents
\end{frame}

\section{Fusão de faixas contíguas ordenadas}

\subsection{Definição}

\begin{frame}

  \frametitle{Ordenação por fusão \textit{merge sort}}
  \framesubtitle{Preâmbulo}

\begin{itemize}

\item Escreva um algoritmo $\proc{Merge}(A, l, m, u)$ tal que
\begin{itemize}
\item $A$ é um arranjo,
\item $l$, $m$, e $u$ são três posições do arranjo, tais que $l \le m \le u$,
\item as sub-faixas $l \twodots m$ e $m+1 \twodots u$ são ordenadas.
\end{itemize}
\item O resultado do algoritmo é que 
\begin{itemize}
\item a sub-faixa $l \twodots u$ de $A$ contenha os elementos inicialmente 
  nas sub-faixas $l \twodots m$ e $m+1 \twodots u$;
\item a sub-faixa $l \twodots u$ de $A$ esteja ordenada;
\item as sub-faixas $1 \twodots l-1$ e $u+1 \twodots n$ permanecem inalteradas.
\end{itemize}
\item Restrição: o algoritmo deve ter complexidade linear.

\end{itemize}

\end{frame}

\subsection{Algoritmo}

\begin{frame}

  \frametitle{Fusão de faixas ordenadas}
  \framesubtitle{Um algoritmo}

\begin{codebox}
\Procname{$\proc{Merge}(A, l, m, u)$}
\li $i \gets 1, j \gets l, k \gets m+1$
\li \While $j \le m$ and $k \le u$ \Do
\li \If $A[j] \le A[k]$
\li   \Then $B[i] \gets A[j], j \gets j+1, i \gets i+1$
\li   \Else $B[i] \gets A[k], k \gets k+1, i \gets i+1$
      \End
    \End
\li \While $j \le m$
\li \Do $B[i] \gets A[j], j \gets j+1, i \gets i+1$
    \End
\li \While $k \le u$
\li \Do $B[i] \gets A[k], k \gets k+1, i \gets i+1$
    \End
\li \For $i \gets l$ \To $u$
\li \Do $A[i] \gets B[i-l+1]$
    \End
\end{codebox}  

\end{frame}

\subsection{Simulação}

\begin{frame}

  \frametitle{Fusão de faixas ordenadas}
  \framesubtitle{Simulação}

\begin{small}
\begin{codebox}
\Procname{$\proc{Merge}(A, l, m, u)$}
\li $i \gets 1, j \gets l, k \gets m+1$
\li \While $j \le m$ and $k \le u$ \Do
\li   \If $A[j] \le A[k]$
\li   \Then $B[i] \gets A[j], j \gets j+1, i \gets i+1$
\li   \Else $B[i] \gets A[k], k \gets k+1, i \gets i+1$
      \End
    \End
\li \While $j \le m$
\li \Do $B[i] \gets A[j], j \gets j+1, i \gets i+1$
    \End
\li \While $k \le u$
\li \Do $B[i] \gets A[k], k \gets k+1, i \gets i+1$
    \End
\li \For $i \gets l$ \To $u$
\li \Do $A[i] \gets B[i-l+1]$
    \End
\end{codebox}  
\end{small}

\alert{$$A = \langle 2, 4, 5, 5, 1, 2, 3, 5 \rangle, l = 1, m = 4, u = 8$$}

\end{frame}

\subsection{Complexidade}

\begin{frame}

  \frametitle{Fusão de faixas ordenadas}
  \framesubtitle{Complexidade}

  \begin{itemize}
  \item Cada um dos três primeiros laços ou incrementa $j$, ou incrementa $k$.
  \item Há, exatamente, $m-l$ incrementos de $j$ e $u-(m+1)$ incrementos de $k$
  \item O custo acumulado dos três primeiros laços é $\Theta(m-l+u-(m+1))=
    \Theta(u-l)$
  \item O custo do quarto laço é $\Theta(u-l)$
  \item Complexidade: $\Theta(u-l)$.
  \end{itemize}

\end{frame}

\subsection{Correção}


\begin{frame}

  \frametitle{Fusão de faixas ordenadas}
  \framesubtitle{Correção}

\begin{definition}[$\id{sorted}$]
$$
  \id{sorted}(A, i, j) \equiv \forall k | i \le k < j \cdot A[k] \le A[k+1]
$$
\end{definition}

\pause

\begin{small}
\begin{codebox}
\Procname{$\proc{Merge}(A, l, m, u)$}
\end{codebox}
\end{small}

  \begin{itemize}
    \item pré-condição:
      \assert{$\begin{array}[t]{l}
      A = \langle a_1 , \ldots , a_n \rangle \land 1 \le l, m, u \le n \land \\
      \id{sorted}(A, l, m) \land \id{sorted}(A, m+1, u)
      \end{array}$}

    \item pós-condição
      \assert{$\begin{array}[t]{l}
      A[1..l-1] = \langle a_1, \ldots, a_{l-1} \rangle \land
      A[u+1..n] = \langle a_{u+1}, \ldots, a_n \rangle \\
      A[l..u] \in \id{permutation}(\langle a_l, \ldots, a_u \rangle) \land \id{sorted}(A, l, u)
      \end{array}$}

  \end{itemize}

\end{frame}

\section{Ordenação por fusão}

\subsection{Algoritmo}

\begin{frame}

  \frametitle{Ordenação por fusão \textit{merge sort}}
  \framesubtitle{Algoritmo}

(Cormen et al. 1990)

\begin{codebox}
\Procname{$\proc{Merge-Sort}(A, l, u)$}
\li \If $l < u$
\li \Then $m \gets l + (u - l) / 2$
\li    $\proc{Merge-Sort}(A, l, m)$
\li    $\proc{Merge-Sort}(A, m+1, u)$
\li    $\proc{Merge}(A, l, m, u)$
\zi \End
\end{codebox}  

$\Longrightarrow \proc{Merge-Sort}(A, 1, \id{length}(A))$

\end{frame}

\subsection{Simulação}

\begin{frame}

  \frametitle{Ordenação por fusão}
  \framesubtitle{Simulação}

\begin{codebox}
\Procname{$\proc{Merge-Sort}(A, l, u)$}
\li \If $l < u$
\li \Then    $m \gets l + (u - l) / 2$
\li    $\proc{Merge-Sort}(A, l, m)$
\li    $\proc{Merge-Sort}(A, m+1, u)$
\li    $\proc{Merge}(A, l, m, u)$
\zi \End
\end{codebox}  

$\Longrightarrow \proc{Merge-Sort}(A, 1, \id{length}(A))$

\alert{$$A = \langle 5, 2, 4, 6, 1, 3, 2, 6 \rangle$$}

\end{frame}

\subsection{Complexidade}

\begin{frame}

  \frametitle{Ordenação por fusão}
  \framesubtitle{Complexidade}

  \begin{itemize}
  \item $T(n) = 2 \times T(n / 2) + \Theta(n)$ \pause
  \item Teorema master \pause
  \item $T(n) = a T(\frac{n}{b}) + f(n)$, onde $a \ge 1, b \ge 1$;
  \item Se existe $k \ge 0$ tal que $f(n) \in \Theta(n^{c} \log^{k} n)$ onde $c = \log_b a$ então $T(n) \in \Theta(n^{c} \log^{k+1} n)$. \pause
  \item $c = \log_2 2 = 1$ \pause
  \item $n = n^1 \times \log^0 n$, logo $k = 0$ satisfaz a condição.\pause
  \item Então $T(n) \in \Theta(n \log n)$.
  \end{itemize}

\end{frame}

\subsection{Complexidade}

\begin{frame}

  \frametitle{Ordenação por fusão}
  \framesubtitle{Correção}

\begin{codebox}
\Procname{$\proc{Merge-Sort}(A, l, u)$}
\li \If $l < u$
\li \Then $m \gets l + (u - l) / 2$
\li    $\proc{Merge-Sort}(A, l, m)$
\li    $\proc{Merge-Sort}(A, m+1, u)$
\li    $\proc{Merge}(A, l, m, u)$
\zi \End
\end{codebox}  

    \begin{itemize}
    \item pré-condição
      \only<2->{
        \assert{$A = \langle a_1, \ldots a_n \rangle \land 1 \le l, u \le n$}
      }
    \item pós-condição
      \only<3->{
        \assert{$\begin{array}[t]{l}
          A[1 \twodots l-1] = \langle a_1, \ldots a_{l-1} \rangle \land A[u+1 \twodots n] = \langle a_{u+1}, \ldots a_{n} \rangle \land \\
          A[l \twodots u] \in \id{permutation}(\langle a_l, \ldots, a_u \rangle) \land
          \id{sorted}(A, l, u)
        \end{array}$}
      }
    \end{itemize}

\end{frame}

\subsection{Verificação}

\begin{frame}

  \frametitle{Ordenação por fusão \textit{merge sort}}
  \framesubtitle{Verificação}

\only<2>{\assert{$\alert{\Gamma_1:} \id{Pre}(\proc{Merge-Sort}(A, l, u)) \land l \ge u \Rightarrow \id{Pos}(\proc{Merge-Sort}(A, l, u)$}}


\begin{small}
\begin{codebox}
\Procname{$\proc{Merge-Sort}(A, l, u)$}
\li \If $l < u$
\li \Then $m \gets l + (u - l) / 2$

\only<3->{\zi \assert{$A_0$}}
\only<4>{\zi \assert{$\alert{\Gamma_2:} 
\begin{array}[t]{rl}
& l < u \land m = l + \frac{u-l}{2} \land \\
& \id{Pre}(\proc{Merge-Sort}(A_0, l, u)) \\
\Rightarrow & \id{Pre}(\proc{Merge-Sort}(A_0, l, m))
\end{array}$}}

\li    $\proc{Merge-Sort}(A, l, m)$

\only<3->{\zi \assert{$A_1$}}

\only<5>{\zi \assert{$\alert{\Gamma_3:} 
\begin{array}[t]{rl}
& l < u \land m = l + \frac{u-l}{2} \land \\
& \id{Pre}(\proc{Merge-Sort}(A_0, l, u)) \land \\
& \id{Pos}(\proc{Merge-Sort}(A_1, l, m)) \\
\Rightarrow & \id{Pre}(\proc{Merge-Sort}(A_1, m+1, u))
\end{array}$}}

\li    $\proc{Merge-Sort}(A, m+1, u)$

\only<3->{\zi \assert{$A_2$}}

\only<6>{\zi \assert{$\alert{\Gamma_4:} 
\begin{array}[t]{rl}
& l < u \land m = l + \frac{u-l}{2} \land \\
& \id{Pre}(\proc{Merge-Sort}(A_0, l, u)) \land \\
& \id{Pos}(\proc{Merge-Sort}(A_1, l, m)) \land \\
& \id{Pos}(\proc{Merge-Sort}(A_2, m+1, u)) \\
\Rightarrow & \id{Pre}(\proc{Merge}(A_2, l, m, u))
\end{array}$}}

\li    $\proc{Merge}(A, l, m, u)$

\only<3->{\zi \assert{$A_3$}}

\only<7>{\zi \assert{$\alert{\Gamma_5:} 
\begin{array}[t]{rl}
& l < u \land m = l + \frac{u-l}{2} \land \\
& \id{Pre}(\proc{Merge-Sort}(A_0, l, u)) \land \\
& \id{Pos}(\proc{Merge-Sort}(A_1, l, m)) \land \\
& \id{Pos}(\proc{Merge-Sort}(A_2, m+1, u)) \land \\
& \id{Pos}(\proc{Merge}(A_3, l, m, u)) \\
\Rightarrow & \id{Pos}(\proc{Merge-Sort}(A_3, l, u))
\end{array}$}}

\zi \End
\end{codebox}  
\end{small}

\end{frame}

\begin{frame}

\only<1-2>{
$$
\begin{array}[t]{lrl}
\alert{\Gamma_1:} 
&      & A = \langle a_1, \ldots a_n \rangle \land 1 \le l, u \le n  \\
& \land & l \ge u \\
& \Rightarrow & A[1 \twodots l-1] = \langle a_1, \ldots a_{l-1} \rangle \\
& \land & A[u+1 \twodots n] = \langle a_{u+1}, \ldots a_{n} \rangle \land \\
& \land & A[l \twodots u] \in \id{permutation}(\langle a_l, \ldots, a_u \rangle) \\
& \land & \id{sorted}(A, l, u)
\end{array}
$$
}

\only<2>{$H \Rightarrow P \land Q \equiv H \Rightarrow P, H \Rightarrow Q$}

\only<3-8>{
$$
\begin{array}[t]{lrl}
\only<3-4>{\alert{\Gamma_{1,1}:} 
& & \alert<4>{A = \langle a_1, \ldots a_n \rangle} \land 1 \le l, u \le n  \\
& \land & l \ge u \\
& \Rightarrow & \alert<4>{A[1 \twodots l-1] = \langle a_1, \ldots a_{l-1} \rangle}
\\
\hline}
\only<3-5>{\alert{\Gamma_{1,2}:} 
& & \alert<5>{A = \langle a_1, \ldots a_n \rangle} \land 1 \le l, u \le n  \\
& \land & l \ge u \\
& \Rightarrow & \alert<5>{A[u+1 \twodots n] = \langle a_{u+1}, \ldots a_{n} \rangle}
\\
\hline}
\alert{\Gamma_{1,3}:} 
& & A = \langle a_1, \ldots a_n \rangle \land 1 \le l, u \le n  \\
& \land & l \ge u \\
& \Rightarrow & A[l \twodots u] \in \id{permutation}(\langle a_l, \ldots, a_u \rangle)
\\
\hline
\alert{\Gamma_{1,4}:} 
& & A = \langle a_1, \ldots a_n \rangle \land 1 \le l, u \le n  \\
& \land & l \ge u \\
& \Rightarrow & \id{sorted}(A, l, u)
\end{array}
$$
}
\only<7-8>{$l \ge u \Leftrightarrow l = u \lor l > u$}
\only<8>{$H \land (P_1 \lor P_2) \Rightarrow Q \equiv H \land P_1 \Rightarrow Q, H \land P_2 \Rightarrow Q$}

\only<9->{
$$
\begin{array}[t]{lrl}
\only<9-16>{\alert{\Gamma_{1,3,1}:} 
& & \alert<12>{A = \langle a_1, \ldots a_n \rangle} \land 1 \le l, u \le n  \\
& \land & \alert<10>{l = u} \\
& \Rightarrow & \only<9-10>{A[l \twodots u] \in \id{permutation}(\langle a_l, \ldots, a_u \rangle)}
\only<11-12>{A[l \twodots l] \in \id{permutation}(\langle a_l \rangle)}
\only<13-14>{\alert<14>{\langle a_1, \ldots a_n \rangle [l \twodots l]} \in \id{permutation}(\langle a_l \rangle)}
\only<15>{\langle a_l \rangle \in \id{permutation}(\langle a_l \rangle)}
\only<16>{\id{true}}
\\
\hline}
\only<9-19>{\alert{\Gamma_{1,3,2}:} 
& & \alert<17>{A = \langle a_1, \ldots a_n} \rangle \land 1 \le l, u \le n  \\
& \land & \alert<17>{l > u} \\
& \Rightarrow & \alert<17>{\only<9-17>{A[l \twodots u] \in \id{permutation}(\langle a_l, \ldots, a_u \rangle)}}
\only<18>{\langle \rangle \in \id{permutation}(\langle \rangle)}
\only<19>{\id{true}}
\\
\hline}
\alert{\Gamma_{1,4,1}:} 
& & A = \langle a_1, \ldots a_n \rangle \land 1 \le l, u \le n  \\
& \land & l = u \\
& \Rightarrow & \id{sorted}(A, l, u)
\\
\hline
\alert{\Gamma_{1,4,2}:} 
& & A = \langle a_1, \ldots a_n \rangle \land 1 \le l, u \le n  \\
& \land & l > u \\
& \Rightarrow & \id{sorted}(A, l, u)
\end{array}
$$
}
\only<20>{\alert{idem}}
\end{frame}

\begin{frame}

  \frametitle{Exercício}

  \begin{enumerate}
  \item A ordenação por fusão lança mão de um arranjo auxiliar $B$ na
    operação de fusão, em cada nível da recursão.
    \begin{enumerate}
      
    \item Determine a complexidade espacial do algoritmo $\proc{Merge-Sort}$.
    
    \item Descreve as adaptações necessárias para ter um algoritmo
      similar com complexidade espacial \emph{linear}.

    \end{enumerate}

  \item Escreva por extenso as obrigações de prova $\Gamma_2$, 
    $\Gamma_3$, $\Gamma_4$, $\Gamma_5$.

    Verifique se elas são passíveis de serem provadas válidas.
  \end{enumerate}
\end{frame}

\end{document}
