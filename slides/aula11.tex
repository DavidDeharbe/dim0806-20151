%%
%% TODO 
%% # Recursion trees
%%  - best case
%%  - constant proportion (9/1)
%%  - alternating worst/best
%% # Algorithm simulation
%%  - Division
%%
%% IMPROVE
%% - define loop invariant in division
%% - improve simulation of quick-sort
%%

\documentclass{beamer}
\setbeamertemplate{footline}[frame number]
%\documentclass{beamer}

\usepackage[utf8x]{inputenc}
\usepackage[brazil,british]{babel}
\usetheme{default} 
\usecolortheme{beaver}
\newtranslation[to=brazil]{Theorem}{Teorema}
\newtranslation[to=brazil]{Definition}{Definição}
\newtranslation[to=brazil]{Example}{Exemplo}
\newtranslation[to=brazil]{Problem}{Exercício}
\newtranslation[to=brazil]{Solution}{Resolução}
\newtranslation[to=brazil]{Lemma}{Lema}
\newtranslation[to=brazil]{Corollary}{Corolário}
\logo{\includegraphics[width=1cm]{../img/logo-ppgsc-icon-text.png}}

\usepackage{graphicx}
\usepackage{clrscode3e}
\usepackage{hyperref}

\usepackage{pgf}
\usepackage{tikz}
\newcommand{\assert}[1]{\textcolor{blue}{#1}}

\usepackage{xspace}

\newcommand{\semester}[0]{2015.2}


\usepackage{tikz}
\usetikzlibrary{arrows,positioning, calc} 
\tikzstyle{vertex}=[draw,fill=black!15,circle,minimum size=18pt,inner sep=0pt]

\title{Aula 11: Algoritmos de ordenação em arranjos}
\subtitle{O algoritmo \textit{Quick-Sort\/}}
\author{David Déharbe \\
  Programa de Pós-graduação em Sistemas e Computação \\
  Universidade Federal do Rio Grande do Norte \\
  Centro de Ciências Exatas e da Terra \\
  Departamento de Informática e Matemática Aplicada}
\date{23 de março de 2015}


\begin{document}
\selectlanguage{brazil}
\begin{frame}
  \titlepage
  Download me from \url{http://ufrn.academia.edu/DavidDeharbe}.
\end{frame}

\begin{frame}
  \frametitle{Plano da aula}
  \tableofcontents
\end{frame}

\section{Introdução}

\begin{frame}
  \frametitle{Introdução}

  Ref: Cormen et al. Seção 4.1 (método da substituição), Capítulo 8 (\textit{quicksort}.

  \begin{itemize}
    \item Sem arranjo auxiliar (como ordenação por inserção);
    \item Baseado no conceito de divisão e conquista.
      \begin{itemize}
        \item Processamento é realizado \emph{antes\/} da divisão.
      \end{itemize}
    \item No pior caso, ordenação em $\Theta(n^2)$;
    \item Em média, ordenação em $\Theta(n \lg n)$;
    \item Provavelmente o mais utilizado nas aplicações.
      \pause
    \item C.A.R. Hoare, 1960.
  \end{itemize}

\end{frame}

\section{O algoritmo}

\begin{frame}

  \frametitle{O algoritmo}
  \framesubtitle{Princípio}

  Seja $A$ um arranjo, para ordenar a faixa $A[l \twodots u]$ ($l < u$):
  \begin{enumerate}
    \item Escolhe um valor na faixa: $A[l]$ (\alert{pivô\/}).
    \item Remaneje os valores da faixa em duas sub-faixas $A[l \twodots m-1]$
      e $A[m \twodots u]$ tais que
      \begin{itemize}
        \item todos os valores na faixa $A[l \twodots m-1]$ são menores que $A[l]$;
        \item todos os valores na faixa $A[m \twodots u]$ são maiores ou iguais a
          $A[l]$;
        \item o índice $m$ é um dos resultados do remanejamento.
        \item caso particular: $A[l]$ é o menor valor da faixa, e neste caso
          $m=l+1$.
      \end{itemize}
    \item Aplique recursivamente o mesmo processamento às duas sub-faixas
      $A[l \twodots m-1]$ e $A[m \twodots u]$.
  \end{enumerate}

\end{frame}

\begin{frame}

  \frametitle{O algoritmo \textit{Quick-Sort}}
  \framesubtitle{Simulação (chamadas recursivas com faixa de um elemento são omitidas)}

  $$
  \begin{array}{cccc}
    A & l & u & m \\
    \langle \assert{\alert{37}, 41, 21, 14, 18, 25, 50, 11} \rangle & 1 & 8 & \\
    \only<2->{\langle 11, 21, 14, 18, 25, 50, 41, 37 \rangle & & & 6 } \\
    \only<3->{\langle \assert{\alert{11}, 21, 14, 18, 25}, 50, 41, 37 \rangle & 1 & 5 & } \\
    \only<3->{\langle 11, 21, 14, 18, 25, 50, 41, 37 \rangle & & & 2 } \\
    \only<4->{\langle 11, \assert{\alert{21}, 14, 18, 25}, 50, 41, 37 \rangle & 2 & 5 & } \\
    \only<5->{\langle 11, 18, 14, 25, 21, 50, 41, 37 \rangle & & & 4} \\
    \only<6->{\langle 11, \assert{\alert{18}, 14}, 25, 21, 50, 41, 37 \rangle & 2 & 3 & } \\
    \only<7->{\langle 11, 14, 18, \assert{\alert{25}, 21}, 50, 41, 37 \rangle & 4 & 5 & } \\
    \only<8->{\langle 11, 14, 18, 21, 25, \assert{\alert{50}, 41, 37} \rangle & 6 & 8 & } \\
    \only<9->{\langle 11, 14, 18, 21, 25, 37, 41, 50 \rangle & & & 8 } \\
    \only<11->{\langle 11, 14, 18, 21, 25, \assert{\alert{37}, 41}, 50 \rangle & 6 & 7 & } \\
    \only<12->{\langle 11, 14, 18, 21, 25, 37, 41, 50 \rangle &  &  & 7 }
  \end{array}
  $$

\end{frame}

\begin{frame}

  \frametitle{O algoritmo \textit{Quick-Sort}}

\begin{codebox}
\Procname{$\proc{Quick-Sort}(A, l, u)$}
\li \If $l < u$
\li \Then $m \gets \proc{Division}(A, l, u)$
\li   $\proc{Quick-Sort}(A, l, m-1)$
\li   $\proc{Quick-Sort}(A, m, u)$
    \End
\end{codebox}

\end{frame}

\begin{frame}

  \frametitle{O algoritmo \textit{Quick-Sort}}
  \framesubtitle{O algoritmo de divisão}

  \begin{itemize}
  \item Dê uma especificação ao algoritmo de divisão (remanejamento dos
    elementos da faixa).
  \end{itemize}
\end{frame}

\begin{frame}

  \frametitle{O algoritmo \textit{Quick-Sort}}
  \framesubtitle{O algoritmo de divisão --- especificação}

  \begin{itemize}
  \item Pré-condição
    \assert{$A = \langle a_1, \ldots, a_n \rangle \land 1 \le l < u \le n$}
    \pause
  \item Pós-condição
    \assert{$\begin{array}[t]{rl}
        & A[1 \twodots l-1] = \langle a_1, \ldots a_{l-1} \rangle \\
        \land & A[u+1 \twodots n] = \langle a_{u+1}, \ldots a_n \rangle \\
        \land & A[l \twodots u] \in \mathit{permutation}\langle a_l, \ldots, a_u \rangle \\
        \land & l < m \le u \\
        \land & \forall i, j | l \le i < m \le j \le u \cdot A[i] \le A[j]
      \end{array}$}
  \end{itemize}

\end{frame}

\begin{frame}

  \frametitle{O algoritmo \textit{Quick-Sort}}
  \framesubtitle{O algoritmo de divisão}

  \begin{itemize}
  \item Dê uma definição do algoritmo de divisão.
  \end{itemize}
\end{frame}

\begin{frame}

  \frametitle{O algoritmo \textit{Quick-Sort}}
  \framesubtitle{O algoritmo de divisão}

\begin{minipage}[l]{.45\textwidth}
\begin{codebox}
\Procname{$\proc{Division}(A, l, u)$}
\li $\id{pivot} \gets \alert<4>{A[l]}$
\li $i \gets l-1$
\li $j \gets u+1$
\li \While $\const{True}$
\li   \Do \Repeat
\li          $j \gets j-1$
\li       \Until $\alert<4>{A[j]} \le \id{pivot}$
\li       \Repeat 
\li          $i \gets i+1$
\li       \Until $\alert<4>{A[i]} \ge \id{pivot}$
\li       \If $i < j$
\li       \Then \alert<4>{$\proc{Swap}(A, i, j)$}
\li       \Else \Return \alert<4>{$i$}
          \End
      \End
\end{codebox}
\end{minipage}
\hfill
\only<2>{\begin{minipage}[l]{.5\textwidth}
Simulação 
$$A[l..u] = \langle 5, 3, 2, 6, 4, 1, 3, 7 \rangle$$
\end{minipage}}
\only<3-6>{
\begin{minipage}[l]{.5\textwidth}
  Condições de correção \only<3>{\alert{???}}\only<4->{:
  \begin{enumerate}
  \item $A$ sempre é acessado em uma posição válida
    \only<5->{
    \item $j \neq u$ quando termina o algoritmo
    }
    \only<6>{
    \item cada elemento de $A[l \twodots j]$ é menor que cada elemento de $A[j+1 \twodots u]$ quando termina o algoritmo
    }
  \end{enumerate}}
\end{minipage}
}
\only<7-8>{
\begin{minipage}[l]{.5\textwidth}
  \begin{itemize}
  \item Qual é a complexidade do algoritmo de divisão?
    \only<8>{
    \begin{itemize}
    \item Observe que $i$ nunca ultrapassa $u$ e $j$ nunca fica menor que $l$.
    \item O número de incrementos e decrementos pertence a \alert{$\Theta(u-l)$}.
    \end{itemize}
    }
  \end{itemize}
\end{minipage}
}

\end{frame}

\section{Análise}

\begin{frame}

  \frametitle{O algoritmo \textit{Quick-Sort}}
  \framesubtitle{Alguns questionamentos}

  \begin{itemize}
  \item Adaptar o algoritmo para tratar elementos repetidos
    \only<2>{
    \item Dividir a faixa em três sub-faixas: 
      \begin{enumerate}
      \item elementos menores que o pivô, 
      \item elementos iguais ao pivô, 
      \item elementos maiores que o pivô.
      \end{enumerate}
    }
  \end{itemize}

\end{frame}

\begin{frame}

  \frametitle{O algoritmo \textit{Quick-Sort}}
  \framesubtitle{Alguns questionamentos}

  \begin{itemize}
  \item Qual arranjo inicial corresponde ao pior caso deste algoritmo?
    \only<2>{
\begin{minipage}[c]{\textwidth}
      \begin{minipage}[l]{.7\textwidth}
        \includegraphics[width=\textwidth]{fig/quick-sort-worst-case-recursion-tree.pdf}
      \end{minipage}
      \hfill
      \begin{minipage}[l]{.25\textwidth}
      $\Longrightarrow$ Quando o arranjo é inicialmente ordenado.
      \end{minipage}
      \end{minipage}
    }
    \only<3->{
    \item Como modificar o algoritmo para evitar isto?
      \only<4>{
        
      $\Longrightarrow$ Escolher outro elemento que o primeiro.
    }
  }
  \end{itemize}
\end{frame}

\begin{frame}

  \frametitle{Análise de complexidade}
  \framesubtitle{Melhor caso: divisão perfeita}

  \begin{itemize}
    \item No melhor caso, cada divisão de uma faixa de $n$ elementos
      produz duas sub-faixas de tamanho $n/2$.
    \item O custo é $T(n) = 2.T(n/2) + \Theta(n)$.
    \item Teorema mestre (caso 2): $T(n) = \Theta(n \lg n)$.
  \end{itemize}
\end{frame}

\begin{frame}

  \frametitle{Análise de complexidade}
  \framesubtitle{Divisão proporcional}

  \begin{itemize}

    \item Objetivo: argumentar que o caso médio se aproxima do melhor caso.

    \item Exemplo: imagine que a divisão produz uma repartição de 9 para 1.

    \item Qual a complexidade neste caso?

    \item $T(n) = T(9n/10) + T(n/10) + \Theta(n)$

    \item O custo de cada nível de recursão é $O(n)$.

    \item Há $\log_{10/9} n \in \Theta(\lg n)$ níveis.

    \item Logo, $T(n) \in O(n \lg n)$.

  \end{itemize}

\end{frame}

\begin{frame}

  \frametitle{Análise de complexidade}
  \framesubtitle{Comportamento médio}

  \begin{itemize}

    \item Objetivo: mostrar que o caso médio se aproxima do melhor caso.

    \item Exemplo: imagine que a divisão ora produz a pior repartição possível,
      ora produz a melhor repartição possível.

    \item Qual a complexidade neste caso?

    \item $T(n) = T(1) + T(n-1/2) + T(n-1/2) + \Theta(n) = 2T(n-1/2) + \Theta(n)$

    \item Logo, $T(n) \in O(n \lg n)$.

  \end{itemize}

\end{frame}

\section{Estratégias para evitar o pior caso}

\begin{frame}

  \frametitle{Estratégias para evitar o pior caso}
  \framesubtitle{Randomização}

  \begin{itemize}

    \item Assumimos a existência de uma sub-rotina $\proc{Random}(l, u)$ que
      retorna um número $k$, entre $l \le k \le u$ com probabilidade
      $1/(u-l+1)$, e de custo $\Theta(1)$.

    \item Solução 1: permutar aleatoriamente os elementos da faixa antes de
      fazer a divisão

      \pause Defina um algoritmo de complexidade linear que faz esta
      permutação

      \pause

    \item Solução 2: permutar o primeiro elemento da faixa com um elemento
      qualquer da faixa.

\begin{codebox}
\Procname{$\proc{Random-Division}(A, l, u)$}
\li $\proc{Swap}(A, l, \proc{Random}(l, u))$
\li $\proc{Division}(A, l, u)$
\end{codebox}

  \end{itemize}

\end{frame}

\begin{frame}

  \frametitle{Análise no caso médio}

  \begin{itemize}

    \item Análise informal: se a divisão separar uma proporção mínima de
      elementos,
      \begin{itemize}
        \item a profundidade da recursão é $\Theta(\lg n)$, e
        \item o custo em cada nível é $\Theta n$, logo
        \item o custo total é $\Theta(n \lg n)$.
      \end{itemize}

    \item Análise rigorosa da versão randomizada:

      \begin{itemize}

        \item análise da divisão

        \item estabelecer recorrência do tempo médio para ordenar um
          arranjo de $n$ elementos

        \item resolução da recorrência

      \end{itemize}

  \end{itemize}

\end{frame}

\begin{frame}

  \frametitle{Análise da divisão, caso inicial}

  \begin{itemize}

    \item Seja $n$ o tamanho da faixa.

    \item Seja $\id{Pos}(p)$ a posição do pivô na ordem dos elementos da faixa.

    \item Se $\id{Pos}(p) = 1$, então a primeira sub-faixa tem tamanho 1 (\alert{probabilidade: $1/n$}).

    \item Se $\id{Pos}(p) = i > 1$, então a primeira sub-faixa tem tamanho
      $1, 2, \ldots n-1$ (\alert{probabilidade: $1/n$}).

    \item Probabilidade da primeira sub-faixa ter tamanho 1: $2/n$.

    \item Probabilidade da primeira sub-faixa ter tamanho $1 < i < n$: $1/n$.

  \end{itemize}

\end{frame}

\begin{frame}

  \frametitle{Análise do algoritmo randomizado}

  \begin{itemize}

    \item Seja $T(n)$ o custo de ordenar uma faixa de tamanho $n$.

    \item Caso de base $T(1) \in \Theta(1)$.

    \item Em geral:

      \begin{small}
      \begin{eqnarray*}
      T(n) & = &
      \frac{1}{n}\left(\sum_{m=1}^{n-1} (T(m) + T(n-m)) \right) + \Theta(n) \\
      T(n) & = &
      \frac{2}{n}\left(\sum_{m=1}^{n-1} T(m) \right) + \Theta(n)
      \end{eqnarray*}
      \end{small}

    \item Resolução pelo \alert{método de substituição} ($\approx$ indução)

  \end{itemize}

\end{frame}

\begin{frame}

\frametitle{( O método de substituição )}
\framesubtitle{Princípios}
\begin{itemize}
  \item Dedução de limite superior de funções definidas por recorrência.
  \item Se aplica se temos uma ideia da fórmula fechada da recorrência:
    \alert{conjectura}.
  \item A conjectura vira \alert{teorema} se
    \begin{itemize}
    \item no(s) caso(s) de base, a fórmula é correta;
    \item no caso geral, substituindo os termos menores na relação, 
      pode-se provar que o termo maior satisfaz a fórmula fechada.
    \end{itemize}
\end{itemize}
\alert{Cormen, 4.1}
\end{frame}

\begin{frame}

\frametitle{( O método de substituição )}
\framesubtitle{Exemplo}
\begin{itemize}
\item $T(1) = 1$ e $T(n) = 2 T (\lfloor n/2 \rfloor) + n$, se $n>1$.
\item Conjectura: $T(n) \le c n \lg n$ para alguma constante $c > 0$.
\item Prova:
  \begin{itemize}
  \item Caso geral:
    $\begin{array}[t]{rcl}
    T(n) & = & 2 T (\lfloor n/2 \rfloor) + n \\
    & \le & 2 c \lfloor n/2 \rfloor \lg \lfloor n / 2 \rfloor + n \\
    & \le & c n \lg(n / 2) + n \\
    & = & c n \lg n - c n \lg 2 + n \\
    & = & c n \lg n - c n + n \\
    & \le & c n \lg n \mbox{ escolhendo $c > 1$}
    \end{array}$
    \pause
  \item Caso de base:
    Supondo $T(1) = 1$, então como $c \lg 1 = 0$, a prova \alert{fracassa}.
  \end{itemize}
\end{itemize}
\end{frame}

\begin{frame}

\frametitle{( O método de substituição )}
\framesubtitle{Exemplo: segunda tentativa}
\begin{itemize}
\item $T(1) = 1$ e $T(n) = 2 T (\lfloor n/2 \rfloor) + n$, se $n>1$.
\item Conjectura: $\forall n \ge n_0 \cdot T(n) \le c n \lg n$ para algumas constantes $c > 0$, 
  e $n_0 \ge 1$.
\end{itemize}

Prova:
\begin{itemize}
\item Caso geral: idem
  \pause
\item Caso de base: vamos escolher $c$ e $n_0$ para funcionar!
  \begin{itemize}
  \item $T(1) = 1$, então $T(2) = 4$, e $T(3) = 5$
  \item Para $n_0 = 2$, escolhemos $c$ tal que $T(2) = 4 \le c 2 \lg 2 = 2c$
    \pause \assert{$\Longrightarrow c = 2$}
  \end{itemize}
\end{itemize}
\end{frame}

\begin{frame}

  \frametitle{Análise do algoritmo randomizado}

  \begin{itemize}

    \item Conjectura: \alert<3-4>{$T(n) \le a n \lg n + b$}, com $a > 0$ e $b > 0$ a determinar.

\only<1>{    \item Caso de base $T(1) \in \Theta(1)$.

\begin{itemize}
  \item \assert{OK}
\end{itemize}
}

\only<2->{
    \item Em geral:

      \begin{small}
      \begin{eqnarray*}
      T(n) 
\only<2-3>{ & = & \frac{1}{n}\left(\sum_{m=1}^{n-1} T(m) + T(n-m)\right) + \Theta(n) \\}
\only<3-4>{ & = & \frac{2}{n}\left(\sum_{m=1}^{n-1} T(m) \right) + \Theta(n) \\}
\only<4-5>{ & \le & \frac{2}{n}\left(\sum_{m=1}^{n-1} a m \lg m + b \right) + \Theta(n) \\}
\only<5-6>{ & \le & \frac{2a}{n} \sum_{m=1}^{n-1} m \lg m + \frac{2b}{n}(n-1) + \Theta(n) \\}
\only<6-7>{ & \le & \frac{2a}{n} \left( \frac{1}{2} n^2 \lg n - \frac{1}{8}n^2 \right) + \frac{2b}{n}(n-1) + \Theta(n) \\}
\only<7-8>{ & \le & a n \lg n - \frac{a}{4}n + 2b + \Theta(n) \\}
\only<8-9>{ & \le & a n \lg n + b + \left(\Theta(n) + b - \frac{a}{4}n \right) \\}
\only<9-10>{ & \le & a n \lg n + b.}
     \end{eqnarray*}
      \end{small}
}

  \end{itemize}

\only<6>{\alert{$\sum_{m=1}^{n-1} m \lg m \le \frac{1}{2} n^2 \lg n - \frac{1}{8}n^2.$}}
\only<9>{Tomando $a$ grande o suficiente para que $\frac{a}{4}n > \Theta(n) + b$.}
\only<10>{O que conclui a prova pelo método de substituição.}
\end{frame}

\begin{frame}

  \frametitle{Análise do algoritmo randomizado}
  \framesubtitle{Justificando $\sum_{m=1}^{n-1} m \lg m \le \frac{1}{2} n^2 \lg n - \frac{1}{8}n^2$}

  $$
  \begin{array}{rcl}
    \sum_{m=1}^{n-1} m \lg m & = & \only<2->{\sum_{m=1}^{\lceil n/2 \rceil -1} m \lg m + \sum_{m=\lceil n/2 \rceil}^{n-1} m \lg m} \\
    \only<3->{ & \le & (\lg n - 1) \sum_{m=1}^{\lceil n/2 \rceil -1} m +
      \lg n \sum_{m = \lceil n/2 \rceil}^{n-1} m} \\
    \only<4->{ & = & \lg n \sum_{m=1}^{n-1} m - \sum_{m = 1}^{\lceil n/2 \rceil-1} m}  \\
    \only<5->{ & \le & \frac{1}{2}n(n-1)\lg n - \frac{1}{2}(\frac{n}{2} - 1)\frac{n}{2} }\\
    \only<6->{ & \le & \assert{\frac{1}{2}n^2\lg n - \frac{1}{8}N^2}}
    \end{array}
  $$
  \only<2>{\begin{itemize}
      \item separação em duas partes do somatório
    \end{itemize}}
  \only<3>{
    \begin{itemize}
      \item substituição por um termo maior ou igual: $\lg(n/2)$ no
        primeiro somatório, e $\lg n$ no segundo;
      \item fatoração;
      \item propriedade do logaritmo: $\lg(n/2) = \lg n - \lg 2 = \lg n - 1$.
    \end{itemize}
  }
  \only<4>{
    \begin{itemize}
      \item remanejamento dos termos (comutatividade da adição);
      \item junção dos somatórios.
    \end{itemize}
  }
  \only<5>{
    \begin{itemize}
      \item simplificação da soma dos termos de progressões aritméticas.
    \end{itemize}
  }
  
\end{frame}
\section{Aperfeiçoamentos práticos}

\begin{frame}

  \frametitle{Estratégias para evitar o pior caso}
  \framesubtitle{Valor mediano}

  \begin{itemize}

    \item Pivô escolhido é o elemento com valor mediano entre
      $A[l]$, $A[u]$ e $A[l+u/2]$.


\begin{codebox}
\Procname{$\proc{Med-3}(A, l, u)$}
\li $m \gets l + (u-l)/2$
\li \If $A[l] < A[m]$
\li \Then \If $A[m] < A[u]$ \> \> \> \Then \Return $m$
\li       \Else \If $A[l] < A[u]$ \> \> \> \Then \Return $u$
\li         \Else \Return $l$
            \End
          \End
\li \Else \If $A[m] > A[u]$ \> \> \> \Then \Return $m$
\li    \Else \If $A[l] < A[u]$ \> \> \> \Then \Return $l$
\li      \Else \Return $u$
         \End
       \End
    \End
\end{codebox}

\item Se a faixa for grande, pode ser escolhido o valor mediano
  entre mais que três posições).

\item Ler uma implementação da função \textit{qsort} da biblioteca padrão da
  linguagem C.
  \end{itemize}

\end{frame}

\end{document}

