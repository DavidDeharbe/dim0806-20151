\documentclass{beamer}
\setbeamertemplate{footline}[frame number]
%\documentclass{beamer}

\usepackage[utf8x]{inputenc}
\usepackage[brazil,british]{babel}
\usetheme{default} 
\usecolortheme{beaver}
\newtranslation[to=brazil]{Theorem}{Teorema}
\newtranslation[to=brazil]{Definition}{Definição}
\newtranslation[to=brazil]{Example}{Exemplo}
\newtranslation[to=brazil]{Problem}{Exercício}
\newtranslation[to=brazil]{Solution}{Resolução}
\newtranslation[to=brazil]{Lemma}{Lema}
\newtranslation[to=brazil]{Corollary}{Corolário}
\logo{\includegraphics[width=1cm]{../img/logo-ppgsc-icon-text.png}}

\usepackage{graphicx}
\usepackage{clrscode3e}
\usepackage{hyperref}

\usepackage{pgf}
\usepackage{tikz}
\newcommand{\assert}[1]{\textcolor{blue}{#1}}

\usepackage{xspace}

\newcommand{\semester}[0]{2015.2}


\title{Aula 15: Tabelas de espalhamento}
\subtitle{Tabelas de dispersão, tabelas \textit{hash\/}}
\author{David Déharbe \\
  Programa de Pós-graduação em Sistemas e Computação \\
  Universidade Federal do Rio Grande do Norte \\
  Centro de Ciências Exatas e da Terra \\
  Departamento de Informática e Matemática Aplicada}
\date{13 de abril de 2015}


\begin{document}
\selectlanguage{brazil}
\begin{frame}
  \titlepage
  Download me from \url{http://ufrn.academia.edu/DavidDeharbe}.
\end{frame}

\begin{frame}
  \frametitle{Plano da aula}
  \tableofcontents

  \alert{Referência}: Cormen et al. Capítulo 12.
\end{frame}

\section{Preâmbulo}

\begin{frame}

  \frametitle{Introdução}

  \begin{itemize}

    \item Coleção dinâmica de dados

    \item Cada dado $x$ tem um atributo \alert{chave} $\attrib{x}{key}$
      \alert{único}

      $x \neq y \Rightarrow \attrib{x}{key} \neq \attrib{y}{key}$

    \item Operações

      \begin{itemize}

        \item Inserção de um dado;

        \item Remoção de um dado;

        \item Busca na coleção de um dado com uma determinada chave

          \begin{itemize}

            \item pode ser bem sucedida ou mal sucedida

          \end{itemize}

        \item Custo $\Theta(1)$ em média, $\Theta(n)$ no pior caso.

      \end{itemize}

  \end{itemize}

\end{frame}

\section{Tabelas de indexação direta}

\begin{frame}

  \frametitle{Tabelas de indexação direta}

  \begin{itemize}

    \item $U$: conjunto das chaves possíveis

    \item $U$ é \alert{pequeno}

    \item Encontrar função bijetora $i$ de $U$ para $1 \twodots | U |$

    \item Manter uma tabela $A$ de tamanho $| U |$, onde $A[ i(\attrib{x}{key}) ]$ 

      \begin{itemize}

        \item é $x$, ou uma referência para $x$, quando o dado $x$ está na
          coleção;

        \item é o valor especial $\const{Nil}$, caso contrário.

      \end{itemize}

  \end{itemize}

\end{frame}

\begin{frame}

  \frametitle{Operações}
  \framesubtitle{Tabelas de indexação direta}

  \begin{codebox}
    \Procname{$\proc{Buscar}(A, k)$}
    \zi \Return $A[ i( k )]$
  \end{codebox}

  \begin{codebox}
    \Procname{$\proc{Inserir}(A, x)$}
    \zi $A[ i( \attrib{x}{key} )] \gets x$
  \end{codebox}

  \begin{codebox}
    \Procname{$\proc{Remover}(A, x)$}
    \zi $A[ i( \attrib{x}{key} )] \gets \const{Nil}$
  \end{codebox}

\end{frame}

\begin{frame}

  \frametitle{Exercício}
  \framesubtitle{Tabelas de indexação direta}

  \begin{enumerate}
    \item Assumindo 
      \begin{itemize}
      \item as mesmas hipóteses que para as tabelas de indexação
        direta, e 
      \item que não estamos interessados em representar $x$ na
      coleção, mas apenas registrar a presença ou ausência dele, 
      \end{itemize}
      encontrar uma estrutura de dados mais econômica em memória que
      a tabela.

    \item Descrever um procedimento para encontrar o dado com maior chave
      em uma tabela de indexação direta.

      Qual o custo deste procedimento?
  \end{enumerate}

\end{frame}

\section{Tabelas de espalhamento}

\begin{frame}

  \frametitle{Tabelas de espalhamento}

  \begin{itemize}

    \item O tamanho de uma tabela de indexação direta é $\Theta(| U |)$

    \item Quando $U$ não é pequeno, tabelas de indexação direta não são
      viáveis.

      \begin{itemize}

        \item Em um compilador, como representar a tabela dos símbolos?

      \end{itemize}

    \item Mesmo se $| U |$ pode ser representado em memória, se o tamanho de
      $K$, o conjunto dos dados efetivamente presentes da coleção é muito menor
      que $| U |$, há desperdício de memória.

    \item Quando $| K |$ é muito menor que $| U |$, recomenda-se considerar
      \alert{tabelas de espalhamento} ao invés de tabelas de indexação direta.

  \end{itemize}

\end{frame}

\begin{frame}

  \frametitle{Considerações sobre a complexidade}
  \framesubtitle{Tabelas de espalhamento}

  \begin{itemize}

    \item O \alert{custo médio} das operações é $\Theta(1)$

      \begin{itemize}

        \item no caso de tabelas de indexação direta, é o custo no \alert{pior caso}

      \end{itemize}
   
    \item A quantidade de memória necessária é $\Theta( | K | )$.

      \begin{itemize}

        \item O tamanho de uma tabela de indexação direta é $\Theta(| U |)$

      \end{itemize}

    \item A função de indexação $h$ de $U$ para $1 \twodots m$ ($m$ tamanho da tabela
      de espalhamento) é \alert{sobrejetora}

      \begin{itemize}

        \item A função da tabela de indexação direta é sobrejetora.

        \item A função $h$ é chamada \alert{função de espalhamento}

          função de dispersão, função \textit{hash\/}

      \end{itemize}

  \end{itemize}
      
\end{frame}

\begin{frame}

  \frametitle{Colisões}
  \framesubtitle{Tabelas de espalhamento}

  \begin{itemize}

    \item Problema: \alert{colisões}

      \begin{itemize}

      \item dados $x$ e $x'$, com chaves $k$ e $k'$ são tais que $h(k) = h(k')$,
         são inseridos na tabela.

      \end{itemize}

    \item Como resolver as colisões?

      \begin{itemize}

        \item Encadeamento externo

        \item Política de endereçamento aberto

      \end{itemize}

    \item Como reduzir a probabilidade de colisões?

      \begin{itemize}

      \item Funções de espalhamento

      \item A função ideal mapearia cada chave para uma posição
        diferente.

      \item Como $| U | > m$, a função ideal não existe.

      \end{itemize}

  \end{itemize}
      
\end{frame}

\subsection{Tratamento de colisões por encadeamento externo}

\begin{frame}

\frametitle{Tratamento de colisões por encadeamento externo}

\begin{itemize}

\item Cada posição contem uma referência para a primeira célula de uma lista
  encadeada

  A lista na posição $p$ armazena $\{ x \, | \, h(\attrib{x}{key}) = p \}$;

\item ou $\const{Nil}$ se a coleção não contem dado $x$ tal que
  $h(\attrib{x}{key}) = p$.

\end{itemize}

\end{frame}

\begin{frame}

  \frametitle{Operações}
  \framesubtitle{Tabelas de espalhamento com encadeamento externo}

  \begin{codebox}
    \Procname{$\proc{Buscar}(A, k)$}
    \zi \Comment buscar um dado com chave $k$ na lista $A[ h( k )]$
  \end{codebox}

  \begin{codebox}
    \Procname{$\proc{Inserir}(A, x)$}
    \zi \Comment inserir o dado na cabeça da lista $A[ h( \attrib{x}{key} )] $
  \end{codebox}

  \begin{codebox}
    \Procname{$\proc{Remover}(A, x)$}
    \zi \Comment remover o dado $x$ da lista $A[ h( \attrib{x}{key} )] $
  \end{codebox}

  \pause
  \alert{não esquecer de atualizar $A[h(\attrib{x}{key})]$ na inserção e na remoção}

\end{frame}

\begin{frame}

  \frametitle{Custo das operações}
  \framesubtitle{Tabelas de espalhamento com encadeamento externo}

  \begin{description}

    \item[Buscar] complexidade linear no tamanho da lista $A[ h( k )]$


    \item[Inserir] $\Theta(1)$

    \item[Remover] 

      \begin{itemize}

        \item $\Theta(1)$ se a lista for duplamente encadeada e o dado $x$ contem os atributos
          $\id{next}$ e $\id{prev}$.

        \item complexidade linearmente proporcional ao tamanho da lista $A[ h(\attrib{x}{key}) ]$
          caso contrário.

      \end{itemize}

  \end{description}

  \pause 

  O que podemos dizer sobre o tamanho das listas?

\end{frame}

\begin{frame}

  \frametitle{Análise de espalhamento com encadeamento externo}


  \begin{center}
    \begin{tabular}{rcl}
      $n$ & : & número de elementos na tabela \\
      $m$ & : & tamanho da tabela
    \end{tabular}
  \end{center}

  \begin{itemize}

  \item \alert{fator de carga}: $\alpha = n / m$.


  \item pior caso: 

    \begin{itemize}

      \item todos os elementos estão na mesma posição $\Theta(n)$

      \item lista encadeada

      \item função de espalhamento \alert{não} espalha!

    \end{itemize}


  \item caso médio depende das propriedades da função de espalhamento

  \end{itemize}

\end{frame}

\begin{frame}

  \frametitle{Análise de espalhamento com encadeamento externo}
  \framesubtitle{Caso médio}

  Hipóteses:

  \begin{enumerate}

  \item \alert{espalhamento uniforme simples}: para qualquer chave $k$ a
    probabilidade de $h(k) = i$ é $1/m$, para $1 \le i \le m$

  \item o custo de calcular $h(k)$ é sempre $\Theta(1)$.

  \item fator de carga $\alpha$

  \item tratamento de colisões por encadeamento externo

  \end{enumerate}

  Cenários analizados:

  \begin{itemize}

    \item busca mal-sucedida

    \item busca bem-sucedida

  \end{itemize}

\end{frame}

\begin{frame}

  \frametitle{Análise de espalhamento com encadeamento externo}
  \framesubtitle{Caso médio de busca mal-sucedida}

  Hipóteses:

  \begin{theorem}

    Sob as hipóteses enunciadas, o custo de uma busca mal-sucedida é $\Theta(1+\alpha)$.

  \end{theorem}

  \pause

  \begin{proof}

  O custo é
  \begin{itemize}

    \item o de uma busca mal-sucedida em uma lista: $\Theta(| \text{lista} |)$ ( $= \Theta( \alpha )$ em média).

    \item acrescentado do cálculo de $h(k)$, e do acesso ao arranjo ($= \Theta(1)$).

  \end{itemize}
  ou seja $\Theta(1 + \alpha)$.

  \end{proof}
\end{frame}

\begin{frame}

  \frametitle{Análise de espalhamento com encadeamento externo}
  \framesubtitle{Caso médio de busca bem-sucedida}

  Hipóteses:

  \begin{theorem}

    Sob as hipóteses enunciadas, o custo de uma busca bem-sucedida é $\Theta(1+\alpha)$.

  \end{theorem}

  \pause

  \begin{proof}

  O custo médio é
  \begin{itemize}

    \item o de uma busca bem-sucedida em uma lista: $\Theta(| \text{lista} | /2)$ ( $= \Theta( \alpha / 2)$ em média).

    \item acrescentado do cálculo de $h(k)$, e do acesso ao arranjo ($= \Theta(1)$).

  \end{itemize}
  ou seja $\Theta(1 + \alpha/2) = \Theta(1 + \alpha)$.

  \end{proof}
\end{frame}

\begin{frame}

  \frametitle{Análise de espalhamento com encadeamento externo}
  \framesubtitle{Conclusão}

  \begin{itemize}

  \item Sob as hipóteses enunciadas, e

  \item se $m$ é pelo menos proporcionalmente linear a $n$ ($m \in \Omega(n)$), 

  \item então $n \in O(m)$ e $\alpha = n/m \in O(m)/m = O(1)$.

  \item Logo o custo da busca é $O(1)$.

  \item Para a inserção, o custo é $O(1)$.

  \item Para a remoção, o custo é 

    \begin{itemize}

      \item $O(1)$ se a lista for duplamente encadeada e $x$ inclui os attributos de encadeamento;

      \item $O(1+\alpha) = O(1)$ caso contrário (pelo mesmos motivos que a busca).

    \end{itemize}

  \end{itemize}

\end{frame}


\begin{frame}
  \frametitle{Exercícios}

  \begin{enumerate}

    \item Simule inserir sucessivamente os valores com chaves 5, 28, 19, 15, 20,
      33, 12, 17, 10 em uma tabela de espalhamento inicialmente vazia, com 9
      posições, e função de espalhamento $h = \lambda k . 1 + k \mod 9$.

    \item Sob as hipóteses enunciadas, estime a complexidade média da operação de inserção caso a mesma
      seja realizadda:
      \begin{itemize}

        \item sempre no final da lista

        \item em uma lista ordenada, mantendo a propriedade de ordenação

      \end{itemize}

    \item Sob a hipótese de espalhamento uniforme simples, quantas colisões há
      quando $n$ chaves distintas são inseridos em uma tabela de espalhamento de
      tamanho $m$?

  \end{enumerate}

\end{frame}

\section{Funções de espalhamento}

\begin{frame}

\frametitle{Funções de espalhamento: características desejadas}

\begin{tabular}{rcl}
  $m$ & : & tamanho da tabela de espalhamento \\
  $h$ & : & função de espalhamento \\
  $U$ & : & universo das chaves \\
\end{tabular}

\begin{itemize}

  \item A função de espalhamento \alert{ideal} é uniforme:

    \begin{itemize}

    \item Todas as listas encadeadas tem o mesmo tamanho $n / m$.

    \item A probabilidade de qualquer operação de inserção ocorrer na posição $j$ seja $1 / m$.

    \item Seja $P(k)$ a probabilidade de um dado ter a chave $k \in U$.

      $$\sum_{\{ k | h(k) = j \}} P(k) = 1/m$$

    \end{itemize}

  \item Exemplo: $U = \{ k \in \mathbb{R} \,|\, 0 \le k \le 1 \}$, $P$ é uniforme.

    $h(k) = 1 + \lfloor k \cdot m \rfloor$

\end{itemize}

\end{frame}

\begin{frame}

\frametitle{Funções de espalhamento: características desejadas}

\begin{itemize}

  \item A distribuição das chaves \alert{geralmente} não é conhecida.

  \item Existe heurísticas com resultado prático satisfatório.

    \begin{itemize}

      \item análise estatística do domínio de aplicação

      \item análise qualitativa do domínio de aplicação

      \item o resultado do espalhamento deve ser independente de qualquer padrão
        que possa ocorrer nos dados

        Exemmplo de chaves comum em tabela de símbolos:
        \texttt{"i"}, \texttt{"j"}, \texttt{"pt"}, \texttt{"ptr"}, \texttt{"pt1"}

    \end{itemize}

\end{itemize}

\end{frame}

\begin{frame}

  \frametitle{Tratando chaves como números naturais}

  \begin{itemize}

    \item É comum funções de espalhamente considerar que as chaves são números naturais.

    \item É simples satisfazer esta hipótese:

      \begin{itemize}

        \item qualquer chave tem uma representação binária

        \item por interpretação como número na base 2, qualquer código binário é
          mapeado  para um número natural.

        \item Exemplo (codificação ASCII, base 128): 

          \begin{itemize}

            \item \texttt{"pt"} é interpretada como o par de naturais (112, 116), através da norma ASCII,
              a qual tem 128 códigos diferentes.

            \item Logo \texttt{"pt"} é mapeado para $112 \cdot 128^1 + 116 \cdot 128^0 = 14452$.

          \end{itemize}

          \item O mapeamento é uma função bijetora.
      \end{itemize}

  \end{itemize}

\end{frame}

\begin{frame}

  \frametitle{Espalhamento pelo método da divisão}

  \begin{tabular}{rcl}
    $m$ & : & tamanho da tabela de espalhamento \\
    $h$ & : & função de espalhamento \\
  \end{tabular}
  
  \begin{itemize}
    \item $h = \lambda k \cdot k \mod m$

    \item Potências de 2 são valores a evitar para $m$

      \begin{itemize}

        \item Motivo: $h(k)$ só depende de $\log_2 m$ bits de $k$.

      \end{itemize}

    \item Potências de 10 são valores a evitar para $m$ se as chaves são números decimais

    \item Receita para umm $m$ geralmente bom:

      \begin{itemize}

        \item número primo

        \item não vizinho de uma potência de 2

      \end{itemize}

    \item Sempre verificar experimentalmente a escolha com dados representativos da aplicação.
      
\end{itemize}

\end{frame}

\begin{frame}

  \frametitle{Espalhamento por multiplicação}

  \begin{tabular}{rcl}
    $k$ & : & chave \\
    $m$ & : & tamanho da tabela de espalhamento \\
    $h$ & : & função de espalhamento \\
  \end{tabular}
  
  \begin{itemize}
    
  \item Escolher uma constante $A \in \mathbb{R}$ tal que $0 < A < 1$

  \item Seja $\varphi(k) = k \cdot A - \lfloor k \cdot A \rfloor \rfloor$ a parte
    fracionária do produto da chave por $A$.

  \item $h = \lambda k . 1 + \lfloor m \cdot \varphi(A) \rfloor$ é a função de espalhamento obtida.

  \item O valor de $m$ não é crítico: pode ser uma potência de 2.

  \item A escolha ótima do valor de $A$ depende da aplicação

    \begin{itemize}

    \item O valor $(\sqrt{5} - 1)/2 = 0.61803...$ foi sugerido por ter boa probabilidade de funcionar bem (Knuth).

    \end{itemize}

  \item $h = \lambda k . k \times \lfloor A \cdot 2^w \rfloor / 2^{w-p}$, onde $w$
    é o tamanho da palavra do computador e $2^p = m$.

  \end{itemize}

\end{frame}

%% incluir desenho figure 12.4 do Cormen

\begin{frame}

  \frametitle{Espalhamento universal}
  \framesubtitle{Carter \& Wegman, 1979}
  \begin{itemize}
    
  \item O usuário maliocoso pode escolher uma série de dados que vão todos 
    ser ``espalhados'' para a mesma posição.

  \item O desempenho da aplicação será afetado negativamente

  \item Espalhamento universal é uma solução a este problema

  \item Há uma coleção de funções de espalhamento.

  \item Em tempo de execução uma das funções de espalhamento é escolhida aleatoriamente.

  \item Reduz a probabilidade do desempenho ser ruim.

  \item Uma coleção $H$ de funções de espalhamento é \alert{universal} quando, para
    cada par de chaves distintas $x, y$, $h(x) = h(y)$ com probabilidade $|H|/m$. 

    \begin{itemize}

      \item Escolhendo uma função de espalhamento $h$ ao acaso em $H$, a 
        chance de colisão entre $x$ e $y$ é $1/m$. 

      \item Esta probabilidade é a mesma que se $h(x)$ e $h(y)$ ter sido
        escolhidos aleatoriamente em $1 \twodots m$.

    \end{itemize}

  \end{itemize}

\end{frame}

\begin{frame}

  \frametitle{Espalhamento universal}

  \begin{theorem}

    Se $h$ é escolhido de uma coleção universal de funções de espalhamento
    para espalhar $n \le m$ chaves, o número esperado de colisões para
    uma chave $x$ é menos que 1.

  \end{theorem}

  \begin{proof}

    \begin{itemize}

    \item Seja $x \neq y$ duas chaves quaisquer.

    \item Por definição, a probabilidade de colisão é $1 / m$.

    \item Se há $n$ chaves distintas $x_1, \ldots x_n$, 

      \begin{itemize}

        \item a probabilidade de colisão de 
          $x_1$ com $x_2$ é $1 / m$, com $x_2$ é $1 / m$, etc.

        \item a probabilidade global de colisão é $(n-1)/m$;

        \item se $n \le m$, a probabilidade de colisão  $(n-1)/m < 1$.

      \end{itemize}

    \end{itemize}

  \end{proof}

\end{frame}

\begin{frame}

\frametitle{Projeto de classe universal de funções de espalhamento}

\begin{itemize}

\item $m$ é um número primo

\item cada chave $k$ é considerada como uma sequência $\langle k_1, \ldots, k_r$
  de $r$ cadeias de $t$ bits (condição $2^t < m$)

\item Seja $a = \langle a_1, \ldots, a_r \rangle$ uma sequênciada formada $r$
  elementos escolhidos do conjunto $1 \twodots m$

\item A função de espalhamento $h_a = \lambda k . \sum_{i=1}^{r} a_i k_i$ 

\item $H = \bigcup_a \{ h_{a} \}$ tem $m^r$ elementos.
\end{itemize}

\begin{theorem}

A classe $H$ é universal.

\end{theorem}

\end{frame}

\begin{frame}

\frametitle{Projeto de classe universal de funções de espalhamento}

\begin{proof}

Seja $x \neq y$ e $h(x) = h(y)$. Assumimos que $x_1 \neq y_1$ (poderia ser qualquer sub-sequência).

\begin{itemize}

\item $h_{a}(x) = h_{a}(y) \Longrightarrow \sum_{i=1}^{r} a_i x_i \mod m = \sum_{i=1}^{r} a_i y_i \mod m$.

\item Logo $\sum_{i=1}^{r} a_i (x_i - y_i) \mod m = 0$.

\item Logo $a_1(x_1 - y_1) \equiv \sum_{i=1}^{r} a_i (x_i - y_i) \mod m = 0$.

\item Teoria dos números: como $x_1 - y_1 \neq 0$, possui um inverso multiplicativo módulo $m$.

\item Logo $a_1 = - \sum_{i=2}^r a_i (x_i - y_i) \cdot (x_1 - y_1)^{-1} \mod m$

... e existe um único $a_0 \mod m$ tal que $h(x) = h(y)$.

\item Tem $m^{r-1}$ valores de $a$ (uma para cada $\langle a_2, \ldots, a_r \rangle$) tais que $x$ e $y$ colidem.

\item De $m^r$ combinações possíveis, há $m^{r-1}$ colisões. A probabilidade é
  $m^{r-1}/m^r = 1/m$: $H$ é universal.

\end{itemize}

\end{proof}

\end{frame}

\begin{frame}

\frametitle{Exercícios}

\begin{enumerate}

\item Aplicando o espalhamento por multiplicação com $A = (\sqrt{5} - 1)/2$ e $m
  = 1000$, quais são as posições para as chaves 61, 62, 63, 64 e 65?

\item Em uma aplicação onde comparar duas chaves é custoso, como adaptar as estruturas de dados
  envolvidas em uma tabela de espalhamento para acelerar as operações?

\end{enumerate}

\end{frame}

\section{Endereçamento aberto}

\begin{frame}

\frametitle{Resolução de colisões por endereçamento aberto}

\begin{itemize}

  \item Sem encadeamento externo

  \item Se houver uma colisão, uma nova posição é calculada

  \item Função de espalhamento: $h(k, i)$

    \begin{itemize}

      \item $k$ chave

      \item $i$ \alert{tentativa}

      até $m$ tentativas

    \end{itemize}

  \item capacidade limitada ($\neq$ encadeamento externo)

\end{itemize}

\end{frame}

\begin{frame}

\frametitle{Inserção}
\framesubtitle{Resolução de colisões por endereçamento aberto}

\begin{codebox}
    \Procname{$\proc{Inserir}(T, x)$}
    \li $i \gets 1$
    \li \Repeat 
    \li   $p \gets h(\attrib{x}{key}, i)$ 
    \li   \If $T[p] \isequal \const{Nil}$
    \li   \Then $T[p] \gets x$
    \li     \Return $p$
    \li   \Else $i \gets i+1$
          \End
    \li \Until $i \isequal m+1$
    \li \Comment tratamento de tabela cheia
\end{codebox}

\end{frame}

\begin{frame}

\frametitle{Busca}
\framesubtitle{Resolução de colisões por endereçamento aberto}

\begin{codebox}
    \Procname{$\proc{Buscar}(T, k)$}
    \li $i \gets 1$
    \li \Repeat 
    \li   $p \gets h(k, i)$ 
    \li   \If $\attrib{T[p]}{key} \isequal k$
    \li   \Then \Return $p$
          \End
    \li   \If $T[p] \isequal \const{Nil}$
    \li   \Then \Return $\const{Nil}$
          \End
    \li   $i \gets i+1$
          \End
    \li \Until $i \isequal m+1$
    \li \Return $\const{Nil}$
\end{codebox}

\end{frame}

\begin{frame}

\frametitle{Remover}
\framesubtitle{Resolução de colisões por endereçamento aberto}

\begin{codebox}
    \Procname{$\proc{Remover}(T, x)$}
    \li $i \gets 1$
    \li \Repeat 
    \li   $p \gets h(\attrib{x}{key}, i)$ 
    \li   \If $T[p] \isequal k$
    \li   \Then $T[p] \gets \const{Nil}$
                \Return
          \End
    \li   $i \gets i+1$
          \End
    \li \Until $i \isequal m+1$
    \li \Comment tratar erro 
\end{codebox}

\pause

\begin{itemize}

\item Assumindo $m = 10$, $h = \lambda k, i \cdot 1 + ((k + i - 1) \mod m)$, $\attrib{x}{key} = x$,
\item simular $\proc{Inserir}(5)$, $\proc{Inserir}(15)$, $\proc{Remover}(5)$, $\proc{Buscar}(15)$
\pause
\item \alert{...problema...} \pause solução: $\const{Deleted}$
\end{itemize}
\end{frame}

\begin{frame}

\frametitle{Inserção}
\framesubtitle{Resolução de colisões por endereçamento aberto}

\begin{codebox}
    \Procname{$\proc{Inserir}(A, x)$}
    \li $i \gets 1$
    \li \Repeat 
    \li   $p \gets h(\attrib{x}{key}, i)$ 
    \li   \If $T[p] \isequal \const{Nil}$ or $T[p] \isequal \const{Deleted}$
    \li   \Then $T[p] \gets x$
    \li     \Return $p$
    \li   \Else $i \gets i+1$
          \End
    \li \Until $i \isequal m+1$
    \li \Comment tratamento de tabela cheia
\end{codebox}

\end{frame}

\begin{frame}

\frametitle{Busca}
\framesubtitle{Resolução de colisões por endereçamento aberto}

\begin{codebox}
    \Procname{$\proc{Buscar}(A, k)$}
    \li $i \gets 1$
    \li \Repeat 
    \li   $p \gets h(k, i)$ 
    \li   \If $\attrib{T[p]}{key} \isequal k$
    \li   \Then \Return $p$
          \End
    \li   \If $T[p] \isequal \const{Nil}$
    \li   \Then \Return $\const{Nil}$
          \End
    \li   $i \gets i+1$
          \End
    \li \Until $i \isequal m+1$
    \li \Return $\const{Nil}$
\end{codebox}

\end{frame}

\begin{frame}

\frametitle{Remover}
\framesubtitle{Resolução de colisões por endereçamento aberto}

\begin{codebox}
    \Procname{$\proc{Remover}(A, x)$}
    \li $i \gets 1$
    \li \Repeat 
    \li   $p \gets h(\attrib{x}{key}, i)$ 
    \li   \If $T[p] \isequal k$
    \li   \Then $T[p] \gets \const{Deleted}$
    \li     \Return
          \End
    \li   $i \gets i+1$
          \End
    \li \Until $i \isequal m+1$
    \li \Comment tratar erro 
\end{codebox}

\end{frame}

\subsection{Tentativas lineares}

\begin{frame}

\frametitle{Tentativas lineares}

\begin{tabular}{rcl}
$h$ & : & função de espalhamento da chave \\
$m$ & : & tamanho da tabela
\end{tabular}

$$h_L = \lambda k, i \cdot 1 + (h(k) + (i - 1) \mod m)$$

\begin{itemize}

\item tentativas são realizadas em posições sucessivas

\item problema: \alert{agrupamentos primários}

\begin{itemize}

  \item quando há uma sub-faixa ocupada de tamanho $n < m$, 

  \item a probabilidade de colisão é $n/m$, 

  \item cria uma sub-faixa de tamanho $n' \ge n+1$

  \item a probabilidade de colisão nesta sub-faixa torna-se $n'/m$.

\end{itemize}

\item não é uma boa sub-solução: desempenho degrada-se rapidamente.

\end{itemize}

\end{frame}

\subsection{Tentativas quadráticas}

\begin{frame}

\frametitle{Tentativas quadráticas}

\begin{tabular}{rcl}
$h$ & : & função de espalhamento da chave \\
$m$ & : & tamanho da tabela
\end{tabular}

$$h_Q = \lambda k, i \cdot 1 + (h(k) + c_1 \cdot (i-1) + c_2 \cdot(i-1)^2 \mod m)$$

$c_1 \neq 0, c_2 \neq 0$ são constantes

\begin{itemize}
\item elimina o problema dos agrupamentos lineares
\item para que todas as posições sejam visitadas em $m$ tentativas, os valores
  de $m$, $c_1$ e $c_2$ não podem ser quaisquer
\item note que se $h_Q(k_1, 1) = h_Q(k_2, 1)$ então $h_Q(k_1, i) = h_Q(k_2, i)$ para $i > 1$
\item a mesma sequência de tentativas é calculada para chaves com o mesmo valor de espalhamento inicial
\begin{itemize}
\item aparecem \alert{agrupamentos secundários}
\end{itemize}
\end{itemize}

\end{frame}

\subsection{Espalhamento duplo}

\begin{frame}

\frametitle{Espalhamento duplo}

\begin{tabular}{rcl}
$h_1$, $h_2$ & : & função de espalhamento da chave \\
$m$ & : & tamanho da tabela
\end{tabular}

$$h_D = \lambda k, i \cdot 1 + (h_1(k) + (i-1)\cdot h_2(k) \mod m)$$

\begin{itemize}
\item a sequência de tentativas depende de $k$ para a posição inicial, 

\item o deslocamento para as próximas tentativas também depende de $k$.

\item para $m$ tentativas visitem $m$ posiçõs diferentes,
  o valor de $h_2(k)$ deve ser relativamente primo com relação a $m$.

\begin{itemize}

\item dica 1: $m = 2^p$ e $\forall k \cdot h_2(k)$ é ímpar.

\item dica 2: $m$ primo, e $\forall k \cdot h_2(k) < m$ é ímpar

  exemplo: $h_1 = \lambda k \cdot 1 + (k \mod m)$, $h_2 = \lambda k \cdot 1 + (k mod m')$,
  com $m' = m - 1$.

\end{itemize}

\end{itemize}

\end{frame}

\begin{frame}

\frametitle{Comparação das tentativas}

\begin{itemize}

\item tentativa linear: $m$ sequências de tentativas possíveis

\item tentativa quadrática: $m$ sequências de tentativas possíveis

\item espalhamento duplo: $m^2$ sequência de tentativas possíveis

\item ideal: $m!$ sequências de tentativas possíveis

\end{itemize}

\alert{$\Longrightarrow$} espalhamento duplo é teoricamente bem superior

\end{frame}

%% TODO \subsection{Análise do endereçamento aberto} pp.237-239

\begin{frame}

\frametitle{Exercícios}

\begin{enumerate}

\item Seja uma tabela de dispersão de tamanho $m = 11$ com endereçamento
  aberto. Os valores $10, 22, 31, 4, 15, 28, 17, 88, 59$ são inseridos, nesta
  ordem.  Qual o resultado destas operações quando é usada a função de
  espalhamento $h = \lambda k \cdot 1 + k \mod m$, e as colisões são tratadas
  por:
\begin{itemize}
\item tentativas lineares
\item tentativas quadráticas
\item espalhamento duplo, e $h_2 = \lambda k \cdot 1 + (k \mod (m - 1))$,
  $c_1 = 1$, $c_2 = 3$.
\end{itemize}

\end{enumerate}

\end{frame}

\end{document}

