\documentclass{beamer}
\setbeamertemplate{footline}[frame number]

\usepackage[utf8x]{inputenc}
\usepackage[brazil,british]{babel}
\usetheme{default} 
\usecolortheme{beaver}
\newtranslation[to=brazil]{Theorem}{Teorema}
\newtranslation[to=brazil]{Definition}{Definição}
\newtranslation[to=brazil]{Example}{Exemplo}
\newtranslation[to=brazil]{Problem}{Exercício}
\newtranslation[to=brazil]{Solution}{Resolução}
\newtranslation[to=brazil]{Lemma}{Lema}
\newtranslation[to=brazil]{Corollary}{Corolário}
\logo{\includegraphics[width=1cm]{../img/logo-ppgsc-icon-text.png}}

\usepackage{graphicx}
\usepackage{clrscode3e}
\usepackage{hyperref}

\usepackage{pgf}
\usepackage{tikz}
\newcommand{\assert}[1]{\textcolor{blue}{#1}}

\usepackage{xspace}

\newcommand{\semester}[0]{2015.2}


\usepackage{pgf}
\usepackage{tikz}
\usetikzlibrary{shapes.multipart}
\usetikzlibrary{decorations.pathmorphing}

\title{Aula 23: Grafos: algoritmos elementares (I)}
\author{David Déharbe \\
  Programa de Pós-graduação em Sistemas e Computação \\
  Universidade Federal do Rio Grande do Norte \\
  Centro de Ciências Exatas e da Terra \\
  Departamento de Informática e Matemática Aplicada}
\date{15 de maio de 2015}

\tikzstyle{vertex}=[circle,fill=black!25,minimum size=15pt,inner sep=0pt]
\tikzstyle{selected vertex} = [vertex, fill=red!24]
\tikzstyle{path} = [draw,thin,decorate,decoration={snake,amplitude=.5mm,segment length=2mm}]
\tikzstyle{edge} = [draw,thick,-]
\tikzstyle{arrow} = [draw,thick,->]
\tikzstyle{weight} = [font=\small]
\tikzstyle{selected edge} = [draw,line width=3pt,-,red!50]
\tikzstyle{ignored edge} = [draw,line width=3pt,-,black!20]

\begin{document}
\selectlanguage{brazil}

\begin{frame}
  \titlepage
\end{frame}

\begin{frame}
  \frametitle{Plano}
  \tableofcontents

Referência: Cormen, cap 23.
\end{frame}

\section{Introdução}

\begin{frame}
\frametitle{Motivação}

\begin{itemize}
\item Grafos: modelagem de dados e relacionamentos 
\item Aplicações em todos os domínios da computação
\item Problemas clássicos: 
\begin{itemize}
\item ordenação topológica
\item componentes conexos
\item árvore de espalhamento máximo
\item caminhos de custo mínimo 
\item fluxo máximo
\end{itemize}
\end{itemize}
\end{frame}

\section{Definições}

\begin{frame}
\frametitle{Grafo não dirigido}

\begin{definition}[Grafo não dirigido, vértice, aresta]
Um \emph{grafo não dirigido} $G$ é um par $(V, E)$, onde $V$ é o conjunto dos
\emph{vértices}, e $E \subseteq V \times V$, é uma relação binária sobre $V$, é
o conjunto das \emph{arestas}\footnote{Existe uma aresta $e$ entre $v$ e $v'$
  sse $(v, v') \in E$ ou $(v', v) \in E$.}.
\end{definition}

\begin{center}
\begin{tikzpicture}[scale=.8, auto,swap]
% Draw a 7,11 network
% First we draw the vertices
\foreach \pos/\name in {{(2,3)/0}, {(0,1.5)/1}, {(4,1.5)/2},
{(2,0)/3}}
\node[vertex] (\name) at \pos {$\name$};
% Connect vertices with edges and draw weights
\foreach \source/ \dest in {0/1, 0/2, 0/3, 1/3, 2/3}
\path[edge] (\source) -- (\dest);
\end{tikzpicture}
$$
\begin{array}{rcl}
V & = & {0, 1, 2, 3}, \\
E & = &\{ (0,1), (0,2), (0,3), (1, 3), (2,3) \}
\end{array}
$$ %
\end{center}

\end{frame}

\begin{frame}
\frametitle{Grafo não dirigido, com pesos}

\begin{definition}[Grafo não dirigido com pesos]
Um \emph{grafo não dirigido com pesos} $G$ é um par $(V, E, w)$, onde $(V, E)$ é
um grafo não dirigido e $w: E \rightarrow \mathbb{R}$ é o \emph{peso} das
arestas.
\end{definition}

\begin{center}
\begin{tikzpicture}[scale=.8, auto,swap]
% Draw a 7,11 network
% First we draw the vertices
\foreach \pos/\name in {{(2,3)/0}, {(0,1.5)/1}, {(4,1.5)/2},
{(2,0)/3}}
\node[vertex] (\name) at \pos {$\name$};
% Connect vertices with edges and draw weights
\foreach \source/ \dest/ \weight in {0/1/1, 2/0/3, 0/3/9, 1/3/2, 3/2/2}
\path[edge] (\source) -- node[weight]{$\weight$} (\dest);
\end{tikzpicture}
\end{center}
$$w = \{ (0, 1) \mapsto
1, (0, 2) \mapsto 3, (0,3) \mapsto 9, (1,3) \mapsto 2, (2, 3) \mapsto 2 \}
$$

\end{frame}

\begin{frame}
\frametitle{Grafo dirigido}

\begin{definition}[Grafo, vértice, seta]
Um \emph{grafo dirigido} $G$ é um par $(V, E)$, onde $V$ é o conjunto dos \emph{vértices}, e
$E \subseteq V \times V$, é uma relação binária sobre $V$, é o conjunto das \emph{setas}\footnote{Existe uma seta $e$ de $v$ para $v'$ sse $(v, v') \in E$.}.
\end{definition}

\begin{center}
\begin{tikzpicture}[scale=.8, auto,swap]
% Draw a 7,11 network
% First we draw the vertices
\foreach \pos/\name in {{(2,3)/0}, {(0,1.5)/1}, {(4,1.5)/2},
{(2,0)/3}}
\node[vertex] (\name) at \pos {$\name$};
% Connect vertices with edges and draw weights
\foreach \source/ \dest in {0/1, 0/2, 3/0, 1/3, 2/3}
\path[arrow] (\source) -- (\dest);
\end{tikzpicture}
$$
\begin{array}{rcl}
V & = & {0, 1, 2, 3}, \\
E & = &\{ (0,1), (0,2), (3,0), (1, 3), (2,3) \}
\end{array}
$$ %
\end{center}

\end{frame}

\begin{frame}
\frametitle{Grafo dirigido, com pesos}

\begin{definition}[Grafo dirigido com pesos, vértice, seta]
Um \emph{grafo dirigido com pesos} $G$ é uma tripla $(V, E, w)$, onde $(V, E)$ é um grafo dirigdo e $w: V \rightarrow \mathbb{R}$ é o peso de cada seta.
\end{definition}

\begin{center}
\begin{tikzpicture}[scale=.8, auto,swap]
% Draw a 7,11 network
% First we draw the vertices
\foreach \pos/\name in {{(2,3)/0}, {(0,1.5)/1}, {(4,1.5)/2},
{(2,0)/3}}
\node[vertex] (\name) at \pos {$\name$};
% Connect vertices with edges and draw weights
\foreach \source/ \dest/ \weight in {0/1/1, 2/0/3, 3/0/9, 1/3/2, 3/2/2}
\path[arrow] (\source) -- node[weight]{$\weight$} (\dest);
\end{tikzpicture}
\end{center}
$$w = \{ (0, 1) \mapsto
1, (0, 2) \mapsto 3, (3,0) \mapsto 9, (1,3) \mapsto 2, (2, 3) \mapsto 2 \}
$$

\end{frame}

\section{Representação}

\begin{frame}

\frametitle{Representação computacional}

\begin{enumerate}

\item lista de adjacência (matriz esparsa) $\Theta(V+E)$

  (listas encadeadas)

\item matriz de adjacência (matriz densa, ou pequena suficiente) $\Theta(V^2)$

\end{enumerate}

\end{frame}

\begin{frame}

\frametitle{Grafo não dirigido sem pesos}

\begin{center}
\begin{tikzpicture}[scale=.8, auto,swap]
% Draw a 7,11 network
% First we draw the vertices
\foreach \pos/\name in {{(2,3)/0}, {(0,1.5)/1}, {(4,1.5)/2},
{(2,0)/3}}
\node[vertex] (\name) at \pos {$\name$};
% Connect vertices with edges and draw weights
\foreach \source/ \dest in {0/1, 0/2, 0/3, 1/3, 2/3}
\path[edge] (\source) -- (\dest);
\end{tikzpicture}
\end{center}

\begin{description}

\item[listas de adjacência] $0 \mapsto \langle 1, 2, 3 \rangle; 1 \mapsto \langle 0, 3 \rangle;
2 \mapsto \langle 0, 3 \rangle, 3 \mapsto \langle 0, 1, 2\rangle$

\item[matriz de adjacência] $\left( 
\begin{array}{cccc}
0 & 1 & 1 & 1 \\
1 & 0 & 0 & 1 \\
1 & 0 & 0 & 1 \\
1 & 1 & 1 & 0
\end{array}
\right)
$
\end{description}
\end{frame}

\begin{frame}
\frametitle{Grafo não dirigido com pesos}

\begin{center}
\begin{tikzpicture}[scale=.8, auto,swap]
% Draw a 7,11 network
% First we draw the vertices
\foreach \pos/\name in {{(2,3)/0}, {(0,1.5)/1}, {(4,1.5)/2},
{(2,0)/3}}
\node[vertex] (\name) at \pos {$\name$};
% Connect vertices with edges and draw weights
\foreach \source/ \dest/ \weight in {0/1/1, 2/0/3, 0/3/9, 1/3/2, 3/2/2}
\path[edge] (\source) -- node[weight]{$\weight$} (\dest);
\end{tikzpicture}
\end{center}

\begin{description}

\item[listas de adjacência] $0 \mapsto \langle 1, 2, 3 \rangle; 1 \mapsto \langle 0, 3 \rangle;
2 \mapsto \langle 0, 3 \rangle, 3 \mapsto \langle 0, 1, 2\rangle$

\item[matriz de adjacência] $\left( 
\begin{array}{cccc}
0 & 1 & 3 & 9 \\
1 & 0 & \infty & 2 \\
3 & \infty & 0 & 2 \\
9 & 2 & 2 & 0
\end{array}
\right)
$
\end{description}
\end{frame}

\begin{frame}
\frametitle{Grafo dirigido}

\begin{center}
\begin{tikzpicture}[scale=.8, auto,swap]
% Draw a 7,11 network
% First we draw the vertices
\foreach \pos/\name in {{(2,3)/0}, {(0,1.5)/1}, {(4,1.5)/2},
{(2,0)/3}}
\node[vertex] (\name) at \pos {$\name$};
% Connect vertices with edges and draw weights
\foreach \source/ \dest in {0/1, 0/2, 3/0, 1/3, 2/3}
\path[arrow] (\source) -- (\dest);
\end{tikzpicture}
\end{center}

\begin{description}

\item[listas de adjacência] $0 \mapsto \langle 1, 2 \rangle; 1 \mapsto \langle 3 \rangle;
2 \mapsto \langle 3 \rangle, 3 \mapsto \langle 0\rangle$

\item[matriz de adjacência] $\left( 
\begin{array}{cccc}
0 & 1 & 1 & 0 \\
0 & 0 & 0 & 1 \\
0 & 0 & 0 & 1 \\
1 & 0 & 0 & 0
\end{array}
\right)
$
\end{description}
\end{frame}

\begin{frame}
\frametitle{Grafo dirigido com pesos}

\begin{center}
\begin{tikzpicture}[scale=.8, auto,swap]
% Draw a 7,11 network
% First we draw the vertices
\foreach \pos/\name in {{(2,3)/0}, {(0,1.5)/1}, {(4,1.5)/2},
{(2,0)/3}}
\node[vertex] (\name) at \pos {$\name$};
% Connect vertices with edges and draw weights
\foreach \source/ \dest/ \weight in {0/1/1, 2/0/3, 3/0/9, 1/3/2, 3/2/2}
\path[arrow] (\source) -- node[weight]{$\weight$} (\dest);
\end{tikzpicture}
\end{center}

\begin{description}

\item[listas de adjacência] $\begin{array}{ll}
  0 \mapsto \langle (1, 1) \rangle; & 1 \mapsto \langle (2, 3) \rangle; \\
  2 \mapsto \langle (1, 3) \rangle, & 3 \mapsto \langle (0, 9), (2, 2) \rangle
  \end{array}$

\item[matriz de adjacência] $\left( 
\begin{array}{cccc}
0 & 1 & \infty & \infty \\
\infty & 0 & \infty & 2 \\
3 & \infty & 0 & \infty \\
9 & \infty & 2 & 0
\end{array}
\right)
$
\end{description}
\end{frame}

\begin{frame}

\frametitle{Exercícios}

\begin{itemize}
\item A matriz de adjacência de um grafo não dirigido é igual à sua
  transposta.  A diagonal não carrega informação. Assim, mais da
  metade da representação é redundante ou inútil. Para reduzir o
  tamanho da representação, pode se utilizar um arranjo simples para
  armazenar apenas as entradas abaixo da diagonal (ou apenas as
  entradas acima da diagonal).

\item Ilustrando esta ideia com o nosso exemplo de grafo simple, a
  matriz de adjacência pode ser compactada no arranjo seguinte:
  \begin{center}
  \begin{tabular}{|c|c|c|c|c|c|}
    \hline
    1 & 1 & 0 & 1 & 1 & 1 \\
    \hline
  \end{tabular}
  \end{center}

  A mesma coisa pode ser feita para grafos com pesos:
  \begin{center}
  \begin{tabular}{|c|c|c|c|c|c|}
    \hline
    1 & 3 & $\infty$ & 9 & 2 & 2 \\
    \hline
  \end{tabular}
  \end{center}

\item Dado dos vértices $u$ e $v$, tal que $u < v$, qual a posição
  correspondente no arranjo de adjacência?
\end{itemize}

\end{frame}

\begin{frame}

\frametitle{Exercícios}

\begin{enumerate}
\item O grau de saída (de entrada) um vértice é o número de arestas que tem este vértice como fonte (destinho).
\begin{itemize}
\item Na representação com listas de adjacência, como calcular o grau de saída e de entrada de um vértice? Qual a complexidade destas operações?
\item E com matrizes de adjacência?
\end{itemize}
\item Considere o problema de inverter as arestas de um grafo dirigido.
\begin{itemize}
\item Como fazer isto com a representação por listas de adjacência? Qual a complexidade?
\item E com matrizes de adjacência?
\item Relacione este problema com o de calcular a matriz transposta de uma matriz.
\end{itemize}
\item Dado um grafo $G = (V, E)$, o grafo $G^2$ é tal que $G^2 = (V, E')$, onde
$E = \{ (u, w) \mid \exists v \cdot (u, v) \in E \land (v, w) \in E \}$.
\begin{itemize}
\item Como calcular $G^2$ com a representação por listas de adjacência? Qual a complexidade?
\item E com matrizes de adjacência?
\end{itemize}
\end{enumerate}

\end{frame}

\section{Busca em largura}

\begin{frame}
\frametitle{Busca em largura}

\begin{itemize}
\item algoritmo elementar para visitar os vértices
\item base para outros algoritmos importantes
\begin{itemize}
\item algoritmo de Dijkstra
\item algoritmo de Prim
\end{itemize}
\item entrada: $G = (V, E)$ e $s \in V$
\begin{itemize}
\item dirigido ou não
\end{itemize}
\item resultado:
\begin{itemize}
\item distância (menor número de arestas) de $s$ para cada $v \in V$
\item árvore ``em largura primeiro'' de raiz $s$ dos vértices alcançáveis
\end{itemize}
\item \alert{largura}: processa vértices de distância $k$ de $s$ antes dos
vértices de distância $k+1$ de $s$.
\end{itemize}
\end{frame}

\begin{frame}

\frametitle{Princípios}

\begin{enumerate}
\item Coloração dos vértices: 
\begin{itemize}
\item branco: a visitar
\item cinza: sendo visitado
\item preto: visitado
\item branco $\rightarrow$ cinza $\rightarrow$ preto
\item chegada a um vértice: branco $\rightarrow$ cinza
\item cinza: pode possuir vizinhos brancos
\item preto: todos os vértices adjacentes são da cor cinza ou preto.
\end{itemize}
\item Árvore de processamento:
\begin{itemize}
\item se $v$ foi descoberto a partir de $u$: $u$ é \alert{predecessor} de $v$
\end{itemize}
\end{enumerate}
\end{frame}

\begin{frame}
\frametitle{Implementação: dados}

\begin{itemize}
\item $\attrib{b}{color}$: cor do vértice $v$
\begin{itemize}
\item inicialmente: branco
\end{itemize}
\item $\attrib{v}{d}$: distância de $s$ a $v$ 
\begin{itemize}
\item inicialmente: $\infty$
\end{itemize}
\item $\attrib{v}{up}$: predecessor de $v$ 
\begin{itemize}
\item inicialmente: $\const{Nil}$
\end{itemize}
\end{itemize}
\end{frame}

\begin{frame}
\frametitle{Implementação: algoritmo}
\framesubtitle{Parte 1: inicialização}

\begin{codebox}
\Procname{$\proc{BFS}(G, s)$}
\zi \Comment{Inicialização}
\zi \For $v \in V - \{ s \}$ 
\zi \Do $\attrib{v}{color} \gets \const{White}$
\zi   $\attrib{v}{up} \gets \const{Nil}$
\zi   $\attrib{v}{d} \gets \infty$
\zi \End
\zi $\attrib{s}{color} \gets \const{Gray}$
\zi $\attrib{s}{up} \gets \const{Nil}$
\zi $\attrib{s}{d} \gets 0$
\zi $\proc{Enqueue}(Q, s)$
\end{codebox}
\end{frame}

\begin{frame}
\frametitle{Implementação: algoritmo}
\framesubtitle{Parte 2: processamento}

\begin{codebox}
\zi \Comment{Processamento}
\zi \While $\neg \proc{Empty}(Q)$
\zi \Do $u \gets \proc{Head}(Q)$
\zi   \For $v \in \attrib{u}{adj}$
\zi   \Do \If $\attrib{v}{color} \isequal \const{White}$
\zi     \Then $\attrib{v}{color} \gets \const{Gray}$
\zi       $\attrib{v}{d} \gets \attrib{u}{d}+1$
\zi       $\attrib{v}{up} \gets u$
\zi       $\proc{Enqueue}(Q, v)$
     \End
   \End
\zi   $\proc{Dequeue}(Q)$
\zi   $\attrib{u}{color} \gets \const{Black}$
\zi \End
\end{codebox}
\end{frame}

\begin{frame}
\frametitle{Algoritmo}
\framesubtitle{Ilustração}

\tikzstyle{todo}=[circle solidus,fill=white,minimum size=30pt,inner sep=0pt,draw,thick]
\tikzstyle{doing}=[circle solidus,fill=black!35,minimum size=30pt,inner sep=0pt]
\begin{center}
\begin{tikzpicture}[scale=.8, auto,swap]
% Draw a 7,11 network
% First we draw the white vertices
\foreach \name/ \id/ \pos in {{0/v/(1,1)}, {1/w/(3,1)}, {2/x/(5,1)}, {3/y/(7,1)}, 
  {4/r/(1, 3)}, {6/t/(5, 3)}, {7/u/(7, 3)}}
\node[todo] (\name) at \pos {$\id$ \nodepart{lower} $\infty$};
\foreach \name/ \id/ \pos in {{5/s/(3,3)}}
\node[doing] (5) at \pos {$\id$ \nodepart{lower} $0$};
% Connect vertices with edges and draw weights
\foreach \source/ \dest in {0/4, 1/2, 1/5, 1/6, 2/3, 3/7, 4/5, 6/7}
\path[edge] (\source) -- (\dest);
\node[rectangle, minimum size=20pt, draw] at (10,1) {$s$};
\node[rectangle split, rectangle split parts=8, rectangle split horizontal=true, draw] at (1,-1) {
$\mathsf{Nil}$
\nodepart{two}$\mathsf{Nil}$
\nodepart{three}$\mathsf{Nil}$
\nodepart{four}$\mathsf{Nil}$
\nodepart{five}$\mathsf{Nil}$
\nodepart{six}$\mathsf{Nil}$
\nodepart{seven}$\mathsf{Nil}$
\nodepart{eight}$\mathsf{Nil}$
};
\end{tikzpicture}
\end{center}
\end{frame}

\begin{frame}
\frametitle{Algoritmo}
\framesubtitle{Ilustração}

\tikzstyle{todo}=[circle solidus,fill=white,minimum size=30pt,inner sep=0pt,draw,thick]
\tikzstyle{doing}=[circle solidus,fill=black!35,minimum size=30pt,inner sep=0pt]
\tikzstyle{done}=[circle solidus,text=white,fill=black!90,minimum size=30pt,inner sep=0pt,draw,thick]
\begin{center}
\begin{tikzpicture}[scale=.8, auto,swap]
% Draw a 7,11 network
% First we draw the white vertices
\foreach \name/ \id/ \pos in {{0/v/(1,1)}, {2/x/(5,1)}, {3/y/(7,1)}, 
  {6/t/(5, 3)}, {7/u/(7, 3)}}
\node[todo] (\name) at \pos {$\id$ \nodepart{lower} $\infty$};
% Next we draw the gray vertices
\foreach \name/ \id/ \pos/ \dist in {{1/w/(3,1)/1},{4/r/(1, 3)/1}}
\node[doing] (\name) at \pos {$\id$ \nodepart{lower} $\dist$};
% Next we draw the black vertices
\foreach \name/ \id/ \pos/ \dist in {{5/s/(3,3)/0}}
\node[done] (\name) at \pos {$\id$ \nodepart{lower} $\dist$};
% Connect vertices with edges and draw weights
\foreach \source/ \dest in {0/4, 1/2, 1/5, 1/6, 2/3, 3/7, 4/5, 6/7}
\path[edge] (\source) -- (\dest);
\node[rectangle split, rectangle split parts=2, rectangle split horizontal=true, draw] at (10,1) {$w$ \nodepart{two} $r$};
\node[rectangle split, rectangle split parts=8, rectangle split horizontal=true, draw] at (1,-1) {
$\mathsf{s}$
\nodepart{two}$\mathsf{Nil}$
\nodepart{three}$\mathsf{Nil}$
\nodepart{four}$\mathsf{Nil}$
\nodepart{five}$\mathsf{Nil}$
\nodepart{six}$\mathsf{s}$
\nodepart{seven}$\mathsf{Nil}$
\nodepart{eight}$\mathsf{Nil}$
};
\end{tikzpicture}
\end{center}
\end{frame}

\begin{frame}
\frametitle{Algoritmo}
\framesubtitle{Ilustração}

\tikzstyle{todo}=[circle solidus,fill=white,minimum size=30pt,inner sep=0pt,draw,thick]
\tikzstyle{doing}=[circle solidus,fill=black!35,minimum size=30pt,inner sep=0pt]
\tikzstyle{done}=[circle solidus,text=white,fill=black!90,minimum size=30pt,inner sep=0pt,draw,thick]
\begin{center}
\begin{tikzpicture}[scale=.8, auto,swap]
% Draw a 7,11 network
% First we draw the white vertices
\foreach \name/ \id/ \pos in {{0/v/(1,1)}, {3/y/(7,1)}, 
  {7/u/(7, 3)}}
\node[todo] (\name) at \pos {$\id$ \nodepart{lower} $\infty$};
% Next we draw the gray vertices
\foreach \name/ \id/ \pos/ \dist in {{4/r/(1, 3)/1},{2/x/(5,1)/2},{6/t/(5, 3)/2}}
\node[doing] (\name) at \pos {$\id$ \nodepart{lower} $\dist$};
% Next we draw the black vertices
\foreach \name/ \id/ \pos/ \dist in {{1/w/(3,1)/1},{5/s/(3,3)/0}}
\node[done] (\name) at \pos {$\id$ \nodepart{lower} $\dist$};
% Connect vertices with edges and draw weights
\foreach \source/ \dest in {0/4, 1/2, 1/5, 1/6, 2/3, 3/7, 4/5, 6/7}
\path[edge] (\source) -- (\dest);
\node[rectangle split, rectangle split parts=3, rectangle split horizontal=true, draw] at (10,1) {$r$ \nodepart{two} $t$ \nodepart{three} $x$};
\node[rectangle split, rectangle split parts=8, rectangle split horizontal=true, draw] at (1,-1) {
$\mathsf{s}$
\nodepart{two}$\mathsf{Nil}$
\nodepart{three}$\mathsf{w}$
\nodepart{four}$\mathsf{Nil}$
\nodepart{five}$\mathsf{Nil}$
\nodepart{six}$\mathsf{s}$
\nodepart{seven}$\mathsf{w}$
\nodepart{eight}$\mathsf{Nil}$
};
\end{tikzpicture}
\end{center}
\end{frame}

\begin{frame}
\frametitle{Algoritmo}
\framesubtitle{Ilustração}

\tikzstyle{todo}=[circle solidus,fill=white,minimum size=30pt,inner sep=0pt,draw,thick]
\tikzstyle{doing}=[circle solidus,fill=black!35,minimum size=30pt,inner sep=0pt]
\tikzstyle{done}=[circle solidus,text=white,fill=black!90,minimum size=30pt,inner sep=0pt,draw,thick]
\begin{center}
\begin{tikzpicture}[scale=.8, auto,swap]
% Draw a 7,11 network
% First we draw the white vertices
\foreach \name/ \id/ \pos in {{3/y/(7,1)},{7/u/(7, 3)}}
\node[todo] (\name) at \pos {$\id$ \nodepart{lower} $\infty$};
% Next we draw the gray vertices
\foreach \name/ \id/ \pos/ \dist in {{2/x/(5,1)/2},{6/t/(5, 3)/2},{0/v/(1,1)/2}}
\node[doing] (\name) at \pos {$\id$ \nodepart{lower} $\dist$};
% Next we draw the black vertices
\foreach \name/ \id/ \pos/ \dist in {{4/r/(1, 3)/1},{1/w/(3,1)/1},{5/s/(3,3)/0}}
\node[done] (\name) at \pos {$\id$ \nodepart{lower} $\dist$};
% Connect vertices with edges and draw weights
\foreach \source/ \dest in {0/4, 1/2, 1/5, 1/6, 2/3, 3/7, 4/5, 6/7}
\path[edge] (\source) -- (\dest);
\node[rectangle split, rectangle split parts=3, rectangle split horizontal=true, draw] at (10,1) {$t$ \nodepart{two} $x$ \nodepart{three} $v$};
\node[rectangle split, rectangle split parts=8, rectangle split horizontal=true, draw] at (1,-1) {
$\mathsf{s}$
\nodepart{two}$\mathsf{Nil}$
\nodepart{three}$\mathsf{w}$
\nodepart{four}$\mathsf{Nil}$
\nodepart{five}$\mathsf{r}$
\nodepart{six}$\mathsf{s}$
\nodepart{seven}$\mathsf{w}$
\nodepart{eight}$\mathsf{Nil}$
};
\end{tikzpicture}
\end{center}
\end{frame}

\begin{frame}
\frametitle{Algoritmo}
\framesubtitle{Ilustração}

\tikzstyle{todo}=[circle solidus,fill=white,minimum size=30pt,inner sep=0pt,draw,thick]
\tikzstyle{doing}=[circle solidus,fill=black!35,minimum size=30pt,inner sep=0pt]
\tikzstyle{done}=[circle solidus,text=white,fill=black!90,minimum size=30pt,inner sep=0pt,draw,thick]
\begin{center}
\begin{tikzpicture}[scale=.8, auto,swap]
% Draw a 7,11 network
% First we draw the white vertices
\foreach \name/ \id/ \pos in {{3/y/(7,1)}}
\node[todo] (\name) at \pos {$\id$ \nodepart{lower} $\infty$};
% Next we draw the gray vertices
\foreach \name/ \id/ \pos/ \dist in {{2/x/(5,1)/2},{0/v/(1,1)/2},{7/u/(7, 3)/3}}
\node[doing] (\name) at \pos {$\id$ \nodepart{lower} $\dist$};
% Next we draw the black vertices
\foreach \name/ \id/ \pos/ \dist in {{6/t/(5, 3)/2},{4/r/(1, 3)/1},{1/w/(3,1)/1},{5/s/(3,3)/0}}
\node[done] (\name) at \pos {$\id$ \nodepart{lower} $\dist$};
% Connect vertices with edges and draw weights
\foreach \source/ \dest in {0/4, 1/2, 1/5, 1/6, 2/3, 3/7, 4/5, 6/7}
\path[edge] (\source) -- (\dest);
\node[rectangle split, rectangle split parts=3, rectangle split horizontal=true, draw] at (10,1) {$x$ \nodepart{two} $v$ \nodepart{three} $u$};
\node[rectangle split, rectangle split parts=8, rectangle split horizontal=true, draw] at (1,-1) {
$\mathsf{s}$
\nodepart{two}$\mathsf{Nil}$
\nodepart{three}$\mathsf{w}$
\nodepart{four}$\mathsf{u}$
\nodepart{five}$\mathsf{r}$
\nodepart{six}$\mathsf{s}$
\nodepart{seven}$\mathsf{w}$
\nodepart{eight}$\mathsf{Nil}$
};
\end{tikzpicture}
\end{center}
\end{frame}

\begin{frame}
\frametitle{Algoritmo}
\framesubtitle{Ilustração}

\tikzstyle{todo}=[circle solidus,fill=white,minimum size=30pt,inner sep=0pt,draw,thick]
\tikzstyle{doing}=[circle solidus,fill=black!35,minimum size=30pt,inner sep=0pt]
\tikzstyle{done}=[circle solidus,text=white,fill=black!90,minimum size=30pt,inner sep=0pt,draw,thick]
\begin{center}
\begin{tikzpicture}[scale=.8, auto,swap]
% Draw a 7,11 network
% No more white vertices
% Next we draw the gray vertices
\foreach \name/ \id/ \pos/ \dist in {{0/v/(1,1)/2},{7/u/(7, 3)/3},{3/y/(7,1)/3}}
\node[doing] (\name) at \pos {$\id$ \nodepart{lower} $\dist$};
% Next we draw the black vertices
\foreach \name/ \id/ \pos/ \dist in {{2/x/(5,1)/2},{6/t/(5, 3)/2},{4/r/(1, 3)/1},{1/w/(3,1)/1},{5/s/(3,3)/0}}
\node[done] (\name) at \pos {$\id$ \nodepart{lower} $\dist$};
% Connect vertices with edges and draw weights
\foreach \source/ \dest in {0/4, 1/2, 1/5, 1/6, 2/3, 3/7, 4/5, 6/7}
\path[edge] (\source) -- (\dest);
\node[rectangle split, rectangle split parts=3, rectangle split horizontal=true, draw] at (10,1) {$v$ \nodepart{two} $u$ \nodepart{three} $y$};
\node[rectangle split, rectangle split parts=8, rectangle split horizontal=true, draw] at (1,-1) {
$\mathsf{s}$
\nodepart{two}$\mathsf{Nil}$
\nodepart{three}$\mathsf{w}$
\nodepart{four}$\mathsf{u}$
\nodepart{five}$\mathsf{r}$
\nodepart{six}$\mathsf{s}$
\nodepart{seven}$\mathsf{w}$
\nodepart{eight}$\mathsf{x}$
};
\end{tikzpicture}
\end{center}
\end{frame}

\begin{frame}
\frametitle{Algoritmo}
\framesubtitle{Ilustração}

\tikzstyle{todo}=[circle solidus,fill=white,minimum size=30pt,inner sep=0pt,draw,thick]
\tikzstyle{doing}=[circle solidus,fill=black!35,minimum size=30pt,inner sep=0pt]
\tikzstyle{done}=[circle solidus,text=white,fill=black!90,minimum size=30pt,inner sep=0pt,draw,thick]
\begin{center}
\begin{tikzpicture}[scale=.8, auto,swap]
% Draw a 7,11 network
% No more white vertices
% Next we draw the gray vertices
\foreach \name/ \id/ \pos/ \dist in {{7/u/(7, 3)/3},{3/y/(7,1)/3}}
\node[doing] (\name) at \pos {$\id$ \nodepart{lower} $\dist$};
% Next we draw the black vertices
\foreach \name/ \id/ \pos/ \dist in {{0/v/(1,1)/2},{2/x/(5,1)/2},{6/t/(5, 3)/2},{4/r/(1, 3)/1},{1/w/(3,1)/1},{5/s/(3,3)/0}}
\node[done] (\name) at \pos {$\id$ \nodepart{lower} $\dist$};
% Connect vertices with edges and draw weights
\foreach \source/ \dest in {0/4, 1/2, 1/5, 1/6, 2/3, 3/7, 4/5, 6/7}
\path[edge] (\source) -- (\dest);
\node[rectangle split, rectangle split parts=2, rectangle split horizontal=true, draw] at (10,1) {$u$ \nodepart{two} $y$};
\node[rectangle split, rectangle split parts=8, rectangle split horizontal=true, draw] at (1,-1) {
$\mathsf{s}$
\nodepart{two}$\mathsf{Nil}$
\nodepart{three}$\mathsf{w}$
\nodepart{four}$\mathsf{u}$
\nodepart{five}$\mathsf{r}$
\nodepart{six}$\mathsf{s}$
\nodepart{seven}$\mathsf{w}$
\nodepart{eight}$\mathsf{x}$
};
\end{tikzpicture}
\end{center}
\end{frame}

\begin{frame}
\frametitle{Algoritmo}
\framesubtitle{Ilustração}

\tikzstyle{todo}=[circle solidus,fill=white,minimum size=30pt,inner sep=0pt,draw,thick]
\tikzstyle{doing}=[circle solidus,fill=black!35,minimum size=30pt,inner sep=0pt]
\tikzstyle{done}=[circle solidus,text=white,fill=black!90,minimum size=30pt,inner sep=0pt,draw,thick]
\begin{center}
\begin{tikzpicture}[scale=.8, auto,swap]
% Draw a 7,11 network
% No more white vertices
% Next we draw the gray vertices
\foreach \name/ \id/ \pos/ \dist in {{3/y/(7,1)/3}}
\node[doing] (\name) at \pos {$\id$ \nodepart{lower} $\dist$};
% Next we draw the black vertices
\foreach \name/ \id/ \pos/ \dist in {{7/u/(7, 3)/3},{0/v/(1,1)/2},{2/x/(5,1)/2},{6/t/(5, 3)/2},{4/r/(1, 3)/1},{1/w/(3,1)/1},{5/s/(3,3)/0}}
\node[done] (\name) at \pos {$\id$ \nodepart{lower} $\dist$};
% Connect vertices with edges and draw weights
\foreach \source/ \dest in {0/4, 1/2, 1/5, 1/6, 2/3, 3/7, 4/5, 6/7}
\path[edge] (\source) -- (\dest);
\node[rectangle, draw] at (10,1) {$y$};
\node[rectangle split, rectangle split parts=8, rectangle split horizontal=true, draw] at (1,-1) {
$\mathsf{s}$
\nodepart{two}$\mathsf{Nil}$
\nodepart{three}$\mathsf{w}$
\nodepart{four}$\mathsf{u}$
\nodepart{five}$\mathsf{r}$
\nodepart{six}$\mathsf{s}$
\nodepart{seven}$\mathsf{w}$
\nodepart{eight}$\mathsf{x}$
};
\end{tikzpicture}
\end{center}
\end{frame}

\begin{frame}
\frametitle{Algoritmo}
\framesubtitle{Ilustração}

\tikzstyle{todo}=[circle solidus,fill=white,minimum size=30pt,inner sep=0pt,draw,thick]
\tikzstyle{doing}=[circle solidus,fill=black!35,minimum size=30pt,inner sep=0pt]
\tikzstyle{done}=[circle solidus,text=white,fill=black!90,minimum size=30pt,inner sep=0pt,draw,thick]
\begin{center}
\begin{tikzpicture}[scale=.8, auto,swap]
% Draw a 7,11 network
% No more white vertices
% No more gray vertices
\foreach \name/ \id/ \pos/ \dist in {}
\node[doing] (\name) at \pos {$\id$ \nodepart{lower} $\dist$};
% Next we draw the black vertices
\foreach \name/ \id/ \pos/ \dist in {{3/y/(7,1)/3},{7/u/(7, 3)/3},{0/v/(1,1)/2},{2/x/(5,1)/2},{6/t/(5, 3)/2},{4/r/(1, 3)/1},{1/w/(3,1)/1},{5/s/(3,3)/0}}
\node[done] (\name) at \pos {$\id$ \nodepart{lower} $\dist$};
% Connect vertices with edges and draw weights
\foreach \source/ \dest in {0/4, 1/2, 1/5, 1/6, 2/3, 3/7, 4/5, 6/7}
\path[edge] (\source) -- (\dest);
\node[rectangle split, rectangle split parts=8, rectangle split horizontal=true, draw] at (1,-1) {
$\mathsf{s}$
\nodepart{two}$\mathsf{Nil}$
\nodepart{three}$\mathsf{w}$
\nodepart{four}$\mathsf{u}$
\nodepart{five}$\mathsf{r}$
\nodepart{six}$\mathsf{s}$
\nodepart{seven}$\mathsf{w}$
\nodepart{eight}$\mathsf{x}$
};
\end{tikzpicture}
\end{center}
\end{frame}

\begin{frame}
\frametitle{Complexidade}

\hspace*{-1cm}
\begin{tabular}{ll}
\begin{minipage}{.6\textwidth}
\begin{itemize}
\item dois laços aninhados
\begin{itemize}
\item um vértice entra em $Q$ quando é branco
\item é imediatemente alterado para cinza
\item logo entra no máximo uma vez em $Q$
\end{itemize}
\item a cada iteração do laço mais externo, um vértice é eliminado
\item logo, o laço mais externo é executado no máximo $|V|$ vezes,
\item o laço mais interno enumera as arestas de um nó
\item cada aresta é enumerada no máximo duas vezes
\item a complexidade é $O(|V|+|E|)$
\end{itemize}
\end{minipage}
&
\begin{minipage}{.35\textwidth}
\small
\begin{codebox}
\zi \Comment{Processamento}
\zi \While $\neg \proc{Empty}(Q)$
\zi \Do $u \gets \proc{Head}(Q)$
\zi   \For $v \in \attrib{u}{adj}$
\zi   \Do \If $\attrib{v}{color} \isequal \const{White}$
\zi     \Then $\attrib{v}{color} \gets \const{Gray}$
\zi       $\attrib{v}{d} \gets \attrib{u}{d}+1$
\zi       $\attrib{v}{up} \gets u$
\zi       $\proc{Enqueue}(Q, v)$
     \End
   \End
\zi   \proc{Dequeue}(Q)
\zi   $\attrib{u}{color} \gets \const{Black}$
\zi \End
\end{codebox}
\end{minipage}
\end{tabular}
\end{frame}

\begin{frame}
\frametitle{Correção}

\begin{quote}
É correto o algoritmo $\proc{BFS}$?
\end{quote}
\pause
Roteiro da demonstração:
\begin{enumerate}
\item Definições: distância de menor caminho, menor caminho.
\item Propriedade da distância de menor caminho.
\item Propriedade sobre a fila $\id{Q}$ no algoritmo $\proc{BFS}$
\item Propriedade sobre o atributo $\id{d}$ do algoritmo $\proc{BFS}$
\begin{itemize}
\item maior ou igual à distância de menor caminho
\item igual à distância de menor caminho
\end{itemize}
\item Teorema: correção de $\proc{BFS}$
\end{enumerate}

\end{frame}

\begin{frame}
\frametitle{Distância de menor caminho, menor caminho}
\framesubtitle{Correção}

\begin{definition}[Distância de menor caminho]
A distância de menor caminho $\delta(s, v)$ entre dois vértices $s$ e $v$
é o menor número de arestas entre todos os caminhos de $s$ até $v$, se 
existir algum caminho. Caso contrário é $\infty$.
\end{definition}

\begin{definition}[Menor caminho]
Um caminho entre $s$ e $v$ é um menor caminho se tem $\delta(s, v)$ arestas.
\end{definition}

\end{frame}

\begin{frame}
\frametitle{Propriedade da distância de menor caminho}
\framesubtitle{Correção}

\begin{lemma}[Lema da distância de menor caminho]
\label{lemma:dist-menor-caminho}
Seja $G = (V, E)$ um grafo (dirigido ou não) e $s \in V$. 
Para qualquer aresta $(u, v) \in E$, temos: \alert{$\delta(s, v) \le \delta(s, u)+1$}.
\end{lemma}
\pause
\begin{center}
\begin{tikzpicture}[scale=.7, auto,swap]
\node[vertex] (s) at (1,2) {$s$};
\node[vertex] (u) at (0,0) {$u$};
\node[vertex] (v) at (2,0) {$v$};
\path[edge] (u) -- (v);
\path[path] (s) -- (u) node [left,align=right,midway] {$\delta(s, u)$};
\path[path] (s) -- (v) node [right,align=left,midway] {$\delta(s, v)$};
\end{tikzpicture}
\end{center}
\begin{proof}
  Se há um caminho de $s$ até $u$, também há um até $v$. 
  O menor caminho até $v$ não pode ser maior que o menor caminho até $u$ seguido de $(u, v)$.
  
  Se não existe caminho de $s$ até $u$, então $\delta(s, v) = \delta(s, u) = \infty$.
\end{proof}

\end{frame}

\begin{frame}
\frametitle{Propriedade da fila no algoritmo $\proc{BFS}$}
\framesubtitle{Correção}

Mostramos então uma propriedade sobre a fila $Q$

\begin{lemma}[fila no algoritmo $\proc{BFS}$]
Durante a execução de $\proc{BFS}(G, s)$, a fila $Q = \langle v_1, v_2, \ldots v_r \rangle$
($v_1$ cabeça). Então $\attrib{v_r}{d} \le \attrib{v_1}{d}+1$ e $\attrib{v_i}{d} \le \attrib{v_{i+1}}{d}$, para $i \in 1 \twodots r-1$.
\end{lemma}
\pause
\begin{proof}
\begin{description}
\item[] Indução sobre operações da fila.
\only<2>{
\item[base] $Q = \langle s \rangle$ e a propriedade é satisfeita.
}
\only<3>{
\item[em geral] prova por caso ($\proc{Enqueue}$ e $\proc{Dequeue}$)
}
\only<4>{
\item[$\proc{Dequeue}$] A nova cabeça é $v_2$. Por hipótese, temos $\attrib{v_r}{d} \le \attrib{v_1}+1 \le \attrib{v_2}+1$, pela hipótese de indução. As outras desigualdades permanecem.
}
\only<5>{
\item[$\proc{Enqueue}$] Um vértice $v$ é inserido, tornando-se $v_{r+1}$. Neste caso, a variável $u$ tem
como valor $v_1$. Logo $(v_1, v_{r+1})$ é uma aresta, e $\attrib{v_{r+1}}{d} = \attrib{v_1}{d} + 1$.
Temos que $\attrib{v_r}{d} \le \attrib{v_1}{d}+1 = \attrib{v_{r+1}}{d}$. As outras desigualdades permanecem.
}
\end{description}
\end{proof}

\end{frame}

\begin{frame}
\frametitle{Propriedade do atributo $\id{d}$ no algoritmo $\proc{BFS}$}
\framesubtitle{Correção}

Prova que $\delta(s, v) \le \attrib{v}{d}$ \ldots
\begin{lemma}[atributo $d$ no algoritmo $\proc{BFS}$]
Seja $G = (V, E)$ um grafo (dirigido ou não), $s \in V$ um vértice
qualquer. Aplicamos o algoritmo $\proc{BFS}(G, s)$. 

Quando o algoritmo termina, para qualquer $v$: $\attrib{v}{d} \ge \delta(s, v)$.
\end{lemma}
\pause
\begin{proof}
  Por indução no número de aplicações de $\proc{Enqueue}$.

  \begin{enumerate}
  \item[base] (após $\proc{Enqueue}(s)$) $\attrib{v}{d} = 0 = \delta(s, s)$, e ,
    se $v \neq s$, $\attrib{v}{d} = \infty \ge \delta(s, v)$.
  \item[caso geral] seja $v$ um vértice branco alcançado na visita de $u$.
    \begin{itemize}
      \item por hipótese de indução, $\attrib{u}{d} \ge \delta(s, u)$
      \item o algoritmo diz que $\attrib{v}{d} \gets \attrib{u}{d} + 1$
      \item logo $\attrib{v}{d} \ge \delta(s, u)+1$
      \item e, pelo lema da distância de menor caminho, $\attrib{v}{d} \ge \delta(s, v)$.
    \end{itemize}
  \item[] o valor de $\attrib{v}{d}$ não é mais alterado
  \end{enumerate}
\end{proof}

\end{frame}

\begin{frame}
\frametitle{O algoritmo $\proc{BFS}$ calcula as distâncias de menor caminho}
\framesubtitle{Correção}

Mostramos enfim que $\proc{BFS}$ calcula as distâncias de menor caminho.

\begin{theorem}[Correção do algoritmo $\proc{BFS}$]
Durante a execução de $\proc{BFS}(G, s)$, para cada vértice $v \in V$ alcançável a partir de $s$:
\begin{itemize}
\item $v$ é processado,
\item no término, $\attrib{v}{d} = \delta(s, v)$.
\item se $v \neq s$, um dos menores caminhos de $s$ até $v$ é composto por um dos menores caminhos de $s$ até
$\attrib{v}{up}$ e pela aresta $(\attrib{v}{up}, v)$.
\end{itemize}
\end{theorem}
\end{frame}

\begin{frame}
\frametitle{Prova do teorema principal}
\framesubtitle{Correção}

\begin{proof}[Por casos]

\only<1>{
\begin{itemize}
\item $v$ não é alcançável a partir de $s$
\item $v$ é alcançável a partir de $s$
\end{itemize}
}
\only<2>{
$v$ não é alcançável a partir de $s$:

\begin{itemize}

\item $\delta(s, v) = \infty$.

\item Pelo lema do atributo $\id{d}$, $\attrib{v}{d} \ge \delta(s, v)$,

\item como não é possível atribuir $\infty$ a $\attrib{v}{d}$ no laço principal de $\proc{BFS}$,

\item logo o laço só processa nós alcançáveis a partir de $s$.

\item Então, no término do algoritmo $\attrib{v}{d} = \infty = \delta(s, v)$.
\end{itemize}
}
\only<3>{
$v$ é alcançável a partir de $s$:

\begin{itemize}

\item Seja $V_k = \{ v \in V \mid \delta(s, v) = k \}$

\item prova por indução sobre $k$.

\item propriedade sobre $k$: se $v \in V_k$, existe um ponto durante a execução tal que
\begin{enumerate}
\item $\attrib{v}{color} = \const{Gray}$
\item $\attrib{v}{d}$ é atribuído $k$
\item se $v \neq s$, então $\attrib{v}{up}$ é atribuído $u$, onde $u \in V_{k-1}$
\item $v$ é inserido em $Q$
\end{enumerate}
\item obs.: se esse ponto existir, necessariamente é unico ($\id{color}$,
$\id{d}$ e $\id{up}$ são atribuídos no máximo uma vez).
\end{itemize}
}

\end{proof}
\end{frame}

\begin{frame}
\frametitle{Prova do teorema principal}
\framesubtitle{Correção}

Caso de base: $k = 0$

\begin{itemize}

\item Por definição, $V_0 = \{ s \}$. 

\item Na fase de inicialização do algoritmo:

\begin{enumerate}
\item $\attrib{s}{color}$ é atribuído $\const{Gray}$
\item $\attrib{s}{d}$ é atribuído $0$
\item $s$ é inserido em $Q$
\end{enumerate}

\end{itemize}
\end{frame}

\begin{frame}
\frametitle{Prova do teorema principal}
\framesubtitle{Correção}

Em geral: $k > 0$

\begin{itemize}
\item $Q \neq \langle \rangle$ antes do término.
\item pelo lema da fila, se os vértices são inseridos em ordem $v_1, v_2, \ldots v_r$ então
$\attrib{v_1}{d} \le \attrib{v_2}{d} \le \ldots \attrib{v_r}{d}$.
\item seja $v \in V_k$, $k \ge 1$,
\item $v$ só pode ser encontrado quando todos os vértices de $V_{k-1}$ tiverem entrados na fila.
\item $v \in V_k \Rightarrow \delta(s, v) = k$, logo há um menor caminho de $k$ arestas entre $s$ e $v$, passando por $u \in V_{k-1}$, e $(u, v) \in E$.
\item seja $u$ o primeiro desses vértices que foi encontrado: $u$ deve ter entrado em $Q$
\item $u$ necessariamente torna-se a cabeça de $Q$ em um momento
\item $v$ deve ser descoberto quando processa as arestas de $u$, e
\begin{enumerate}
\item $\attrib{v}{color}$ é atribuído $\const{Gray}$
\item $\attrib{v}{d}$ é atribuído $\attrib{u}{d}+1 = \delta(s, u)+1 = (k-1)+1 =k$
\item $\attrib{v}{up}$ é atribuído $u$
\item $v$ é inserido em $Q$
\end{enumerate}
\end{itemize}

\end{frame}

\begin{frame}
\frametitle{Prova do teorema principal}
\framesubtitle{Correção}

\begin{itemize}
\item Isso conclui a prova por indução
\item Além disto, quando $v \in V_k$, então $\attrib{v}{up} \in V_{k-1}$.
\item podemos formar um menor caminho entre $s$ e $v$ com o menor
caminho entre $s$ e $\attrib{v}{up}$, seguido da aresta $(\attrib{v}{up}, v)$.
\end{itemize}

\end{frame}

\begin{frame}
\frametitle{$\proc{BFS}$ e árvore de processamento em largura}

\begin{itemize}
\item $\proc{BFS}$ calcula também uma árvore, formada pelas arestas
$(\attrib{v}{up}, v)$.
\item Esta árvore é uma árvore de processamento em largura.
\end{itemize}
\begin{definition}[Sub-grafo dos predecessores]
Seja $G=(V, E)$ um grafo, o sub-grafo dos predecessores é o
grafo $G_\pi = (V_\pi, E_\pi)$, onde
\begin{itemize}
\item $V_\pi = \{s\} \cup \{ v \in V \mid \attrib{v}{up} \neq \const{Nil} \}$, e
\item $E_\pi = \{ (\attrib{v}{up}, v) \mid v \in V_\pi - \{ s \} \}$
\end{itemize}
\end{definition}

\begin{definition}[Árvore de processamento em largura]
Um sub-grafo dos predecessores é uma árvore de processamento em largura
quando $V_\pi$ é composto por $s$ e por todos os vértices alcançáveis
$v \in V_\pi$, existe um único caminho entre $s$ e $v$ em $G_\pi$. Este
caminho é um menor caminho entre $s$ e $v$ em $G$.
\end{definition}

\end{frame}

\begin{frame}
\frametitle{Exercícios}
\begin{enumerate}
\item Escreva um algoritmo que, dado um grafo $G=(V, E)$, dois vértices $u$ e
  $v$, imprime um menor caminho entre $u$ e $v$, se esse existir.
\item Qual seria a complexidade de $\proc{BFS}$, caso seja adotada uma matriz de
  adjacência, uma vez feitas as adaptações necessárias?
\item Mostre que o valor atribuído a $\attrib{u}{d}$ é independente da ordem dos
  vértices nas listas de adjacência.
\item Um grafo é bipartido se o conjunto de vértices pode ser separado em dois
  conjuntos, digamos $V_1$ e $V_2$, tal que todas as arestas relacionam um
  vértice de $V_1$ com um vértice de $V_2$.

  Forneça um algoritmo eficiente que testa se um grafo é bipartido.
\end{enumerate}
\end{frame}

\section{Busca em profundidade}

\begin{frame}
\frametitle{Busca em profundidade}

\begin{itemize}
\item algoritmo elementar para visitar os vértices
\item base para outros algoritmos importantes
\item entrada: $G = (V, E)$
\begin{itemize}
\item dirigido ou não
\end{itemize}
\item resultado:
\begin{itemize}
\item floresta de árvores ``em profundidade primeiro'' dos vértices de $G$
\end{itemize}
\item \alert{profundidade}: prioriza vértices mais distantes do ponto de partida.
\end{itemize}
\end{frame}

\begin{frame}
\frametitle{Princípios}

\begin{itemize}
\item inicia em um vértice
\item visita os vértices alcançáveis, indo o mais ``profundo'' possível
\item volte e explora outro caminho repetidamente
\item recomeça a busca a partir de outro vértice não visitado
\item gera uma floresta
\item cada vértice $v$ tem atributo $\attrib{v}{up}$.
\item Sub-grafo dos predecessores: $G_\pi = \{ (\attrib{v}{up}, v) \mid v \in V \land \attrib{v}{up} \neq \const{Nil} \}$.
\item A mesma convenção de coloração é utilizada
\item Cada vértice tem dois marcadores: 
\begin{itemize}
\item $\attrib{v}{d}$: quando $v$ é descoberto
\item $\attrib{v}{f}$: quando $f$ é finalizado
\item $\attrib{v}{color} = \const{Black} \Rightarrow \attrib{v}{d} < \attrib{v}{f}$
\item variável global $\id{time}$ marca as etapas
\end{itemize}
\end{itemize}
\end{frame}

\begin{frame}
\frametitle{Algoritmo}

\begin{codebox}
\Procname{$\proc{Tick}$}
\zi $\id{time} \gets \id{time} + 1$
\zi \Return $\id{time}$
\end{codebox}

\begin{codebox}
\Procname{$\proc{DFS}(G)$}
\zi \For $v \in \attrib{G}{V}$
\zi \Do $\attrib{v}{color} \gets \const{White}$
\zi   $\attrib{v}{up} \gets \const{Nil}$
    \End
\zi $\id{time} \gets 0$
\zi \For $v \in \attrib{G}{V}$
\zi \Do \If $\attrib{v}{color} \isequal \const{White}$
\zi   \Then $\proc{DFS-visit}(v)$
      \End
    \End
\end{codebox}

\end{frame}

\begin{frame}
\frametitle{Algoritmo}

\begin{codebox}
\Procname{$\proc{DFS-visit}(v)$}
\zi $\attrib{v}{color} \gets \const{Gray}$
\zi $\attrib{v}{d} \gets \proc{Tick}$
\zi \For $w \in \attrib{v}{adj}$
\zi \Do \If $\attrib{w}{color} \isequal \const{White}$
\zi   \Then $\attrib{w}{up} \gets v$
\zi   $\proc{DFS-visit}(w)$
      \End
    \End
\zi $\attrib{v}{color} \gets \const{Black}$
\zi $\attrib{v}{f} \gets \proc{Tick}$
\end{codebox}

\end{frame}

\begin{frame}
\frametitle{Ilustração}

\only<1>{
\begin{center}
\includegraphics[width=.8\textwidth]{fig/dfs-1.pdf}
\end{center}
}

\only<2>{
\begin{center}
\includegraphics[width=.8\textwidth]{fig/dfs-2.pdf}
\end{center}
}

\only<3>{
\begin{center}
\includegraphics[width=.8\textwidth]{fig/dfs-3.pdf}
\end{center}
}

\only<4>{
\begin{center}
\includegraphics[width=.8\textwidth]{fig/dfs-4.pdf}
\end{center}
}

\only<5>{
\begin{center}
\includegraphics[width=.8\textwidth]{fig/dfs-5.pdf}
\end{center}
}

\only<6>{
\begin{center}
\includegraphics[width=.8\textwidth]{fig/dfs-6.pdf}
\end{center}
}

\only<7>{
\begin{center}
\includegraphics[width=.8\textwidth]{fig/dfs-7.pdf}
\end{center}
}

\only<8>{
\begin{center}
\includegraphics[width=.8\textwidth]{fig/dfs-8.pdf}
\end{center}
}

\only<9>{
\begin{center}
\includegraphics[width=.8\textwidth]{fig/dfs-9.pdf}
\end{center}
}

\only<10>{
\begin{center}
\includegraphics[width=.8\textwidth]{fig/dfs-10.pdf}
\end{center}
}

\only<11>{
\begin{center}
\includegraphics[width=.8\textwidth]{fig/dfs-11.pdf}
\end{center}
}

\only<12>{
\begin{center}
\includegraphics[width=.8\textwidth]{fig/dfs-12.pdf}
\end{center}
}

\only<13>{
\begin{center}
\includegraphics[width=.8\textwidth]{fig/dfs-13.pdf}
\end{center}
}

\only<14>{
\begin{center}
\includegraphics[width=.8\textwidth]{fig/dfs-14.pdf}
\end{center}
}

\only<15>{
\begin{center}
\includegraphics[width=.8\textwidth]{fig/dfs-15.pdf}
\end{center}
}

\only<16>{
\begin{center}
\includegraphics[width=.8\textwidth]{fig/dfs-16.pdf}
\end{center}
}

\only<17>{
\begin{center}
\includegraphics[width=.8\textwidth]{fig/dfs-17.pdf}
\end{center}
}

\end{frame}

\begin{frame}

\frametitle{Complexidade do algoritmo}

\only<1>{
\begin{codebox}
\Procname{$\proc{DFS}(G)$}
\zi \For $v \in \attrib{G}{V}$ \assert{\Comment $\Theta(|V|)$}
\zi \Do $\attrib{v}{color} \gets \const{White}$
\zi   $\attrib{v}{up} \gets \const{Nil}$
    \End
\zi $\id{time} \gets 0$
\zi \For $v \in \attrib{G}{V}$ \assert{\Comment $\Theta(|V|)$}
\zi \Do \If $\attrib{v}{color} \isequal \const{White}$
\zi   \Then $\proc{DFS-visit}(v)$
    \End
  \End
\end{codebox}
}
\only<2>{
\begin{codebox}
\Procname{$\proc{DFS-visit}(v)$}
\zi \assert{\Comment{$\Theta(|V|)$ chamadas}}
\zi $\attrib{v}{color} \gets \const{Gray}$
\zi $\attrib{v}{d} \gets \proc{Tick}$
\zi \For $w \in \attrib{v}{adj}$ \assert{\Comment{total: $\Theta(|E|)$}}
\zi \Do \If $\attrib{w}{color} \isequal \const{White}$
\zi   \Then $\attrib{w}{up} \gets v$
\zi   $\proc{DFS-visit}(w)$
      \End
    \End
\zi $\attrib{v}{color} \gets \const{Black}$
\zi $\attrib{v}{f} \gets \proc{Tick}$
\end{codebox}
}
\only<3>{
$$
\Theta(|V|+|E|)
$$
}

\end{frame}

\begin{frame}
\frametitle{Propriedades}

\begin{enumerate}
\item a relação predecessor forma uma floresta de árvores
\item estrutura de aninhamento da busca: teorema dos parênteses
\item condição necessária e suficiente de alcançabilidade: teorema do caminho branco
\item classificação das arestas na busca em profundidade
\item busca em profundidade em grafos não dirigidos
\end{enumerate}

\end{frame}

\begin{frame}
\frametitle{A relação predecessor}

\begin{itemize}
\item O grafo dos predecessores $G_\pi$ é uma floresta de árvores.

\item as arestas correspondem às chamadas de $\proc{DFS-visit}$

\end{itemize}

\end{frame}

\begin{frame}
\frametitle{Estrutura aninhada}

\begin{itemize}

\item Etapas de descoberta e finalização formam uma estrutura de parênteses.

\begin{itemize}

\item $\attrib{v}{d} \gets ...$ $\Longrightarrow$ $(v$

\item $\attrib{v}{f} \gets ...$ $\Longrightarrow$ $v)$

\item exemplo $(u (v (y (x x) y) v) u) (w (z z) w)$

\includegraphics[width=.4\textwidth]{fig/dfs-1}

\end{itemize}

\end{itemize}

\end{frame}

\begin{frame}
\frametitle{Teorema dos parênteses}

\begin{theorem}[Teorema dos parênteses]
No percurso em profundidade de um grafo $G$, para qualquer par de vértices $u$ e $v$, uma
das três condições seguintes é satisfeita:
\begin{enumerate}
\item os intervalos $[\attrib{u}{d}, \attrib{u}{f}]$ e $[\attrib{v}{d}, \attrib{v}{f}]$ não têm superposição;
\item o intervalo $[\attrib{u}{d}, \attrib{u}{f}]$ é inteiramente contido em $[\attrib{v}{d}, \attrib{v}{f}]$;
\item o intervalo $[\attrib{u}{d}, \attrib{u}{f}]$ contem inteiramente $[\attrib{v}{d}, \attrib{v}{f}]$.
\end{enumerate}
\end{theorem}

\begin{corollary}[Aninhamento dos intervalos dos descendentes]
Se $v$ é um descendente próprio de $u$ na floresta de busca em profundidade, então
$$\attrib{u}{d} < \attrib{v}{d} < \attrib{v}{f} < \attrib{u}{f}$$
\end{corollary}

\end{frame}

\begin{frame}
\frametitle{Teorema dos parênteses}
\framesubtitle{Demonstração}

\begin{proof}
Por casos:
\begin{enumerate}
\item se $\attrib{u}{d} < \attrib{v}{d}$
\begin{enumerate}
\item se $\attrib{v}{d} < \attrib{u}{f}$: 
\begin{itemize}
\item $v$ é descoberto quando $u$ é cinza, logo $v$ é descendente de $u$.
\item $v$ é mais recente que $u$ e as arestas saindo de $v$ são processadas 
  antes de finalizar as de $u$, logo $\attrib{v}{f} < \attrib{u}{f}$. 
\item O intervalo para $v$ é contido no intervalo para $u$.
\end{itemize}
\item se $\attrib{u}{f} < \attrib{v}{d}$: 
\begin{itemize}
\item $\attrib{v}{d} < \attrib{v}{f}$
\item o intervalo para $u$ vem antes do intervalo para $v$. 
\item não há sobreposição.
\end{itemize}
\end{enumerate}
\item caso simétrico: raciocínio simétrico
\end{enumerate}
\end{proof}

\end{frame}

\begin{frame}
\frametitle{Teorema do caminho branco}

\begin{theorem}[Teorema do caminho branco]
Na floresta de profundidade de um grafo, dirigido ou não, o vértice $v$ é descendente de $u$
s e somente se na etapa $\attrib{u}{d}$, $v$ pode ser alcançado a partir de $u$ por um caminho
composto apenas por vértices brancos.
\end{theorem}

\pause
\begin{proof}
\only<2>{
$\Rightarrow$ 
\begin{itemize}
\item $v$ é um descendente de $u$, 
\item $w$ é um vértice no caminho de $u$ até $v$,
\item pelo corolário do aninhamento, $\attrib{u}{d} < \attrib{w}{d}$.
\item logo, na etapa $\attrib{w}{d}$, $w$ é branco.
\end{itemize}
}
\only<3>{
$\Leftarrow$
\begin{itemize}
\item na etapa $\attrib{u}{d}$, $v$ é alcançável por um caminho branco a partir de $u$
\item supondo que $v$ não é descendente de $u$ na árvore em profundidade
\item supondo que os demais vértices no caminho tornam-se descendentes na árvore em profundidade
\item seja $w$ o predecessor de $v$
\item temos que $\attrib{w}{f} < \attrib{u}{f}$
\item $v$ deve ser processado após $u$, mas antes que $w$ seja finalizado
\item logo $\attrib{u}{d} < \attrib{v}{d} < \attrib{w}{f} < \attrib{u}{f}$
\item pelo teorema dos parênteses, o intervalo para $v$ é contido no intervalo para $f$
\item pelo corolário do aninhamento, $v$ deve ser um descendente de $u$
\end{itemize}
}
\end{proof}

\end{frame}

\begin{frame}
\frametitle{Classificação das arestas}

\begin{itemize}
\item arestas de árvore: são as arestas $(u, v)$ processadas em $\proc{DFS-visit}$ tais que $v$ é encontrado pela primeira vez. \only<3>{\alert{$\attrib{v}{color} = \const{White}$}}
\item arestas para trás: são as arestas processadas em $\proc{DFS-visit}$ que conectam $v$ a um ancestro de $v$
\only<4>{\alert{$\attrib{v}{color} = \const{Gray}$}}
\item arestas para frente: são as arestas processadas em $\proc{DFS-visit}$ que conectam $v$ a um descendente de $v$ já visitado \only<5>{\alert{$\attrib{v}{color} = \const{Black}$}} \only<5-6>{\alert{$\attrib{u}{d} < \attrib{v}{d}$}}
\item arestas cruzadas: as demais arestas. \only<5-6>{\alert{$\attrib{v}{color} = \const{Black}$}} \only<6>{\alert{$\attrib{u}{d} > \attrib{v}{d}$}}
\begin{itemize}
\item entre vértices da mesma árvore, tais que nenhuma é ancestro de outra;
\item entre vértices de árvores diferentes.
\end{itemize}
\end{itemize}
\pause
\alert{Como o algoritmo $\proc{DFS-visit}$ pode identificar o tipo de uma aresta?}
\end{frame}

\begin{frame}
\frametitle{Arestas em grafos não dirigidos}

\begin{theorem}
Na busca em profundidade de um grafo não dirigido, todas as arestas são de árvore ou para trás.
\end{theorem}

\begin{proof}
\begin{itemize}
\item Seja $(u, v)$ uma aresta em um grafo não dirigido.
\item Se $\attrib{u}{d} < \attrib{v}{d}$:
\begin{itemize}
\item Então $v$ deve ser finalizado antes de $u$, pois $(u, v)$ pertence à $\attrib{u}{adj}$. 
\item Se $(u, v)$ é processada primeiro em $\proc{BFS-visit}(u)$, então é de árvore.
\item Se $(u, v)$ é processada primeiro em $\proc{BFS-visit}(v)$, então é para trás.
\end{itemize}
\item Se $\attrib{v}{d} < \attrib{u}{d}$: argumento simétrico.
\end{itemize}
\end{proof}

\end{frame}

\begin{frame}
\frametitle{Exercícios}

\begin{enumerate}
\item Desenha uma tabela 3 por 3 com $\id{White}$, $\id{Gray}$ $\id{Black}$ como
  cabeçalho das linhas e colunas.  

  Indique em $i, j$ se $\proc{DFS-visit}$ pode encontrar uma aresta entre
  vértices da cor $i$ e $j$ em um grafo dirigido, e o tipo de aresta correspondente.

  Repita o exercício para grafos não dirigidos.

\item Considere a afirmação: Em um grafo dirigido, se há um caminho
de $u$ para $v$, e $\attrib{u}{d} < \attrib{v}{d}$ em uma busca em profundidade,
então $v$ será um descendente de $u$ na floresta correspondente.

Dê um contra-exemplo que invalida esta proposição.

\item Altere $\proc{DFS-visit}$ para imprimir as arestas visitadas e seus tipos. 

É necessário modificações para tratar grafos não dirigidos?

\item $u$ é um vértice com arestas entrando e saindo. Em um percurso em
  profundidade, $u$ encontra-se o único vértice de uma das árvores da floresta
  produzida. Como isto é possível?
\end{enumerate}

\end{frame}

\end{document}
