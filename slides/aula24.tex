\documentclass{beamer}
\setbeamertemplate{footline}[frame number]

\usepackage[utf8x]{inputenc}
\usepackage[brazil,british]{babel}
\usetheme{default} 
\usecolortheme{beaver}
\newtranslation[to=brazil]{Theorem}{Teorema}
\newtranslation[to=brazil]{Definition}{Definição}
\newtranslation[to=brazil]{Example}{Exemplo}
\newtranslation[to=brazil]{Problem}{Exercício}
\newtranslation[to=brazil]{Solution}{Resolução}
\newtranslation[to=brazil]{Lemma}{Lema}
\newtranslation[to=brazil]{Corollary}{Corolário}
\logo{\includegraphics[width=1cm]{../img/logo-ppgsc-icon-text.png}}

\usepackage{graphicx}
\usepackage{clrscode3e}
\usepackage{hyperref}

\usepackage{pgf}
\usepackage{tikz}
\newcommand{\assert}[1]{\textcolor{blue}{#1}}

\usepackage{xspace}

\newcommand{\semester}[0]{2015.2}


\title{Aula 24: Grafos: algoritmos elementares (II)}
\author{David Déharbe \\
  Programa de Pós-graduação em Sistemas e Computação \\
  Universidade Federal do Rio Grande do Norte \\
  Centro de Ciências Exatas e da Terra \\
  Departamento de Informática e Matemática Aplicada}
\date{18 de maio de 2015}

\begin{document}
\selectlanguage{brazil}

\begin{frame}
  \titlepage
\end{frame}

\begin{frame}
  \frametitle{Plano}
  \tableofcontents

Referência: Cormen, cap 23.
\end{frame}

\section{Ordenação topológica}

\begin{frame}
\frametitle{Ordenação topológica}

\begin{itemize}
\item Grafos dirigidos acíclicos indicam uma relação de precedência
\begin{itemize}
\item $u \prec v$: $u$ vem antes de $v$, aresta $(u, v)$
\item relação parcial
\item exemplos: pré-requisitos entre componentes curriculares, entre tarefas, etc.
\end{itemize}
\item Ordenação topológica:
\begin{itemize}
\item entrada: uma relação de precedência
\item saída: um ``escalonamento'' dos vértices que respeita a precedência
\end{itemize}
\item exemplo: grade curricular
\end{itemize}
\end{frame}

\begin{frame}
\frametitle{Definição}

\begin{definition}[Grafo dirigido acíclico, DAG]
Um \emph{grafo dirígido acíclico}, ou \emph{DAG}\footnote{Do inglês \textit{Directed Acyclic Graph}.}  é um grafo dirigido $G$ tal que se existe uma aresta dirigda de $(u, v)$, então não existe caminho de $v$ até $u$.
\end{definition}

Não há cíclo em DAGs.

\begin{center}
\includegraphics[height=.5\textheight]{fig/dag-1.pdf}
\end{center}

\end{frame}

\begin{frame}
\frametitle{Exemplo}
\framesubtitle{Ordenação topológica}
\begin{center}
\includegraphics[height=.5\textheight]{fig/dag-1.pdf}
\end{center}

\begin{center}
\includegraphics[height=.1\textheight]{fig/dag-2.pdf}
\end{center}

\end{frame}

\begin{frame}
\frametitle{Algoritmo}
\framesubtitle{Ordenação topológica}

\begin{itemize}
\item Aplicar $\proc{DFS}$;
\item Retornar a lista dos vértices em ordem decrescente do atributo $\id{f}$.
\end{itemize}

\begin{codebox}
\Procname{$\proc{Topological-Sort}(G)$}
\zi aplique $\proc{DFS}$ a $G$
\zi inserindo cada $v$ na cabeça de uma lista quando é finalizado
\zi \Return a lista dos vértices
\end{codebox}

\end{frame}

\begin{frame}
\frametitle{Algoritmo detalhado}
\framesubtitle{Ordenação topológica}

\begin{codebox}
\Procname{$\proc{Topological-Sort}(G)$}
\zi \For $v \in \attrib{G}{V}$
\zi \Do $\attrib{v}{visited} \gets \const{False}$
    \End
\zi $l \gets \const{Empty-List}$
\zi \For $v \in \attrib{G}{V}$
\zi \Do \If $\neg \attrib{v}{visited}$
\zi   \Then $\proc{Topological-Sort-Visit}(v)$
      \End
    \End
\zi \Return $l$
\end{codebox}

\begin{codebox}
\Procname{$\proc{Topological-Sort-Visit}(v, l)$}
\zi $\attrib{v}{visited} \gets \const{True}$
\zi \For $w \in \attrib{v}{adj}$
\zi \Do \If $\neg \attrib{w}{visited}$
\zi   \Then $\proc{Topological-Sort-Visit}(v, l)$
      \End
    \End
\zi $l \gets \proc{Push-Front}(v, l)$
\end{codebox}

\end{frame}

\begin{frame}
\frametitle{Algoritmo}
\framesubtitle{Exemplo}

\begin{center}
\includegraphics[height=.5\textheight]{fig/dag-1.pdf}
\end{center}

\end{frame}

\begin{frame}
\frametitle{Complexidade}
\framesubtitle{Ordenação topológica}

\begin{itemize}
\item $\proc{DFS}$: $\Theta(|V|+|E|)$
\item + $\Theta(1)$ cada inserção de vértice $\equiv$ $\Theta(|V|)$
\item = $\Theta(|V|+|E|)$
\end{itemize}
\end{frame}

\begin{frame}
\frametitle{Correção}
\framesubtitle{Ordenação topológica}

\begin{enumerate}
\item Lema: caracterização de DAG por arestas
\item Teorema: correção do algoritmo proposto
\end{enumerate}

\end{frame}

\begin{frame}
\frametitle{Aciclicidade e arestas}

\begin{lemma}[Arestas e DAG]
Um grafo dirigido $G = (V, E)$ é acíclico (um DAG) se e somente se qualquer aplicação de $\proc{DFS}(G)$ encontra nenhuma aresta de volta.
\end{lemma}
\pause
Plano de prova
\begin{itemize}
\item ($\Leftarrow$) nenhuma aresta de volta $\Rightarrow$ nenhum ciclo
\item ($\Rightarrow$) nenhum ciclo $\Rightarrow$ nenhuma aresta de volta
\end{itemize}
\end{frame}

\begin{frame}
\frametitle{Aciclicidade e arestas}
\framesubtitle{Demonstração}

\begin{proof}
\only<1>{
\begin{tabular}{c}
($\Leftarrow$) nenhuma aresta de volta $\Rightarrow$ nenhum ciclo \\
$\equiv$
ciclo $\Rightarrow$ aresta de volta
\end{tabular}
\begin{itemize}
\item $G$ possui um ciclo $c$
\item seja $v$ o primeiro vértice de $c$ encontrado em uma busca em profundidade
\item seja $(u, v)$ a aresta de $c$ chegando em $v$
\item pelo teorema do caminho branco: na etapa $\attrib{v}{d}$, há um caminho branco até $u$ 
\item $u$ torna-se um descendente de $v$ na floresta em profundidade
\item logo, $(u, v)$ é uma aresta de volta.
\end{itemize}
}
\only<2>{
\begin{tabular}{c}
($\Rightarrow$) nenhum ciclo $\Rightarrow$ nenhuma aresta de volta \\
$\equiv$ \\
aresta de volta $\Rightarrow$ ciclo
\end{tabular}
\begin{itemize}
\item seja $(u, v)$ uma aresta de volta.
\item logo $v$ é um ancestro de $u$ na floresta de profundidade.
\item então há um caminho de $v$ até $v$, passando por $u$: é um ciclo
\end{itemize}
}
\end{proof}
\end{frame}

\begin{frame}
\frametitle{Correção de $\proc{Topological-Sort}$}

\begin{theorem}[Correção do algoritmo $\proc{Topological-Sort}$]
$\proc{Topological-Sort}(G)$ produz uma ordenação topológica de um grafo dirigido acíclico $G$.
\end{theorem}
\end{frame}

\begin{frame}
\frametitle{Correção de $\proc{Topological-Sort}$}

\begin{proof}
\begin{itemize}
\item Basta mostrar que: em um DAG, se há uma aresta $(u, v)$, então $\attrib{v}{f} < \attrib{u}{f}$.
\item $(u, v)$ não é uma aresta de volta
\item Quando encontrado, $v$ é branco ou preto
\begin{itemize}
\item se $v$ for branco, é um descendente de $u$ e $\attrib{v}{f} < \attrib{u}{f}$
\item se $v$ for preto, já foi finalizado e $\attrib{v}{f} < \attrib{u}{d} < \attrib{u}{f}$.
\end{itemize}
\end{itemize}
\end{proof}
\end{frame}

\section{Componentes fortemente conectados}

\begin{frame}
\frametitle{Componentes fortemente conectados}
\begin{itemize}
\item \alert{SCC}: \textit{Strongly connected components}
\item Decomposição de um grafo dirigido em grafos menores
\item Permite, para alguns problemas de grafos, aplicar estratégia de
  divisão e conquista.
\end{itemize}
\end{frame}

\begin{frame}

\frametitle{Definição}

\begin{itemize}
\item Notação: $u \leadsto v$ existe um caminho de $u$ até $v$
\end{itemize}

\begin{definition}[Componente fortemente contectado]
Um \emph{componente fortemente conectado} de um grafo $G = (V, E)$ é um
conjunto máximo de vértices $U \subseteq V$ tal que para qualquer $(u, v) \in U^2$,
então $u \leadsto v$ e $v \leadsto u$.
\end{definition}

\begin{center}
\includegraphics[height=.4\textheight]{fig/scc-1.pdf}
\end{center}
\end{frame}

\begin{frame}

\frametitle{Matriz transposta}

O algoritmo para encontrar os SCC de $G$ utiliza o grafo transposto de $G$.

\begin{definition}[Grafo transposto]
O \emph{grafo transposto} de um grafo dirigido $G = (V, E)$ é 
o grafo dirigido $G^T = (V, E^T)$, onde $E^T = \{ (u, v) | (v, u) \in E\}$.

\end{definition}


\begin{center}
\includegraphics[height=.4\textheight]{fig/scc-4.pdf}
\end{center}

\only<2-3>{\alert{Complexidade?}}\only<3>{$Theta(|V|+|E|)$}
\end{frame}

%%%%%%%%%%%%%%%%%%%%%%%%%%%%%%%%%%%%%%%%%%%%%%%%%%%%%%%%%%%%%%%%%%%%%%%%%%%%%%%

\begin{frame}
\frametitle{Algoritmo}
\framesubtitle{As etapas principais}

\begin{codebox}
\Procname{$\proc{Graph-SCC}(G)$}
\li $\proc{DFS}(G)$
\li $G^T \gets \proc{Graph-Transpose}(G)$
\li $\proc{DFS}(G^T)$ t.~q. laço principal processa vértices por $\id{f}$ decrescente
\li cada árvore da floresta de profundidade resultado é um SCC
\end{codebox}

\end{frame}

%%%%%%%%%%%%%%%%%%%%%%%%%%%%%%%%%%%%%%%%%%%%%%%%%%%%%%%%%%%%%%%%%%%%%%%%%%%%%%%

\begin{frame}
\frametitle{Ilustração}
\framesubtitle{Algoritmo}

\only<1-2>{
\begin{center}
\includegraphics{fig/scc-1.pdf}
\end{center}
}

\only<2-4>{
\begin{center}
\includegraphics{fig/scc-3.pdf}
\end{center}
}

\only<3>{
\begin{center}
\includegraphics{fig/scc-4.pdf}
\end{center}
}

\only<4>{
\begin{center}
\includegraphics{fig/scc-5.pdf}
\end{center}
}

\end{frame}

%%%%%%%%%%%%%%%%%%%%%%%%%%%%%%%%%%%%%%%%%%%%%%%%%%%%%%%%%%%%%%%%%%%%%%%%%%%%%%%

\begin{frame}
\frametitle{Complexidade}
\framesubtitle{Algoritmo}

\begin{codebox}
\Procname{$\proc{Graph-SCC}(G)$}
\li $\proc{DFS}(G)$ \assert{\Comment $\Theta(|V|+|E|)$}
\li $G^T \gets \proc{Graph-Transpose}(G)$ \assert{\Comment $\Theta(|V|+|E|)$}
\li $\proc{DFS}(G^T)$ \assert{\Comment $\Theta(|V|+|E^T|) = \Theta(|V|+|E|)$}
\end{codebox}
\pause
$$
\Theta(|V|+|E|)
$$
\end{frame}

%%%%%%%%%%%%%%%%%%%%%%%%%%%%%%%%%%%%%%%%%%%%%%%%%%%%%%%%%%%%%%%%%%%%%%%%%%%%%%%

\begin{frame}
\frametitle{Correção (roteiro)}
\framesubtitle{Algoritmo}

\begin{enumerate}
\item propriedade dos caminhos entre vértices de um mesmo SCC
\item propriedade sobre busca em profundidade e vértices de um SCC
\item noção de \alert{antepassado} de um vértice em uma busca em profundidade
\item relação entre antepassado na busca e ancestro no grafo
\item propriedade sobre antepassado e SCC
\item propriedade sobre antepassados dos vértices de um mesmo SCC
\item correção do algoritmo
\end{enumerate}
\end{frame}

%%%%%%%%%%%%%%%%%%%%%%%%%%%%%%%%%%%%%%%%%%%%%%%%%%%%%%%%%%%%%%%%%%%%%%%%%%%%%%%

\begin{frame}
\frametitle{Caminhos entre vértices de um mesmo SCC}
\framesubtitle{Correção}

\begin{lemma}[Caminhos entre vértices de um mesmo SCC]
Seja $u$ e $v$ dois vértices quaisqueres de um mesmo SCC. Então todos
os vértices nos caminhos entre $u$ e $v$ estão neste mesmo SCC.
\end{lemma}
\pause
\begin{proof}
Seja $w$ um vértice no caminho de $u$ até $v$. Logo $u \leadsto w$. Precisamos verificar que $w \leadsto u$.
\begin{itemize}
\item Como $w$ está no camino de $u$ até $v$, então $w \leadsto v$.
\item Como $u$ e $v$ estão no mesmo SCC, $v \leadsto u$.
\end{itemize}
Logo $w \leadsto u$.
\end{proof}
\end{frame}

%%%%%%%%%%%%%%%%%%%%%%%%%%%%%%%%%%%%%%%%%%%%%%%%%%%%%%%%%%%%%%%%%%%%%%%%%%%%%%%

\begin{frame}
\frametitle{Busca em profundidade e vértices de um SCC}
\framesubtitle{Correção}

\begin{theorem}[Busca em profundidade e vértices de um SCC]
Em qualquer busca em profundidade, todos os vértices em um SCC
encontram-se em uma mesma árvore da busca em profundidade.
\end{theorem}
\pause
\begin{proof}
Seja $u$ o primeiro vértice do SCC encontrado na busca em profundidade,
e $v$ qualquer outro vértice do $SCC$.
\begin{itemize}
\item na etapa $\attrib{u}{d}$, todos os demais vértices do SCC estão brancos
\item pelo teorema do caminho branco, $v$ é um descendente de $u$ na floresta
de profundidade
\end{itemize}
Logo, $u$ e $v$ estão na mesma árvore de profundidade.
\end{proof}
\end{frame}
%%%%%%%%%%%%%%%%%%%%%%%%%%%%%%%%%%%%%%%%%%%%%%%%%%%%%%%%%%%%%%%%%%%%%%%%%%%%%%%

\begin{frame}
\frametitle{Antepassado na busca em profundidade}
\framesubtitle{Correção}

\begin{itemize}
\item $\attrib{v}{d}$ e $\attrib{v}{f}$ são os valores obtidos em $\proc{DFS}(G)$
\end{itemize}

\begin{definition}[Antepassado em uma busca em profundidade]
Em uma busca em profundidade, para qualquer vértice $u$, o antepassado
de $u$, denotado $\phi(u)$, é o vértice $v$ tal que $u \leadsto v$ com o
maior valor de $\id{f}$.
\end{definition}

\only<2>{
\begin{itemize}
\item um antepassado por SCC ($\approx$ representante do componente)
\item primeiro vértice do SCC descoberto na busca em profundidade de $G$
\item último a ser finalizado
\item é a raiz da árvore de profundidade na busca em profundidade de $G^T$
\end{itemize}
}
\only<3>{
Temos:
\begin{enumerate}
\item $\attrib{u}{f} \le \attrib{\phi(u)}{f}$
\item $u \leadsto v \Rightarrow \attrib{\phi(v)}(f) \le \attrib{\phi(u)}{f}$
\item $\phi(\phi(u)) = \phi(u)$
\end{enumerate}
}
\end{frame}


%%%%%%%%%%%%%%%%%%%%%%%%%%%%%%%%%%%%%%%%%%%%%%%%%%%%%%%%%%%%%%%%%%%%%%%%%%%%%%%

\begin{frame}
\frametitle{Antepassado na busca em profundidade}
\framesubtitle{Correção}

\alert{$\attrib{u}{f} \le \attrib{\phi(u)}{f}$}

\begin{itemize}
\item Pois $u \leadsto u$, e $\forall v | u \leadsto v \cdot \attrib{v}{f} \le \attrib{\phi(u)}{f}$
\end{itemize}

\alert{$u \leadsto v \Rightarrow \attrib{\phi(v)}(f) \le \attrib{\phi(u)}{f}$}

\begin{itemize}
\item Pois $\{ w \cdot v \leadsto w\} \subseteq \{ w \cdot u \leadsto w\}$
\end{itemize}

\alert{$\phi(\phi(u)) = \phi(u)$}

\begin{itemize}
\item Pois $\attrib{\phi(u)}{f} \le \attrib{\phi(\phi(u))}{f}$,
\item e, como $u \leadsto \phi(u)$, então $\attrib{\phi(\phi(u))}{f} \le \attrib{\phi(u)}{f}$,
\item temos $\attrib{\phi(u)}{f} = \attrib{\phi(\phi(u))}{f}$, e
\item cada vértice tem um valor de $\id{f}$ diferente.
\end{itemize}

\end{frame}


%%%%%%%%%%%%%%%%%%%%%%%%%%%%%%%%%%%%%%%%%%%%%%%%%%%%%%%%%%%%%%%%%%%%%%%%%%%%%%%

\begin{frame}
\frametitle{Antepassados e ancestros}
\framesubtitle{Correção}

\begin{theorem}[Antepassado é ancestro]
Em um grafo dirigido $G = (V, E)$, o antepassado $\phi(u)$ de qualquer
$u \in V$ em qualquer busca em profundidadede $G$, sempre é um
ancestro de $u$.
\end{theorem}
\pause
\begin{proof}
Por caso sobre a cor de $\phi(u)$ na etapa $\attrib{d}{u}$
\only<3-5>{
\begin{itemize}
\item 
\only<3>{Se for $\const{Gray}$, então $\phi(u)$ é ancestro de $u$.
}
\only<4>{Se for $\const{Black}$, então foi finalizado, e $\attrib{\phi(u)}{f} < \attrib{u}{f}$. \alert{Contradiz} $\attrib{u}{f} \le \attrib{\phi(u)}{f}$.
}
\only<5>{
Se for $\const{White}$, consideramos a cor dos vértices no caminho
de $u$ até $\phi(u)$
\begin{itemize}
\item todos são brancos, logo $\phi(u)$ é descendente de $u$, $\attrib{\phi(u)}{f} < \attrib{u}{f}$: \alert{contradição}
\item senão, seja $t$ o último vértice não branco do caminho:
\begin{itemize}
\item $t$ não pode ser preto, pois tem um sucessor branco
\item então há um caminho branco de $t$ até $\phi(u)$, e 
\item $\phi(u)$ é descendente de $t$ (teorema do caminho branco)
\item logo $\attrib{t}{f} > \attrib{\phi(u)}{f}$
\item \alert{contradição} por definição de $\phi(u)$ e $u \leadsto t$.
\end{itemize}
\end{itemize}
}
\end{itemize}
}
\end{proof}

\end{frame}

%%%%%%%%%%%%%%%%%%%%%%%%%%%%%%%%%%%%%%%%%%%%%%%%%%%%%%%%%%%%%%%%%%%%%%%%%%%%%%%

\begin{frame}
\frametitle{Antepassado e SCC}
\framesubtitle{Correção}

\begin{corollary}[Antepassado e SCC]
Em qualquer busca em profundidade de um grafo dirigido $G = (V, E)$, 
para qualquer $u \in V$, ambos $u$ e $\phi(u)$ pertencem ao mesmo SCC.
\end{corollary}
\pause
\begin{proof}
\begin{itemize}
\item Por definição de $\phi$, temos $u \leadsto \phi(u)$.
\item Pelo teorema ``antepassado é ancestro'', $\phi(u) \leadsto u$.
\end{itemize}
\end{proof}
\end{frame}

%%%%%%%%%%%%%%%%%%%%%%%%%%%%%%%%%%%%%%%%%%%%%%%%%%%%%%%%%%%%%%%%%%%%%%%%%%%%%%%

\begin{frame}
\frametitle{Antepassados dos vértices de um SCC}
\framesubtitle{Correção}

\begin{theorem}[Antepassados dos vértices de um SCC]
Em um grafo dirigido $G = (V, E)$, os vértices $u$ e $v$ pertencem ao
mesmo SCC se, e somente se, possuem o mesmo antepassado na busca em
profundidade de $G$.
\end{theorem}

\pause
\begin{proof}
\begin{itemize}
\item $(\Rightarrow)$ \only<3>{$u$ e $v$ pertencem ao mesmo SCC:
\begin{itemize}
\item logo $\{ w \cdot u \leadsto u \} = \{ w \cdot v \leadsto w\}$.
\item por definição de $\phi$, $\phi(u) = \phi(v)$.
\end{itemize}
}
\item $(\Leftarrow)$ 
\only<4>{$\phi(u) = \phi(v)$:
\begin{itemize}
\item pelo corolário ``Antepassado e SCC'' $u$ e $\phi(u)$ estão no mesmo SCC
\item $v$ e $\phi(v)$ estão no mesmo SCC,
\item logo $u$ e $v$ estão no mesmo SCC.
\end{itemize}
}
\end{itemize}
\end{proof}
\end{frame}

%%%%%%%%%%%%%%%%%%%%%%%%%%%%%%%%%%%%%%%%%%%%%%%%%%%%%%%%%%%%%%%%%%%%%%%%%%%%%%%

\begin{frame}
\frametitle{Intuição do algoritmo}

\begin{itemize}
\item $\proc{DFS}(G)$ marca os antepassados com o maior valor de
  $\id{f}$ (e o menor valor de $\id{d}$) de cada SCC.
\item $\proc{DFS}(G^T)$ 
\begin{itemize}
\item começa com um vértice $v_1$ de maior $\id{f}$, que é um antepassado
\item visita todos os vértices do SCC de $v_1$
\item continua com um vértice $v_2$ de maior $\id{f}$ entre os não visitados,
também é um antepassado
\item visita todos os vértices do SCC de $v_2$
\item e assim sucessivamente
\end{itemize}
\end{itemize}
\end{frame}

%%%%%%%%%%%%%%%%%%%%%%%%%%%%%%%%%%%%%%%%%%%%%%%%%%%%%%%%%%%%%%%%%%%%%%%%%%%%%%%

\begin{frame}
\frametitle{Teorema da correção}
\framesubtitle{Correção}

\begin{theorem}[Correção de $\proc{Graph-SCC}$]
Seja $G$ um grafo dirigido qualquer, $\proc{Graph-SCC}(G)$ calcula
corretamente os componentes fortemente conectados de $G$.
\end{theorem}

\pause

Roteiro da demonstração:
\begin{itemize}
\item indução 
\item número de árvores de profundidade encontrados em $\proc{DFS}(G^T)$.
\item mostramos que, assumindo que as árvores anteriores são SCC, 
  cada nova árvore formada é um SCC
\item trivial para a primeira árvore (não existe árvores anteriores)
\end{itemize}

\end{frame}

%%%%%%%%%%%%%%%%%%%%%%%%%%%%%%%%%%%%%%%%%%%%%%%%%%%%%%%%%%%%%%%%%%%%%%%%%%%%%%%

\begin{frame}
\frametitle{Prova do teorema da correção}
\framesubtitle{Correção}

\begin{proof}
\begin{itemize}
\item Seja $T$ uma árvore de profundidade de raiz $r$ produzida por $\proc{DFS}(G^T)$.
\item Seja $C(r) = \{ w \cdot \phi(w) = r\}$ (é um SCC)
\item Mostramos que $u$ é incluído em $T$ se e somente se $u \in C(r)$
\begin{itemize}
\item $(\Leftarrow)$ \only<2>{
\begin{itemize}
\item teorema ``busca em profundidade e vértices de um SCC''
\item cada vértice em $C(r)$ termina em uma mesma árvore de profundidade
\item como $r \in C(r)$, e $r$ é a raiz de $T$, então cada elemento de $C(r)$
termina precisamente em $T$.
\end{itemize}
}
\item $(\Rightarrow)$ \only<3-5>{Mostramos que, para um vértice $w$, se
$\attrib{\phi(w)}{f} < \attrib{r}{f}$, ou 
$\attrib{\phi(w)}{f} > \attrib{r}{f}$, $w$ não pertence a $T$
\only<4>{
\begin{itemize}
\item $\attrib{\phi(w)}{f} < \attrib{r}{f}$
\item se $w$ for colocado em $T$, então $w \leadsto r$
\item logo, $\attrib{\phi(w)}{f} \le \attrib{\phi(r)}{f} = \attrib{r}{f}$
\item \alert{contradição}
\end{itemize}
}
\only<5>{
\begin{itemize}
\item $\attrib{\phi(w)}{f} > \attrib{r}{f}$
\item por hipótese de indução, quando $r$ for selecionado em $\proc{DFS}(G^T)$,
$\attrib{w}{f}$ terá sido inserido na árvore de raiz $\phi(w)$.
\item um vértice é inserido exatamente em uma árvore.
\end{itemize}
}
}
\end{itemize}
\end{itemize}
\end{proof}

\end{frame}

%%%%%%%%%%%%%%%%%%%%%%%%%%%%%%%%%%%%%%%%%%%%%%%%%%%%%%%%%%%%%%%%%%%%%%%%%%%%%%%

\begin{frame}
\frametitle{Teorema da correção}
\framesubtitle{Correção}

\begin{theorem}[Correção de $\proc{Graph-SCC}$]
Seja $G$ um grafo dirigido qualquer, $\proc{Graph-SCC}(G)$ calcula
corretamente os componentes fortemente conectados de $G$.
\end{theorem}

\pause

\begin{proof}
\begin{itemize}
\item indução 
\item número de árvores de profundidade encontrados em $\proc{DFS}(G^T)$.
\item mostramos que, assumindo que as árvores anteriores são SCC, 
  cada nova árvore formada é um SCC
\item trivial para a primeira árvore (não existe árvores anteriores)
\end{itemize}
\end{proof}
\end{frame}

%%%%%%%%%%%%%%%%%%%%%%%%%%%%%%%%%%%%%%%%%%%%%%%%%%%%%%%%%%%%%%%%%%%%%%%%%%%%%%%

\begin{frame}
\frametitle{Exercícios}

\begin{enumerate}
\item Como pode evoluir a quantidade de SCC em um grafo quando uma
  aresta é adicionada? removida?
\item Seja $G$ um grafo dirigido. O grafo dos componentes de $G$ é o grafo
  $G^{SCC} = (V^{SCC}, E^{SCC})$, tal que $V^{SCC}$ contem um vértice por
  SCC de $G$, e $E^{SCC}$ contem a aresta $(u, v)$ se existe uma aresta entre
  um vértice de $u$ e um vértice de $v$ no grafo $G$.

  Mostre que o grafo dos componentes conectados de $G$ é um DAG.
\item Escreva um algoritmo que calcula o grafo dos componentes de
  $G=(V, E)$, com complexidade $O(|V|+|E|)$. 

  Nota: o grafo resultado deve ter, ao máximo, uma aresta entre cada
  par de vértices.

\item Um grafo é semi-conectado se, para qualquer par de vértices $(u, v)$,
  ou $u \leadsto v$ ou $v \leadsto u$.

  Escreva um algoritmo eficiente que testa se um grafo é semi-conectado.
  Mostre que seu algoritmo é correto.
\end{enumerate}

\end{frame}


\end{document}
