%% TODO: illustrations
\documentclass{beamer}
\setbeamertemplate{footline}[frame number]

\usepackage[utf8x]{inputenc}
\usepackage[brazil,british]{babel}
\usetheme{default} 
\usecolortheme{beaver}
\newtranslation[to=brazil]{Theorem}{Teorema}
\newtranslation[to=brazil]{Definition}{Definição}
\newtranslation[to=brazil]{Example}{Exemplo}
\newtranslation[to=brazil]{Problem}{Exercício}
\newtranslation[to=brazil]{Solution}{Resolução}
\newtranslation[to=brazil]{Lemma}{Lema}
\newtranslation[to=brazil]{Corollary}{Corolário}
\logo{\includegraphics[width=1cm]{../img/logo-ppgsc-icon-text.png}}

\usepackage{graphicx}
\usepackage{clrscode3e}
\usepackage{hyperref}

\usepackage{pgf}
\usepackage{tikz}
\newcommand{\assert}[1]{\textcolor{blue}{#1}}

\usepackage{xspace}

\newcommand{\semester}[0]{2015.2}


\title{Aula 28: Algoritmos gulosos}
\subtitle{\textit{Greedy}}
\author{David Déharbe \\
  Programa de Pós-graduação em Sistemas e Computação \\
  Universidade Federal do Rio Grande do Norte \\
  Centro de Ciências Exatas e da Terra \\
  Departamento de Informática e Matemática Aplicada}
\date{27 de maio de 2015}

\begin{document}
\selectlanguage{brazil}

%%%%% SLIDE 01 %%%%%

\begin{frame}
  \titlepage
\end{frame}

\begin{frame}
  \frametitle{Plano}

  \tableofcontents
Referência: Cormen, cap 17.
\end{frame}

\section{Introdução}

%%%%% SLIDE 02 %%%%%

\begin{frame}
\frametitle{Introdução}

\begin{itemize}

\item Otimização

\item Alternativa à programação dinâmica

\item Em geral, mais simples

\item Nem sempre possível

\item Exemplos:

\begin{itemize}

\item algoritmo de menor caminho (Dijkstra)

\item árvore geradora mínima

\end{itemize}

\end{itemize}

\end{frame}

%%%%% SLIDE 03 %%%%%

\begin{frame}
\frametitle{Princípios}

\begin{itemize}
\item a solução é calculada de forma incremental
\item cada incremento é realizado com a melhor escolha naquele momento
\item a sequência de decisões localmente ótimas é globalmente ótima
\item adequado para problemas exibindo certas propriedades
\begin{itemize}
\item sub-estrutura ótima
\item propriedade da escolha gulosa
\end{itemize}

\end{itemize}

\end{frame}


\section{Um exemplo simples}

%%%%% SLIDE 04 %%%%%

\begin{frame}
\frametitle{Problema de escalonamento de atividades}
\framesubtitle{Exemplo introdutório}

\begin{itemize}
\item Temos $n$ atividades
\item As atividades necessitam de um recurso compartilhado
\item Cada atividade $i$ tem uma hora de início $s_i$
e uma hora de término $f_i$
\item A cada $t$, $s_i \le t < f_i$, a atividade $i$ precisa
do acesso exclusivo ao recurso para funcionar.
\item Problema:
\begin{itemize}
\item Escalonar a maior quantidade possível de atividades.
\end{itemize}
\end{itemize}

\end{frame}

%%%%% SLIDE 05 %%%%%

\begin{frame}
\frametitle{Algoritmo}
\framesubtitle{Escalonamento de atividades}

\begin{codebox}
\Procname{\proc{Greedy-Scheduler}(s, f)}
\zi \Comment{$f_1 \le f_2 \le \ldots f_n$}
\li $n \gets \attrib{s}{length}$
\zi \Comment{$A$: atividades escalonadas}
\li $A \gets \{ 1 \}$  \Comment{atividade terminando mais cedo é escalonada}
\zi \Comment{$j$: última atividade escalonada}
\li $j \gets 1$
\li \For $i \gets 2 \To n$ \Comment{escolhe se a atividade $i$ é escalonada}
\li \Do \If $s_i \ge f_j$
\li   \Then $A \gets A \cup \{ i \}$
\li     $j \gets i$
      \End
    \End
\li \Return $A$
\end{codebox}

\end{frame}

%%%%% SLIDE 06-07 %%%%%

\begin{frame}
\frametitle{Ilustração}

$$
\begin{array}{|c|c|c|c|c|c|c|c|c|c|c|c|}
i & 1 & 2 & 3 & 4 & 5 & 6 & 7 & 8 & 9 & 10 & 11 \\
\hline
\hline
s & 1 & 3 & 0 & 5 & 3 & 5 & 6 & 8 & 8 & 2 & 12 \\
\hline
f & 4 & 5 & 6 & 7 & 8 & 9 & 10 & 11 & 12 & 13 & 14
\end{array}
$$

\pause
Por qual razão este algoritmo é qualificado de \alert{guloso}?

\end{frame}


%%%%% SLIDE 08-09 %%%%%

\begin{frame}
\frametitle{Complexidade}
\framesubtitle{Escalonamento de atividades}

\begin{codebox}
\Procname{\proc{Greedy-Scheduler}(s, f)}
\li $n \gets \attrib{s}{length}$
\li $A \gets \{ 1 \}$
\li $j \gets 1$
\li \For $i \gets 2 \To n$ \Comment{\assert{$\Theta(n)$ repetições}}
\li \Do \If $s_i \ge f_i$ \Comment{\assert{$O(1)$ por repetição}}
\li   \Then $A \gets A \cup \{ i \}$
\li     $j \gets i$
      \End
    \End
\li \Return $A$
\end{codebox}
\pause
\alert{$$\Theta(n)$$}
\end{frame}

%%%%% SLIDE %%%%%

\begin{frame}
\frametitle{Correção}
\framesubtitle{Escalonamento de atividades}

\begin{itemize}
\item O algoritmo sempre tenta maximizar o tempo que o recurso ficará disponível
\item Será que é correto? Justifique.
\end{itemize}

\begin{theorem}
O algoritmo $\proc{Greedy-Scheduler}$ produz soluções de tamanho máximo para o
problema do escalonamento de atividades.
\end{theorem}

\end{frame}

%%%%% SLIDE %%%%%

\begin{frame}
\frametitle{Correção}
\framesubtitle{Escalonamento de atividades}

\begin{proof}[Por indução]
\begin{itemize}
\item Seja $A = \{ A_1, \ldots A_k \} $ uma solução ótima, ordenada por 
tempo de término crescente.
\item Queremos mostrar que existe um escalonamento ótimo com $1$
\item Se $A_1 \neq 1$, seja $B = A - \{ A_1 \} \cup \{ 1 \}$
\item $B$ é um escalonamento correto das atividades, pois $f_1 \le f_{A_1}$
\item Logo $B$ é um escalonamento ótimo que começa com $1$.
\item Sempre existe um escalonamento ótimo começando por $1$.
\item Uma vez $1$ escolhido, deve-se escalonar a maior quantidade possível
de tarefas a partir de $f_1$.
\begin{itemize}
\item Instância menor do mesmo problema!
\item Podemos repetir a escolha gulosa.
\item Por indução a solução é correta.
\end{itemize}
\end{itemize}
\end{proof}

\end{frame}

%%%%% SLIDE %%%%%

\begin{frame}
\frametitle{Exercícios}
\begin{enumerate}
\item Desenvolver um algoritmo para o problema do escalonamento da atividades utilizando \alert{programação dinâmica}.
\begin{itemize}
\item Dica: o algoritmo deve calcular $m_i$, para $1 \le i \le n$, onde $m_i$ é 
a solução para as atividades $\{1, 2, \ldots i\}$.
\item Compare a complexidade do algoritmo obtido com a de $\proc{Greedy-Schedule}$
\end{itemize}
\item A abordagem gulosa nem sempre funciona:
\begin{itemize}
\item Encontre um exemplo que mostra por que não funciona se for escolhida a atividade de menor duração no problema do escalonamento de atividades.
\item Encontre um exemplo que mostra por que não funciona se for escolhida a atividade com menos conflitos no problema do escalonamento de atividades.
\end{itemize}
\end{enumerate}
\end{frame}

\section{Considerações sobre algoritmos gulosos}

%%%%% SLIDE %%%%%

\begin{frame}
\frametitle{Estratégias gulosas}

\begin{itemize}
\item Como determinar se uma estratégia gulosa é correta?
\item Deve-se demonstrar que uma solução globalmente ótima pode ser
calculada a partir de soluções localmente ótimas
\begin{itemize}
\item Exemplo: o teorema sobre $\proc{Greedy-Scheduler}$
\item Considera uma solução globalmente ótima ($A$).
\item Mostra que pode se construir uma solução globalmente ótima a partir
da escolha gulosa ($A - \{ A_1 \} \cup \{ 1 \}$).
\item Mostrar por indução que a escolha gulosa pode ser repetida.
\item uma escolha localmente ótima produz uma instância menor do
mesmo problema de otimização.
\item \alert{sub-estrutura ótima}
\end{itemize}
\end{itemize}

\end{frame}

%%%%% SLIDE %%%%%

\begin{frame}
\frametitle{Programação dinâmica e algoritmos gulosos}
\framesubtitle{Comparação}

\begin{itemize}
\item Compartilham a propriedade de sub-estrutura ótima
\item Porém não são equivalentes!
\item Exemplos:
\begin{itemize}
\item problema da mochila 0-1 (ver aula 27)
\item problema da mochila: itens tem pesos e valores, mas é possível fracionar
  cada item.
\end{itemize}
\item Uma instância: \item $K = 50$, $v_1 = 60$, $w_1 = 10$, $v_2 = 100$, $w_2 = 20$, $v_3 = 120$, $w_1 = 30$, 
\begin{itemize}
\item Qual a solução ótima no caso da mochila $0-1$?
\item Qual a solução ótima no caso da mochila?
\end{itemize}
\item Qual pode ser resolvido apenas com programação dinâmica, por que?
\pause
\item Como resolver o problema da mochila com um algoritmo guloso?
\end{itemize}
\end{frame}

%%%%% SLIDE %%%%%

\begin{frame}
\frametitle{Exercícios}
\begin{enumerate}
\item Demonstrar que o problema da mochila pode ser resolvido de forma gulosa.
\item Projetar um algoritmo guloso para solucionar o problema da mochila.
\end{enumerate}
\end{frame}

%%%%% SLIDE %%%%%
\section{Exemplo avançado: códigos de Huffman}
\begin{frame}
\frametitle{Códigos de Huffman}
\framesubtitle{Um exemplo avançado}

\begin{itemize}
\item Utilizado para compactar dados
\item Dados: sequências de símbolos
\item Necessita conhecer a frequência de cada símbolo
\item Atribui código de tamanho menor para símbolos mais frequentes
\item Alocação de código: algoritmo guloso
\end{itemize}

\end{frame}

%%%%% SLIDE %%%%%
\begin{frame}
\frametitle{Ilustração}
\framesubtitle{Um exemplo avançado}

\begin{itemize}
\item arquivo de tamanho 100.000 símbolos
\item seis símbolos diferentes
\item hipótese: codificação binária
\item 3 bits por símbolo
\item custo total (sem compactação): 300 kbits
\end{itemize}
\pause
\begin{center}
\begin{tabular}{r|c|c|c|c|c|c|}
& a & b & c & d & e & f \\
\hline
frequência & 45 & 13 & 12 & 16 & 9 & 5 \\
código de tamanho fixo & 000 & 001 & 010 & 011 & 100 & 101 \\
código de tamanho variável & 0 & 101 & 100 & 111 & 1101 & 1100
\end{tabular}
Custo com código de tamanho variável: 224 kbits ($\approx 25\%$)
\end{center}
\end{frame}

%%%%% SLIDE %%%%%

\begin{frame}
\frametitle{Códigos livres de prefixo}

\begin{center}
\begin{tabular}{r|c|c|c|c|c|c|}
& a & b & c & d & e & f \\
\hline
código de tamanho variável & 0 & 101 & 100 & 111 & 1101 & 1100
\end{tabular}
\end{center}

\begin{definition}[Código livre de prefixo]
Um código é livre de préfixo se, para qualquer par de símbolos $s_1, s_2$
a codificação de $s_1$ não é prefixo da codificação de $s_2$.
\end{definition}

\begin{itemize}
\item O código na tabela é livre de prefixo
\item Qual texto representa $0101100$? e $101001101$?
\item Por que é possível decodificar?
\end{itemize}
\end{frame}

%%%%% SLIDE %%%%%

\begin{frame}
\frametitle{Códigos livres de prefixo}

\begin{itemize}
\item Decodificação é simples
\item Qual estrutura de dados utilizar para uma decodificação eficiente?
\pause
\item Árvore binária
\begin{itemize}
\item sem sub-árvore: símbolo decodificado no vértice
\item 0: selecionar sub-árvore esquerda
\item 1: selecionar sub-árvore direita
\item Exemplo:
\begin{center}
\begin{tabular}{r|c|c|c|c|c|c|}
& a & b & c & d & e & f \\
\hline
frequência & 45 & 13 & 12 & 16 & 9 & 5 \\
código de tamanho fixo & 000 & 001 & 010 & 011 & 100 & 101 \\
código de tamanho variável & 0 & 101 & 100 & 111 & 1101 & 1100
\end{tabular}
\end{center}
\end{itemize}
\end{itemize}
\end{frame}

%%%%% SLIDE %%%%%

\begin{frame}
\frametitle{Códigos livres de prefixo}
\framesubtitle{}
\begin{itemize}
\item codificação ótima: árvore binária cheia
\begin{itemize}
\item cada vértice ou é uma folha, ou tem duas sub-árvores não vazias
\end{itemize}
\item Seja $T$ uma árvore de codificação ótima. Dada a frequência $f(A, c)$ de cada
  caractere $c$ em um arquivo $A$, qual o tamanho da representação de $A$ com
  $T$?  \pause
$$ B(A, T) = \sum_{c} f(A, c) d_T(c) \text{ onde $d_T(c)$ é a profundidade de
    $c$ em $T$} $$
é o \alert{custo} de $T$.
\pause
\item Dada $f(A, c)$ para cada $c$, como construir uma árvore de codificação ótima? \pause \alert{Códificação de Huffman}

\end{itemize}
\end{frame}

%%%%% SLIDE %%%%%

\begin{frame}
\frametitle{Codificação de Huffman}
\framesubtitle{Um pouco de história}

\begin{itemize}
\item David Huffman inventou a codificação de Huffman em 1952
\item doutorando no M.I.T.
\item avaliação da disciplina \emph{Teoria da Informação}: 
\begin{itemize}
\item exame escrito, ou
\item escrever um artigo original sobre o tema ``codificação binária ótima''
\end{itemize}
\item desenvolveu a ideia, que era melhor que as existentes (inclusive do
  docente da disciplina)
\end{itemize}
\end{frame}

%%%%% SLIDE %%%%%

\begin{frame}
\frametitle{Construção do código de Huffman}
\framesubtitle{Princípios}
\begin{itemize}
\item construção incremental da árvore
\item abordagem ascendente
\item inicia com uma floresta de $|C|$ folhas
\item realiza $|C|-1$ etapas de ``junção'' para criar a árvore final:
\begin{itemize}
\item \alert<2>{escolher} duas árvores $T_i$, $T_j$ da floresta
\item criar uma nova sub-árvore que tem $T_i$ e $T_j$ como sub-árvores.
\end{itemize}
\pause
\item \alert{como escolher?}
\end{itemize}
\end{frame}

%%%%% SLIDE %%%%%

\begin{frame}
\frametitle{Construção do código de Huffman}
\framesubtitle{Algoritmo}

\begin{itemize}
\item $\attrib{v}{f}$ frequência dos caracteres da sub-árvore enraizada em $v$
\item $\attrib{v}{c}$ caractere no vértice-folha $v$.
\end{itemize}
\begin{small}
\begin{codebox}
\Procname{$\proc{Huffman-Code}(C)$}
\zi \assert{\Comment $Q$: fila de prioridade com chave $\attrib{v}{f}$}
\li \For $c \in C$
\li \Do $v \gets \proc{Alloc-Vertex}()$
\li   $\attrib{v}{f} \gets \attrib{c}{f}$
\li   $\attrib{v}{val} \gets c$
\li   $\proc{Push-Back}(Q, v)$
    \End
\li $\proc{Make-Heap}(Q, v)$
\li \For $i \gets 1 \To |C|-1$
\li \Do $v \gets \proc{Alloc-Vertex}()$
\li   $l \gets \attrib{v}{left} \gets \proc{Extract-Min}(Q)$
\li   $r \gets \attrib{v}{right} \gets \proc{Extract-Min}(Q)$
\li   $\attrib{v}{f} \gets \attrib{l}{f} + \attrib{r}{f}$
\li   $\proc{Insert}(Q, v)$
    \End
\li \Return $\proc{Extract-Min}(Q)$
\end{codebox}
\end{small}
\end{frame}

%%%%% SLIDE %%%%%

\begin{frame}
\frametitle{Ilustração}

\begin{center}
\begin{tabular}{r|c|c|c|c|c|c|}
& a & b & c & d & e & f \\
\hline
frequência & 45 & 13 & 12 & 16 & 9 & 5
\end{tabular}
\end{center}

\begin{center}
quadro
\end{center}
\end{frame}
%%%%% SLIDE %%%%%

\begin{frame}
\frametitle{Complexidade}
\framesubtitle{Construção do código de Huffman}

\only<1>{
\begin{codebox}
\Procname{$\proc{Huffman-Code}(C)$}
\zi \assert{\Comment{$Q$: fila de prioridade com chave $\attrib{v}{f}$}}
\li \For $c \in C$ \assert{\Comment{$\Theta(|C|)$ iterações}}
\li \Do $v \gets \proc{Alloc-Vertex}()$
\li   $\attrib{v}{f} \gets \attrib{c}{f}$
\li   $\attrib{v}{val} \gets c$
\li   $\proc{Push-Back}(Q, v)$
    \End
\li $\proc{Make-Heap}(Q, v)$  \assert{\Comment{\textit{heap\/} binário $\Theta(|C|)$}}
\li \For $i \gets 1 \To |C|-1$ \assert{\Comment{$|C|$ iterações}}
\li \Do $v \gets \proc{Alloc-Vertex}()$
\li   $l \gets \attrib{v}{left} \gets \proc{Extract-Min}(Q)$ \assert{\Comment{$\Theta(\lg |C|)$}}
\li   $r \gets \attrib{v}{right} \gets \proc{Extract-Min}(Q)$ \assert{\Comment{$\Theta(\lg |C|)$}}
\li   $\attrib{v}{f} \gets \attrib{l}{f} + \attrib{r}{f}$
\li   $\proc{Insert}(Q, v)$ \assert{\Comment  $\Theta(\lg |C|)$}
    \End
\li \Return $\proc{Extract-Min}(Q)$ \assert{\Comment{$\Theta(\lg |C|)$}}
\end{codebox}
}
\only<2>{
\begin{center}
\alert{$\Theta |C| \lg |C|$}
\end{center}
}
\end{frame}
%%%%% SLIDE %%%%%

\begin{frame}
\frametitle{Construção do código de Huffman}
\framesubtitle{Correção}

\begin{lemma}
Seja $C$ um alfabeto, e $\attrib{c}{f}$ a frequência de cada caractere em
$C$.  Se $x$ e $y$ são os caracteres com a menor frequência em $C$, então existe
um código livre de prefixo ótimo para $C$, onde a códificação de $x$ e a de $y$
tem mesmo comprimento e diferem apenas no último bit.
\end{lemma}

Roteiro da prova
\begin{itemize}
\item Considerar uma árvore $T$ representando um código livre de prefixo ótimo
  \alert{qualquer}
\item Construir uma árvore $T'$ a partir de $T$ onde as codificações de $x$ e $y$ têm
  as propriedades enunciadas.
\item $x$ e $y$ devem ser irmãos de profundidade máxima em $T'$.
\end{itemize}

\end{frame}

%%%%% SLIDE %%%%%

\begin{frame}
\frametitle{Construção do código de Huffman}
\framesubtitle{Correção}

\begin{proof}
\begin{itemize}
\only<1>{\item Seja uma árvore $T$ representando um código livre de prefixo ótimo qualquer.}
\item Seja $b$ e $c$ dois vizinhos na profundidade máxima, $\attrib{b}{f} \le \attrib{c}{f}$

\item Seja $x$ e $y$ os dois caracteres com a menor frequência, $\attrib{x}{f} \le \attrib{y}{f}$

\item Temos $\attrib{x}{f} \le \attrib{b}{f}$ e $\attrib{y}{f} \le \attrib{c}{f}$

\item $T'$ é obtido a partir de $T$ trocando $x$ e $b$.
\pause
\only<2>{
{\small
$
\begin{array}{l}
B(T) - B(T') \\
\begin{array}{rcl}
\quad & = & \sum_c \attrib{c}{f} \cdot d_T(c) - \sum_c \attrib{c}{f} \cdot d_{T'}(c) \\
& = & \attrib{x}{f} \cdot d_T(x) + \attrib{b}{f} \cdot d_T(b) - \attrib{x}{f}\cdot d_{T'}(x) + \attrib{b}{f} \cdot d_{T'}(b) \\
& = & \attrib{x}{f} \cdot d_T(x) + \attrib{b}{f} \cdot d_T(b) - \attrib{x}{f} \cdot d_{T}(b) + \attrib{b}{f} \cdot d_{T}(x) \\
& = & (\attrib{b}{f} - \attrib{x}{f})\cdot(d_T(b) - d_{T}(x)) \\
& = & \underbrace{(\attrib{b}{f} - \attrib{x}{f})}{\ge 0}\cdot\underbrace{(d_T(b) - d_{T}(x))}{\ge 0} \\
& \ge & 0
\end{array}
\end{array}
$}
}
\only<3>{
\item $T''$ é obtido a partir de $T'$ trocando $y$ e $c$.
\item Similarmente, mostramos que $B(T'') \le B(T') \le B(T)$.
\item Mas $B$ tem custo mínimo, logo $B(T) \le B(T'')$.
\item Conclusão: $B(T'') = B(T)$ e $T''$, onde a códificação de $x$ e a
  de $y$ tem mesmo comprimento e diferem apenas no último bit, representa um
  código livre de prefixo ótimo.
}
\end{itemize}
\end{proof}
\end{frame}

%%%%% SLIDE %%%%%

\begin{frame}
\frametitle{Construção do código de Huffman}
\framesubtitle{Correção}

\begin{lemma}
Seja $T$ uma árvore binária cheia representando um código livre de prefixo ótimo
para um alfabeto $C$, com frequência $\attrib{c}{f}$ para cada $c$ em $C$.  

Se $x$ e $y$ são duas folhas vizinhas em $T$, e $z$ o ascendente imediato
dessas folhas. Então, considerando $z$ como um caractere de frequência $\attrib{x}{f} + \attrib{y}{f}$,
a árvore $T' = T - \{ x, y \}$ representa código livre de prefixo ótimo para
o alfabeto $C'= C - \{ x, y \} \cup \{ z \}$.

\end{lemma}

Roteiro da prova
\begin{itemize}
\item Relacionar o custo de $T$ com o custo de $T'$
\item Mostrar, por contradição, que $T'$ é representa um código livre de prefixo ótimo.
\end{itemize}

\end{frame}

%%%%% SLIDE %%%%%

\begin{frame}
\frametitle{Construção do código de Huffman}
\framesubtitle{Correção}

\begin{proof}
\begin{itemize}
\item hipótese inicial: $T$ representa um código livre de prefixo ótimo;
\item Relacionar o custo de $T$ com o custo de $T'$:
$
\begin{array}{rcl}
B(T) & = & 
\only<1>{
\sum_{c \in T} \attrib{c}{f} \cdot d_T(c) \\
  & = & \sum_{c \in T-\{x, y\}} \attrib{c}{f} \cdot d_T(c) + \attrib{x}{f} \cdot d_T(x) + \attrib{y}{f} \cdot d_T(y) \\
  & = & (\sum \cdots) + \attrib{x}{f} \cdot (d_{T'}(z)+1) + \attrib{y}{f} \cdot (d_{T'}(z)+1) \\
  & = & (\sum \cdots) + (\attrib{x}{f} + \attrib{y}{f}) \cdot d_{T'}(z) + \attrib{x}{f} + \attrib{y}{f} \\
  & = & (\sum \cdots) + \attrib{x}{f} + \attrib{y}{f} \\
  & = & \sum_{c \in T'} \attrib{c}{f} \cdot d_{T'}(c) + \attrib{x}{f} + \attrib{y}{f} \\
  & = & 
}
B(T') + \attrib{x}{f} + \attrib{y}{f}
\end{array}
$
\only<2>{
\item hipótese adicional: $T'$ não representa um código livre de prefixo ótimo:
\begin{itemize}
\item Logo, existe $T''$ ótimo para $C'$, tal que $B(T'') < B(T')$.
\item $z$ é uma folha em $T'$ e em $T''$.
\item Substituindo $z$ por $x$ e $y$ em $T''$, obtemos uma nova árvore para $C$, que tem custo menor que $T$.
\alert{Contradição}
\end{itemize}
}
\end{itemize}
\end{proof}
\end{frame}

%%%%% SLIDE %%%%%

\begin{frame}
\frametitle{Construção do código de Huffman}
\framesubtitle{Correção}

\begin{theorem}
O algoritmo $\proc{Huffman-Code}$ produz uma árvore que representa um código livre de prefixo ótimo para $C$.
\end{theorem}

\begin{proof}
A partir dos lemas anteriores.
\end{proof}

\end{frame}

%%%%% SLIDE %%%%%

\begin{frame}
\frametitle{Exercícios}

\begin{enumerate}
\item Explique por quais motivos $\proc{Huffman-Code}$ é um algoritmo guloso.
\item Demonstrar que uma árvore binária não cheia não pode representar um código livre de prefixo ótimo.
\item Construir um código livre de prefixo ótimo para o alfabeto e as frequências seguintes:
$$
a: 1, b: 1, c: 2, d: 3, e: 5, f: 8, g: 13, h: 21
$$
\item Construir um código livre de prefixo ótimo para um alfabeto cujas
  frequências correspondem à sequência de Fibonacci.
\item Caracterizar instâncias do problema tal que todas as codificações do
  código livre de prefixo ótimo tenham o mesmo comprimento.
\end{enumerate}

\end{frame}

\end{document}

