\documentclass[handout]{beamer}
%\documentclass{beamer}
\setbeamertemplate{footline}[frame number]

\usepackage[utf8x]{inputenc}
\usepackage[brazil,british]{babel}
\usetheme{default} 
\usecolortheme{beaver}
\newtranslation[to=brazil]{Theorem}{Teorema}
\newtranslation[to=brazil]{Definition}{Definição}
\newtranslation[to=brazil]{Example}{Exemplo}
\newtranslation[to=brazil]{Problem}{Exercício}
\newtranslation[to=brazil]{Solution}{Resolução}
\newtranslation[to=brazil]{Lemma}{Lema}
\newtranslation[to=brazil]{Corollary}{Corolário}
\logo{\includegraphics[width=1cm]{../img/logo-ppgsc-icon-text.png}}

\usepackage{graphicx}
\usepackage{clrscode3e}
\usepackage{hyperref}

\usepackage{pgf}
\usepackage{tikz}
\newcommand{\assert}[1]{\textcolor{blue}{#1}}

\usepackage{xspace}

\newcommand{\semester}[0]{2015.2}


\usepackage{pgf}
\usepackage{tikz}

\title{Aula 03: Análise de algoritmos --- melhor caso, pior caso e caso médio}
\author{David Déharbe \\
  Programa de Pós-graduação em Sistemas e Computação \\
  Universidade Federal do Rio Grande do Norte \\
  Centro de Ciências Exatas e da Terra \\
  Departamento de Informática e Matemáica Aplicada}
\date{23 de fevereiro de 2015}

\begin{document}
\selectlanguage{brazil}

\begin{frame}
  \titlepage
\end{frame}

\begin{frame}
  \frametitle{Plano da aula}
  \tableofcontents
\end{frame}

\section{Pior caso, melhor caso, caso médio, análise amortizada}

\begin{frame}
\frametitle{Pior caso, melhor caso, caso médio}

\begin{itemize}
\item A complexidade do algoritmo de \emph{ordenação por inserção} depende não
  só do tamanho da entrada, mas também do valor desta entrada.
\item A complexidade do algoritmo de \emph{busca linear} depende não só do
  tamanho da entrada, mas também do valor desta entrada.
\end{itemize}
\pause
\begin{description}
\item[pior caso] é a função que relaciona o tamanho da entrada $n$ ao maior tempo de execução possível para tratar uma entrada de tamanho $n$.
\item[melhor caso] é a função que relaciona o tamanho da entrada $n$ ao menor tempo de execução possível para tratar uma entrada de tamanho $n$.
\item[caso médio] é a função que relaciona o tamanho da entrada $n$ ao tempo médio de execução possível para tratar uma entrada de tamanho $n$, \emph{assumindo uma distribuição probabilística} das entradas possíveis.
\end{description}
\end{frame}

\begin{frame}
\frametitle{Pior caso}

\begin{block}{Como calcular?}
Determinar uma entrada, de tamanho $n$, tal que a operação básica é executada a maior quantidade de vezes possível.
\end{block}

\begin{block}{Exemplo}
\begin{codebox}
\Procname{$\proc{Linear-Search}(A, v)$}
\li $j \gets 1$
\li \While \textcolor{blue}{$A[j] \neq v$ and $j \le \id{length}(A)$}
\li \Do
      $j \gets j+1$
    \End
\li \If $j \le \id{length}(A)$
\li \Then
      \Return $j$
\li \Else
      \Return \const{nil}
    \End
\end{codebox}
No pior caso, $v \not\in A$ e o número de vezes que a operação básica é
executada é $n+1$.
\end{block}

\end{frame}

\begin{frame}
\frametitle{Melhor caso}

\begin{block}{Como calcular?}
Determinar uma entrada, de tamanho $n$, tal que a operação básica é executada a menor quantidade de vezes possível.
\end{block}

\begin{block}{Exemplo}
\begin{codebox}
\Procname{$\proc{Linear-Search}(A, v)$}
\li $j \gets 1$
\li \While \textcolor{blue}{$A[j] \neq v$ and $j \le \id{length}(A)$}
\li \Do
      $j \gets j+1$
    \End
\li \If $j \le \id{length}(A)$
\li \Then
      \Return $j$
\li \Else
      \Return \const{nil}
    \End
\end{codebox}
No melhor caso, $v = A[1]$ e o número de vezes que a operação básica é
executada é $1$.
\end{block}

\end{frame}

\begin{frame}
\frametitle{Complexidade média}

\begin{block}{Como calcular?}
  O cálculo é feito assumindo uma distribuição probabilística das entradas (ou
  classes de entradas) possíveis de tamanho $n$. É então realizada uma média
  ponderada do custo para aquela entrada (ou classe de entrada) com a
  probabilidade correspondente.
\end{block}

\begin{block}{Exemplo}
\begin{codebox}
\Procname{$\proc{Linear-Search}(A, v)$}
\li $j \gets 1$
\li \While \textcolor{blue}{$A[j] \neq v$ and $j \le \id{length}(A)$}
\li \Do
      $j \gets j+1$
    \End
\li \If $j \le \id{length}(A)$
\li \Then
      \Return $j$
\li \Else
      \Return \const{nil}
    \End
\end{codebox}
\end{block}
\end{frame}

\begin{frame}
\frametitle{Complexidade média (busca linear)}

Assume:
\begin{enumerate}
\item $0 \le p \le 1$ é a probabilidade de $v$ estar em $A$, 
\item $p/n$ é a probabilidade de $v$ estar em cada $A[i]$.
\end{enumerate}

\begin{itemize}
\item O custo do algoritmo quando $v$ está em $A[i]$ é $i$.
\item O custo do algoritmo quando $v$ não está em $A$ é $n+1$.
\end{itemize}

A complexidade média é:
\begin{eqnarray*}
(1-p).(n+1) + \sum_{i=1}^{n} i.\frac{p}{n}
& = & (1-p).(n+1) + \frac{p}{n}\times(\sum_{i=1}^{n} i) \\
& = & (1-p).(n+1) + \frac{p}{n}\times\frac{n.(n+1)}{2} \\
& = & (1-p).(n+1) + \frac{p.(n+1)}{2} \\
& = & \frac{(n+1).(2 - p)}{2}
\end{eqnarray*}

\end{frame}

\begin{frame}
\frametitle{Complexidade média (busca linear)}

$$\frac{(n+1).(2 - p)}{2}$$

Note que
\begin{itemize}
\item se $p=1$, a busca é sempre bem-sucedida, e o valor $\frac{n+1}{2}$ faz sentido.
\item se $p=0$, a busca é sempre mal-sucedida, e o valor $n+1$ faz sentido.
\end{itemize}

\end{frame}

\begin{frame}
\frametitle{Considerações sobre essas medidas de complexidade}

\begin{block}{Pontos chaves}
  \begin{itemize}
  \item A complexidade média é geralmente a mais importante, e a mais difícil
    de ser analisada.

  \item A complexidade no pior caso é também muito importante pois não raramente
    se aproxima da complexidade em média. Ela é essencial

  \item A complexidade no melhor caso não é tão importante, mas pode ser
    suficiente para \emph{descartar} um algoritmo.

  \item A complexidade média \alert{não} é a média da complexidade no pior caso
    e da complexidade no melhor caso.
  \end{itemize}
\end{block}

\end{frame}

\begin{frame}
\frametitle{Exercício}

Como um algoritmo de ordenação qualquer pode ser alterado de tal forma que seu
custo no melhor caso seja $n-1$ para ordenar uma sequência de tamanho $n$?

\end{frame}

\begin{frame}
\frametitle{Análise amortizada}

\begin{itemize}
\item A complexidade amortizada considera uma série de execuções de um algoritmo
sobre alguma \emph{estrutura de dados}.
\item Tem como objetivo determinar o custo médio por operação.
\item Não necessita de nenhuma distribuição probabilística.
\end{itemize}

\pause
Métodos de resolução
\begin{itemize}
\item método agregado \only<3>{\alert{$\leftarrow$}}
\item método do contador
\item método do potencial
\end{itemize}

\end{frame}

\begin{frame}

\frametitle{Método agregado}

\begin{itemize}
\item Considere uma série de $n$ operações em alguma estrutura de dados, $n$
  qualquer.

\item Determina o custo $T(n)$ desta série de operações.

\item O custo médio de cada operação é $T(n)/n$.

\item Exemplo: contador binário
\end{itemize}

\end{frame}

\begin{frame}

\frametitle{Método agregado: um exemplo}

  \begin{codebox}
    \Procname{$\proc{Increment}(A)$}
    \zi \Comment $A$ é um arranjo de $k$ bits
    \li $i \gets 0$
    \li \While $i < \id{length}[A] \mbox{ and } A[i] = 1$
    \li \Do 
    \li   $A[i] \gets 0$
    \li   $i \gets i + 1$
        \End
    \li \If $i < \id{length}[A]$
    \li \Then
    \li    $A[i] \gets 1$
        \End
  \end{codebox}

\begin{itemize}
\item O tamanho da entrada é $k$;
\item O custo da execução da $\proc{Increment}$ é proporcional ao número de bits
  invertidos.
\item No pior caso, a complexidade é proporcional a $k$;
\item Logo, uma sequência de $n$ operações é proporcional a $n \times k$.
\pause
\item Podemos prover uma análise mais precisa?
\end{itemize}

\end{frame}

\begin{frame}

\frametitle{Método agregado: um exemplo}

Em uma série de $n$ execuções de $\proc{Increment}$:
\begin{itemize}
\item $A[0]$ é invertido $n/2$ vezes;
\item $A[1]$ é invertido $n/4$ vezes;
\item $A[2]$ é invertido $n/8$ vezes;
\item $A[k-1]$ é invertido $n/2^{k-1}$ vezes;
\end{itemize}
O custo total para $n$ operações é 
\begin{eqnarray*}
T(n) & = & \sum_{i=1}^{\lfloor \lg n \rfloor} n/2^i \\
T(n) & < & \sum_{i=1}^{\infty} n/2^i \\
T(n) & < & n \times \sum_{i=1}^{\infty} 1/2^i = n
\end{eqnarray*}

Logo o custo amortizado de $\proc{Increment}$ é $T(n)/n = 1$: é \alert{constante}.

\end{frame}

\section{Conclusão}

\begin{frame}
\frametitle{Síntese}
\begin{itemize}
\item A complexidade temporal tanto quanto a complexidade espacial são
funções do tamanho da entrada.
\item A complexidade temporal, ou complexidade, é determinado pelo número
de vezes que a operação básica do algoritmo é executada.
\item A complexidade de um algoritmo pode variar consideravelmente para
entradas de mesmo tamanho. 

Nestes casos, deve-se distinguir melhor caso, pior caso, e complexidade média.
\item O arcabouço teórico da análise de algoritmos baseia-se na noção de
  crescimento assintótico de funções.
\end{itemize}
\end{frame}

\end{document}
