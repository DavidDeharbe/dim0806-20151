\documentclass{beamer}
\setbeamertemplate{footline}[frame number]
%\documentclass{beamer}

\usepackage[utf8x]{inputenc}
\usepackage[brazil,british]{babel}
\usetheme{default} 
\usecolortheme{beaver}
\newtranslation[to=brazil]{Theorem}{Teorema}
\newtranslation[to=brazil]{Definition}{Definição}
\newtranslation[to=brazil]{Example}{Exemplo}
\newtranslation[to=brazil]{Problem}{Exercício}
\newtranslation[to=brazil]{Solution}{Resolução}
\newtranslation[to=brazil]{Lemma}{Lema}
\newtranslation[to=brazil]{Corollary}{Corolário}
\logo{\includegraphics[width=1cm]{../img/logo-ppgsc-icon-text.png}}

\usepackage{graphicx}
\usepackage{clrscode3e}
\usepackage{hyperref}

\usepackage{pgf}
\usepackage{tikz}
\newcommand{\assert}[1]{\textcolor{blue}{#1}}

\usepackage{xspace}

\newcommand{\semester}[0]{2015.2}


\usepackage{tikz}
\usetikzlibrary{arrows,positioning, calc} 
\tikzstyle{vertex}=[draw,fill=black!15,circle,minimum size=18pt,inner sep=0pt]

\title{Aula 14: Estruturas de dados básicas, parte 2}
\author{David Déharbe \\
  Programa de Pós-graduação em Sistemas e Computação \\
  Universidade Federal do Rio Grande do Norte \\
  Centro de Ciências Exatas e da Terra \\
  Departamento de Informática e Matemática Aplicada}
\date{8 de abril de 2015}


\begin{document}
\selectlanguage{brazil}
\begin{frame}
  \titlepage
  Download me from \url{http://ufrn.academia.edu/DavidDeharbe}.
\end{frame}

\begin{frame}
  \frametitle{Plano da aula}
  \tableofcontents
\end{frame}

\section{Listas encadeadas}

\subsection{Introdução}

\begin{frame}
  \frametitle{Listas encadeadas}

  \begin{itemize}
  \item simples
  \item coleções homogêneas de dados
  \item muitas variações
    \begin{enumerate}
      \item ordenada ou não ordenadas
      \item simplesmente encadeada ou duplamente encadeada
      \item circular ou não circular
      \item sem sentinela ou com sentinela
    \end{enumerate}
  \item implementação:
    \begin{itemize}
      \item registros e ponteiros
      \item arranjos
    \end{itemize}
  \end{itemize}

\end{frame}

\subsection{Listas simplesmente encadeadas}

\begin{frame}
  \frametitle{Listas simplesmente encadeadas}
  \framesubtitle{Formalização 1/2}

  \begin{itemize}
    \item coleção homogênea: $\langle v_1, \ldots , v_n \rangle$
    \item possivelmente vazia: $n \ge 0$
  \end{itemize}

  \begin{itemize}
    \item a representação é um conjunto de \alert{células} de lista
    \item cada valor é armazenado em uma célula
      $$
      \forall i \mid 1 \le i \le n \cdot \attrib{\id{cell}(v_i)}{val} = v_i
      $$
    \item cada valor é armazenado em uma célula \alert{diferente}
      $$
      \forall i, j \mid 1 \le i < j \le n \cdot \id{cell}(v_i) \neq \id{cell}(v_j)
      $$
    \item as células são \alert{encadeadas}
      \begin{itemize}
      \item cada célula possui uma referência para a célula seguinte:
        $$
        \forall i \mid 1 \le i < n \cdot \attrib{\id{cell}(v_i)}{next} = \id{cell}(v_{i+1})
        $$
      \item com exceção da última que referência um valor especial:
        $$
        \attrib{\id{cell}(v_n)}{next} = \const{Nil}
        $$
      \end{itemize}
  \end{itemize}
  
\end{frame}

\begin{frame}
  \frametitle{Listas simplesmente encadeadas}
  \framesubtitle{Formalização 2/2}

  \begin{itemize}
    \item coleção homogênea: $\langle v_1, \ldots , v_n \rangle$
    \item possivelmente vazia: $n \ge 0$
  \end{itemize}

  Seja $\id{hd}$ a representação da coleção
  \begin{itemize}
    
    \item quando não vazia, o acesso à coleção é realizado pela primeira célula
      $$
      \id{hd} = \id{cell}(v_1)
      $$
    \item a coleção vazia é representada por um valor especial
      $$
      \id{hd} = \const{Nil}
      $$
  \end{itemize}
  
  Uma posição na coleção (iterador) na coleção é a célula que guarda o valor nesta
  posição.
\end{frame}

\begin{frame}
  \frametitle{Listas simplesmente encadeadas}
  \framesubtitle{Ilustração}

  $\langle 12, 99, 37 \rangle$

  \begin{center}
    \includegraphics[width=.7\textwidth]{fig/singly-linked-list.pdf}
  \end{center}

\end{frame}

\begin{frame}
  \frametitle{Listas simplesmente encadeadas}
  \framesubtitle{Implementação usando registros e ponteiros}

  \begin{center}
    \begin{tabular}{rl}
    \includegraphics[width=.7\textwidth]{fig/singly-linked-list.pdf}
    &
    \includegraphics[width=.25\textwidth]{fig/singly-linked-list-memory.pdf}
    \end{tabular}
  \end{center}

\end{frame}

\begin{frame}
  \frametitle{Processamento}
  \framesubtitle{Listas simplesmente encadeadas}
  
  \begin{codebox}
    \zi $\id{it} \gets \id{hd}$
    \zi \While $\id{it} \neq \const{Nil}$
    \zi \Do
          \Comment process $\attrib{\id{it}}{val}$
    \zi   $\id{it} \gets \attrib{\id{it}}{next}$
        \End
  \end{codebox}
\end{frame}

\begin{frame}
  \frametitle{Inserção}
  \framesubtitle{Listas simplesmente encadeadas}

  \begin{center}
    \includegraphics[width=.7\textwidth]{fig/singly-linked-list.pdf}
  \end{center}
  \begin{itemize}
    \item caso geral: \alert{após} uma dada posição
    \item caso especial: em primeira posição
  \end{itemize}
\end{frame}

\begin{frame}
  \frametitle{Inserção: caso geral}
  \framesubtitle{Listas simplesmente encadeadas}

  \begin{center}
    \begin{tabular}{ll}
      & \raisebox{-.5\height}{\includegraphics[width=.4\textwidth]{fig/singly-linked-list-insert-after-1.pdf}} \pause \\
      $\attrib{\id{c}}{next} \gets \attrib{\id{it}}{next}$ \pause 
      & \raisebox{-.5\height}{\includegraphics[width=.4\textwidth]{fig/singly-linked-list-insert-after-2.pdf}} \pause\\
      $\attrib{\id{it}}{next} \gets \id{c}$ \pause 
      & \raisebox{-.5\height}{\includegraphics[width=.4\textwidth]{fig/singly-linked-list-insert-after-3.pdf}}
    \end{tabular}
  \end{center}

\end{frame}

\begin{frame}
  \frametitle{Inserção: caso especial}
  \framesubtitle{Listas simplesmente encadeadas}

  \begin{center}
    \begin{tabular}{ll}
      & \raisebox{-.5\height}{\includegraphics[width=.4\textwidth]{fig/singly-linked-list-insert-first-1.pdf}} \pause \\
      $\attrib{\id{c}}{next} \gets \id{hd}$ \pause 
      & \raisebox{-.5\height}{\includegraphics[width=.4\textwidth]{fig/singly-linked-list-insert-first-2.pdf}} \pause\\
      $\id{hd} \gets \id{c}$ \pause 
      & \raisebox{-.5\height}{\includegraphics[width=.4\textwidth]{fig/singly-linked-list-insert-first-3.pdf}}
    \end{tabular}
  \end{center}

\end{frame}

\begin{frame}
  \frametitle{Remoção}
  \framesubtitle{Listas simplesmente encadeadas}

  \begin{center}
    \includegraphics[width=.7\textwidth]{fig/singly-linked-list.pdf}
  \end{center}
  \begin{itemize}
    \item caso geral: \alert{após} uma dada posição
    \item caso especial: em primeira posição
  \end{itemize}
\end{frame}

\begin{frame}
  \frametitle{Remoção: caso geral}
  \framesubtitle{Listas simplesmente encadeadas}

  \begin{center}
    \begin{tabular}{ll}
      & \raisebox{-.5\height}{\includegraphics[width=.4\textwidth]{fig/singly-linked-list-remove-after-1.pdf}} \pause \\
      $\attrib{\id{it}}{next} \gets \attrib{\attrib{\id{it}}{next}}{next}$ \pause 
      & \raisebox{-.5\height}{\includegraphics[width=.4\textwidth]{fig/singly-linked-list-remove-after-2.pdf}}
    \end{tabular}
  \end{center}
  \pause
  \begin{itemize}
    \item se for necessário gerenciar os recursos de memória:
  \begin{codebox}
    \zi $\id{tmp} \gets \attrib{\id{it}}{next}$
    \zi $\attrib{\id{it}}{next} \gets \attrib{\attrib{\id{it}}{next}}{next}$
    \zi $\proc{Free}(\id{tmp})$
  \end{codebox}

  \end{itemize}

\end{frame}

\begin{frame}
  \frametitle{Remoção: caso especial}
  \framesubtitle{Listas simplesmente encadeadas}

  \begin{center}
    \begin{tabular}{ll}
      & \raisebox{-.5\height}{\includegraphics[width=.4\textwidth]{fig/singly-linked-list-remove-first-1.pdf}} \pause \\
      $\id{hd} \gets \attrib{\id{hd}}{next}$ \pause 
      & \raisebox{-.5\height}{\includegraphics[width=.4\textwidth]{fig/singly-linked-list-remove-first-2.pdf}}
    \end{tabular}
  \end{center}

  \pause
  \begin{itemize}
    \item se for necessário gerenciar os recursos de memória:
  \begin{codebox}
    \zi $\id{tmp} \gets \id{hd}$
    \zi $\id{hd} \gets \attrib{\id{hd}}{next}$
    \zi $\proc{Free}(\id{tmp})$
  \end{codebox}

  \end{itemize}

\end{frame}

\begin{frame}

  \frametitle{Exercícios}
  \framesubtitle{Listas encadeadas}

  \begin{itemize}
    \item Escrever um algoritmo de inserção
      \begin{itemize}
        \item assumindo a lista não ordenada
        \item assumindo a lista ordenada
      \end{itemize}
    \item Escrever um algoritmo de remoção
      \begin{itemize}
        \item assumindo a lista não ordenada
        \item assumindo a lista ordenada
      \end{itemize}
    \item Escrever um algoritmo de busca
      \begin{itemize}
        \item assumindo a lista não ordenada
        \item assumindo a lista ordenada
      \end{itemize}
  \end{itemize}

\end{frame}

\subsection{Listas simplesmente encadeadas circulares}

\begin{frame}

  \frametitle{Listas simplesmente encadeadas circulares}

  \begin{itemize}

    \item a última célula referência a primeira
      $$
      \attrib{\id{cell}(v_n)}{next} = \id{cell}(v_1)
      $$
  \end{itemize}

\end{frame}

\begin{frame}
  \frametitle{Listas simplesmente encadeadas circulares}
  \framesubtitle{Implementação usando registros e ponteiros}

  \begin{center}
    \begin{tabular}{rl}
    \includegraphics[width=.7\textwidth]{fig/singly-linked-list-circular.pdf}
    &
    \includegraphics[width=.25\textwidth]{fig/singly-linked-list-circular-memory.pdf}
    \end{tabular}
  \end{center}

\end{frame}

\begin{frame}
  \frametitle{Processamento}
  \framesubtitle{Listas simplesmente encadeadas circulares}
  
  \begin{codebox}
    \zi \If $\id{hd} \neq \const{Nil}$
    \zi \Then $\id{it} \gets \id{hd}$
    \zi   \Repeat
    \zi     \Comment process $\attrib{\id{it}}{val}$
    \zi     $\id{it} \gets \attrib{\id{it}}{next}$
    \zi   \Until $\id{it} \isequal \id{hd}$
        \End
  \end{codebox}
\end{frame}

\begin{frame}
  \frametitle{Exercícios}
  \framesubtitle{Listas simplesmente encadeadas circulares}

  Adapte os algoritmos de inserção, remoção e busca desenvolvidos para as
  listas simplesmente encadeadas circulares.

\end{frame}

\begin{frame}

  \frametitle{Listas duplamente encadeadas}

  \begin{itemize}

    \item acesso direto à posição anterior

    \item remoção do item na posição 

  \end{itemize}

  \begin{center}
    \includegraphics[width=.7\textwidth]{fig/doubly-linked-list.pdf}
  \end{center}

\end{frame}

\begin{frame}

  \frametitle{Implementação usando registros e ponteiros}
  \framesubtitle{Listas duplamente encadeadas}

  \begin{center}
    \begin{tabular}{cc}
    \raisebox{-.5\height}{\includegraphics[width=.7\textwidth]{fig/doubly-linked-list.pdf}}
    &
    \raisebox{-.5\height}{\includegraphics[width=.25\textwidth]{fig/doubly-linked-list-memory.pdf}}
    \end{tabular}
  \end{center}

\end{frame}

\subsection{Listas duplamente encadeadas}

\begin{frame}
  \frametitle{Formalização}
  \framesubtitle{Listas duplamente encadeadas}

  \begin{itemize}
    \item as células são \alert{duplamente} encadeadas
      \begin{itemize}
      \item cada célula possui uma referência para a célula anterior:
        $$
        \forall i \mid 1 < i \le n \cdot \attrib{\id{cell}(v_i)}{prev} = \id{cell}(v_{i-1})
        $$
      \item com exceção da primeira que referência um valor especial:
        $$
        \attrib{\id{cell}(v_1)}{prev} = \const{Nil}
        $$
      \end{itemize}
  \end{itemize}
  
\end{frame}

\begin{frame}
  \frametitle{Inserção após a posição atual}
  \framesubtitle{Listas duplamente encadeadas}

  \begin{tabular}{cc}
    & \raisebox{-.5\height}{\includegraphics[width=.7\textwidth]{fig/doubly-linked-list-insert-after-1.pdf}} \\
    $\attrib{\id{c}}{next} \gets \attrib{\id{it}}{next}$ & 
    \raisebox{-.5\height}{\includegraphics[width=.7\textwidth]{fig/doubly-linked-list-insert-after-2.pdf}}
  \end{tabular}
\end{frame}

\begin{frame}
  \frametitle{Inserção após a posição atual}
  \framesubtitle{Listas duplamente encadeadas}

  \begin{tabular}{cc}
    $\attrib{\id{c}}{next} \gets \attrib{\id{it}}{next}$ & 
    \raisebox{-.5\height}{\includegraphics[width=.7\textwidth]{fig/doubly-linked-list-insert-after-2.pdf}} \pause \\
    $\attrib{\id{c}}{prev} \gets \id{it}$ & 
    \raisebox{-.5\height}{\includegraphics[width=.7\textwidth]{fig/doubly-linked-list-insert-after-3.pdf}}
  \end{tabular}
\end{frame}

\begin{frame}
  \frametitle{Inserção após a posição atual}
  \framesubtitle{Listas duplamente encadeadas}

  \begin{tabular}{cc}
    $\attrib{\id{c}}{next} \gets \attrib{\id{it}}{next}$ & \\
    $\attrib{\id{c}}{prev} \gets \id{it}$ & 
    \raisebox{-.5\height}{\includegraphics[width=.7\textwidth]{fig/doubly-linked-list-insert-after-3.pdf}} \\
    $\attrib{\id{it}}{next} \gets \id{c}$ & 
    \raisebox{-.5\height}{\includegraphics[width=.7\textwidth]{fig/doubly-linked-list-insert-after-4.pdf}}
  \end{tabular}
\end{frame}

\begin{frame}
  \frametitle{Inserção após a posição atual}
  \framesubtitle{Listas duplamente encadeadas}

  \begin{tabular}{cc}
    $\attrib{\id{c}}{next} \gets \attrib{\id{it}}{next}$ & \\
    $\attrib{\id{c}}{prev} \gets \id{it}$ & \\ 
    $\attrib{\id{it}}{next} \gets \id{c}$ & 
    \raisebox{-.5\height}{\includegraphics[width=.7\textwidth]{fig/doubly-linked-list-insert-after-4.pdf}} \\
    \begin{tabular}{l}
      \If $\attrib{\id{c}}{next} \neq \const{Nil}$ \\
      \quad $\attrib{\attrib{\id{c}}{next}}{prev} \gets \id{c}$ 
    \end{tabular} & 
    \raisebox{-.5\height}{\includegraphics[width=.7\textwidth]{fig/doubly-linked-list-insert-after-5.pdf}} \\
  \end{tabular}
\end{frame}

\begin{frame}
  \frametitle{Inserção após a posição atual}
  \framesubtitle{Listas duplamente encadeadas}

  \begin{codebox}
    \zi $\attrib{\id{c}}{next} \gets \attrib{\id{it}}{next}$
    \zi $\attrib{\id{c}}{prev} \gets \id{it}$
    \zi $\attrib{\id{it}}{next} \gets \id{c}$
    \zi \If $\attrib{\id{c}}{next} \neq \const{Nil}$ 
    \zi \Then $\attrib{\attrib{\id{c}}{next}}{prev} \gets \id{c}$ 
    \End
  \end{codebox}

\end{frame}

\begin{frame}
  \frametitle{Inserção antes da posição atual}
  \framesubtitle{Listas duplamente encadeadas}

  \begin{codebox}
    \zi \only<1>{$\attrib{\id{c}}{\textcolor{gray}{next}} \gets \attrib{\id{it}}{\textcolor{gray}{next}}$}
    \only<2->{$\attrib{\id{c}}{prev} \gets \attrib{\id{it}}{prev}$}
    \zi \only<1-2>{$\attrib{\id{c}}{\textcolor{gray}{prev}} \gets \id{it}$}
    \only<3->{$\attrib{\id{c}}{next} \gets \id{it}$}
    \zi \only<1-3>{$\attrib{\id{it}}{\textcolor{gray}{next}} \gets \id{c}$}
    \only<4->{$\attrib{\id{it}}{prev} \gets \id{c}$}
    \zi \If \only<1-4>{$\attrib{\id{c}}{\textcolor{gray}{next}} \neq \const{Nil}$}
            \only<5->{$\attrib{\id{c}}{prev} \neq \const{Nil}$}
    \zi \Then \only<1-4>{$\attrib{\attrib{\id{c}}{\textcolor{gray}{next}}}{\textcolor{gray}{prev}} \gets \id{c}$}
            \only<5->{$\attrib{\attrib{\id{c}}{prev}}{next} \gets \id{c}$}
        \End
    \only<6->{
    \zi \If $\id{it} \isequal \id{hd}$
    \zi \Then  $\id{hd} \gets \id{c}$
        \End
    }
  \end{codebox}

  \only<1>{\includegraphics[width=.7\textwidth]{fig/doubly-linked-list-insert-before-1.pdf}}
  \only<2>{\includegraphics[width=.7\textwidth]{fig/doubly-linked-list-insert-before-2.pdf}}
  \only<3>{\includegraphics[width=.7\textwidth]{fig/doubly-linked-list-insert-before-3.pdf}}
  \only<4>{\includegraphics[width=.7\textwidth]{fig/doubly-linked-list-insert-before-4.pdf}}
  \only<5>{\includegraphics[width=.7\textwidth]{fig/doubly-linked-list-insert-before-5.pdf}}

\end{frame}

\subsection{Listas duplamente encadeadas circulares}

\begin{frame}

  \frametitle{Listas duplamente encadeadas circulares}

  \begin{itemize}

    \item a primeira célula referência a última
      $$
      \attrib{\id{cell}(v_1)}{prev} = \id{cell}(v_n)
      $$
  \end{itemize}

\end{frame}

\begin{frame}
  \frametitle{Implementação usando registros e ponteiros}
  \framesubtitle{Listas duplamente encadeadas circulares}

  \begin{center}
    \begin{tabular}{rl}
    \includegraphics[width=.7\textwidth]{fig/doubly-linked-list-circular.pdf}
    &
    \includegraphics[width=.25\textwidth]{fig/doubly-linked-list-circular-memory.pdf}
    \end{tabular}
  \end{center}

\end{frame}

\begin{frame}
  \frametitle{Inserção}
  \framesubtitle{Listas duplamente encadeadas circulares}

  \begin{itemize}
    \item Todas as células possuem sucessor e predecessor
    \item Algoritmos mais simples:
    \item Inserção de $\id{c}$ após a posição $\id{it}$
      \begin{codebox}
        \zi $\attrib{\id{c}}{next} \gets \attrib{\id{it}}{next}$
        \zi $\attrib{\id{c}}{prev} \gets \id{it}$
        \zi $\attrib{\id{it}}{next} \gets \id{c}$
        \zi $\attrib{\attrib{\id{c}}{next}}{prev} \gets \id{c}$
      \end{codebox}
    \item Inserção de $\id{c}$ antes da posição $\id{it}$
      \begin{codebox}
        \zi $\attrib{\id{c}}{prev} \gets \attrib{\id{it}}{prev}$
        \zi $\attrib{\id{c}}{next} \gets \id{it}$
        \zi $\attrib{\id{it}}{prev} \gets \id{c}$
        \zi $\attrib{\attrib{\id{c}}{prev}}{next} \gets \id{c}$
      \end{codebox}
      \pause
    \item Exercício: identifique as dependências entre os quatro comandos formando cada operação de inserção.
  \end{itemize}
  
\end{frame}

\begin{frame}
  \frametitle{Remoção}
  \framesubtitle{Listas duplamente encadeadas circulares}

  \begin{itemize}
    \item Remoção da célula na posição $\id{it}$
      \begin{codebox}
        \zi \Comment quando a lista tem mais de um elemento
        \zi \If $\attrib{\id{it}}{next} \neq \id{it}$
        \zi \Then 
              $\attrib{\attrib{\id{c}}{next}}{prev} \gets \attrib{\id{c}}{prev}$
        \zi   $\attrib{\attrib{\id{c}}{prev}}{next} \gets \attrib{\id{c}}{next}$
        \zi   \If $\id{it} \isequal \id{hd}$
        \zi   \Then
                $\id{hd} \gets \attrib{\id{it}}{next}$
              \End
        \zi \Comment quando a lista tem um único elemento
        \zi \Else
              $\id{hd} \gets \const{Nil}$
            \End
        \zi \textcolor{gray}{$\proc{Free}(\id{it})$}
      \end{codebox}
  \end{itemize}
  
\end{frame}

\subsection{Célula sentinela}

\begin{frame}

  \frametitle{Elemento sentinela}

  \begin{itemize}
    \item célula que não contem valor
    \item localizada no início ou no fim da lista
    \item simplifica os algoritmos de algumas operações
    \item consumo extra de memória
  \end{itemize}

\end{frame}

\begin{frame}

  \frametitle{Formalização}
  \framesubtitle{Elemento sentinela}
  
  
  \begin{itemize}
    \item $l = \langle v_1, \ldots, v_n \rangle$
    \item $\forall i \cdot \id{cell}(v_i) \neq \id{nil}(l)$
    \item se $n \ge 1$
      \begin{itemize}
        \item $\attrib{\id{nil}(l)}{next} = \id{cell}(v_1)$
        \item $\attrib{\id{nil}(l)}{prev} = \id{cell}(v_n)$
        \item $\attrib{\id{cell}(v_1)}{prev} = \id{nil}(l)$
        \item $\attrib{\id{cell}(v_n)}{next} = \id{nil}(l)$
      \end{itemize}
    \item se $n = 0$
      \begin{itemize}
        \item $\attrib{\id{nil}(l)}{next} = \id{nil}(l)$
        \item $\attrib{\id{nil}(l)}{prev} = \id{nil}(l)$
      \end{itemize}
  \end{itemize}

\end{frame}

\begin{frame}

  \frametitle{Ilustração}
  \framesubtitle{Elemento sentinela}
  

  \begin{tabular}{cc}
  $l = \langle 12, 99, 37 \rangle$
    &
    \raisebox{-.5\height}{\includegraphics[width=.7\textwidth]{fig/doubly-linked-list-sentinel-circular.pdf}}
    \\
    \hline
    $l = \langle \rangle$
    &
    \raisebox{-.5\height}{\includegraphics[width=.4\textwidth]{fig/doubly-linked-list-sentinel-circular-empty.pdf}}
  \end{tabular}

\end{frame}

\begin{frame}
  \frametitle{Processamento}
  \framesubtitle{Listas duplamente encadeadas circulares com sentinela}

  \begin{codebox}
    \zi $\id{it} \gets \attrib{\id{Nil}(l)}{next}$
    \zi \While $\id{it} \neq \id{Nil}(l)$
    \zi   \Do \Comment Processar $\attrib{\id{it}}{val}$
    \zi       $\id{it} \gets \attrib{\id{it}}{next}$
          \End
  \end{codebox}

\end{frame}

\begin{frame}
  \frametitle{Remoção de um item na posição atual}
  \framesubtitle{Listas duplamente encadeadas circulares com sentinela}

  \begin{codebox}
    \zi \only<2->{$\attrib{\attrib{\id{it}}{prev}}{next} \gets \attrib{\id{it}}{next}$}
    \zi \only<3->{$\attrib{\attrib{\id{it}}{next}}{prev} \gets \attrib{\id{it}}{prev}$}
    \zi \only<4->{$\proc{Free}(\id{it})$}
  \end{codebox}

  \only<1>{\includegraphics[width=.7\textwidth]{fig/doubly-linked-list-sentinel-circular-remove-1.pdf}}
  \only<2>{\includegraphics[width=.7\textwidth]{fig/doubly-linked-list-sentinel-circular-remove-2.pdf}}
  \only<3>{\includegraphics[width=.7\textwidth]{fig/doubly-linked-list-sentinel-circular-remove-3.pdf}}
  \only<4>{\includegraphics[width=.7\textwidth]{fig/doubly-linked-list-sentinel-circular-remove-4.pdf}}

\end{frame}

\begin{frame}

  \frametitle{Exercícios}

  \begin{itemize}
    \item Identifique qual tipo de lista é mais adequado para representar pilhas, filas e deques.
    \item Projete um algoritmo que, dada duas listas, concatena o conteúdo da segunda no fim da primeira.
      \begin{itemize}
        \item Escolhe o tipo de lista que considerar mais adequado.
      \end{itemize}
  \end{itemize}
\end{frame}

\section{Listas encadeadas e arranjos}

\begin{frame}

  \frametitle{Listas encadeadas e arranjos}

  \begin{itemize}
    \item O princípio de encadeamento não precisa ser restrito a ponteiros.
    \item Os atributos $\id{next}$ (e $\id{prev}$) podem ser índices de um arranjo.
    \item Observação:
      \begin{itemize}
        \item a memória do computador nada mais é que um arranjo, onde
        \item os ponteiros são os índices deste arranjo.
      \end{itemize}
    \item Nem todas as posições do arranjo são ocupadas
      \begin{itemize}
        \item manter uma lista de posições livres.
      \end{itemize}
  \end{itemize}
\end{frame}

\begin{frame}

  \frametitle{Ilustração: lista simplesmente encadeada}
  \framesubtitle{Listas encadeadas e arranjos}

  $\begin{array}{ccc}
    \langle 37, 12, 99 \rangle &
    \begin{array}{c||c|c|c||}
      \mbox{índice} & \mbox{valor} & \mbox{next} \\
      \cline{2-3}
      1 & & 8 \\
      \cline{2-3}
      2 & & 1 \\
      \cline{2-3}
      3 & 12 & 7\\
      \cline{2-3}
      4 & & 5 \\
      \cline{2-3}
      5 & & 0 \\
      \cline{2-3}
      6 & 37 & 3 \\
      \cline{2-3}
      7 & 99 & 0 \\
      \cline{2-3}
      8 & & 4 \\
      \cline{2-3}
  \end{array}
    &
    \begin{array}{c}
      \id{hd} = 6 \\
      \id{free} = 2
    \end{array}
  \end{array}$

  \begin{codebox}
    \zi \If $\id{free} \isequal 0$
    \zi \Then \Return
      \End
    \zi $\id{n} \gets \id{free}$
    \zi $\id{free} \gets \attrib{A[\id{free}]}{next}$
    \zi $\attrib{A[n]}{next} \gets \id{hd}$
    \zi $\attrib{A[n]}{value} \gets \id{v}$
    \zi $\id{hd} \gets \id{n}$
  \end{codebox}

\end{frame}

\begin{frame}

  \frametitle{Exercício}

  \begin{itemize}
    \item Escrever algoritmo para calcular tamanho de uma lista
    \item Escrever algoritmo para remover um valor
    \item Escrever um algoritmo para compactar a representação de uma lista em um
      arranjo: ao término, as posições ocupadas devem estar nas posições 1 até $i$.
  \end{itemize}
\end{frame}

\end{document}

