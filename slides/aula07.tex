\documentclass{beamer}
\setbeamertemplate{footline}[frame number]
%\documentclass{beamer}

\usepackage[utf8x]{inputenc}
\usepackage[brazil,british]{babel}
\usetheme{default} 
\usecolortheme{beaver}
\newtranslation[to=brazil]{Theorem}{Teorema}
\newtranslation[to=brazil]{Definition}{Definição}
\newtranslation[to=brazil]{Example}{Exemplo}
\newtranslation[to=brazil]{Problem}{Exercício}
\newtranslation[to=brazil]{Solution}{Resolução}
\newtranslation[to=brazil]{Lemma}{Lema}
\newtranslation[to=brazil]{Corollary}{Corolário}
\logo{\includegraphics[width=1cm]{../img/logo-ppgsc-icon-text.png}}

\usepackage{graphicx}
\usepackage{clrscode3e}
\usepackage{hyperref}

\usepackage{pgf}
\usepackage{tikz}
\newcommand{\assert}[1]{\textcolor{blue}{#1}}

\usepackage{xspace}

\newcommand{\semester}[0]{2015.2}


\title{Aula 07: Algoritmos de busca em arranjos lineares}
\author{David Déharbe \\
  Programa de Pós-graduação em Sistemas e Computação \\
  Universidade Federal do Rio Grande do Norte \\
  Centro de Ciências Exatas e da Terra \\
  Departamento de Informática e Matemática Aplicada}
\date{}

\begin{document}
\selectlanguage{brazil}
\begin{frame}
  \titlepage
\end{frame}

\section{Introdução}

\begin{frame}

  \frametitle{Algoritmos}
  \framesubtitle{Busca em arranjo}

  \begin{description}
    \item[Entrada]
      \begin{itemize}
        \item uma sequência linear $A = \langle A_1, \dots A_n \rangle$,
        \item um valor $v$
      \end{itemize}
    \item[Saída] 
      \begin{itemize}
        \item se $\exists j \mid 1 \le j \le n \cdot A_j = v$ então o menor $1 \le i \le n$ tal que $A_i = v$, ou
        \item se $\forall j \mid 1 \le j \le n \cdot A_j \neq v$ então $\const{nil}$.
      \end{itemize}

    \item[Algoritmos]
      \begin{itemize}
      \item busca linear em arranjo 
      \item busca em arranjo ordenado
        \begin{itemize}
        \item busca linear
        \item busca binária
        \end{itemize}
      \end{itemize}

  \end{description}

\end{frame}

\section{Busca linear}

\begin{frame}
  \frametitle{Algoritmo de busca linear}

\begin{codebox}
\Procname{$\proc{Linear-Search}(A, v)$}
\zi \Comment \assert{$\{ A = \langle a_1, a_2, \ldots, a_n \rangle \}$}
\li $j \gets 1$
\li \While $A[j] \neq v$ and $j \le \id{length}(A)$
\li \Do
      $j \gets j+1$
    \End
\li \If $j \le \id{length}(A)$
\li \Then
      \Return $j$
\li \Else
      \Return \const{nil}
    \End
\zi \Comment \assert{$\{ \begin{array}[t]{l}
      \left(1 \le r \le n \Leftrightarrow a_r = v \land \left(\forall i : 1 \le i < r \Rightarrow a_i \neq v \right) \right) \land \\
      \left(r = \const{nil} \Leftrightarrow \left(\forall i : 1 \le i \le n \Rightarrow a_i \neq v \right)\right) \}
      \end{array}$}
\end{codebox}  

\end{frame}

\begin{frame}
  \frametitle{Algoritmo de busca linear}

\begin{itemize}

  \item Complexidade no melhor caso

  \item Complexidade no pior caso

  \item Complexidade no caso médio

\end{itemize}

\end{frame}

\begin{frame}
  \frametitle{Algoritmo de busca linear}
  \framesubtitle{Assumindo o arranjo ordenado}

\begin{codebox}
\Procname{$\proc{Linear-Search}(A, v)$}
\zi \Comment \assert{$\{ A = \langle a_1, a_2, \ldots, a_n \rangle \land
\left( \forall i \mid 1 \le i < n \cdot a_i \le a_{i+1} \right) \} $}
\li $j \gets 1$
\li \While $j \le \id{length}(A)$ and $A[j] \alert{<} v$ 
\li \Do
      $j \gets j+1$
    \End
\li \If $j \le \id{length}(A)$ \alert{and $A[j] = v$}
\li \Then
      \Return $j$
\li \Else
      \Return \const{nil}
    \End
\zi \Comment \assert{$\{ \begin{array}[t]{l}
      \left(1 \le r \le n \Leftrightarrow a_r = v \land \left(\forall i : 1 \le i < r \Rightarrow a_i \neq v \right) \right) \land \\
      \left(r = \const{nil} \Leftrightarrow \left(\forall i : 1 \le i \le n \Rightarrow a_i \neq v \right)\right) \}
      \end{array}$}
\end{codebox}  

\end{frame}

\begin{frame}
  \frametitle{Algoritmo não recursivo de busca binária em arranjo ordenado}
  \framesubtitle{Assumindo o arranjo ordenado}

\begin{small}
\begin{codebox}
\Procname{$\proc{Binary-Search}(A, v)$}
\zi \Comment \assert{$\{ A = \langle a_1, a_2, \ldots, a_n \rangle \land
\left( \forall i \mid 1 \le i < n \cdot a_i \le a_{i+1} \right) \} $}
\li $l \gets 1$
\li $u \gets \id{length}(A)$
\li $m \gets \lfloor (\id{length}(A)/2 \rfloor$
\li \While $l \le u$ and $v \neq A[m]$
\li \Do
       \If $v < A[m]$
\li    \Then
         $u \gets m-1$
\li    \Else
         $l \gets m+1$
       \End
\only<1-2>{\li    $m \gets (u + l)/2$} \only<2>{\alert{$\Longleftarrow$ Bug (aritmética em hardware)}}
\only<3>{\li $m \gets l + \lfloor (u + 1 - l)/2 \rfloor$}
    \End
\li \If $v = A[m]$
\li \Then
      \Return $m$
\li \Else
      \Return \const{nil}
    \End
\zi \Comment \assert{$\{ \begin{array}[t]{l}
      \left(1 \le r \le n \Leftrightarrow a_r = v \right) \land \\
      \left(r = \const{nil} \Leftrightarrow \left(\forall i : 1 \le i \le n \Rightarrow a_i \neq v \right)\right) \}
      \end{array}$}
\end{codebox}  
\end{small}
\end{frame}

\begin{frame}
  \frametitle{Algoritmo não recursivo de busca binária}

\begin{itemize}

  \item Complexidade no melhor caso

  \item Complexidade no pior caso

\end{itemize}

\end{frame}

\begin{frame}
  \frametitle{Algoritmo recursivo de busca binária}
  \framesubtitle{Assumindo o arranjo ordenado}

\begin{codebox}
\Procname{$\proc{Binary-Search}(A, v)$}
\zi \Comment \assert{$\{ A = \langle a_1, a_2, \ldots, a_n \rangle \land
\left( \forall i \mid 1 \le i < n \cdot a_i \le a_{i+1} \right) \} $}
\li \Return $\proc{Binary-Search-Range}(A, v, 1, \id{length}(A))$
\zi \Comment \assert{$\{ \begin{array}[t]{l}
      \left(1 \le r \le n \Leftrightarrow a_r = v \right) \land \\
      \left(r = \const{nil} \Leftrightarrow \left(\forall i : 1 \le i \le n \Rightarrow a_i \neq v \right)\right) \} \end{array}$}
\end{codebox}  

\end{frame}

\begin{frame}
  \frametitle{Algoritmo de busca binária}
  \framesubtitle{Assumindo o arranjo ordenado}
\begin{small}
\begin{codebox}
\Procname{$\proc{Binary-Search-Range}(A, v, l, u)$}
\zi \Comment \assert{$\begin{array}[t]{l}
\{ A = \langle a_1, a_2, \ldots, a_n \rangle \land \left( \forall i \mid 1 \le i < n \cdot a_i \le a_{i+1} \right) \} \land \\
1 \le u, l \le n \end{array}$}
\li \If $l > u$
\li \Then
      \Return \const{nil}
\li \Else
\li   $m \gets l + \lfloor (u + 1 - l)/2 \rfloor$
\li   \If $v = A[m]$
\li   \Then \Return $m$
\li   \ElseIf $v < A[m]$
\li   \Then \Return $\proc{Binary-Search-Range}(A, v, l, m-1)$
\li   \ElseNoIf
\li \Return $\proc{Binary-Search-Range}(A, v, m+1, u)$
      \End
    \End
\zi \Comment \assert{$\{ \begin{array}[t]{l}
      \left(l \le r \le u \Leftrightarrow a_r = v \right) \land \\
      \left(r = \const{nil} \Leftrightarrow \left(\forall i : l \le i \le u \Rightarrow a_i \neq v \right)\right)  \} \end{array}$}
\end{codebox}  
\end{small}
\end{frame}

\begin{frame}
  \frametitle{Algoritmo de busca binária}
  \framesubtitle{Assumindo o arranjo ordenado}

\begin{itemize}

  \item Complexidade no melhor caso

  \item Complexidade no pior caso

    \begin{itemize}

    \item $T(n) = T(n/2) + f(n)$, onde $f \in \Theta(1)$.

    \item Teorema Master, caso 2

    \item $T(n) \in \Theta(\log n)$

    \end{itemize}

\end{itemize}

\end{frame}

\begin{frame}

\frametitle{Exercício}

Considere o problema de contar o número de ocorrências de um valor $v$ em uma sequência $A$.

\begin{enumerate}

\item Assume $A$ não ordenado. 
\begin{enumerate}
\item Escreva um algoritmo que solucione o problema. 
\item Determine a complexidade no melhor caso, e no pior caso.
\end{enumerate}
\item Assume $A$ ordenado.
\begin{enumerate}
\item Adapte o algoritmo anterior para levar esta hipótese em conta.
\item Determine a complexidade no melhor caso, e no pior caso.
\item Escreva um novo algoritmo, inspirado do algoritmo da busca binária.
\item Determine a complexidade, no pior caso, do novo algoritmo.
\end{enumerate}
\end{enumerate}


\end{frame}

\end{document}
