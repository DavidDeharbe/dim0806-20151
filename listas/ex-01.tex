\documentclass[a4wide]{article}

\usepackage[utf8]{inputenc}
\usepackage[brazil]{babel}

\title {DIM0806 - Estruturas de Dados e Algoritmos \\
  Lista de exercícios 1}
\author{David Déharbe\thanks{\textbf{Fonte:} \textit{Introduction to Algorithms}. Thomas H. Cormen, 
Charles E. Leiserson, Ronald R. Rivest. Capítulo 1, seções 1 e 2.
}}

\newcounter{ExCounter}
\newcommand{\exercicio}[0]{%
\stepcounter{ExCounter}
\paragraph{ex \theExCounter.}
}

\begin{document}

\maketitle


\exercicio\label{ex:algo-poli}

Considere o problema de calcular o valor de uma função polinomial em 
um dado ponto. Dados $n$ coeficientes $a_0, a_1, \ldots a_{n-1}$ e um número
real $x$, queremos calcular $\sum_{i=0}^{n-1} a_i x^i$. 

Escreva um algoritmo em pseudo-código para este problema.

\exercicio\label{ex:algo-selection}

Queremos ordenar $N$ números armazenados em um arranjo $A$ da seguinte
forma.  Primeiro encontre o menor número em $A$ e coloque ele na
primeira posição. Em seguida, encontre o segundo menor número em $A$,
e coloque ele na segunda menor posição. Repita para cada um dos $N$
elementos de $A$.

Escreva um algoritmo em pseudo-código para este algoritmo, conhecido
como ordenação por seleção (\textit{selection sort}).

\exercicio

Expresse a função $n^{3}/1000 - 100n^{2} - 100n + 3$ usando a notação $\Theta$.

\exercicio

Como podemos modificar qualquer algoritmo para ter um tempo de
execução no melhor caso da ordem de $\Theta(n)$, onde $n$ é o tamanho
da entrada?

\exercicio

Considere o algoritmo em pseudo-código que você desenvolveu no exercício~2. Dê o
tempo de execução deste algoritmo no melhor caso e no pior caso utilizando a
notação $\Theta$.

\exercicio

Considere o algoritmo da busca linear dos slides. Quantos elementos da
sequência de entrada devem ser testados em média, assumindo que o
elemento sendo pesquisado tem a mesma probabilidade de estar em
qualquer posição da sequência? O que caracteriza o pior caso, e qual
a quantidade de comparações necessárias?

Quais são o pior tempo e o tempo médio de execução do algoritmo
utilizando a notação $\Theta$? Justifique.

\exercicio

Considere o problema de determinar se há números repetidos em uma
sequência qualquer (não necessariamente ordenada). 

Mostre que isto pode ser realizado em $\Theta(n \lg n)$.

\exercicio

Expresse o tempo de execução do algoritmo que você desenvolveu na
questão~1 utilizando a notação
$\Theta$?\footnote{Assumimos que os números são implantados com uma
  representação de ponto-flutuante, e que o custo de cada operação
  aritmética é constante.}.

Considere agora a seguinte forma de calcular o mesmo resultado:
\begin{eqnarray*}
\sum_{i=0}^{n-1} a_i x^i & = & (\cdots(a_{n-1}x + a_{n-2})x + \cdots + a_1)x + a_0.
\end{eqnarray*}
Utilize este método (apelidado de regra de Horner) para escrever um
algoritmo em pseudo-código para este mesmo problema, cujo tempo de
execução é $\Theta(n)$.

\end{document}
